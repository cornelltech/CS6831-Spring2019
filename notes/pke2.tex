%!TEX root = main.tex
%%%%%%%%%%%%%%%%%%%%%%%%%%%%%%%%%%%%%%%%%%%%%%%%%%%%%%%%%%%%%%%%%%%%%%%%%%%%%%%%
\section{Key Exchange and Public Key Tools}
\label{sec:pke2}

\scribenote{add motivation for key exchange}

\subsection{Diffie-Hellman Key Exchange}

In this section, we will go over one method to construct a secure key exchange protocol. The Diffie-Hellman key exchange protocol was created by Whitfield Diffie and Martin Hellman. This protocol requires some algebra and number theory that we will first go over. \scribenote{talk about how this is specifically an anonymous key exchange protocol}

Let $p$ be a large prime number. We fix the group $\group = \Z_p^*=\{1,2,3,\ldots,p-1\}$, with group operation multiplication mod $p$. Recall that $\Z_p^*$ is the set of nonzero elements of $\Z_p$. We then know that $\group$ is \textit{cyclic}. This means that one can provide a group member $g \in \group$, called the \textit{generator}, such that $\group = \{g^0, g^1, g^2, \ldots, g^{p-1}\}$. 

\begin{example}
	Let $p=7$. Is 2 or 3 a generator for $\Z_7^*$?
	
	First let us note that $\Z_7^* = \{1,2,3,4,5,6\}$. If either 2 or 3 is a generator, then exponentiating this value will produce all the elements in $\Z_7^*$. Looking at the table below, we can see that 2 only produces the values $\{1,2,4\}$, while 3 does indeed produce all elements of $\Z_7^*$. We can therefore conclude that 3 is a generator for $\Z_7^*$. 
	\begin{center}
	\begin{tabular}{|c|c|c|c|c|c|c|c|}
		\hline
		$x$ & 0 & 1 & 2 & 3 & 4 & 5 & 6 \\
		\hline \hline
		$2^x \mod 7$ & 1 & 2 & 4 & 1 & 2 & 4 & 1 \\
		\hline
		$3^x \mod 7$ & 1 & 3 & 2 & 6 & 4 & 5 & 1 \\
		\hline
	\end{tabular}
	\end{center}
\end{example}

\begin{figure}
	\center
	\begin{tikzpicture}
		\node (rect1) [draw]  {
			\begin{minipage}[t][4cm]{2cm}
				{\centering\underline{\textbf{Alice}} \\}
				$x \getsr \Z_{|\group|}$ \\
				$X \gets g^x$
			\end{minipage}
		};
		\node (rect2) [draw, right of=rect1, node distance=6cm] {
			\begin{minipage}[t][4cm]{2cm}
				{\centering\underline{\textbf{Bob}} \\}
				$y \getsr \Z_{|\group|}$ \\
				$Y \gets g^y$
			\end{minipage}
		};
		\node (group1) [above of=rect1, node distance=3cm] {$\group,g$};
		\node (group2) [above of=rect2, node distance=3cm] {$\group,g$};
		\path[->,>=stealth'] (group1) edge (rect1);
		\path[->,>=stealth'] (group2) edge (rect2);
		
		\path[->,>=stealth'] (rect1.10) edge node[anchor=south] {$X$} ( rect2.west|-rect1.10); 
		\path[<-,>=stealth'] (rect1.-40) edge node[anchor=south] {$Y$} (rect2.west|-rect1.-40);

		\node (key1) at (0,-1.75) {$K \gets \hash(Y^x)$};
		\node (key2) at (6,-1.75) {$K \gets \hash(X^y)$};
	\end{tikzpicture}
	\caption{The Diffie-Hellman anonymous key exchange protocol for cyclic group $\group$ with generator $g$.}
	\label{fig:DHKE}
\end{figure}

\scribenote{The notes fix G but then the protocol shown generalizes G. In the protocol, should I just use $\Z_p^*$?}

\paragraph{Diffie-Hellman key exchange.} We now describe the details of the Diffie-Hellman anonymous key exchange protocol. We assume both parties have access to public values $\group$ and $g\in\group$. It executes as follows, as shown in \figref{fig:DHKE}:
\begin{enumerate}
	\item Alice chooses $x$ uniformly at random from $\Z_{|\group|}$, computes $X \gets g^x$, and sends $X$ to Bob.
	
	\item Bob chooses $y$ uniformly at random from $\Z_{|\group|}$, computes $Y \gets g^y$, and sends $Y$ to Alice.
	
	\item Upon receiving $X$, Alice computes $K \gets \hash(Y^x)$, where $\hash$ is a collision-resistant hash function. 
	
	\item Upon receiving $Y$, Bob computes $K \gets \hash(X^y)$. 
\end{enumerate}

The secret key $K$ shared by Alice and Bob must be the same since
\begin{equation*}
	Y^x = g^{yx} = g^{xy} = X^y.
\end{equation*}

\paragraph{Security of Diffie-Hellman key exchange.} Notice that if an adversary could easily compute $x$ given $X \gets g^x$, then the adversary could certainly compute $K$, rendering this protocol insecure. The function that calculates this value is called the \textbf{discrete logarithm function}, usually denoted as $\dlog$, and is the inverse of the exponentiation function. Thus, for Diffie-Hellman key exchange to have any chance of being secure, we must find a group in which it is difficult to compute $\dlog$. In particular, for $p$ at least 2048-bits and $q = |\group|$ at least 256-bits, where $p$ and $q$ are both primes, the discrete logarithm function is believed to be hard to compute in the order $q$ subgroup of $Z_p^*$ \cite{BonehShoupBook}. This is called the \textbf{discrete logarithm assumption}.

However, a group that only meets the discrete logarithm assumption on its own is not sufficient to guarantee the security of the Diffie-Hellman key exchange. In fact, this protocol is only secure if and only if the \textbf{computational Diffie-Hellman assumption} holds, which says that given $g^\alpha, g^\beta \in \group$, where $\alpha \getsr \Z_{|\group|}$ and $\beta \getsr \Z_{|\group|}$, it is hard to compute $g^{\alpha\beta} \in \group$ \cite{BonehShoupBook}. We will now take a more formal look at the discrete logarithm assumption, the computational Diffie-Hellman assumption, and other related assumptions. 

\subsection{The Discrete Logarithm and Related Assumptions}

\begin{figure}
	\center
	\begin{tabular}{|c|c|c|}
		\hline
		Problem & Given & Compute \\
		\hline \hline
		Discrete logarithm (DL) & $g, g^x$ & $x$ \\
		\hline
		Computational Diffie-Hellman (CDH) & $g,g^x,g^y$ & $g^{xy}$ \\
		\hline
		Decisional Diffie-Hellman (DDH) & $g, g^x, g^y, g^z$ & Is $z \equiv xy\ (\textrm{mod}\ |\group|)$? \\
		\hline
	\end{tabular}
	\caption{A summary of the discrete logarithm related problems over a cyclic group $\group$ with generator $g$ from this section. For each row, we state the problem, the values given to the adversary, and what the adversary must provide to solve the problem.}
	\label{fig:DL}
\end{figure}

\paragraph{The Discrete Logarithm.} For $X \in \group$, the discrete logarithm function $\dlog(X)$ finds the unique value $x \in \Z_{|\group|}$ such that $g^x = X$. We show game $\DL_{\group, g}$, where given $X = g^x$ adversary $\advA$ must find $x$. 

\begin{center}
	\fpage{.18}{
		\underline{$\DL^\advA_{\group, g}$} \\
		$x \getsr \Z_{|\group|} \ ; \ X \gets g^x$ \\
		$\hat{x} \getsr \advA(X)$ \\
		Return $\hat{x} = x$ 
	}
\end{center}

The DL-advantage of $\advA$ is defined as 
\begin{equation*}
\AdvDL{\group, g}{\advA} = \Prob{\DL^\advA_{\group, g}\Rightarrow\true}.
\end{equation*}

Consider the following adversary $\advA'$ for game $\DL_{\group, g}$:

\begin{center}
	\mpage{.25}{
		\underline{adversary $\advA'(X)$} \\
		For $i=2,\ldots,|G|-1$ do \\
		\ind If $X = g^i$ then \\
		\ind \ind Return $i$
	}
\end{center}

$\advA'$ simply brute-force searches through every possible value to find the correct exponent $x$. While $\AdvDL{\group, g}{\advA'}=1$, the running time of $\advA'$ is $\bigO(|G|)$, which becomes very slow for large groups. Other algorithms have been developed that are more efficient, such as Baby-step giant-step, whose running time is $\bigO(|G|^{0.5})$. However, this is still quite slow, and while certain groups might have more efficient algorithms, nothing faster is known for other groups. For such a group $\group$ in which the discrete logarithm problem is hard, we refer to this as the discrete logarithm assumption for group $\group$. This is formalized below.

\begin{definition}
	The \textbf{discrete logarithm (DL) assumption} holds for $\group$ if $\AdvDL{\group, g}{\advA}$ is negligible for all efficient adversaries $\advA$.
\end{definition}

\paragraph{Computational Diffie-Hellman.} A related problem to the discrete logarithm problem is the computational Diffie-Hellman problem, which says that given $g^x, g^y \in \group$, where $x \getsr \Z_{|\group|}$ and $y \getsr \Z_{|\group|}$, it is hard to compute $g^{xy} \in \group$ \cite{BonehShoupBook}. We formalize this with game $\CDH_{\group, g}$ shown below.  

\begin{center}
	\fpage{.26}{
		\underline{$\CDH^\advA_{\group, g}$} \\
		$x,y \getsr \Z_{|\group|}$ \\
		$X \gets g^x \ ; \ Y \gets g^y \ ; \ Z \gets g^{xy}$ \\
		$\hat{Z} \getsr \advA(X,Y)$ \\
		Return $\hat{Z} = Z$ 
	}
\end{center}

The CDH-advantage of $\advA$ is defined as 
\begin{equation*}
\AdvCDH{\group, g}{\advA} = \Prob{\CDH^\advA_{\group, g}\Rightarrow\true}.
\end{equation*}

Similar to the DL assumption, the computational Diffie-Hellman assumption for group $\group$ tells us that the CDH problem is hard in $\group$.

\begin{definition}
	The \textbf{computational Diffie-Hellman (CDH) assumption} holds for $\group$ if $\AdvCDH{\group, g}{\advA}$ is negligible for all efficient adversaries $\advA$.
\end{definition}

\paragraph{Decisional Diffie-Hellman.} The decisional Diffie-Hellman problem for group $\group$ and generator $g$ says that for $x \getsr \Z_{|\group|}$, $y \getsr \Z_{|\group|}, z \getsr \Z_{|\group|}$, when given $g^x,g^y \in \group$ and either $g^z \in \group$ or $g^{xy} \in \group$, it is hard to distinguish between getting $g^z$ and $g^{xy}$. This is formalized in game $\DDH_{\group,g}$ below.  

\begin{center}
	\fpage{.20}{
		\underline{$\DDH_{\group,g}^\advA$}\\
		$b \getsr \bits$\\
		$x,y,z \getsr \Z_{|G|}$\\
		$Z_0 \gets g^z$\\
		$Z_1 \gets g^{xy}$\\
		$b' \getsr \advA(g^x,g^y,Z_b)$\\
		Return $(b' = b)$
	}

%	\hfpages{.25}{
%		\underline{$\DDH1_{\group, g}^\advA$}\\
%		$x,y,z \getsr \Z_{|\group|}$ \\
%		$X \gets g^x \ ; \ Y \gets g^y \ ; \ Z \gets g^{xy}$ \\
%		$b \getsr \advA(X,Y,Z)$ \\
%		Return $b$ 
%	}{
%		\underline{$\DDH0_{\group, g}^\advA$}\\
%		$x,y,z \getsr \Z_{|\group|}$ \\
%		$X \gets g^x \ ; \ Y \gets g^y \ ; \ Z \gets g^z$ \\
%		$b \getsr \advA(X,Y,Z)$ \\
%		Return $b$
%	}
\end{center}

The DDH-advantage of $\advA$ is defined as 
\begin{equation*}
\AdvDDH{\group, g}{\advA} = 2 \cdot \Prob{\DDH_{\group, g}^\advA\Rightarrow\true} - 1.
\end{equation*}

The decisional Diffie-Hellman assumption for group $\group$ tells us that the DDH problem is hard in $\group$. 

\begin{definition}
	The \textbf{decisional Diffie-Hellman (DDH) assumption} holds for $\group$ if $\AdvDDH{\group, g}{\advA}$ is negligible for all efficient adversaries $\advA$.
\end{definition}

We summarize the three problems mentioned in this section in \figref{fig:DL}. 

\subsection{ElGamal Encryption}

\begin{figure}
	\center
	\hfpages{.2}{
		\underline{$\enc(X,M)$:}\\
		$y \getsr \Z_{|\group|}$ \\
		$C_1 \gets g^y$\\
		$Z \gets X^y$\\
		$C_2 \gets Z \cdot M$\\
		Return $(C_1,C_2)$
	}{
		\underline{$\dec(x,(C_1,C_2))$:}\\
		$M \gets C_2 \cdot C_1^{-x}$\\
		Return $M$
	}
	\caption{The ElGamal encryption scheme.}
	\label{fig:elgamal}
\end{figure}

\begin{wrapfigure}{r}{1.5in}
	\center
	\fpage{.20}{
		\underline{$G_0$ \;\;\; \fbox{$G_1$}}\\
		$b \getsr \bits$\\
		$x \getsr \Z_{|\group|}$\\
		$X \gets g^x$\\
		$b' \getsr \advA^\EncOracle(g,X)$\\
		Ret $(b' = b)$\medskip
		
		\underline{$\EncOracle(M_0,M_1)$}\\
		$C_1 \gets g^y$\\
		$Z \gets g^{xy}$\\
		\fbox{$z \getsr \Z_{|\group|}$ \;;\; $Z \gets g^z$}\\
		$C_2 \gets Z\cdot M_b$\\
		Ret $(C_1,C_2)$
	}
	\caption{Games for the proof of \thref{proof:elgamal}.}
	\label{fig:elgamal-games}
\end{wrapfigure}

We will now go over a well-known public key encryption scheme called \textbf{ElGamal encryption}. Let $\group$ be a cyclic group and let $g$ be a generator for $\group$. During key generation, secret key $x$ is chosen at random from $\Z_{|\group|}$, and the public key is computed as $X \gets g^x$. The encryption algorithm $\enc$ takes in public key $X$ and message $M$. It then chooses $y\getsr \Z_{|\group|}$ and computes $C_1 \gets g^y$ and $Z \gets X^y$. It then returns $(C_1, Z \cdot M)$, which is equal to $(g^y, g^{xy} \cdot M)$. The decryption algorithm $\dec$ takes in this ciphertext value in addition to the secret key $x$. It computes $M \gets C_2 \cdot C_1^{-x}$. Notice that
\begin{equation*}
	C_2 \cdot C_1^{-x} = g^{xy} \cdot M \cdot (g^y)^{-x} = M
\end{equation*}
which gives us the original message back as desired. 

The ElGamal scheme can be proven IND-CPA secure if DDH holds in $\group$, which we show with the following theorem. 

\begin{theorem}
\label{proof:elgamal}
	Let $\AEnc$ be the ElGamal scheme over group $\group$ with generator $g$. 
	Let $\advA$ be an $\INDCPA_\AEnc$-adversary. Then we give a $\DDH_{\group,g}$ 
	adversary $\advB$ such that 
	\bnm
	\AdvINDCPA{\AEnc}{\advA} \le 2\cdotsm \AdvDDH{\group,g}{\advB} \;.
	\enm
	Adversary $\advB$ runs in time that of $\advA$. 
\end{theorem} 
	
\begin{proof}
	We define the games in \figref{fig:elgamal-games}. Game $G_0$ has an identical output distribution as game $\INDCPA_{\AEnc}$, so $\Prob{G_0\Rightarrow\true} = \Prob{\INDCPA_{\AEnc}^\advA\Rightarrow\true}$. Game $G_1$ is the same as $G_0$ except now instead of assigning $Z \gets g^{xy}$, we choose a random value $z \getsr \Z_{|\group|}$ and assign $Z \gets g^z$. Now consider the following adversary $\advB$ playing game $\DDH_{\group, g}$. 
	
	\begin{center}
		\fpage{.20}{
			\underline{Adversary $\advB(X,Y,Z)$}\\
			$b \getsr \bits$\\
			$b' \getsr \advA^\EncSim(g,X)$\\
			If $(b'=b)$ then return 1 \\
			Else return 0 \medskip
			
			\underline{$\EncSim(M_0,M_1)$}\\
			$C_1 \gets Y$\\
			$C_2 \gets Z\cdot M_b$\\
			Ret $(C_1,C_2)$
		}
	\end{center}
	
	It takes inputs $X,Y,Z$ and must determine which value of $Z$ it was given. Recall that $X = g^x$ and $Y = g^y$. $\advB$ first chooses a bit $b$ at random. It then runs adversary $\advA$ with a simulation of its encryption oracle, called $\EncSim$, and gives $\advA$ both $g$ and $X$, which acts as the public key. If $\advA$ returns the correct bit, then $\advB$ returns 1; otherwise, it returns 0. We now have that
	\begin{align*}
		\AdvDDH{\group,g}{\advB} &= 2 \cdot \Prob{\DDH_{\group, g}^\advB\Rightarrow\true} - 1 \\
		&=
		\begin{aligned}
			2 \cdot (&\Prob{\DDH_{\group, g}^\advB\Rightarrow\true \ | \ b=1}\cdot\Prob{b=1} + \\
			&\Prob{\DDH_{\group, g}^\advB\Rightarrow\true \ | \ b=0}\cdot\Prob{b=0}) - 1 .
		\end{aligned}
	\end{align*} 
	
	To find $\Prob{\DDH_{\group, g}^\advB\Rightarrow\true \ | \ b=1}$, first notice that when $b=1$, adversary $\advB$ gets $Z = g^{xy}$. This means encryption during $\EncSim$ has the same output distribution as $\EncOracle$ in game $G_0$. Furthermore, in this case game $\DDH_{\group, g}$ returns $\true$ when $\advB$ returns 1, which occurs when $b'=b$. We know that $b'=b$ when $\advA$ successfully determines the correct message encrypted. This happens with the same probability as $\advA$ winning in game $G_0$ and thus $\Prob{\DDH_{\group, g}^\advB\Rightarrow\true \ | \ b=1} = \Prob{G_0 \Rightarrow\true}$. 
	
	Likewise, to find $\Prob{\DDH_{\group, g}^\advB\Rightarrow\true \ | \ b=0}$, we again notice that when $b=0$, adversary $\advB$ gets $Z = g^z$. This now means that $\EncSim$ has the same output distribution as $\EncOracle$ in game $G_1$. $\DDH_{\group, g}$ returns $\true$ when $\advB$ returns 1, which occurs when $b'\neq b$. We know that $b' \neq b$ when $\advA$ fails to determine the correct message encrypted. This happens exactly when $\advA$ does not win in game $G_1$ and thus $\Prob{\DDH_{\group, g}^\advB\Rightarrow\true \ | \ b=0} = 1 - \Prob{G_1 \Rightarrow\true}$. 
	
	We now put this together to get
	\begin{align*}
		\AdvDDH{\group,g}{\advB} &= 2 \cdot \left(\Prob{G_0 \Rightarrow\true} \cdot \frac{1}{2} + (1 - \Prob{G_1 \Rightarrow\true})\cdot\frac{1}{2}\right) - 1 \\
		&= \Prob{G_0 \Rightarrow\true} - \Prob{G_1 \Rightarrow\true}.
	\end{align*}
	
	Lastly, in game $G_1$ since $Z$ is a random value that is multiplied with the message, $C_2$ is also a random value and thus no information is leaked about $b$. We then know that the success of $\advA$ is that of a random coin flip, so $\Prob{G_1 \Rightarrow\true} = \frac{1}{2}$. We then have that
	\begin{align*}
		\AdvINDCPA{\AEnc}{\advA} 
		&= 2\cdotsm\Prob{\INDCPA_{\AEnc}^\advA\Rightarrow\true} - 1\\
		&= 2\cdotsm\Prob{G_0\Rightarrow\true} - 1\\
		&= 2\cdotsm\left(\Prob{G_1\Rightarrow\true} + \AdvDDH{G,g}{\advB})\right) - 1\\
		&= 2\cdotsm\left(\frac{1}{2}+ \AdvDDH{G,g}{\advB})\right) - 1 \\
		&= 2 \cdot \AdvDDH{G,g}{\advB}.
	\end{align*}
\end{proof}

\paragraph{ElGamal in the group $Z_p^*$.} 

To see the proof of this, we point the interested reader to read about Euler's criterion. 


\section*{Exercises}

\begin{enumerate}[label=\textbf{Exercise \thesection.\arabic*}, wide=0pt]
	\item show that DHKE is secure given CDH assumption?
	
	\item Let $\group$ be a cyclic group and let $g$ be a generator of $\group$. Let $\advA$ be an adversary against the DL problem. Show that there exists an adversary $\advB$ such that 
	\begin{equation*}
		\AdvDL{\group,g}{\advA} \leq \AdvCDH{\group,g}{\advB}
	\end{equation*}
	where the running time of $\advB$ is that of $\advA$ plus the time to do one exponentiation in $\group$. 
	
	\item Let $\group$ be a cyclic group and let $g$ be a generator of $\group$. Let $\advB$ be an adversary against the CDH problem. Show that there exists an adversary $\advC$ such that 
	\begin{equation*}
	\AdvCDH{\group,g}{\advB} \leq \AdvDDH{\group,g}{\advC} + \frac{1}{|\group|}
	\end{equation*}
	where the running time of $\advC$ is that of $\advB$.
\end{enumerate}