%!TEX root = ../main.tex
%%%%%%%%%%%%%%%%%%%%%%%%%%%%%%%%%%%%%%%%%%%%%%%%%%%%%%%%%%%%%%%%%%%%%%%%%%%%%%%%
\subsection{Building PRPs from PRFs}
We've seen how it is hard to distinguish between a PRF and a PRP.
In the previous section, we showed that it is hard to distinguish between a PRF and a PRP.
However, as in the use case of block ciphers for length-preserving encryption and decryption, we specifically require a PRP for correctness purposes (in fact, we want an invertible PRP).
This is to ensure that a single ciphertext does not have ambiguous decryptions.
A natural question to ask is does the existence of a PRF imply the existence of a PRP, or more constructively, can we build a PRP given a PRF?

In this section, we will show that it is possible to build a PRP from a PRF.
We will examine one such construction called a Feistel network which is used in the construction of many early block ciphers, including DES (Data Encryption Standard), the first standardized block cipher.

\paragraph{Feistel networks}
A Feistel round transforms an arbitrary function into a permutation.
Define $\cipher:\braces*{0,1}^{2n}\rightarrow\braces*{0,1}^{2n}$ as the permuation constructed using one Feistel round with function $\prf:\braces*{0,1}^k\times\braces*{0,1}^n\rightarrow\braces*{0,1}^n$.
The function $\prf$ is known as the round function of the Feistel network.
\begin{wrapfigure}{r}{2in}
\center
\begin{tikzpicture}
    \node(r0) at ($(0,0)$)  {$R_0$};
    \node (l0) [left of = r0, node distance=3.0cm] {$L_0$};
    \node[draw,thick,minimum width=0.75cm] (\prf) [below of = r0, node distance=1.1cm]  {$\prf_{\prfkey}$};
    \node (xor) [XOR, below of = \prf, node distance = 0.75cm, scale=1.0] {};
    \node (junction0) [below of = r0, node distance = 0.5cm] {};
    \node (junction1) [below of = l0, node distance = 0.75cm] {};
    \node (junction2) [below of = l0, node distance = 1.5cm] {};
    \node (junction3) [left of = xor, node distance = 0.5cm] {};
    \path[->]
      (r0) edge[thick] node {} (\prf)
      (\prf) edge[thick] node {} (xor)
      (junction3.center) edge[thick] node {} (xor);
    \path[-]
      (l0.south) edge[thick] node {} (junction1.center)
      (junction0.center) edge[thick] node {} (junction2.center)
      (junction1.center) edge[thick] node {} (junction3.center);
    \node (r1) [below of = xor, node distance=0.75cm] {$R_1$};
    \node (l1) [left of = r1, node distance=3.0cm] {$L_1$};
    \path[->]
      (xor) edge[thick] node {} (r1)
      (junction2.center) edge[thick] node {} (l1);
\end{tikzpicture}
\caption{A single round of a Feistel network.}
\label{fig:feistel-1}
\end{wrapfigure}
This construction is depicted in Figure~\ref{fig:feistel-1} where $2n$ length inputs and outputs are split into left and right parts of length $n$ notated $L$ and $R$, respectively.
Function $\prf$ is notated with fixed key $\prfkey$ as $\prf_{\prfkey}:\braces*{0,1}^n\rightarrow\braces*{0,1}^n$.
The Feistel round is then
\begin{align*}
L_1 &\gets R_0\\
R_1 &\gets L_0 \oplus \prf_{\prfkey}(R_0)\,.
\end{align*}

\begin{example}[1-round Feistel network is a permutation]
  First, let us see why the Feistel construction builds a permutation.
  Notice that as long as the round function $\prf_{\prfkey}$ is defined on all inputs from domain $\braces*{0,1}^n$ then $\cipher$ is defined on all inputs from domain $\braces*{0,1}^{2n}$.
  Next, we show that any output of $\cipher$, $L_1\concat R_1$, corresponds to a unique input $L_0\concat R_0$.
  Even though $\prf_{\prfkey}$ is not assumed to be invertible, since the input to $\prf_{\prfkey}$, $R_0$, is passed directly to the output as $L_1$, we can invert as follows
  \begin{align*}
    L_0 &\gets R_1\oplus \prf_{\prfkey}(L_1)\\
    R_0 &\gets L_1\,.
  \end{align*}
  These two properties together mean $\cipher$ is a complete permutation on $\braces*{0,1}^{2n}$.
\end{example}

\begin{example}[1-round Feistel network is not a PRP]
  Next, we show that a 1-round Feistel network $\cipher$ is not a PRP.
  Consider the following PRP adversary $\advA$ for domain $\braces*{0,1}^{2n}$:
  \begin{center}
    \fpage{.2}{
    \underline{\textbf{adversary }$\advA^{\Fn}$}\\[2pt]
    $(L_1,R_1)\gets\Fn(0^{2n})$\\
    Return $L_1 = 0^{n}$
    }
  \end{center}

  In $\PRP1$, where the oracle $\Fn$ returns $\cipher_{\prfkey}(0^{2n})$, $\advA$ returns 1 with probability 1.
  In $\PRP0$, where oracle $\Fn$ returns a random value, the probability $\advA$ returns 1 is the probability that the first $n$ bits of the random output are 0.
  Thus,
\begin{align*}
\AdvPRP{\cipher}{\advA} &= \absv*{\Prob{\PRP1_\cipher^\advA\Rightarrow 1} - \Prob{\PRP0_\cipher^\advA\Rightarrow1} }\\
&\geq 1 - \frac{1}{2^n}\,.
\end{align*}
\end{example}

We see that it is trivial to distinguish a 1-round Feistel network from a random permutation, since the second half of the input is mapped directly to the first half of the output.
Does the randomness of the permutation improve by performing numerous Feistel rounds?
One way to construct a multi-round Feistel network by feeding the outputs of the previous round as inputs to the next round, and where each round is keyed with a different unique key.
To use only a single key, we consider an alternate construction, where the round function $\prf$ in each Feistel round has a different domain, $\prf:\braces{0,1}^{k}\times\braces{0,1}^{2n}\rightarrow\braces{0,1}^n$.
As before, the round function will take in the second half of the input, but it will also take in an $n$-bit round counter.
To stack Feistel rounds into a multi-round Feistel network, we simply ensure that each round uses a different round counter.
Figure~\ref{fig:feistel-3} depicts a 3-round Feistel network constructed in this manner.

\begin{wrapfigure}{r}{2.5in}
\center
\begin{tikzpicture}
    \node(r0) at ($(0,0)$)  {$R_0$};
    \node (l0) [left of = r0, node distance=3.0cm] {$L_0$};
    \def\lastz{0}
    \foreach \z[remember=\z as \lastz] in {1, 2,...,3} {
      \node[draw,thick,minimum width=0.75cm] (\prf\z) [below of = r\lastz, node distance=1.1cm]  {$\prf_{\prfkey}$};
      \node (ctr\z) [right of = \prf\z, node distance=1.25cm] {$\langle\z\rangle$};
      \node (xor\z) [XOR, below of = \prf\z, node distance = 0.75cm, scale=1.0] {};
      \node (junction0\z) [below of = r\lastz, node distance = 0.5cm] {};
      \node (junction1\z) [below of = l\lastz, node distance = 0.75cm] {};
      \node (junction2\z) [below of = l\lastz, node distance = 1.5cm] {};
      \node (junction3\z) [left of = xor\z, node distance = 0.5cm] {};
      \node (r\z) [below of = xor\z, node distance=0.75cm] {$R_\z$};
      \node (l\z) [left of = r\z, node distance=3.0cm] {$L_\z$};
      \path[->]
        (r\lastz) edge[thick] node {} (\prf\z)
        (\prf\z) edge[thick] node {} (xor\z)
        (junction3\z.center) edge[thick] node {} (xor\z)
        (xor\z) edge[thick] node {} (r\z)
        (junction2\z.center) edge[thick] node {} (l\z)
        (ctr\z) edge[thick] node {} (\prf\z.east);
        \path[-]
        (l\lastz.south) edge[thick] node {} (junction1\z.center)
        (junction0\z.center) edge[thick] node {} (junction2\z.center)
        (junction1\z.center) edge[thick] node {} (junction3\z.center);
    }
\end{tikzpicture}
\caption{A 3-round Feistel network.}
\label{fig:feistel-3}
\end{wrapfigure}

\begin{example} A 2-round Feistel network is not a PRP.
  \scribenote{Add as homework problem}
\end{example}

\paragraph{3-round Feistel network is a PRP}
Next, we show that given PRF security of the underlying round function, a 3-round Feistel network is a PRP.
\scribenote{Intuition?}

\begin{theorem}[Luby-Rackoff]\label{thm:luby-rackoff}
Let $\Feistel$ be the 3-round Feistel cipher using round function
$\prf\Colon\bits^k\times\bits^{2n}\rightarrow \bits^n$. For any
$\PRP_\cipher$-adversary $\advA$ making at most $q$ queries
we give an $\PRF_{\prf}$-adversary $\advB$ making at most $3q$ queries such that
\bnm
  \AdvPRP{\Feistel}{\advA} \le \AdvPRF{\prf}{\advB} + \frac{2q^2}{2^n} +
  \frac{q^2}{2^{2n}} \;.
\enm
\end{theorem}
\scribenote{How are we doing cites? (Luby-Rackoff)}

\begin{proof}
Without loss of generality, we consider $\advA$ that make only unique queries to oracle $\Fn$.
We bound the advantage of $\PRP$ adversary $\advA$ by bounding the advantage of each of a series of game hops.
Pseudocode for the games is given in Figure~\ref{fig:games-luby-rackoff}.
Intuitively, the proof will follow the intuition that if the round function can be considered as a random function, then barring \scribenote{Add intuition}.

Game $\G0$ is constructed exactly as $\PRP1^{\advA}_{\Feistel}$; the oracle pseudocode expands out the computation of the 3-round Feistel network $\Feistel$.
\bnm
\Prob{\PRP1_\Feistel^\advA\Rightarrow 1} = \Prob{\G0\Rightarrow 1}
\enm
Game $\G1$ replaces the round function $\prf$ with a random function $\randfn$ mapping from $\braces{0,1}^{2n}\rightarrow\braces{0,1}^n$.
\scribenote{Should PRF security game be defined with respect to different domain and range space, $m$ and $n$?}
We bound the ability to distinguish between $\G0$ and $\G1$ by the PRF security of $\prf$.
Consider the following PRF adversary $\advB$ which runs $\advA$ with a simulated oracle $\FnSim$.

\begin{center}
\fpage{.22}{
\underline{$\advB^\Fn$}\\[2pt]
$K \getsr \bits^k$\\
$b' \getsr \advA^\FnSim$\\
Return $b'$\medskip

\underline{$\FnSim(\msg)$}\\
$L_1 \gets R_0$\\
$R_1 \gets L_0 \oplus \Fn(\langle 1\rangle \concat R_0)$\\
$L_2 \gets R_1$\\
$R_2 \gets L_1 \oplus \Fn(\langle 2\rangle \concat R_1)$\\
$L_3 \gets R_2$\\
$R_3 \gets L_2 \oplus \Fn(\langle 3\rangle \concat R_2)$\\
Return $L_3 \concat R_3$
}
\end{center}

Adversary $\advB$ simulates $\FnSim$ by running the 3-round Feistel network but replacing the round function with a call to its own oracle $\Fn$.
In \PRF0, where $\advB$'s oracle $\Fn$ acts as the round function $\prf$, $\advB$ runs exactly $\G0$.
In \PRF1, where $\Fn$ acts as a random function $\rho$, $\advB$ runs exactly $\G1$.
Thus, we have
\begin{align*}
\AdvPRF{\prf}{\advB} &= \absv*{\Prob{\PRF1_\prf^\advB\Rightarrow 1} - \Prob{\PRF0_\prf^\advB\Rightarrow1}}\\
&= \absv*{\Prob{\G1\Rightarrow 1} - \Prob{\G0\Rightarrow1}}\\
\end{align*}

In games $\G2$ and $\G3$, we use a similar trick to as in our proof of the PRF-PRP Switching Lemma.
Game $\G2$ replaces the random function $\rho$ with a lazy evaluation of a random function using a table $\TabF$.
When the random function is queried on an input, the value in $\TabF$ keyed by the input is returned.
If there is no such value, meaning the input has not been previously queried, a new random value is generated, returned, and stored in $\TabF$ for future queries.
Thus, lazily building out $\TabF$ is equivalent to using random function $\rho$.
\bnm
\Prob{\G1\Rightarrow 1} = \Prob{\G2\Rightarrow 1}
\enm

Game $\G3$ generates fresh random values on every input to the second and third round functions.
This is in contrast to $\G2$ where a fresh random value is only returned for inputs that have never been seen.
Thus, the difference between $\G2$ and $\G3$ occur when repeat inputs are used with the second or third round functions, which corresponds to repeat values of $R_1$ and $R_2$, respectively.
Intuitively, since we are only considering adversaries $\advA$ that make unique queries to $\Fn$, finding repeat values of $R_1$ and $R_2$ is hard because they both include randomness from the random function.
The pseudocode in Figure~\ref{fig:games-luby-rackoff} captures the event of a repeat query by a setting a $\bad$ flag.
The only difference between $\G2$ and $\G3$ occur after the $\bad$ flag is set; $\G2$ returns the consistent value from the look-up table $\TabF$, while $\G3$ returns a fresh random value.
Then by the fundamental lemma of game-playing,
\bnm
\absv*{\Prob{\G3\Rightarrow 1} - \Prob{\G2\Rightarrow1}} \le \Prob{\G3 \setsbad}
\enm

If we assume $\advA$ makes at most $q$ queries to $\Fn$, we can bound the probability $\G3\setsbad$.
The probability of $R_1 = L_0 \xor \TabF[1,R_0]$ colliding with a previous $R_1$ on query $i$ is bounded by $q/2^n$.
\scribenote{Problem: I think it is a much more complicated argument to say $R_1$ is random, since $\TabF[1,R_0]$ is not fresh every time. Need to argue something about no information of $\TabF[1,R_0]$ is output.}
\scribenote{Probably has connection to Luby-Rackoff Revisited [Naor-Reingold]}
\scribenote{Boneh-Shoup book seems to address this challenge with an independence argument.}
Similarly, the probability of $R_2 = L_0 \xor X_2$ colliding with a previous $R_2$ on query $i$ is bounded by $q/2^n$ since $X_2$ is a fresh random string.
Thus, by two applications of the union bound, we have
\begin{align*}
  \Prob{\G3 \setsbad} &= \Prob{R_1 \text{ collides } \vee R_2 \text{ collides }}
\end{align*}

Game $\G4$ is the same as $\G3$ except $R_2$ and $R_3$ are set to fresh random values.
In $\G3$,

\begin{align*}
\AdvPRP{\cipher}{\advA}
    &= \left|\Prob{\PRP1^\advA_\cipher} - \Prob{\PRP0^\advA_\cipher}\right|\\
    &= \left|\Prob{\G0} - \Prob{\PRP0^\advA_\cipher}\right|\\
    &\le \left|\Prob{\G1} + \AdvPRF{F}{\advB} - \Prob{\PRP0^\advA_\cipher}\right|\\
    &=   \left|\Prob{\G2} + \AdvPRF{F}{\advB} - \Prob{\PRP0^\advA_\cipher}\right|\\
    &\le \left|\Prob{\G3} + \Prob{\bad_3} + \AdvPRF{F}{\advB} - \Prob{\PRP0^\advA_\cipher}\right|\\
    &= \left|\Prob{\G4} + \Prob{\bad_4} + \AdvPRF{F}{\advB} - \Prob{\PRP0^\advA_\cipher}\right|\\
    &\le \left|\Prob{\PRP0^\advA_\cipher} + \frac{q^2}{2^{2n}} + \Prob{\bad_4} + \AdvPRF{F}{\advB} - \Prob{\PRP0^\advA_\cipher}\right|\\
    &= \frac{q^2}{2^{2n}} + \Prob{\bad_4} + \AdvPRF{F}{\advB}\\
    &\le \frac{q^2}{2^{2n}} + \frac{2q^2}{2^n} + \AdvPRF{F}{\advB}\\
\end{align*}

\end{proof}


\begin{figure}[p]
\hfpagessss{.20}{.20}{.20}{.25}{
\underline{$\G0$}\\[2pt]
$K \getsr \bits^k$\\
$b' \getsr \advA^\Fn$\\
Return $b'$\medskip

\underline{$\Fn(\msg)$}\\
$(L_0, R_0) \gets \msg$\\
$L_1 \gets R_0$\\
$R_1 \gets L_0 \oplus F_K(\langle 1\rangle \concat R_0)$\\
$L_2 \gets R_1$\\
$R_2 \gets L_1 \oplus F_K(\langle 2 \rangle \concat R_1)$\\
$L_3 \gets R_2$\\
$R_3 \gets L_2 \oplus F_K(\langle 3 \rangle \concat R_2)$\\
Return $L_3 \concat R_3$
}{
\underline{$\G1$}\\[2pt]
$\rho \getsr \Func(2n,n)$\\
$b' \getsr \advA^\Fn$\\
Return $b'$\medskip

\underline{$\Fn(\msg)$}\\
$(L_0, R_0) \gets \msg$\\
$L_1 \gets R_0$\\
$R_1 \gets L_0 \oplus \rho(\langle 1\rangle \concat R_0)$\\
$L_2 \gets R_1$\\
$R_2 \gets L_1 \oplus \rho(\langle 2 \rangle \concat R_1)$\\
$L_3 \gets R_2$\\
$R_3 \gets L_2 \oplus \rho(\langle 3 \rangle \concat R_2)$\\
Return $L_3 \concat R_3$
}{
\underline{$\fbox{\G2}$\;\;\;\G3}\\[2pt]
$b' \getsr \advA^\Fn$\\
Return $b'$\medskip

\underline{$\Fn(\msg)$}\\
$(L_0, R_0) \gets \msg$\\
$L_1 \gets R_0$\\
If $\TabF[1,R_0] = \bot$ then\\
\ind $\TabF[1,R_0] \getsr \bits^n$\\
$R_1 \gets L_0 \oplus \TabF[1,R_0]$\\
$L_2 \gets R_1$\\
$X_2 \getsr \bits^n$\\
If $\TabF[2,R_1] \ne \bot$ then\\
\ind $\badtrue$\\
\ind \fbox{$X_2 \gets \TabF[2,R_1]$}\\
$\TabF[2,R_1] \gets X_2$\\
$R_2 \gets L_1 \oplus X_2$\\
$L_3 \gets R_2$\\
$X_3 \getsr \bits^n$\\
If $\TabF[3,R_2] \ne \bot$ then\\
\ind $\badtrue$\\
\ind \fbox{$X_3 \gets \TabF[3,R_2]$}\\
$\TabF[3,R_2] \gets X_3$\\
$R_3 \gets L_2 \oplus X_3$\\
Return $L_3 \concat R_3$
}{
\underline{$\G3$\;\;\;$\fbox{\G4}$}\\[2pt]
$b' \getsr \advA^\Fn$\\
Return $b'$\medskip

\underline{$\Fn(\msg)$}\\
$(L_0, R_0) \gets \msg$\\
$L_1 \gets R_0$\\
If $\TabF[1,R_0] = \bot$ then\\
\ind $\TabF[1,R_0] \getsr \bits^n$\\
$R_1 \gets L_0 \oplus \TabF[1,R_0]$\\
$L_2 \gets R_1$\\
$X_2 \getsr \bits^n$\\
If $\TabF[2,R_1] \ne \bot$ then\\
\ind $\badtrue$\\
$\TabF[2,R_1] \gets X_2$\\
$R_2 \gets L_1 \oplus X_2$;\;\;\fbox{$R_2 \getsr \bits^n$}\\
$L_3 \gets R_2$\\
$X_3 \getsr \bits^n$\\
If $\TabF[3,R_2] \ne \bot$ then\\
\ind $\badtrue$\\
$\TabF[3,R_2] \gets X_3$\\
$R_3 \gets L_2 \oplus X_3;$\;\;\fbox{$R_3 \getsr \bits^n$}\\
Return $L_3 \concat R_3$
}
\caption{Games for proof of 3-round Feistel network as PRP (Theorem~\ref{thm:luby-rackoff})}
\label{fig:games-luby-rackoff}
\end{figure}

\begin{figure}[p]
\center
\begin{tikzpicture}
    \node(r0) at ($(0,0)$)  {$R_0$};
    \node (l0) [left of = r0, node distance=3.0cm] {$L_0$};
    \def\lastz{0}
    \foreach \z[remember=\z as \lastz] in {1, 2,...,3} {
      \node[draw,thick,minimum width=0.75cm] (\prf\z) [below of = r\lastz, node distance=1.1cm]  {$\prf_{\prfkey}$};
      \node (ctr\z) [right of = \prf\z, node distance=1.25cm] {$\langle\z\rangle$};
      \node (xor\z) [XOR, below of = \prf\z, node distance = 0.75cm, scale=1.0] {};
      \node (junction0\z) [below of = r\lastz, node distance = 0.5cm] {};
      \node (junction1\z) [below of = l\lastz, node distance = 0.75cm] {};
      \node (junction2\z) [below of = l\lastz, node distance = 1.5cm] {};
      \node (junction3\z) [left of = xor\z, node distance = 0.5cm] {};
      \node (r\z) [below of = xor\z, node distance=0.75cm] {$R_\z$};
      \node (l\z) [left of = r\z, node distance=3.0cm] {$L_\z$};
      \path[->]
        (r\lastz) edge[thick] node {} (\prf\z)
        (\prf\z) edge[thick] node {} (xor\z)
        (junction3\z.center) edge[thick] node {} (xor\z)
        (xor\z) edge[thick] node {} (r\z)
        (junction2\z.center) edge[thick] node {} (l\z)
        (ctr\z) edge[thick] node {} (\prf\z.east);
        \path[-]
        (l\lastz.south) edge[thick] node {} (junction1\z.center)
        (junction0\z.center) edge[thick] node {} (junction2\z.center)
        (junction1\z.center) edge[thick] node {} (junction3\z.center);
    }
\end{tikzpicture}
\end{figure}

\paragraph{Connection to card shuffling algorithms}

\scribenote{Additional Exercise Ideas}
\begin{itemize}
  \item 2-round Feistel is not a PRP
  \item Show security of 3-round Feistel with 3 different keys
  \item What happens when same key is used across Feistel rounds? (I don't know)
  \item Show 3-round Feistel is not Strong PRP
  \item Show 4-round Feistel is Strong PRP
\end{itemize}
