%%%%%%%%%%%%%%%%%%%%%%%%%%%%%%%%%%%%%%%%%%%%%%%%%%%%%%%%%%%%%%%%%%%%%%%%%%%%%%%%
\section{Ciphers and Initial Security Notions}
\label{sec:se}

\paragraph{Ciphers.}
We start by defining a cipher. A cipher $\cipher = (\cipherE,\cipherD)$ is
defined by a a pair of deterministic algorithms $\cipherE$ and $\cipherD$.  To
any cipher $\cipher$ we associate sets called the key space $\keyspace$, message
space $\msgspace$, and ciphertext space $\ctxtspace$. We do not surface in the
notation for a cipher these sets, and will require that the association be clear
from context.

The algorithms are two-input. Enciphering takes a key $K \in \keyspace$ and
message $M \in \msgspace$, and outputs a ciphertext $C \in \ctxtspace$. Because
$\cipherE$ is deterministic, we can equally formalize it as a map 
$\cipherE\Colon\keyspace\times\msgspace\rightarrow\ctxtspace$. For a given key
$K$ we let $\cipherE_K\Colon\msgspace\rightarrow\ctxtspace$ be defined by
$\cipherE_K(M) = \cipherE(K,M)$ for all $M \in \msgspace$.  Deciphering takes a key $K\in \keyspace$ and ciphertext
$C \in \ctxtspace$ and outputs a message $M \in \msgspace$. Again, we can view
it as a map $\cipherD\Colon\keyspace\times\ctxtspace\rightarrow\msgspace$. 

Both $\cipherE$ and $\cipherD$ must be efficiently computable for all $K \in
\keyspace$.  (We have not defined efficiently computable, and use the term here
informally.) We require that a cipher be correct, meaning that for
$\cipherD_K(\cipherE_K(M)) = M$ for all $M$. 

\paragraph{Security notions.} What do we intuitively expect of a cipher?
Minimally:
\begin{itemize}
\item Secret key should remain secret
\item Message should remain secret
\end{itemize}
Let's try to formalize these notions.
\begin{itemize}
\item TKR security
\item KR security
\item (ot-)IND security
\end{itemize}


\begin{figure}[p]
\fpage{.45}{
\underline{$\TKR^\advA_\cipher$}\\[1pt]
$K \getsr \keyspace$\\
$K^* \getsr \advA^\Fn$\\
Ret $(K = K^*)$\medskip

\underline{$\Fn(M)$}\\
$C \gets \cipherE_K(M)$\\
Ret $C$
}
\end{figure}

We let $\TKR_\cipher$-advantage of a $\TKR_\cipher$-adversary $\advA$ be defined by 
\bnm
  \AdvTKR{\cipher}{\advA} = \Prob{\TKR^\advA_\cipherE \Rightarrow\true}  \;.
\enm

\begin{figure}[p]
\fpage{.45}{
\underline{$\KR^\advA_\cipher$}\\[1pt]
$K \getsr \keyspace$\\
$K^* \getsr \advA^\Fn$\\
$\win \gets \true$\\ 
For $M \in \calX$:\\
\ind If $\cipherE_{K^*}(M) \ne \cipherE_{K}(M)$ then\\
\ind\ind $\win \gets \false$\\
Ret $\win$\medskip

\underline{$\Fn(M)$}\\
$\calX \gets \calX \cup \{M\}$\\
$C \gets \cipherE_K(M)$\\
Ret $C$
}
\end{figure}

We let $\KR_\cipher$-advantage of a $\KR_\cipher$-adversary $\advA$ be defined by 
\bnm
  \AdvKR{\cipher}{\advA} = \Prob{\KR^\advA_\cipherE \Rightarrow\true}  \;.
\enm


\begin{figure}[p]
\fpage{.45}{
\underline{$\OTIND^\advA_\cipher$}\\[1pt]
$K \getsr \keyspace$\\
$b \getsr \bits$\\
$b' \getsr \advA^\Fn$\\
Ret $(b = b')$\medskip

\underline{$\Fn(M_0,M_1)$}\\
$C \gets \cipherE_K(M_b)$\\
Ret $C$
}
\end{figure}

We let $\OTIND_\cipher$-advantage of a $\OTIND_\cipher$-adversary $\advA$ be defined by 
\bnm
  \AdvOTIND{\cipher}{\advA} = 2\cdotsm\Prob{\OTIND^\advA_\cipherE \Rightarrow\true} - 1  \;.
\enm

\bigskip
\bigskip


\begin{itemize}
\item First example: Our simple OTP cipher is not $\TKR$ secure? Go over example: $\advA$
queries once on arbitrary message, recovers $K$ by computing $M \oplus C$. This
is guaranteed to succeed because $M \oplus C$ uniquely defines $K$. What does
this mean? Isn't OTP considered secure? Shannon said so!
%
\item Second example: Give toy cipher $\cipher$ for which $\TKR_\cipher$ has
$\AdvTKR{\cipher}{\advA} = 0$ for any adversary $\advA$. What is it?
$\cipherE(K,M) = M$. It is correct 
%
\item Third example: Exhaustive key search attack against generic
cipher. Emphasize that lower-bounding the efficacy of this is not possible in
general. Why? Consider toy identity cipher! 
%
\item Discuss the KR definition. Rules out the
identity map as being relevant. Lower-bounding security is 
%
\item Shannon's perfect secrecy (one-time left-or-right indistinguishability). 
\end{itemize}

\begin{theorem}
Let $\cipher$ be a cipher. For any $\TKR_\cipher$-adversary $\advA$, we give a
$\KR_\cipher$-adversary $\advB$ such that 
  $\AdvKR{\cipher}{\advA} = \AdvTKR{\cipher}{\advB}$.
\end{theorem}



\begin{theorem}
Let $\cipher$ be the OTP cipher. Then for any single-query
$\OTIND_\cipher$-adversary $\advA$ it holds that $\AdvOTIND{\cipher}{\advA} = 0$. \end{theorem}


\begin{theorem}
Let $\cipher$ be a cipher defined 
over $(\keyspace,\msgspace,\ctxtspace)$ such that for any $\OTIND_\cipherE$-adversary 
$\advA$ it holds that $\AdvOTIND{\cipher}{\advA} = 0$. Then $|\keyspace| \ge
|\msgspace|$. 
\end{theorem}

