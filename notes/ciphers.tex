%%%%%%%%%%%%%%%%%%%%%%%%%%%%%%%%%%%%%%%%%%%%%%%%%%%%%%%%%%%%%%%%%%%%%%%%%%%%%%%%
\section{Ciphers and Initial Security Notions}
\label{sec:se}

\paragraph{Ciphers.}
We start by defining a cipher. A cipher $\cipher = (\cipherE,\cipherD)$ is
defined by a a pair of deterministic algorithms $\cipherE$ and $\cipherD$.  To
any cipher $\cipher$ we associate sets called the key space $\keyspace$, message
space $\msgspace$, and ciphertext space $\ctxtspace$. We do not surface in the
notation for a cipher these sets, and will require that the association be clear
from context.

The algorithms are two-input. Enciphering takes a key $K \in \keyspace$ and
message $M \in \msgspace$, and outputs a ciphertext $C \in \ctxtspace$. Because
$\cipherE$ is deterministic, we can equally formalize it as a map 
$\cipherE\Colon\keyspace\times\msgspace\rightarrow\ctxtspace$. For a given key
$K$ we let $\cipherE_K\Colon\msgspace\rightarrow\ctxtspace$ be defined by
$\cipherE_K(M) = \cipherE(K,M)$ for all $M \in \msgspace$.  Deciphering takes a key $K\in \keyspace$ and ciphertext
$C \in \ctxtspace$ and outputs a message $M \in \msgspace$. Again, we can view
it as a map $\cipherD\Colon\keyspace\times\ctxtspace\rightarrow\msgspace$. 

Both $\cipherE$ and $\cipherD$ must be efficiently computable for all $K \in
\keyspace$.  (We have not defined efficiently computable, and use the term here
informally.) We require that a cipher be correct, meaning that for
$\cipherD_K(\cipherE_K(M)) = M$ for all $M$. 

\paragraph{Security notions.} What do we intuitively expect of a cipher?
Minimally:
\begin{itemize}
\item Secret key should remain secret
\item Message should remain secret
\end{itemize}
Let's try to formalize these notions.
\begin{itemize}
\item TKR security
\item KR security
\item (ot-)IND security
\end{itemize}


\begin{figure}[p]
\fpage{.45}{
\underline{$\TKR^\advA_\cipher$}\\[1pt]
$K \getsr \keyspace$\\
$K^* \getsr \advA^\Fn$\\
Ret $(K = K^*)$\medskip

\underline{$\Fn(M)$}\\
$C \gets \cipherE_K(M)$\\
Ret $C$
}
\end{figure}

We let $\TKR_\cipher$-advantage of a $\TKR_\cipher$-adversary $\advA$ be defined by 
\bnm
  \AdvTKR{\cipher}{\advA} = \Prob{\TKR^\advA_\cipherE \Rightarrow\true}  \;.
\enm

\begin{figure}[p]
\fpage{.45}{
\underline{$\KR^\advA_\cipher$}\\[1pt]
$K \getsr \keyspace$\\
$K^* \getsr \advA^\Fn$\\
$\win \gets \true$\\ 
For $M \in \calX$:\\
\ind If $\cipherE_{K^*}(M) \ne \cipherE_{K}(M)$ then\\
\ind\ind $\win \gets \false$\\
Ret $\win$\medskip

\underline{$\Fn(M)$}\\
$\calX \gets \calX \cup \{M\}$\\
$C \gets \cipherE_K(M)$\\
Ret $C$
}
\end{figure}

We let $\KR_\cipher$-advantage of a $\KR_\cipher$-adversary $\advA$ be defined by 
\bnm
  \AdvKR{\cipher}{\advA} = \Prob{\KR^\advA_\cipherE \Rightarrow\true}  \;.
\enm


\begin{figure}[p]
\fpage{.45}{
\underline{$\OTIND^\advA_\cipher$}\\[1pt]
$K \getsr \keyspace$\\
$b \getsr \bits$\\
$b' \getsr \advA^\Fn$\\
Ret $(b = b')$\medskip

\underline{$\Fn(M_0,M_1)$}\\
$C \gets \cipherE_K(M_b)$\\
Ret $C$
}
\end{figure}

We let $\OTIND_\cipher$-advantage of a $\OTIND_\cipher$-adversary $\advA$ be defined by 
\bnm
  \AdvOTIND{\cipher}{\advA} = 2\cdotsm\Prob{\OTIND^\advA_\cipherE \Rightarrow\true} - 1  \;.
\enm

\bigskip
\bigskip


\begin{itemize}
\item First example: Our simple OTP cipher is not $\TKR$ secure? Go over example: $\advA$
queries once on arbitrary message, recovers $K$ by computing $M \oplus C$. This
is guaranteed to succeed because $M \oplus C$ uniquely defines $K$. What does
this mean? Isn't OTP considered secure? Shannon said so!
%
\item Second example: Give toy cipher $\cipher$ for which $\TKR_\cipher$ has
$\AdvTKR{\cipher}{\advA} = 0$ for any adversary $\advA$. What is it?
$\cipherE(K,M) = M$. It is correct 
%
\item Third example: Exhaustive key search attack against generic
cipher. Emphasize that lower-bounding the efficacy of this is not possible in
general. Why? Consider toy identity cipher! 
%
\item Discuss the KR definition. Rules out the
identity map as being relevant. Lower-bounding security is 
%
\item Shannon's perfect secrecy (one-time left-or-right indistinguishability). 
\end{itemize}

\begin{theorem}
Let $\cipher$ be a cipher. For any $\TKR_\cipher$-adversary $\advA$, we give a
$\KR_\cipher$-adversary $\advB$ such that 
  $\AdvKR{\cipher}{\advA} = \AdvTKR{\cipher}{\advB}$.
\end{theorem}

In our theorem statements including reductions,  we need to interpret the words
``we give a''. We will focus on concrete, specified reductions. That means that
the adversary $\advB$ not only exists, but is fully specified  --- minus the
details of $\advA$ ---  within the proof.  In particular, if you give someone
$\advA$ then $\advB$ becomes runnable.  Runnable reductions are generally
speaking easier to interpret when it comes to implied security guarantees. They
even allow us to use the human-ignorance model~\cite{rogaway2006formalizing}
which, roughly, states that a reduction even to a mathematically easy assumption
can still be meaningful. (We will revisit this particular issue with an example
in the context of collision resistance.)
An example of a non-runnable $\advB$ would be one that includes some constant
value that we know exists, but don't know an exact value for. This comes up in
various arguments, and can cause problems in interpreting the reduction in terms
of concrete security.  This issue is subtle and we will revisit it.

The takeaway here being that one interprets ``we give a'' to mean runnable
adversaries that are specified fully in the proof. (Or when brevity is at stake,
specified to a leave of detail that the average reader could specify it in
detail easily.)  When we deviate from this convention we should remark on it.




\begin{theorem}
Let $\cipher$ be the OTP cipher. Then for any single-query
$\OTIND_\cipher$-adversary $\advA$ it holds that $\AdvOTIND{\cipher}{\advA} = 0$. \end{theorem}


\begin{theorem}
Let $\cipher$ be a cipher defined 
over $(\keyspace,\msgspace,\ctxtspace)$ such that for any $\OTIND_\cipherE$-adversary 
$\advA$ it holds that $\AdvOTIND{\cipher}{\advA} = 0$. Then $|\keyspace| \ge
|\msgspace|$. 
\end{theorem}


\begin{figure}
\fpage{.45}{
\underline{$\advA_{\textrm{eks}}$}\\[1pt]
$C \gets \Fn(M)$\\
For $K \in \calK$ do:\\
\ind If $C = \cipherE(K,M)$ then\\
\ind\ind Return $K$\\
Return $\bot$
}
\end{figure}

\paragraph{PRPs and PRFs.}

\begin{itemize}
\item Basic cipher definitions, key recovery attacks
\item PRP and PRF definitions
\item PRP/PRF switching lemma
\begin{itemize}
  \item Incorrect conditioning argument 
  \item Correct game-playing argument
\end{itemize}
\item PRP from PRF constructions: Feistel networks
\item Luby-Rackoff proof
\end{itemize}


\begin{figure}
\fpage{.15}{
\underline{$\PRP_{\cipher}^\advA$}\\
$b \getsr \bits$\\
$K \getsr \keyspace$\\
$\pi \getsr \Perm(n)$\\
$b' \getsr \advA^\Fn$\\
Return $(b = b')$\medskip

\underline{$\Fn(M)$}\\
If $b = 1$ then\\
\ind Return $\cipherE_K(M)$\\
Return $\pi(M)$
}
\end{figure}

\bnm
\AdvPRP{\cipher}{\advA} = 2\cdotsm\Prob{\PRP^\advA_\cipher\Rightarrow\true}- 1
\enm

\begin{figure}
\hfpages{.25}{
\underline{$\PRP1_{\cipher}^\advA$}\\
$K \getsr \keyspace$\\
$b' \getsr \advA^\Fn$\\
Return $b'$\medskip

\underline{$\Fn(M)$}\\
Return $\cipherE_K(M)$\\
}{
\underline{$\PRP0_{\cipher}^\advA$}\\
$\pi \getsr \Perm(n)$\\
$b' \getsr \advA^\Fn$\\
Return $b'$\medskip

\underline{$\Fn(M)$}\\
Return $\pi(M)$\\
}


\end{figure}


\begin{figure}
\hfpages{.25}{
\underline{$\PRF1_{\cipher}^\advA$}\\
$K \getsr \keyspace$\\
$b' \getsr \advA^\Fn$\\
Return $b'$\medskip

\underline{$\Fn(M)$}\\
Return $\cipherE_K(M)$\\
}{
\underline{$\PRF0_{\cipher}^\advA$}\\
$\rho \getsr \Func(n,n)$\\
$b' \getsr \advA^\Fn$\\
Return $b'$\medskip

\underline{$\Fn(M)$}\\
Return $\rho(M)$\\
}


\end{figure}



\begin{lemma} Let $\cipher$ be a cipher with ciphertext space $\bits^n$. 
Let $\advA$ be an adversary making at most $q$ queries. Then
\bnm
  \left| \Prob{\PRF0_\cipher^\advA\Rightarrow 1} 
      - \Prob{\PRP0_\cipher^\advA\Rightarrow1} \right| \le \frac{q^2}{2^n}  \;.
\enm
\end{lemma}

\begin{align*}
&\left| \Prob{\PRP0_\cipher^\advA\Rightarrow 1} 
      - \Prob{\PRF0_\cipher^\advA\Rightarrow1} \right| \\ 
     &\myInd\myInd\myInd =  \left|\Prob{\G0} - \Prob{\PRF0_\cipher^\advA\Rightarrow1} \right|  \\
     &\myInd\myInd\myInd  =  \left|\Prob{\G1} - \Prob{\PRF0_\cipher^\advA\Rightarrow1} \right|  \\
     &\myInd\myInd\myInd  \le \left|\Prob{\G2} + \Prob{\bad_2} - \Prob{\PRF0_\cipher^\advA\Rightarrow1} \right|\\
     &\myInd\myInd\myInd  = \Prob{\bad_2}\\
     &\myInd\myInd\myInd  \le \frac{q^2}{2^n}\\
\end{align*}

\begin{lemma} Let $\G$, $\Hgame$ be games that are identical-until-bad and $y$ be any
value. Then
\bnm
  \big| \Prob{\G\Rightarrow y} 
      - \Prob{\Hgame\Rightarrow y} \big| \le \Prob{\Hgame\setsbad} = \Prob{\G\setsbad}  \;.
\enm
\end{lemma}


\begin{figure}
\hfpagess{.20}{.20}{
\underline{$\G0$}\\[2pt]
$b' \getsr \advA^\Fn$\\
Return $b'$\medskip

\underline{$\Fn(M)$}\\
If $\TabF[M] = \bot$ then\\
\ind $\TabF[M] \getsr \bits^n \setminus \TabF$\\
Return $\TabF[M]$
}{
\underline{\fbox{$\G1$} \;\;\; $\G2$}\\[2pt]
$b' \getsr \advA^\Fn$\\
Return $b'$\medskip

\underline{$\Fn(M)$}\\
$C \getsr \bits^n$\\
If $C \in \TabF$ then\\
\ind $\badtrue$\\
\ind \fbox{$C \getsr \bits^n \setminus \TabF$}\\
$\TabF[M] \gets C$\\
Return $\TabF[M]$
}


\end{figure}



\begin{figure}
\hfpagessss{.20}{.20}{.20}{.20}{
\underline{$\G0$}\\[2pt]
$K \getsr \bits^k$\\
$b' \getsr \advA^\Fn$\\
Return $b'$\medskip

\underline{$\Fn(LR)$}\\
$L_1 \gets R$\\
$R_1 \gets L \oplus F_K(\langle 1\rangle \concat R)$\\
$L_2 \gets R_1$\\
$R_2 \gets L_1 \oplus F_K(\langle 2 \rangle \concat R_1)$\\
$L_3 \gets R_2$\\
$R_3 \gets L_2 \oplus F_K(\langle 3 \rangle \concat R_2)$\\
Return $L_3 \concat R_3$
}{
\underline{$\G1$}\\[2pt]
$\rho \getsr \Func(2n,n)$\\
$b' \getsr \advA^\Fn$\\
Return $b'$\medskip

\underline{$\Fn(LR)$}\\
$L_1 \gets R$\\
$R_1 \gets L \oplus \rho(\langle 1\rangle \concat R)$\\
$L_2 \gets R_1$\\
$R_2 \gets L_1 \oplus \rho(\langle 2 \rangle \concat R_1)$\\
$L_3 \gets R_2$\\
$R_3 \gets L_2 \oplus \rho(\langle 3 \rangle \concat R_2)$\\
Return $L_3 \concat R_3$
}{
\underline{$\fbox{\G2}$\;\;\;\G3}\\[2pt]
$b' \getsr \advA^\Fn$\\
Return $b'$\medskip

\underline{$\Fn(LR)$}\\
$L_1 \gets R$\\
If $\TabF[1,R] = \bot$ then\\
\ind $\TabF[1,R] \getsr \bits^n$\\
$R_1 \gets L \oplus \TabF[1,R]$\\
$L_2 \gets R_1$\\
$X_2 \getsr \bits^n$\\
If $\TabF[2,R_1] \ne \bot$ then\\
\ind $\badtrue$\\
\ind \fbox{$X_2 \gets \TabF[2,R_1]$}\\
$\TabF[2,R_1] \gets X_2$\\
$R_2 \gets L_1 \oplus X_2$\\
$L_3 \gets R_2$\\
$X_3 \getsr \bits^n$\\
If $\TabF[3,R_2] \ne \bot$ then\\
\ind $\badtrue$\\
\ind \fbox{$X_3 \gets \TabF[2,R_2]$}\\
$\TabF[3,R_2] \gets X_3$\\
$R_3 \gets L_2 \oplus X_3$\\
Return $L_3 \concat R_3$
}{
\underline{$\G4$}\\[2pt]
$b' \getsr \advA^\Fn$\\
Return $b'$\medskip

\underline{$\Fn(LR)$}\\
$L_1 \gets R$\\
If $\TabF[1,R] = \bot$ then\\
\ind $\TabF[1,R] \getsr \bits^n$\\
$R_1 \gets L \oplus \TabF[1,R]$\\
$L_2 \gets R_1$\\
If $\TabF[2,R_1] \ne \bot$ then\\
\ind $\badtrue$\\
$\TabF[2,R_1] \gets 1$\\
$R_2 \getsr \bits^n$\\
$L_3 \gets R_2$\\
%$X_3 \getsr \bits^n$\\
If $\TabF[3,R_2] \ne \bot$ then\\
\ind $\badtrue$\\
$\TabF[3,R_2] \gets 1$\\
$R_3 \getsr \bits^n$\\
Return $L_3 \concat R_3$
}
\end{figure}

\begin{figure}
\fpage{.25}{
\underline{$\advB^\Fn$}\\[2pt]
$K \getsr \bits^k$\\
$b' \getsr \advA^\FnSim$\\
Return $b'$\medskip

\underline{$\FnSim(LR)$}\\
$L_1 \gets R$\\
$R_1 \gets L \oplus \Fn(\langle 1\rangle \concat R)$\\
$L_2 \gets R_1$\\
$R_2 \gets L_1 \oplus \Fn(\langle 2 \rangle \concat R_1)$\\
$L_3 \gets R_2$\\
$R_3 \gets L_2 \oplus \Fn(\langle 3 \rangle \concat R_2)$\\
Return $L_3 \concat R_3$
}
\end{figure}

\newpage

\begin{theorem}
Let $\Feistel$ be the 3-round Feistel cipher using round function 
$F\Colon\bits^k\times\bits^n\rightarrow \bits^n$. For any
$\PRP_\cipher$-adversary $\advA$ making at most $q$ queries 
we give an $\PRF_F$-adversary $\advB$ making at most $3q$ queries such that
\bnm
  \AdvPRP{\Feistel}{\advA} \le \AdvPRF{F}{\advB} + \frac{2q^2}{2^n} +
  \frac{q^2}{2^{2n}} \;.
\enm
\end{theorem}



\begin{align*}
\AdvPRP{\cipher}{\advA} 
    &= \left|\Prob{\PRP1^\advA_\cipher} - \Prob{\PRP0^\advA_\cipher}\right|\\
    &= \left|\Prob{\G0} - \Prob{\PRP0^\advA_\cipher}\right|\\
    &\le \left|\Prob{\G1} + \AdvPRF{F}{\advB} - \Prob{\PRP0^\advA_\cipher}\right|\\
    &=   \left|\Prob{\G2} + \AdvPRF{F}{\advB} - \Prob{\PRP0^\advA_\cipher}\right|\\
    &\le \left|\Prob{\G3} + \Prob{\bad_3} + \AdvPRF{F}{\advB} - \Prob{\PRP0^\advA_\cipher}\right|\\
    &= \left|\Prob{\G4} + \Prob{\bad_4} + \AdvPRF{F}{\advB} - \Prob{\PRP0^\advA_\cipher}\right|\\
    &\le \left|\Prob{\PRP0^\advA_\cipher} + \frac{q^2}{2^{2n}} + \Prob{\bad_4} + \AdvPRF{F}{\advB} - \Prob{\PRP0^\advA_\cipher}\right|\\
    &= \frac{q^2}{2^{2n}} + \Prob{\bad_4} + \AdvPRF{F}{\advB}\\
    &\le \frac{q^2}{2^{2n}} + \frac{2q^2}{2^n} + \AdvPRF{F}{\advB}\\
\end{align*}

