%%%%%%%%%%%%%%%%%%%%%%%%%%%%%%%%%%%%%%%%%%%%%%%%%%%%%%%%%%%%%%%%%%%%%%%%%%%%%%%%
% title:
% authors:
%%%%%%%%%%%%%%%%%%%%%%%%%%%%%%%%%%%%%%%%%%%%%%%%%%%%%%%%%%%%%%%%%%%%%%%%%%%%%%%%
\def\fullver{0}
\def\submission{1}
\documentclass[11pt]{article}

\usepackage[letterpaper,hmargin=1in,vmargin=1in]{geometry}
\usepackage[backref=true,backend=bibtex]{biblatex}

\DefineBibliographyStrings{english}{%
  backrefpage = {page},% originally "cited on page"
  backrefpages = {pages},% originally "cited on pages"
}


\usepackage{framed}
%\usepackage{cite}
\usepackage[hidelinks]{hyperref}

\usepackage{color}
\usepackage{float}
\usepackage{setspace}
\usepackage{booktabs}
\usepackage{outlines}
\usepackage{enumitem}
\usepackage{epsfig}
\usepackage{wrapfig}
\usepackage{textcomp}
\usepackage{rotating}
\usepackage{xspace}
\usepackage{tikz}
\usetikzlibrary{calc,arrows,arrows.meta,shapes,positioning,matrix,decorations.pathreplacing}
\usepackage{amsthm}
\usepackage{amssymb}
\usepackage{pifont}
\usepackage{amsfonts}
\usepackage{amsmath}
\DeclareMathOperator*{\argmax}{arg\,max}
\DeclareMathOperator*{\argmin}{arg\,min}
\usepackage{cleveref}
\usepackage{bm}
\usepackage{listings}
\usepackage{mathtools}
\allowdisplaybreaks[2]
\usepackage{latexsym}
\usepackage{graphics}
\usepackage{graphicx}
\usepackage{fancyhdr}
\usepackage{url}
%\usepackage{enumerate}
\usepackage{adjustbox}
\usepackage{multirow}
\usepackage[font=small]{caption}
\def\authnotes{1}
\newtheorem{thm}{Theorem} %[section]
\newtheorem*{theorem*}{Theorem}
\newtheorem{lem}{Lemma}

\newtheorem{cor}[thm]{Corollary}
\newtheorem{propo}[thm]{Proposition}
\newtheorem{defn}[thm]{Definition}
\newtheorem{assm}[thm]{Assumption}
\newtheorem{clm}[thm]{Claim}
\newtheorem{rem}[thm]{Remark}
\newtheorem{exa}{Example}

\newenvironment{theorem}{\begin{thm}
    \begin{sl}
    }%
    {
    \end{sl}
  \end{thm}}

\newenvironment{lemma}{\begin{lem}\begin{sl}}%
    {\end{sl}\end{lem}}
\newenvironment{corollary}{\begin{cor}\begin{sl}}%
    {\end{sl}\end{cor}}
\newenvironment{proposition}{\begin{propo}\begin{sl}}%
    {\end{sl}\end{propo}}
\newenvironment{definition}{\begin{defn}\begin{sl}}%
    {\end{sl}\end{defn}}
\newenvironment{assumption}{\begin{assm}\begin{em}}%
    {\end{em}\end{assm}}
\newenvironment{claim}{\begin{clm}\begin{sl}}%
    {\end{sl}\end{clm}}
\newenvironment{remark}{\begin{rem}\begin{em}}%
    {\end{em}\end{rem}}
\newenvironment{example}{\begin{exa}\begin{em}}%
    {\end{em}\end{exa}}

\newcommand{\lemautorefname}{Lemma}
\newcommand{\algorithmautorefname}{Algorithm}
\renewcommand{\subsectionautorefname}{Section}


% Deluxe proof enviroment
\iffalse
    \def\qsym{\vrule width0.6ex height1em depth0ex}
    \def\qedsym{{\hspace{5pt}\rule[-1pt]{3pt}{9pt}}}
    \newcount\proofqeded
    \newcount\proofended
    \def\qed{%\qedsym
    \end{rm}\addtolength{\parskip}{-0pt}
    \setlength{\parindent}{\saveparindent}
    \global\advance\proofqeded by 1 }
    \newenvironment{proof}%
     {\proofstart}%
     {\ifnum\proofqeded=\proofended\qed\fi \global\advance\proofended by 1
      \medskip}
    \makeatletter
    \def\proofstart{\@ifnextchar[{\@oprf}{\@nprf}}
    \def\@oprf[#1]{\begin{rm}\protect\vspace{6pt}\noindent{\bf Proof of #1:\
    }%
    \addtolength{\parskip}{5pt}\setlength{\parindent}{0pt}}
    \def\@nprf{\begin{rm}\protect\vspace{6pt}\noindent{\bf Proof:\ }%
    \addtolength{\parskip}{5pt}\setlength{\parindent}{0pt}}
  \makeatother
\fi


% Extra padding for table entries
\newcommand\Tvsp{\rule{0pt}{2.6ex}}
\newcommand\Bvsp{\rule[-1.2ex]{0pt}{0pt}}
\newcommand{\TabPad}{\hspace*{5pt}}
\newcommand\TabSep{@{\hspace{5pt}}|@{\hspace{5pt}}}
\newcommand\TabSepLeft{|@{\hspace{5pt}}}
\newcommand\TabSepRight{@{\hspace{5pt}}|}



%\def\qedsym{{\ifnum\llncsclass=0  \else {\hspace{5pt}\rule[-1pt]{3pt}{9pt}} \fi}}
\def\qedsym{{\hspace{5pt}\rule[-1pt]{3pt}{9pt}} }

\def\pseudocodesize{ \footnotesize }

\renewcommand{\arraystretch}{1.3}
\DeclareMathAlphabet{\mathsl}{OT1}{cmr}{m}{sl}
\DeclareMathAlphabet{\mathsc}{OT1}{cmr}{m}{sc}


% Reference expansions
\newcommand{\secref}[1]{\mbox{Section~\ref{#1}}}
\newcommand{\subsecref}[1]{\mbox{Subsection~\ref{#1}}}
\newcommand{\apref}[1]{\mbox{Appendix~\ref{#1}}}
\newcommand{\thref}[1]{\mbox{Theorem~\ref{#1}}}
\newcommand{\thmrefshort}[1]{\mbox{\textbf{Th~\ref{#1}}}}
\newcommand{\exref}[1]{\mbox{Example~\ref{#1}}}
\newcommand{\defref}[1]{\mbox{Definition~\ref{#1}}}
\newcommand{\corref}[1]{\mbox{Corollary~\ref{#1}}}
\newcommand{\lemref}[1]{\mbox{Lemma~\ref{#1}}}
\newcommand{\clref}[1]{\mbox{Claim~\ref{#1}}}
\newcommand{\propref}[1]{\mbox{Proposition~\ref{#1}}}
\newcommand{\consref}[1]{\mbox{Construction~\ref{#1}}}
\newcommand{\figref}[1]{\mbox{Figure~\ref{#1}}}
%\newcommand{\eqref}[1]{\mbox{Equation~\ref{#1}}}

\newcommand{\enumref}[1]{\mbox{(\ref{#1})}}

\newcommand{\thmlabel}[1]{\textnormal{\textbf{[#1]}}}

\newcommand{\SetFigFont}[5]{} % This is for xfig issue
\newcommand{\SetFigFontNFSS}[5]{} % This is for xfig issue

\DeclarePairedDelimiter{\ceil}{\lceil}{\rceil}%
\DeclarePairedDelimiter{\floor}{\lfloor}{\rfloor}%
\DeclarePairedDelimiter\absv{\lvert}{\rvert}%
%\DeclarePairedDelimiter\norm{\lVert}{\rVert}%
\DeclarePairedDelimiter\prns{(}{)}%
\DeclarePairedDelimiter\braces{\{}{\}}%
\DeclarePairedDelimiter\bracks{[}{]}%
\DeclarePairedDelimiterX\condprns[2]{(}{)}{\,#1 \;\delimsize\vert\; #2\,}
\DeclarePairedDelimiterX\condbrks[2]{[}{]}{\,#1 \;\delimsize\vert\; #2\,}
\DeclarePairedDelimiterX\condbraces[2]{\{}{\}}{\,#1 \;\delimsize\vert\; #2\,}

% ========================================================================

% Lists

\newcounter{ctr}
\newcounter{savectr}
\newcounter{ectr}

\newlength{\saveparindent}
\setlength{\saveparindent}{\parindent}
\newlength{\saveparskip}
\setlength{\saveparskip}{\parskip}


\newenvironment{newitemize}{%
\begin{list}{\mbox{}\hspace{5pt}$\bullet$\hfill}{\labelwidth=15pt%
\labelsep=5pt \leftmargin=20pt \topsep=3pt%
\setlength{\listparindent}{\saveparindent}%
\setlength{\parsep}{\saveparskip}%
\setlength{\itemsep}{3pt} }}{\end{list}}


\newenvironment{newenum}{%
\begin{list}{{\rm (\arabic{ctr})}\hfill}{\usecounter{ctr} \labelwidth=17pt%
\labelsep=5pt \leftmargin=22pt \topsep=3pt%
\setlength{\listparindent}{\saveparindent}%
\setlength{\parsep}{\saveparskip}%
\setlength{\itemsep}{2pt} }}{\end{list}}

\newenvironment{tiret}{%
\begin{list}{\hspace{2pt}\rule[0.5ex]{6pt}{1pt}\hfill}{\labelwidth=15pt%
\labelsep=3pt \leftmargin=22pt \topsep=3pt%
\setlength{\listparindent}{\saveparindent}%
\setlength{\parsep}{\saveparskip}%
\setlength{\itemsep}{2pt}}}{\end{list}}


\newenvironment{blocklist}{\begin{list}{}{\labelwidth=0pt%
\labelsep=0pt \leftmargin=0pt \topsep=10pt%
\setlength{\listparindent}{\saveparindent}%
\setlength{\parsep}{\saveparskip}%
\setlength{\itemsep}{20pt}}}{\end{list}}

\newenvironment{blocklistindented}{\begin{list}{}{\labelwidth=0pt%
\labelsep=30pt \leftmargin=30pt\topsep=5pt%
\setlength{\listparindent}{\saveparindent}%
\setlength{\parsep}{\saveparskip}%
\setlength{\itemsep}{10pt}}}{\end{list}}

\newenvironment{onelist}{%
\begin{list}{{\rm (\arabic{ctr})}\hfill}{\usecounter{ctr} \labelwidth=18pt%
\labelsep=7pt \leftmargin=25pt \topsep=2pt%
\setlength{\listparindent}{\saveparindent}%
\setlength{\parsep}{\saveparskip}%
\setlength{\itemsep}{2pt} }}{\end{list}}

\newenvironment{twolist}{%
\begin{list}{{\rm (\arabic{ctr}.\arabic{ectr})}%
\hfill}{\usecounter{ectr} \labelwidth=26pt%
\labelsep=7pt \leftmargin=33pt \topsep=2pt%
\setlength{\listparindent}{\saveparindent}%
\setlength{\parsep}{\saveparskip}%
\setlength{\itemsep}{2pt} }}{\end{list}}

\newenvironment{centerlist}{%
\begin{list}{\mbox{}}{\labelwidth=0pt%
\labelsep=0pt \leftmargin=0pt \topsep=10pt%
\setlength{\listparindent}{\saveparindent}%
\setlength{\parsep}{\saveparskip}%
\setlength{\itemsep}{10pt} }}{\end{list}}

\newenvironment{newcenter}[1]{\begin{centerlist}\centering%
\item #1}{\end{centerlist}}

\newenvironment{codecenter}[1]{\begin{small}\begin{centerlist}\centering%
\item #1}{\end{centerlist}\end{small}}

% =========================================================================

% Math

\newlength{\savejot}
\setlength{\jot}{3pt}
\setlength{\savejot}{\jot}

\newenvironment{newmath}{\begin{displaymath}%
\setlength{\abovedisplayskip}{4pt}%
\setlength{\belowdisplayskip}{4pt}%
\setlength{\abovedisplayshortskip}{6pt}%
\setlength{\belowdisplayshortskip}{6pt} }{\end{displaymath}}

\newenvironment{newequation}{\begin{equation}%
\setlength{\abovedisplayskip}{4pt}%
\setlength{\belowdisplayskip}{4pt}%
\setlength{\abovedisplayshortskip}{6pt}%
\setlength{\belowdisplayshortskip}{6pt} }{\end{equation}}

%\newcommand{\headingg}[1]{{\textbf{#1}}}
%\newcommand{\heading}[1]{{\vspace{6pt}\noindent\textbf{#1}}}
\newcommand{\ind}{\hspace*{1.5em}}
\newcommand{\indsm}{\hspace*{.75em}}
\newcommand{\indeqn}{\;\;\;\;\;\;\;}
\newcommand{\bits}{\{0,1\}}
\newcommand{\zon}[1]{\bits^{#1}}
\newcommand{\emptystring}{\varepsilon}
\newcommand{\xor}{{\:\oplus\:}}
%\newcommand{\concat}{\,,\,}
\newcommand{\smidge}{{\kern .05em}}
\newcommand{\Colon}{{\smidge\colon\smidge}}
\newcommand{\norm}[1]{\|#1\|}
\def\poly{\mathop{\rm poly}\nolimits}
\def\div{\mathop{\rm div}\nolimits}
%\newcommand{\ro}{RO}
\newcommand{\advA}{{\mathcal A}}
\newcommand{\advB}{{\mathcal B}}
\newcommand{\advC}{{\mathcal C}}
\newcommand{\advD}{{\mathcal D}}
\newcommand{\advE}{{\mathcal E}}

\newcommand{\calA}{{\cal A}}
\newcommand{\calB}{{\cal B}}
\newcommand{\calC}{{\cal C}}
\newcommand{\calE}{{\cal E}}
\newcommand{\calF}{{\cal F}}
\newcommand{\calG}{{\cal G}}
\newcommand{\calH}{{\cal H}}
\newcommand{\calI}{{\cal I}}
\newcommand{\calJ}{{\cal J}}
\newcommand{\calO}{{\cal O}}
\newcommand{\calR}{{\cal R}}
\newcommand{\calS}{{\cal S}}
\newcommand{\calD}{{\cal D}}
\newcommand{\calK}{{\cal K}}
\newcommand{\calL}{{\cal L}}
\newcommand{\calM}{{\cal M}}
\newcommand{\calN}{{\cal N}}
\newcommand{\calP}{{\cal P}}
\newcommand{\calQ}{{\cal Q}}
\newcommand{\calT}{{\cal T}}
\newcommand{\calU}{{\cal U}}
\newcommand{\calV}{{\cal V}}
\newcommand{\calW}{{\cal W}}
\newcommand{\calX}{{\cal X}}
\newcommand{\calY}{{\cal Y}}

\newcommand{\N}{{{\mathbb N}}}
\newcommand{\Z}{{{\mathbb Z}}}
\newcommand{\G}{{{\textnormal G}}}
\newcommand{\Hgame}{{{\textnormal H}}}
\newcommand{\bigO}{\calO}
\newcommand{\goesto}{{\rightarrow}}
\newcommand{\eqdef}{\stackrel{\rm def}{=}}
\newcommand{\negsmidge}{{\hspace{-0.1ex}}}
\newcommand{\cdotsm}{\negsmidge\negsmidge\negsmidge\cdot\negsmidge\negsmidge\negsmidge}
\def\union{\cup}
\def\sep{\,}
\def\bigunion{\bigcup}
\def\suchthatt{\: :\:}
%\def\next{\hspace{12pt};\hspace{12pt}}
\def\nextt{\hspace{3pt};\hspace{6pt}}
\newcommand{\set}[2]{\{\:#1 \suchthatt #2\:\}}
\newcommand{\card}[1]{|#1|}
\def\leqq{\;\leq\;}
\def\eqq{\;=\;}
\def\geqq{\;\geq\;}
\def\lst{\;<\;}
\def\gst{\;>\;}
\def\prn#1{\left(#1\right)}

% \newcolumntype{y}[1]{%
% >{\hspace{0pt}}p{#1}}%

% \newcolumntype{x}[1]{%
% >{\centering\hspace{0pt}}p{#1}}%




\newcommand{\verylongleftarrow}[1]
      {\setlength{\unitlength}{.01in}
      \begin{picture}(#1,1) \put(#1,0){\vector(-1,0){#1}} \end{picture}}
\newcommand{\verylongrightarrow}[1]             %longleft and rightgoing arrows
      {\setlength{\unitlength}{.01in}           %for protocols
      \begin{picture}(#1,1) \put(0,0){\vector(1,0){#1}} \end{picture}}
\newcommand{\verylongbotharrow}[2]             %longleft and rightgoing arrows
      {\setlength{\unitlength}{.01in}           %for protocols
      \begin{picture}(#2,1) \put(#1,0){\vector(1,0){#1}}
                            \put(#1,0){\vector(-1,0){#1}} \end{picture}}
\newcommand{\leftgoing}[2]{{\stackrel{{\displaystyle #2}} {\verylongleftarrow{#1}}}}
\newcommand{\rightgoing}[2]{{\stackrel{{\displaystyle #2}} {\verylongrightarrow{#1}}}}
\newcommand{\bothgoing}[3]{{\stackrel{{\displaystyle #3}} {\verylongbotharrow{#1}{#2}}}}


\newcommand{\leftgoinga}[1]{\leftgoing{230}{#1} }
\newcommand{\rightgoinga}[1]{\rightgoing{230}{#1} }

\newcommand{\leftgoingb}[1]{\leftgoing{300}{#1} }
\newcommand{\rightgoingb}[1]{\rightgoing{300}{#1} }




\newcommand{\veryshortleftarrow}[1]
      {\setlength{\unitlength}{.01in}
      \begin{picture}(#1,1) \put(#1,0){\vector(-1,0){#1}} \end{picture}}


\newcommand{\getparse}[1]{{\:\stackrel{\raisebox{-0.5em}{{\hspace{0.1em}\mbox{\boldmath$\scriptscriptstyle
            #1$}}}}{\leftarrow}\:}}
\newcommand{\getu}{{\:\stackrel{\raisebox{-0.5em}{{\hspace{0.1em}\mbox{\boldmath$\scriptscriptstyle \cup$}}}}{\leftarrow}\:}}
\newcommand{\getdiff}{{\:\stackrel{{\scriptscriptstyle\hspace{0.2em} /}}{\leftarrow}\:}}
\newcommand{\getsr}{{\:{\leftarrow{\hspace*{-3pt}\raisebox{.75pt}{$\scriptscriptstyle\$$}}}\:}}
\newcommand{\get}[1]{{\:\leftarrow_{#1}\:}}
%\newcommand{\getsr}{{\:{\raisebox{3pt}{\veryshortleftarrow{15}}{\raisebox{1pt}{$\scriptscriptstyle\$$}}}\:}}
%\newcommand{\getsr}{{\:{\xleftarrow{\scriptscriptstyle\$}}\:}}
%\newcommand{\getsr}{{\:\stackrel{\raisebox{-0.5em}{$\scriptscriptstyle \hspace{0.2em}\$$}}{\leftarrow}\:}}
\newcommand{\sendsr}{{\:\stackrel{\scriptscriptstyle \hspace{0.2em}\$}{\rightarrow}\:}}
\renewcommand{\choose}[2]{{{#1}\atopwithdelims(){#2}}}
\newcommand{\abs}[1]{{\displaystyle \left| {#1} \right| }}
\newcommand{\E}{{\mbox{\bf E}}}
\newcommand{\EE}[1]{{\E\left[{#1}\right]}}
\newcommand{\EEE}[2]{{\E_{#1}\left[{#2}\right]}}
\newcommand{\Ex}{{\textbf E}}
\newcommand{\Exx}[1]{{\Ex\left[{#1}\right]}}
\newcommand{\Exxx}[2]{{\Ex_{#1}\left[{#2}\right]}}
\newcommand{\Var}{{\textnormal{Var}}}
\newcommand{\Varr}[1]{{\Var\left[{#1}\right]}}
\newcommand{\Varrr}[2]{{\Var_{#1}\left[{#2}\right]}}
\newcommand{\Prob}[1]{\Pr\left[\: #1 \:\right]}
\newcommand{\CondProb}[2]{{\Pr}\left[\: #1\:\left|\right.\:#2\:\right]}
\newcommand{\CondProbb}[3]{{\Pr}_{#1}\left[\: #2\:\left|\right.\:#3\:\right]}
\newcommand{\ProbExp}[2]{\Pr\left[\: #1 \: : \: #2\: \right]}
\newcommand{\Probb}[2]{{\Pr}_{#1}\left[\: #2 \:\right]}
\newcommand{\Probc}[2]{\Pr\left[\: #1 \:{\left|\right.}\:#2\:\right]}
\newcommand{\Probcc}[3]{{\Pr}_{#1}\left[\: #2 \:\left|\right.\:#3\:\right]}
\newcommand{\suchthat}{{\mbox{s.t.\ }}}
\newcommand{\qquadd}{{\quad}}
\def\d{{\delta}}
\def\e{{\epsilon}}
\newcommand{\ceiling}[1]{\lceil #1\rceil}
\newcommand{\sfrac}[2]{{\textstyle \frac{#1}{#2}}}
\newcommand{\ssum}[2]{{\textstyle \sum_{\,#1}^{\,#2}\,}}
\newcommand{\sprod}[2]{{\textstyle \prod_{\,#1}^{\,#2}\,}}
\def\smax{{\textstyle \max}}
%\def\N{{\sf N}}
\def\R{{\sf R}}
%\def\getsr{\stackrel{\$}{\leftarrow}}
\newcommand{\blockindex}[2]{{\langle#1\rangle}_{#2}}
\def\chv{\raisebox{2pt}{$\chi$}}

\newcommand{\cclass}[1]{{\rm #1}}
\def\P{\cclass{P}}
\def\NP{\cclass{NP}}
\def\BPP{\cclass{BPP}}
\def\coRP{\cclass{coRP}}
\def\NEXP{\cclass{NEXP}}
\def\DES{\mbox{\rm DES}}
\newcommand{\md}{\textsf{md5}}
\newcommand{\MD}{\textnormal{MD}}
\newcommand{\MDb}{\textbf{MD}}
\newcommand{\sha}{\textsf{sha-1}}
\newcommand{\ripemd}{\textsf{ripemd-160}}

\newcommand{\badSet}{\calS}
\newcommand{\bad}{\ensuremath{\mathsf{bad}}}
\newcommand{\cbad}{\bad}
\newcommand{\ctrue}{\true}
\newcommand{\notbad}{\overline{\bad}}
\newcommand{\win}{\ensuremath{\mathsf{win}}}
\newcommand{\outputs}{\:{\Rightarrow}\:}
\newcommand{\cheat}{\mathsf{cheat}}
\newcommand{\cheattrue}{\cheat\gets\true}
\newcommand{\notcheated}{\mathsf{Honest}}
\newcommand{\badtrue}{\bad\gets\true}
\newcommand{\Good}{\mathsf{Good}}
\newcommand{\EvNbadA}{\overline{\mathsf{Bad}}_1}
\newcommand{\EvbadA}{\mathsf{Bad}_1}
\newcommand{\EvNbadB}{\overline{\mathsf{Bad}}_2}
\newcommand{\EvbadB}{\mathsf{Bad}_2}
\newcommand{\badA}{\bad_1}
\newcommand{\badB}{\bad_2}
\newcommand{\badC}{\bad_3}
\newcommand{\badAtrue}{\badA\gets\true}
\newcommand{\badBtrue}{\badB\gets\true}
\newcommand{\badCtrue}{\badC\gets\true}
\newcommand{\setsbad}{\mbox{~\textup{sets} $\bad$}\,}
\newcommand{\setsbadA}{\mbox{~\textup{sets} $\badA$}\,}
\newcommand{\setsbadB}{\mbox{~\textup{sets} $\badB$}\,}
\newcommand{\setsbadzero}{\mbox{~\textup{sets} $\badzero$}\,}
\newcommand{\setsbadone}{\mbox{~\textup{sets} $\badone$}\,}
\newcommand{\setsbadtwo}{\mbox{~\textup{sets} $\badtwo$}\,}
\newcommand{\setsbadthree}{\mbox{~\textup{sets} $\badthree$}\,}
\newcommand{\setsbadfour}{\mbox{~\textup{sets} $\badfour$}\,}
\newcommand{\badzero}{\ensuremath{\mathsf{bad}_0}}
\newcommand{\badone}{\ensuremath{\mathsf{bad}_1}}
\newcommand{\badtwo}{\ensuremath{\mathsf{bad}_2}}
\newcommand{\badthree}{\ensuremath{\mathsf{bad}_3}}
\newcommand{\badfour}{\ensuremath{\mathsf{bad}_4}}
\newcommand{\defined}{\ensuremath{\mathsf{defined}}}
%\newcommand{\undefined}{\ensuremath{\mathsf{undefined}}}
\newcommand{\true}{\ensuremath{\mathsf{true}}}
\newcommand{\false}{\ensuremath{\mathsf{false}}}
\newcommand{\zero}{\ensuremath{\mathsf{zer}}}
\newcommand{\one}{\ensuremath{\mathsf{one}}}
\newcommand{\Initialize}{{\textbf{Initialize}}}
\newcommand{\Finalize}  {{\textbf{Finalize}}}
\newcommand{\Onquery}   {{\textbf{procedure~}}}
\newcommand{\onquery}   {{\textbf{query~}}}
\newcommand{\Update}{{\textnormal{update}}}
%\newcommand{\algorithm}[1]{{\textbf{algorithm~}{#1}}}
\newcommand{\fn}{\footnotesize}

\newcommand{\hash}{\mathcal{H}}

\newcommand{\Hash}{\textbf{Hash}}
\newcommand{\fEval}{\textbf{f-Eval}}
\newcommand{\fReveal}{\textbf{f-Reveal}}

\newcommand{\xRV}{X_{t,i}}
\newcommand{\XRV}{X_t}
\newcommand{\zRV}{Z_{t\concat r}}
\newcommand{\ZRV}{Z}
\newcommand{\const}{\mathsf{c}}

% from /usr/local/lib/TeX+MF/tex/macros/art11.sty

\makeatletter

\def\subsubsection{\@startsection{subsubsection}{3}{\z@}{-2.25ex plus
 -1ex minus -.2ex}{1.5ex plus .2ex}{\sf}}


% =========================================================================


% Definitions for this paper
\newcommand{\secparam}{\kappa}
\newcommand{\ctr}{\mathit{ctr}}
\newcommand{\MAC}{\mathsf{MAC}}
\newcommand{\kwfont}[1]{\textrm{#1}}
\newcommand{\procfont}[1]{\textbf{#1}}
\newcommand{\nfont}[1]{{\footnotesize #1}}
\newcommand{\lnum}[1]{{\footnotesize #1}\;\;}

\newcommand{\ETS}[2]{\cal{ETS}_{#1}^{#2}}

\newcommand{\Exp}{\mathbf{Exp}}
\newcommand{\MUExp}[4]{\mathbf{Exp}^{\mathrm{n \mbox{-}mu\mbox{-}#4\mbox{-}#1}}_{#2}(#3)}
%\newcommand{\SUExp}[2]{\mathbf{Exp}^{\mathrm{su\mbox{-}cpa\mbox{-}#2}}_{#1}}
\newcommand{\SUExp}[3]{\mathbf{Exp}^{\mathrm{ind\mbox{-}{#3}}}_{#1}(#2)}
\newcommand{\ForgeExp}[2]{\mathbf{Exp}^{\mathrm{uf\mbox{-}cma}}_{#1}(#2)}
\newcommand{\LeakExp}[3]{\mathbf{Exp}^{\mathrm{pred\mbox{-}ct\mbox{-}#3}}_{#1}(#2)}
\newcommand{\CompExp}[3]{\mathbf{Exp}^{\mathrm{pred\mbox{-}pt\mbox{-}#3}}_{#1}(#2)}
\newcommand{\FuncExp}[2]{\mathbf{Exp}^{\mathrm{ss\mbox{-}real}}_{#1}(#2)}
\newcommand{\FuncExpReal}[2]{\mathbf{Exp}^{\mathrm{ror\mbox{-}det\mbox{-}real}}_{#1}(#2)}
\newcommand{\FuncExpRand}[2]{\mathbf{Exp}^{\mathrm{ror\mbox{-}det\mbox{-}rand}}_{#1}(#2)}
\newcommand{\FuncExpRealH}[2]{\mathbf{Exp}^{\mathrm{hyb\mbox{-}det\mbox{-}real}}_{#1}(#2)}
\newcommand{\FuncExpRandH}[2]{\mathbf{Exp}^{\mathrm{hyb\mbox{-}det\mbox{-}rand}}_{#1}(#2)}
\newcommand{\FuncExpD}[2]{\mathbf{Exp}^{\mathrm{det\mbox{-}1}}_{#1}(#2)}
\newcommand{\FuncExpS}[3]{\mathbf{Exp}^{\mathrm{priv\mathrm{\mbox{-}#3}\mbox{-}1}}_{#1}(#2)}
\newcommand{\FuncExpDR}[2]{\mathbf{Exp}^{\mathrm{ss\mbox{-}det\mbox{-}rand}}_{#1}(#2)}
\newcommand{\FuncPTExp}[2]{\mathbf{Exp}^{\mathrm{func\mbox{-}pt}}_{#1}(#2)}
\newcommand{\PlainFuncExp}[3]{\mathbf{Exp}^{\mathrm{func\mbox{-}nil\mbox{-}#3}}_{#1}(#2)}
\newcommand{\FuncPredExp}[2]{\mathbf{Exp}^{\mathrm{ss\mbox{-}sim}}_{#1}(#2)}
\newcommand{\FuncPredExpD}[2]{\mathbf{Exp}^{\mathrm{det\mbox{-}0}}_{#1}(#2)}
\newcommand{\FuncPredExpS}[3]{\mathbf{Exp}^{\mathrm{priv\mathrm{\mbox{-}#3}\mbox{-}0}}_{#1}(#2)}
\newcommand{\SSExp}[2]{\mathbf{Exp}^{\mathrm{ss}\mbox{-}\mathrm{real}}_{#1}(#2)}
\newcommand{\SSExpS}[2]{\mathbf{Exp}^{\mathrm{ss\mbox{-}sim}}_{#1}(#2)}
\newcommand{\InvertExp}[2]{\mathbf{Exp}^{\mathrm{pt\mbox{-}pred}}_{#1}(#2)}
\newcommand{\IndOracle}[2]{\mathbf{I}_{#1}(#2)}
\newcommand{\BOracle}{\mathsf{B}}
\newcommand{\EqOracle}{\mathsf{Eq}}
\newcommand{\REC}{\mathsf{REC}}
\newcommand{\notask}{\overline{\mathsc{Ask}}}
\newcommand{\ask}{\mathsc{Ask}}
\newcommand{\MQ}{\mathsf{MsgQueried}}
\newcommand{\MG}{\mathsf{QueryOfMsgGuessed}}
\newcommand{\MC}{\mathsf{MsgChosenAgain}}
\newcommand{\MT}{\mathsf{MsgTargeted}}
\newcommand{\AT}{\mathsf{AdvFindsTarget}}
\newcommand{\ME}{\mathsf{MsgsAreEqual}}
\newcommand{\QM}{\mathsf{AskM}}
\newcommand{\QR}{\mathsf{AskM_r}}
\newcommand{\TG}{\mathsf{CorrGuess}}
\newcommand{\flips}{n_{\text{f}}}
\newcommand{\mprime}{m^{\prime}}

\newcommand{\queries}{q}

\newcommand{\Aguess}{A_{\mathrm{g}}}
\newcommand{\Ad}{A_{\mathrm{d}}}
\newcommand{\Ap}{A_{\mathrm{p}}}
\newcommand{\Af}{A_{\mathrm{f}}}
\newcommand{\Ag}{A_{\mathrm{g}}}
\newcommand{\Am}{A_{\mathrm{m}}}
\newcommand{\Ac}{A_{\mathrm{c}}}
\newcommand{\Ai}{A_{\mathrm{i}}}



\newcommand{\Aone}{A_1}
\newcommand{\Atwo}{A_2}
\newcommand{\Done}{D_1}
\newcommand{\Dtwo}{D_2}

\newcommand{\Agone}{A^*_{\mathrm{g}}}
\newcommand{\Amone}{A^*_{\mathrm{m}}}
\newcommand{\Acone}{A^*_{\mathrm{c}}}

\newcommand{\Astar}{A^*}
\newcommand{\Agstar}{A^*_{\mathrm{g}}}
\newcommand{\Amstar}{A^*_{\mathrm{m}}}
\newcommand{\Acstar}{A^*_{\mathrm{c}}}

\newcommand{\Ig}{I_{\mathrm{g}}}
\newcommand{\Iguess}{\Ig}
\newcommand{\Imsg}{I_{\mathrm{m}}}
\newcommand{\Ic}{I_{\mathrm{c}}}
\newcommand{\Ip}{I_{\mathrm{p}}}
\newcommand{\Is}{I_{\mathrm{s}}}

\newcommand{\Istar}{I^*}
\newcommand{\Igstar}{I^*_{\mathrm{g}}}
\newcommand{\Imstar}{I^*_{\mathrm{m}}}
\newcommand{\Icstar}{I^*_{\mathrm{c}}}


\newcommand{\Jm}{J_{\mathrm{m}}}
\newcommand{\Jg}{J_{\mathrm{g}}}
\newcommand{\Jc}{J_{\mathrm{c}}}
\newcommand{\Jp}{J_{\mathrm{p}}}
\newcommand{\Js}{J_{\mathrm{s}}}

\newcommand{\Dm}{D_{\mathrm{m}}}
\newcommand{\Dc}{D_{\mathrm{c}}}
\newcommand{\Dp}{D_{\mathrm{p}}}
\newcommand{\Dg}{D_{\mathrm{g}}}

\newcommand{\Pg}{P_{\mathrm{g}}}
\newcommand{\Pp}{P_{\mathrm{p}}}
\newcommand{\Pt}{P_{\mathrm{t}}}
\newcommand{\Pguess}{\Pg}
\newcommand{\Pmsg}{P_{\mathrm{m}}}
\newcommand{\Pm}{\Pmsg}
\newcommand{\Pc}{P_{\mathrm{c}}}
\newcommand{\Pd}{P_{\mathrm{d}}}


\newcommand{\Bc}{B_{\mathrm{c}}}
\newcommand{\Bf}{B_{\mathrm{f}}}
\newcommand{\Bg}{B_{\mathrm{g}}}
\newcommand{\Bm}{B_{\mathrm{m}}}
\newcommand{\Bi}{B_{\mathrm{i}}}

\newcommand{\SSAdv}{\calA_{\mathrm{SS}}}
\newcommand{\TrivAdv}{\calA_\emptystr}
%\newcommand{\LegAdv}{\calA_L}
%\newcommand{\EQAdv}{\calA_{\mathrm{E}}}
\newcommand{\BoolAdv}{\calA_{\mathrm{B}}}
\newcommand{\BBAdv}{\calA_{\mathrm{BB}}^\delta}
\newcommand{\MEAdv}{\calA_{\mathrm{ME}}^\mu}
\newcommand{\UMEAdv}{\calA_{\mathrm{UN}}}
\newcommand{\IMAdv}{\calA_{\times}}
\newcommand{\AdvSep}{\calA_{\mathrm{sep}}}

\newcommand{\AdvJell}{\calJ_{\ell}}
\newcommand{\AdvJzero}{\calJ_{0}}
\newcommand{\AdvIell}{\calJ_{\ell}}
\newcommand{\AdvDell}{\calD_{\ell}}

\newcommand{\AdvJme}{\calJ_{\mathrm{ME}}^\mu}
\newcommand{\AdvDme}{\calD_{\mathrm{ME}}^\mu}

\newcommand{\MEAdvv}[1]{\calA_{\mathrm{E}}^{#1}}
\newcommand{\HEAdv}{\calA_{\mathrm{HE}}}
\newcommand{\SMEAdv}{\calA_{\mathrm{SME}}^\mu}

\newcommand{\INDCPA}{\textnormal{IND-CPA}\xspace}
\newcommand{\INDCCA}{\textnormal{IND-CCA}\xspace}
\newcommand{\INDRAND}{\textnormal{IND\$}\xspace}
\newcommand{\INDSIM}{\textnormal{IND-SIM}\xspace}

\newcommand{\INDAdv}{\calI}
\newcommand{\TrivAdvI}{\calI_\emptystr}
\newcommand{\MEAdvI}{\calI_{\mathrm{ME}}^\mu}
\newcommand{\MEAdvvI}[1]{\calI_{\mathrm{ME}}^{#1}}
\newcommand{\HEAdvI}{\calI_{\mathrm{HE}}}
\newcommand{\SMEAdvI}{\calI_{\mathrm{SME}}^\mu}

\newcommand{\setZero}{\mathtt{I}_0}

%\newcommand{\MUExpcca}[2]{\mathbf{Exp}^{\mathrm{n \mbox{-}mu\mbox{-}cca\mbox{-}#2}}_{#1}}
%\newcommand{\SUExpcca}[2]{\mathbf{Exp}^{\mathrm{su\mbox{-}cca}\mbox{-}{#2}}_{#1}}
\newcommand{\DDHRealexp}[2]{\mathbf{Exp}^{\mathrm{ddh\mbox{-}real}}_{#1}(#2)}
\newcommand{\DDHRandexp}[2]{\mathbf{Exp}^{\mathrm{ddh\mbox{-}rand}}_{#1}(#2)}

\newcommand{\HyExp}{\mathbf{ExpH}}
\newcommand{\find}{\mathsf{find}}
\newcommand{\guess}{g}

\newcommand{\wPRF}{\mathsf{wPRF}}

\newcommand{\Succ}{\mathsf{Succ}}
\newcommand{\Adv}{\mathbf{Adv}}
\newcommand{\Advcpa}[1]{\Adv^{\mathrm{ind\mbox{-}cpa}}_{#1}}
\newcommand{\Advcca}[1]{\Adv^{\mathrm{ind\mbox{-}cca}}_{#1}}
\newcommand{\AdvBIND}[2]{\Adv^{\mathrm{r\dash bind}}_{#1}(#2)}
\newcommand{\AdvRBIND}[2]{\Adv^{\mathrm{r\dash bind}}_{#1}(#2)}
\newcommand{\AdvSRBIND}[2]{\Adv^{\mathrm{sr\dash bind}}_{#1}(#2)}

\newcommand{\AdvVFROB}[2]{\Adv^{\mathrm{efrob}}_{#1}(#2)}
\newcommand{\AdvSBIND}[2]{\Adv^{\mathrm{s\dash bind}}_{#1}(#2)}
\newcommand{\AdvDBIND}[2]{\Adv^{\mathrm{t\dash bind}}_{#1}(#2)}
\newcommand{\AdvMRBIND}[2]{\Adv^{\mathrm{mr\dash bind}}_{#1}(#2)}
\newcommand{\AdvVBIND}[2]{\Adv^{\mathrm{v\dash bind}}_{#1}(#2)}
\newcommand{\AdvNAE}[1]{\Adv^{\mathrm{dae}}_{#1}}
\newcommand{\AdvFROB}[2]{\Adv^{\mathrm{frob}}_{#1}(#2)}
\newcommand{\AdvOTROR}[2]{\Adv^{\mathrm{ot \dash ror}}_{#1}(#2)}
\newcommand{\AdvOTCTXT}[2]{\Adv^{\mathrm{scu}}_{#1}(#2)}
\newcommand{\AdvotCTXT}[2]{\Adv^{\mathrm{ot-ctxt}}_{#1}(#2)}
\newcommand{\AdvTBC}[2]{\Adv^{\mathrm{t\dash sprp}}_{#1}(#2)}

\newcommand{\AdvMEANBIND}[1]{\Adv^{\mathrm{mbind}}_{#1}}
\newcommand{\Coins}{\mathsf{Coins}}
\newcommand{\inv}{^{-1}}
 \newcommand{\AdvPOW}[2]{\Adv^{\mathrm{pow}}_{#1}(#2)}
 \newcommand{\ExpPOW}[2]{\mathbf{Exp}^{\mathrm{pow}}_{#1}(#2)}
 \newcommand{\ExpPOWW}[3]{\mathbf{Exp}^{\mathrm{pow\mbox{-}#2}}_{#1}(#3)}

 \newcommand{\AdvPRIV}[2]{\Adv^{\mathrm{priv}}_{#1}(#2)}
 \newcommand{\ExpPRIV}[2]{\mathbf{Exp}^{\mathrm{priv}}_{#1}(#2)}
 \newcommand{\ExpPRIVV}[3]{\mathbf{Exp}^{\mathrm{priv\mbox{-}#2}}_{#1}(#3)}

 \newcommand{\XCSS}{\textnormal{X-CSS}\xspace}
 \newcommand{\XSSS}{\textnormal{X-SSS}\xspace}
 \newcommand{\CSS}{\textnormal{CSS}\xspace}
 \newcommand{\FCSS}{\textnormal{A-CSS}\xspace}
 \newcommand{\BCSS}{\textnormal{B-CSS}\xspace}
 \newcommand{\BBCSS}{\textnormal{BB-CSS}\xspace}
 \newcommand{\BBCSSbf}{\textbf{BBCSS}\xspace}
 \newcommand{\CSSbf}{\textbf{CSS}\xspace}
 \newcommand{\AdvCSS}[2]{\Adv^{\mathrm{css}}_{#1}(#2)}
 \newcommand{\AdvBBCSS}[2]{\Adv^{\mathrm{bbcss}}_{#1}(#2)}
 \newcommand{\ExpCSS}[2]{\mathbf{Exp}^{\mathrm{css}}_{#1}(#2)}
 \newcommand{\ExpCSSS}[3]{\mathbf{Exp}^{\mathrm{css\mbox{-}#2}}_{#1}(#3)}

 \newcommand{\SSS}{\textnormal{SSS}\xspace}
 \newcommand{\SSSbf}{\textbf{SSS}\xspace}
 \newcommand{\FSSS}{\textnormal{A-SSS}\xspace}
 \newcommand{\BSSS}{\textnormal{B-SSS}\xspace}
 \newcommand{\BBSSS}{\textnormal{BB-SSS}\xspace}
 \newcommand{\BBSSSbf}{\textbf{BBSSS}\xspace}
 \newcommand{\AdvSSS}[2]{\Adv^{\mathrm{sss}}_{#1}(#2)}
 \newcommand{\ExpSSS}[2]{\mathbf{Exp}^{\mathrm{sss}}_{#1}(#2)}
 \newcommand{\ExpSSSS}[3]{\mathbf{Exp}^{\mathrm{sss\mbox{-}#2}}_{#1}(#3)}
 \newcommand{\prf}{F}
 \newcommand{\dash}{\mbox{-}}
 \newcommand{\IND}{\textnormal{IND}\xspace}
 \newcommand{\INDbf}{\textbf{IND}\xspace}
 \newcommand{\ExpSPRP}[2]{\mathbf{Exp}^{\mathrm{sprp}}_{#1}(#2)}
\newcommand{\AdvPRPprime}[2]{\Adv^{\mathrm{prp'}}_{#1}(#2)}
 \newcommand{\ExpPRPprime}[2]{\mathbf{Exp}^{\mathrm{prp'}}_{#1}(#2)}
 \newcommand{\AdvNPRG}[2]{\Adv^{\mathrm{prg}}_{#1}(#2)}
 %\newcommand{\AdvOTPRF}[2]{\Adv^{\mathrm{ot\dash prf}}_{#1}(#2)}
 \newcommand{\AdvMUPRF}[2]{\Adv^{\mathrm{mu\dash prf}}_{#1}(#2)}
 \newcommand{\ExpPRF}[2]{\mathbf{Exp}^{\mathrm{prf}}_{#1}(#2)}
 \newcommand{\AdvIND}[2]{\Adv^{\mathrm{ind}}_{#1}(#2)}
 \newcommand{\AdvINDCCA}[2]{\Adv^{\mathrm{ind\dash cca}}_{#1}(#2)}
 \newcommand{\ExpIND}[2]{\mathbf{Exp}^{\mathrm{ind}}_{#1}(#2)}
 \newcommand{\ExpINDD}[3]{\mathbf{Exp}^{\mathrm{ind\dash#2}}_{#1}(#3)}
 \newcommand{\ExpINDCCA}[2]{\mathbf{Exp}^{\mathrm{ind\dash cca}}_{#1}(#2)}
 \newcommand{\ExpINDCCAD}[3]{\mathbf{Exp}^{\mathrm{ind\dash cca\dash#2}}_{#1}(#3)}

 \newcommand{\AdvKEMCCA}[2]{\Adv^{\mathrm{kem\dash cca}}_{#1}(#2)}
 \newcommand{\AdvKEMCPA}[2]{\Adv^{\mathrm{kem}}_{#1}(#2)}
 \newcommand{\ExpKEMCCA}[2]{\mathbf{Exp}^{\mathrm{kem\dash cca}}_{#1}(#2)}
 \newcommand{\ExpKEMCPA}[2]{\mathbf{Exp}^{\mathrm{kem}}_{#1}(#2)}

 \newcommand{\AdvLOR}[2]{\Adv^{\mathrm{lor}}_{#1}(#2)}
 \newcommand{\AdvLORCCA}[2]{\Adv^{\mathrm{lor\dash cca}}_{#1}(#2)}
 \newcommand{\ExpLOR}[2]{\mathbf{Exp}^{\mathrm{lor}}_{#1}(#2)}
 \newcommand{\ExpLORCCA}[2]{\mathbf{Exp}^{\mathrm{lor\dash cca}}_{#1}(#2)}
 \newcommand{\Enc}{\procfont{Enc}}

 \newcommand{\Ectxtspace}{\calC}
 \newcommand{\coinspace}{\calR}
 \newcommand{\ctxtspacebar}{\overline{\calC}}
 \newcommand{\Frank}{\calF}
 \newcommand{\metadata}{\text{MD}}
 \newcommand{\Dec}{\procfont{Dec}}
 \newcommand{\Tag}{\procfont{Tag}}
 \newcommand{\Report}{\procfont{Report}}
 \newcommand{\Chk}{\procfont{Check}}
 \newcommand{\lrproc}{\procfont{Enc}}
 \newcommand{\decproc}{\procfont{Dec}}
 \newcommand{\conceal}{\textsf{conceal}}
 \newcommand{\conopen}{\textsf{open}}
 \newcommand{\CON}{\mathcal{C}}
 \newcommand{\hider}{h}
 \newcommand{\binder}{b}
% \newcommand{\CKgtrans}{\CKg_{\EC, \AEAD}}
% \newcommand{\CEnctrans}{\CEnc_{\EC, \AEAD}}
%  \newcommand{\CDectrans}{\CDec_{\EC, \AEAD}}
%  \newcommand{\CVertrans}{\CVer_{\EC, \AEAD}}
\newcommand{\CKgtrans}{\CKg}
\newcommand{\CEnctrans}{\CEnc}
\newcommand{\CDectrans}{\CDec}
\newcommand{\CEnccmpr}{\CEnc}
\newcommand{\CDeccmpr}{\CDec}
\newcommand{\CEnctbc}{\CEnc}
\newcommand{\CDectbc}{\CDec}
\newcommand{\CVertrans}{\CVer}
\newcommand{\afenc}{\text{FAF\dash Enc}}
\newcommand{\afdec}{\text{FAF\dash Dec}}
\newcommand{\afver}{\text{FAF\dash Ver}}
\newcommand{\GetID}{\text{GetID}}
\newcommand{\AdvWPRF}[2]{\Adv^{\mathrm{wPRF}}_{#1}(#2)}
\newcommand{\AdvINDrCPA}[2]{\Adv^{\mathrm{ind\$\dash cpa}}_{#1}(#2)}
\newcommand{\ExpINDrCPA}[2]{\mathbf{Exp}^{\mathrm{ind\$\dash cpa}}_{#1}(#2)}
\newcommand{\AdvLHINDrCPA}[2]{\Adv^{\mathrm{lh\dash ind\$\dash cpa}}_{#1}(#2)}
\newcommand{\ExpLHINDrCPA}[2]{\mathbf{Exp}^{\mathrm{lh\dash ind\$\dash cpa}}_{#1}(#2)}

\newcommand{\AdvINDCPA}[2]{\Adv^{\mathrm{ind\dash cpa}}_{#1}(#2)}
\newcommand{\ExpINDCPA}[2]{\mathbf{Exp}^{\mathrm{ind\dash cpa}}_{#1}(#2)}
\newcommand{\AdvLHINDCPA}[2]{\Adv^{\mathrm{lh\dash ind\dash cpa}}_{#1}(#2)}
\newcommand{\ExpLHINDCPA}[2]{\mathbf{Exp}^{\mathrm{lh\dash ind\dash cpa}}_{#1}(#2)}

\newcommand{\AdvAE}[2]{\Adv^{\mathrm{ror}}_{#1}(#2)}
\newcommand{\AdvMOROR}[2]{\Adv^{\mathrm{mo\dash ror}}_{#1}(#2)}
\newcommand{\AdvNMOROR}[2]{\Adv^{\mathrm{mo\dash nror}}_{#1}(#2)}
\newcommand{\AdvMORORctxt}[2]{\Adv^{\mathrm{mo\dash ror\dash ctxt}}_{#1}(#2)}
\newcommand{\AdvNMORORctxt}[2]{\Adv^{\mathrm{mo\dash nror\dash ctxt}}_{#1}(#2)}
\newcommand{\AdvSOROR}[2]{\Adv^{\mathrm{ror}}_{#1}(#2)}
\newcommand{\AdvSORORctxt}[2]{\Adv^{\mathrm{ror\dash ctxt}}_{#1}(#2)}
\newcommand{\AdvMOCTXT}[2]{\Adv^{\mathrm{mo\dash ctxt}}_{#1}(#2)}
\newcommand{\AdvNMOCTXT}[2]{\Adv^{\mathrm{mo\dash nctxt}}_{#1}(#2)}
\newcommand{\AdvSOCTXT}[2]{\Adv^{\mathrm{ctxt}}_{#1}(#2)}
\newcommand{\AdvCSCTXT}[2]{\Adv^{\mathrm{cs\dash nm}}_{#1}(#2)}
\newcommand{\AdvCSROR}[2]{\Adv^{\mathrm{cs\dash ror}}_{#1}(#2)}
\newcommand{\AdvAElor}[2]{\Adv^{\mathrm{ae\dash lor}}_{#1}(#2)}
\newcommand{\AdvLHAE}[2]{\Adv^{\mathrm{lh\dash ae}}_{#1}(#2)}
\newcommand{\AdvLHAElor}[2]{\Adv^{\mathrm{lh \dash ae\dash lor}}_{#1}(#2)}

\newcommand{\OTRORReal}{\text{otROR0}}
\newcommand{\OTRORRand}{\text{otROR1}}
\newcommand{\OTCTXT}{\text{SCU}}
\newcommand{\OTROR}{\text{otROR}}
\newcommand{\lens}{\textsf{lens}}
\newcommand{\eccomlen}{\textsf{btlen}}

\newcommand{\LRKAPRF}{\text{RKA-PRF}}
\newcommand{\LRKAPRFReal}{\text{RKA-PRF0}}
\newcommand{\LRKAPRFIdeal}{\text{RKA-PRF1}}
\newcommand{\AdvLRKAPRF}[2]{\Adv^{\mathrm{\oplus \dash prf}}_{#1}(#2)}

\newcommand{\LRKAPRP}{\text{RKA-PRP}}
\newcommand{\LRKAPRPReal}{\text{RKA-PRP0}}
\newcommand{\LRKAPRPIdeal}{\text{RKA-PRP1}}
\newcommand{\AdvLRKAPRP}[2]{\Adv^{\mathrm{\oplus \dash prp}}_{#1}(#2)}

\newcommand{\AEAD}{\textnormal{\textsf{AE}}}
%% \newcommand{\AEADone}{\textnormal{LHAE1}}
%% \newcommand{\AEADzero}{\textnormal{LHAE0}}
%% \newcommand{\AEADstate}{\textnormal{sLHAE}}
%% \newcommand{\AEADstateOne}{\textnormal{sLHAE1}}
%% \newcommand{\AEADstateZero}{\textnormal{sLHAE0}}
%% \newcommand{\AdvLHSTAE}[2]{\Adv^{\mathrm{lh\dash st\dash ae}}_{#1}(#2)}


\newcommand{\ExpAE}[2]{\mathbf{Exp}^{\mathrm{ae}}_{#1}(#2)}
\newcommand{\AdvCRD}[2]{\Adv^{\mathrm{crd}}_{#1}(#2)}
\newcommand{\ExpCRD}[2]{\mathbf{Exp}^{\mathrm{crd}}_{#1}(#2)}
\newcommand{\GameCRD}{\textnormal{CRD}}
\newcommand{\CRD}{\textnormal{CRD}}
\newcommand{\AdvRD}[2]{\Adv^{\mathrm{rp}}_{#1}(#2)}
\newcommand{\AdvRDIND}[2]{\Adv^{\mathrm{rp\dash ind}}_{#1}(#2)}
\newcommand{\ExpRD}[2]{\mathbf{Exp}^{\mathrm{rp}}_{#1}(#2)}
\newcommand{\ExpRDreal}[2]{\mathbf{Exp}^{\mathrm{rp\dash 1}}_{#1}(#2)}
\newcommand{\ExpRDideal}[2]{\mathbf{Exp}^{\mathrm{rp\dash 0}}_{#1}(#2)}
\newcommand{\RP}{\textnormal{RD}}


 \newcommand{\AdvKI}[2]{\Adv^{\mathrm{ki}}_{#1}(#2)}
\newcommand{\AdvINTPTXT}[2]{\Adv^{\mathrm{int\dash ptxt}}_{#1}(#2)}
\newcommand{\AdvPTXT}[2]{\Adv^{\mathrm{ptxt}}_{#1}(#2)}
\newcommand{\ExpINTPTXT}[2]{\mathbf{Exp}^{\mathrm{int\dash ptxt}}_{#1}(#2)}
\newcommand{\AdvINTCTXT}[2]{\Adv^{\mathrm{int\dash ctxt}}_{#1}(#2)}
\newcommand{\AdvCTXT}[2]{\Adv^{\mathrm{ctxt}}_{#1}(#2)}
\newcommand{\ExpINTCTXT}[2]{\mathbf{Exp}^{\mathrm{int\dash ctxt}}_{#1}(#2)}

\newcommand{\CTXT}{\textnormal{CTXT}}
\newcommand{\CTXTce}{\textnormal{CTXT}}
\newcommand{\PTXT}{\textnormal{PTXT}}

\newcommand{\UFCMA}{\textnormal{UF-CMA}}
\newcommand{\MUUFCMA}{\textnormal{MU-UF-CMA}}
\newcommand{\SUFCMA}{\textnormal{SUF-CMA}}

 \newcommand{\AdvMUUFCMA}[2]{\Adv^{\mathrm{mu\dash uf\dash cma}}_{#1}(#2)}
 \newcommand{\AdvUFCMA}[2]{\Adv^{\mathrm{uf\dash cma}}_{#1}(#2)}
 \newcommand{\AdvSUFCMA}[2]{\Adv^{\mathrm{suf\dash cma}}_{#1}(#2)}
 \newcommand{\AdvPRO}[2]{\Adv^{\mathrm{pro}}_{#1}(#2)}
 \newcommand{\AdvPROG}[2]{\Adv^{\mathrm{prog}}_{#1}(#2)}
 \newcommand{\AdvPUBPRO}[2]{\Adv^{\mathrm{pub\dash pro}}_{#1}(#2)}
 \newcommand{\AdvPUBPICF}[2]{\Adv^{\mathrm{pub\dash gpro}}_{#1}(#2)}
 \newcommand{\AdvWPA}[2]{\Adv^{\mathrm{wpra}}_{#1}(#2)}
 \newcommand{\AdvWPAone}[2]{\Adv^{\mathrm{wpra\dash 1}}_{#1}(#2)}
 \newcommand{\AdvINV}[2]{\Adv^{\mathrm{inv}}_{#1}(#2)}
 \newcommand{\AdvPA}[2]{\Adv^{\mathrm{pra}}_{#1}(#2)}
 \newcommand{\AdvVPA}[2]{\Adv^{\mathrm{v\dash pra}}_{#1}(#2)}
 \newcommand{\AdvOnePA}[2]{\Adv^{\mathrm{1\dash pra}}_{#1}(#2)}
 \newcommand{\AdvPAone}[2]{\Adv^{\mathrm{pa\dash 1}}_{#1}(#2)}
 \newcommand{\AdvEXT}[2]{\Adv^{\mathrm{ext}}_{#1}(#2)}
 \newcommand{\AdvEXTone}[2]{\Adv^{\mathrm{pra}}_{#1}(#2)}
 \newcommand{\AdvEXTtwo}[2]{\Adv^{\mathrm{ext}}_{#1}(#2)}
 \newcommand{\AdvCR}[2]{\Adv^{\mathrm{cr}}_{#1}(#2)}
 \newcommand{\AdvePRE}[2]{\Adv^{\mathrm{ePre}}_{#1}(#2)}
 \newcommand{\AdvrOWF}[2]{\Adv^{\mathrm{r\dash owf}}_{#1}(#2)}


\newcommand{\query}[1]{\procfont{query} {#1}:}
\newcommand{\queryl}[1]{\underline{\procfont{query} {#1}:}}
\newcommand{\oracle}[1]{\underline{\procfont{oracle} {#1}:}}
\newcommand{\oraclev}[1]{\underline{\procfont{oracle} {#1}:}\smallskip}
\newcommand{\procedure}[1]{\underline{\procfont{procedure} {#1}:}}
%\newcommand{\procedurev}[1]{\underline{\procfont{procedure} {#1}:}\smallskip}
\newcommand{\procedurev}[1]{\underline{{#1}:}\smallskip}
\newcommand{\subroutine}[1]{\underline{\procfont{subroutine} {#1}:}}
\newcommand{\subroutinev}[1]{\underline{\procfont{subroutine} {#1}:}\smallskip}
\newcommand{\subroutinenl}[1]{{\procfont{subroutine} {#1}:}}
\newcommand{\subroutinenlv}[1]{{\procfont{subroutine} {#1}:}\smallskip}
\newcommand{\adversary}[1]{\underline{\procfont{adversary} {#1}:}}
\newcommand{\adversaryv}[1]{\underline{\procfont{adversary} {#1}:}\smallskip}
\newcommand{\experiment}[1]{\underline{{#1}}}
\newcommand{\experimentv}[1]{\underline{{#1}}\smallskip}

\newcommand{\algorithm}[1]{\underline{\procfont{algorithm} {#1}:}}
\newcommand{\algorithmv}[1]{\underline{\procfont{algorithm} {#1}:}\smallskip}
%\newcommand{\experiment}[1]{\underline{\procfont{Experiment} {#1}}}
%\newcommand{\experimentv}[1]{\underline{\procfont{Experiment} {#1}}\smallskip}

\newcommand{\AdvMU}[3]{\Adv^{\mathrm{n \mbox{-}mu\mbox{-}#3}}_{#1}(#2)}
%\newcommand{\AdvSU}[2]{\Adv^{\mathrm{su\mbox{-}cpa}}_{#1}(#2)}
\newcommand{\AdvSU}[3]{\Adv^{\mathrm{ ind\mbox{-}#3}}_{#1}(#2)}
\newcommand{\AdvSUnok}[2]{\Adv^{\mathrm{ ind\mbox{-}#2}}_{#1}}
\newcommand{\AdvForge}[2]{\Adv^{\mathrm{uf\mbox{-}cma}}_{#1}(#2)}
\newcommand{\AdvPred}[2]{\Adv^{\mathrm{pred\mbox{-}ct}}_{#1}(#2)}
\newcommand{\AdvInvert}[2]{\Adv^{\mathrm{pt\mbox{-}pred}}_{#1}(#2)}
\newcommand{\AdvCompPred}[2]{\Adv^{\mathrm{pred\mbox{-}pt}}_{#1}(#2)}
\newcommand{\AdvFuncPred}[2]{\Adv^{\mathrm{func\mbox{-}pred}}_{#1}(#2)}
\newcommand{\AdvFunc}[2]{\Adv^{\mathrm{ss}\mbox{-}\mathrm{det}}_{#1}(#2)}
\newcommand{\AdvSS}[2]{\Adv^{\mathrm{ss}}_{#1}(#2)}
\newcommand{\AdvFuncD}[2]{\Adv^{\mathrm{det}}_{#1}(#2)}
\newcommand{\AdvFuncSnok}[2]{\Adv^{\mathrm{priv\mathrm{\mbox{-}#2}}}_{#1}}
\newcommand{\AdvFuncS}[3]{\Adv^{\mathrm{priv\mathrm{\mbox{-}#3}}}_{#1}(#2)}
\newcommand{\AdvFuncSS}[2]{\Adv^{\mathrm{priv\mathrm{\mbox{-}#2}}}_{#1}}
\newcommand{\AdvFuncH}[2]{\Adv^{\mathrm{hyb\mbox{-}det}}_{#1}(#2)}
\newcommand{\AdvFuncROR}[2]{\Adv^{\mathrm{ror\mbox{-}det}}_{#1}(#2)}
\newcommand{\AdvPlainFunc}[2]{\Adv^{\mathrm{func\mbox{-}nil}}_{#1}(#2)}
%\newcommand{\AdvMUcca}[2]{\Adv^{\mathrm{n \mbox{-}mu\mbox{-}cca}}_{#1}(#2)}
%\newcommand{\AdvSUcca}[2]{\Adv^{\mathrm{su\mbox{-}cca}}_{#1}(#2)}
\newcommand{\AdvOWF}[2]{\Adv^{\mathrm{owf}}_{{#1}}(#2)}
\newcommand{\AdvOWFnok}[1]{\Adv^{\mathrm{owf}}_{{#1}}}
\newcommand{\ExpOWF}[2]{\mathbf{Exp}^{\mathrm{owf}}_{#1}(#2)}
\newcommand{\AdvPOWF}[2]{\Adv^{\mathrm{powf}}_{{#1}}(#2)}
\newcommand{\AdvDDH}[2]{\Adv^{\mathrm{ddh}}_{#1}(#2)}
\newcommand{\AdvH}[2]{\Adv^{cr}_{\cal H, #1}(#2)}
\newcommand{\Hcollexp}[2]{\mathbf{Exp}^{\mathrm{cr}}_{#1}(#2)}
\newcommand{\Hcolladv}[2]{\Adv^{\mathrm{cr}}_{#1}(#2)}

\newcommand{\EG}{\text{EG}}


\newcommand{\st}{st}
\newcommand{\seqnum}{c}

\newcommand{\init}{{\sf Init}}
\newcommand{\LR}{\mathrm{LR}}
\newcommand{\SEscheme}{\textnormal{\textsf{SE}}}
\newcommand{\SEschemebar}{\overline{\textnormal{\textsf{SE}}}}
\newcommand{\MACscheme}{\mbox{\textsf{MAC}}}
\newcommand{\enc}{{\sf enc}}
\newcommand{\encr}{{\enc}^{\$}}
\newcommand{\dec}{{\sf dec}}
\newcommand{\encsign}{{\cal ES}}
\newcommand{\sign}{{\sf S}}
\newcommand{\mac}{{\sf Mac}}
\newcommand{\verdecc}{{\cal VD}}
\newcommand{\henc}{{\cal HE}}
\newcommand{\hkg}{{\cal HK}}
\newcommand{\hf}{{\cal HF}}
\newcommand{\h}{{\cal H}}
\newcommand{\detenc}{{\cal DE}}
\newcommand{\denc}{{\cal DE}}
\newcommand{\senc}{{\cal SE}}
\newcommand{\sdecc}{{\cal SD}}
\newcommand{\decc}{{\cal D}}
\newcommand{\detdecc}{{\cal DD}}
\newcommand{\hdecc}{{\cal HD}}

\newcommand{\EN}{{\sf CODE}}
\newcommand{\encode}{{\sf encode}}
\newcommand{\decode}{{\sf decode}}
\newcommand{\AEnc}{\textsf{AE}}
\newcommand{\goodcode}{{\sf GENcode}}
\newcommand{\goodencode}{{\sf GENencode}}
\newcommand{\gooddecode}{{\sf GENdecode}}

\newcommand{\kg}{{\sf kg}}
\newcommand{\kgse}{\kg_{\mathrm{se}}}
\newcommand{\kgma}{\kg_{\mathrm{ma}}}
\newcommand{\kgc}{{\cal G}}


\newcommand{\kgSym}{\mathsf{kg_s}}
\newcommand{\encSym}{\mathsf{enc_s}}
\newcommand{\decSym}{\mathsf{dec_s}}

\newcommand{\msgsp}{\mathrm{MsgSp}}
\newcommand{\PKscheme}{\Pi} %{{\cal AE}} %replaced it according to other notation
\newcommand{\DSscheme}{{\cal DS}}
\newcommand{\hPKscheme}{{\cal HPE}}
\newcommand{\dPKscheme}{{\cal DPE}}
\newcommand{\sPKscheme}{{\cal SAE}}
\newcommand{\EGscheme}{{\cal EG}}
\newcommand{\CSscheme}{{\cal CS}}
\newcommand{\OAEPD}{\mathsf{DOAEP}}
\newcommand{\RSAOAEPD}{\mathsf{RSA\mbox{-}DOAEP}}
\newcommand{\cert}{\mathrm{cert}}
\newcommand{\pk}{\mathsl{pk}}
\newcommand{\pkbar}{\overline{\pk}}
\newcommand{\sk}{\mathsl{sk}}
\newcommand{\skbar}{\overline{\sk}}
\newcommand{\ke}{\mathsl{k_{e}}}
\newcommand{\kd}{\mathsl{k_{d}}}
\newcommand{\lr}{\mathrm{LR}}

\newcommand{\SE}{{\mathsf{SE}}}
\newcommand{\Skg}{{\mathsf{kg}}}
\newcommand{\Senc}{{\mathsf{enc}}}
\newcommand{\Sdec}{{\mathsf{dec}}}

\newcommand{\KEM}{\Psi}
\newcommand{\KEMkg}{{\cal KK}}
\newcommand{\KEMenc}{{\cal KE}}
\newcommand{\KEMdec}{{\cal KD}}

\def\next{\:;\:}
\newcommand{\concat}{\: \| \:}
\newcommand{\texp}{T_{\kgc}^{\mathrm{exp}}(k)}
\newcommand{\tgr}{T_{\kgc}^{\mathrm{gr-op}}(k)}
\newcommand{\tmem}{T_{\kgc}^{\mathrm{gr-mem}}(k)}
\newcommand{\tran}{T_{\kgc}^{\mathrm{rand}}(k)}
\newcommand{\qu}{q_{\mathrm{e}}}
\newcommand{\qd}{q_{\mathrm{d}}}
\newcommand{\qr}{q_{\mathrm{r}}}
\newcommand{\qh}{q_{\mathrm{h}}}
\newcommand{\qho}{q_{\mathrm{h_1}}}
\newcommand{\qht}{q_{\mathrm{h_2}}}
\newcommand{\qhi}{q_{\mathrm{h_i}}}
\newcommand{\lh}{l_{\mathrm{h}}}

\newcommand{\Bcpa}[1]{B_{\mathrm{cpa}}^{#1}}
\newcommand{\Bcca}[1]{B_{\mathrm{cca}}^{#1}}
\newcommand{\Dcca}[1]{D_{\mathrm{cca}}^{#1}}
\newcommand{\Acpa}[1]{A_{\mathrm{cpa}}^{#1}}
\newcommand{\Alg}[2]{A^{{#1}{#2}}}
\newcommand{\atk}{\mathrm{atk}}
\newcommand{\cpa}{\mathrm{cpa}}
\newcommand{\cca}{\mathrm{cca}}
\newcommand{\Aatk}[1]{A_{\mathrm{atk}}^{#1}}
\newcommand{\Batk}[1]{B_{\mathrm{atk}}^{#1}}
\newcommand{\Datk}[1]{D_{\mathrm{atk}}^{#1}}
\newcommand{\sasha}[1]{\textit{Sasha: #1}}
\newcommand{\GH}{\mathcal{GH}}
\newcommand{\EH}{\mathcal{EH}}
\newcommand{\GF}{\mathcal{GF}}
\newcommand{\gftwo}[1]{\text{GF}(2^{#1})}
\newcommand{\EF}{\mathcal{EF}}
\newcommand{\tmf}{\overline{\calG}}
\newcommand{\gh}{{\mathit{g}}}
\newcommand{\uh}{{\mathit{u}}}
\newcommand{\NR}{\mathsf{NR}}
\newcommand{\mc}{\text{mc}}
\newcommand{\Inv}{\sf{Inv}}
\newcommand{\rview}{\mathsf{View}}
\newcommand{\Acca}[1]{A^{#1}}

\newcommand{\adam}[1]{\textit{Adam: #1}}

\newcommand{\comment}[1]{\hspace{5pt}\textbf{[}\hspace{2pt}#1\textbf{]}}
% \ \ /$\!$/ {\small\textsl #1}}

\newcommand{\veca}{{{\bf a}}}
\newcommand{\vecb}{{{\bf b}}}
\newcommand{\vect}{{{\bf t}}}
\newcommand{\vecx}{{{\bf x}}}
\newcommand{\vecc}{{{\bf c}}}
\newcommand{\vecK}{{{\bf K}}}
\newcommand{\vecy}{{{\bf y}}}
\newcommand{\vecz}{{{\bf z}}}
\newcommand{\vech}{{{\bf h}}}
\newcommand{\vecomega}{\boldsymbol{\omega}}
\newcommand{\test}{t}
\newcommand{\oguess}{g}
\newcommand{\me}[2]{\mathrm{me}_{#1}(#2)}
\newcommand{\menok}[1]{\mathrm{me}_{#1}}
%\newcommand{\inst}[1]{{}$^{#1}$}
%\newcommand{\email}[1]{\texttt{#1}}

\newcommand{\dqed}{{\hspace{5pt}\mbox{$\square$}}}
\newcommand{\SD}[1]{{\textnormal{SD}\left(#1\right)}}
\newcommand{\thh}{^{\textit{th}}} % th
\newcommand{\emptystr}{\varepsilon}
\newcommand{\emptyalg}{\Lambda}
\newcommand{\dotdot}{{\,..\,}}

\newcommand{\nk}{n}
\newcommand{\nt}{n}
\newcommand{\nm}{v}
\newcommand{\ml}{w}
\newcommand{\skl}{s}
\newcommand{\indless}{\hspace*{1em}}

%---Marc
\newcommand{\AdvPRGo}[2]{\Adv^{\mathrm{prg}}_{{#1}}({#2})}
\newcommand{\AdvPRG}[3]{\Adv^{\mathrm{prg}, #3}_{{#1}}({#2})}
\newcommand{\ExpPRGo}[2]{\mathbf{Exp}^{\mathrm{prg}}_{{#1}}({#2})}
\newcommand{\ExpPRG}[3]{\mathbf{Exp}^{\mathrm{prg},#3}_{{#1}}({#2})}
\newcommand{\AdvPRGv}[3]{\Adv^{\mathrm{prg}\textnormal{-}{#1}}_{#2}(#3)}
\newcommand{\ExpPRGv}[3]{\mathbf{Exp}^{\mathrm{prg}\textnormal{-}{#1}}_{#2}(#3)}
\newcommand{\AdvPRE}[2]{\Adv^{\mathrm{pre}}_{#1}(#2)}
\newcommand{\ExpPRE}[2]{\mathbf{Exp}^{\mathrm{pre}}_{#1}(#2)}
\newcommand{\prg}{\ensuremath{\mathcal{PRG}}}
\newcommand{\prgkeygen}{\ensuremath{\mathcal{GK}}}
\newcommand{\prgeval}{\ensuremath{\mathcal{G}}}
\newcommand{\vecu}{\ensuremath{\mathbf{u}}}
\newcommand{\vecs}{\ensuremath{\mathbf{s}}}
\newcommand{\vecr}{\ensuremath{\mathbf{r}}}
\newcommand{\hb}{\ensuremath{\mathsl{hb}}}
\newcommand{\Angle}[1]{\ensuremath{\left\langle #1\right\rangle}}
\newcommand{\Anglefix}[1]{\ensuremath{\langle #1\rangle}}
\newcommand{\game}[1]{\text{Game #1}}
\newcommand{\recover}{\ensuremath{\textsc{Recover}}}
\newcommand{\marc}[1]{\textbf{mf:} \textit{#1}}
\newcommand{\iter}{n}
%----
\newcommand{\rtab}{\mathrm{R}}
\newcommand{\dtab}{\mathrm{D}}
\newcommand{\khm}{[K,H,M]}
\newcommand{\khc}{[K,H,C]}
\newcommand{\kh}{[K,H]}
\newcommand{\rng}{\mathrm{Rng}}
\newcommand{\dom}{\mathrm{Dom}}

\newcommand{\Dplus}{D^+}
\newcommand{\fiter}{\textsf{Itr}}   %{f_+}
\newcommand{\Itr}{\textsf{Itr}}   %{f_+}
\newcommand{\Itrbf}{\textbf{Itr}}   %{f_+}

\newcommand{\kset}{\mathtt{K}}
\newcommand{\pkdet}{\pk_{\text{d}}}
\newcommand{\skdet}{\sk_{\text{d}}}

\newcommand{\stretchval}{1.2}

\newcommand{\mpage}[2]{\begin{minipage}{#1\textwidth} #2 \end{minipage}}
\newcommand{\fpage}[2]{\framebox{\begin{minipage}{#1\textwidth}\setstretch{\stretchval}\gamesfontsize #2 \end{minipage}}}

\newcommand{\hfpages}[3]{\hfpagess{#1}{#1}{#2}{#3}}
\newcommand{\hfpagess}[4]{
		\begin{tabular}{c@{\hspace*{.5em}}c}
		\framebox{\begin{minipage}[t]{#1\textwidth}\setstretch{\stretchval}\gamesfontsize #3 \end{minipage}}
		&
		\framebox{\begin{minipage}[t]{#2\textwidth}\setstretch{\stretchval}\gamesfontsize #4 \end{minipage}}
		\end{tabular}
	}
\newcommand{\hfpagesss}[6]{
		\begin{tabular}{c@{\hspace*{.5em}}c@{\hspace*{.5em}}c}
		\framebox{\begin{minipage}[t]{#1\textwidth}\setstretch{\stretchval}\gamesfontsize #4 \end{minipage}}
		&
		\framebox{\begin{minipage}[t]{#2\textwidth}\setstretch{\stretchval}\gamesfontsize #5 \end{minipage}}
		&
		\framebox{\begin{minipage}[t]{#3\textwidth}\setstretch{\stretchval}\gamesfontsize #6 \end{minipage}}
		\end{tabular}
	}
\newcommand{\hfpagessss}[8]{
		\begin{tabular}{c@{\hspace*{.5em}}c@{\hspace*{.5em}}c@{\hspace*{.5em}}c}
		\framebox{\begin{minipage}[t]{#1\textwidth}\setstretch{\stretchval}\gamesfontsize #5 \end{minipage}}
		&
		\framebox{\begin{minipage}[t]{#2\textwidth}\setstretch{\stretchval}\gamesfontsize #6 \end{minipage}}
		&
		\framebox{\begin{minipage}[t]{#3\textwidth}\setstretch{\stretchval}\gamesfontsize #7 \end{minipage}}
		&
		\framebox{\begin{minipage}[t]{#4\textwidth}\setstretch{\stretchval}\gamesfontsize #8 \end{minipage}}
		\end{tabular}
	}


\def\codestretch{\stretchval}

\newcommand{\hpagesl}[3]{
	\begin{tabular}{c|c}
	  \begin{minipage}{#1\textwidth}\setstretch{\codestretch} #2 \end{minipage}
	  &
	  \begin{minipage}{#1\textwidth} #3 \end{minipage}
	\end{tabular}
	}

\newcommand{\hpagessl}[4]{
	\begin{tabular}{c|@{\hspace*{.5em}}c}
	   \begin{minipage}[t]{#1\textwidth}\setstretch{\codestretch} #3 \end{minipage}
	   &
	   \begin{minipage}[t]{#2\textwidth}\setstretch{\codestretch} #4 \end{minipage}
	\end{tabular}
	}

\newcommand{\hpages}[3]{
	\begin{tabular}{cc}
	   \begin{minipage}[t]{#1\textwidth}\setstretch{\codestretch} #2 \end{minipage}
	   &
	   \begin{minipage}[t]{#1\textwidth}\setstretch{\codestretch} #3 \end{minipage}
	\end{tabular}
	}

\newcommand{\hpagess}[4]{
    \begin{tabular}{c@{\hspace*{1.5em}}c}
	   \begin{minipage}[t]{#1\textwidth}\setstretch{\codestretch} #3 \end{minipage}
	   &
	   \begin{minipage}[t]{#2\textwidth}\setstretch{\codestretch} #4 \end{minipage}
    \end{tabular}
	}

\newcommand{\hpagesss}[6]{
	\begin{tabular}{ccc}
	\begin{minipage}[t]{#1\textwidth}\setstretch{\codestretch}\gamesfontsize #4 \end{minipage} &
	\begin{minipage}[t]{#2\textwidth}\setstretch{\codestretch}\gamesfontsize #5 \end{minipage} &
	\begin{minipage}[t]{#3\textwidth}\setstretch{\codestretch}\gamesfontsize #6 \end{minipage}
	\end{tabular}}
\newcommand{\hpagesssl}[6]{
	\begin{tabular}{c|c|c}
	\begin{minipage}[t]{#1\textwidth}\setstretch{\codestretch} #4 \end{minipage} &
	\begin{minipage}[t]{#2\textwidth}\setstretch{\codestretch} #5 \end{minipage} &
	\begin{minipage}[t]{#3\textwidth}\setstretch{\codestretch} #6 \end{minipage}
	\end{tabular}}
\newcommand{\hpagessss}[8]{
	\begin{tabular}{cccc}
	\begin{minipage}[t]{#1\textwidth}\setstretch{\codestretch} #5 \end{minipage} &
	\begin{minipage}[t]{#2\textwidth}\setstretch{\codestretch} #6 \end{minipage} &
	\begin{minipage}[t]{#3\textwidth}\setstretch{\codestretch} #7 \end{minipage}
	\begin{minipage}[t]{#4\textwidth}\setstretch{\codestretch} #8 \end{minipage}
	\end{tabular}}



\newcommand{\authnote}[2]{\ifnum\authnotes=1\begin{quote}\textbf{#1 says:} #2\end{quote}\fi}
\newcommand{\authfnote}[2]{\ifnum\authnotes=1\footnote{\textbf{#1 says:} #2}\fi}
\newcommand{\myemph}[1]{\textsl{\textbf{#1}}}
\renewcommand{\paragraph}[1]{\vspace{.6em}\noindent\textbf{#1}\hspace*{.5em}}

\newcounter{mynote}[section]
\newcommand{\notecolor}{blue}
\newcommand{\scribenotecolor}{purple}
\newcommand{\proofreadernotecolor}{green}

\newcommand{\thenote}{\thesection.\arabic{mynote}}
\newcommand{\tnote}[1]{\ifnum\authnotes=1\refstepcounter{mynote}{\bf \textcolor{\notecolor}{$\ll$TCR~\thenote: {\sf #1}$\gg$}}\fi}
\newcommand{\scribenote}[1]{\ifnum\authnotes=1\refstepcounter{mynote}{\bf \textcolor{\scribenotecolor}{$\ll$Scribe~\thenote: {\sf #1}$\gg$}}\fi}
\newcommand{\pfreadernote}[1]{\ifnum\authnotes=1\refstepcounter{mynote}{\bf \textcolor{\proofreadernotecolor}{$\ll$Proofreader~\thenote: {\sf #1}$\gg$}}\fi}
\newcommand{\fixme}[1]{\ifnum\authnotes=1\textbf{\textcolor{red}{[FIXME: #1]}}\fi}



\newcommand{\semi}{\:;\:}
\newcommand{\bitset}{B}
\newcommand{\eq}{\mathrm{eq}}
\newcommand{\TrapPerm}{\mathcal{TP}}
\newcommand{\TPgen}{G}
\newcommand{\TPf}{F}
\newcommand{\TPinv}{\overline{F}}
\newcommand{\Gen}{\mathcal{G}}

\newcommand{\Pibar}{{\overline{\Pi}}}
\newcommand{\kgbar}{{\overline{\kg}}}
\newcommand{\encbar}{{\overline{\enc\raisebox{2.5mm}{}}}}
\newcommand{\decbar}{{\overline{\dec}}}
%\newcommand{\kgbar}{{\overline{\calK}}}
%\newcommand{\encbar}{{\overline{\calE}}}
%\newcommand{\decbar}{{\overline{\calD}}}

\newcommand{\DEone}{\textsf{DE1}}
\newcommand{\DEtwo}{\textsf{DE2}}

\newcommand{\Znstar}{\Z_n^*}

\newcommand{\caseeqndef}[4]{
    \left\{
	\begin{array}{ll}
	    #1 & \textnormal{#2}	\\
        #3 & \textnormal{#4}
	\end{array}\right.}

% ========================================================================

\newcommand{\pubsim}{\calS}
\newcommand{\blenn}{\textsf{blen}}
\newcommand{\rhotable}{\mathtt{P}}
\newcommand{\TabIdx}{\mathtt{Idx}}
\newcommand{\TabB}{\mathtt{B}}
%\newcommand{\TabC}{\mathtt{C}}
\newcommand{\TabC}{\mathtt{ctr}}
\newcommand{\Tabc}{\mathtt{c}}
\newcommand{\TabD}{\mathtt{D}}
\newcommand{\TabE}{\mathtt{E}}
\newcommand{\TabH}{\mathtt{H}}
\newcommand{\TabYtoX}{\mathtt{YtoX}}
\newcommand{\TabX}{\mathtt{X}}
\newcommand{\TabY}{\mathtt{Y}}
\newcommand{\TabP}{\mathtt{P}}
\newcommand{\TabK}{\mathtt{K}}
\newcommand{\TabT}{\mathtt{T}}
\newcommand{\TabR}{\mathtt{R}}
\newcommand{\TabF}{\mathtt{F}}
\newcommand{\Tabf}{\mathtt{f}}
\newcommand{\TabFx}{\mathtt{F_x}}
\newcommand{\TabFy}{\mathtt{F_y}}
\newcommand{\TabV}{\mathtt{V}}
\newcommand{\TabZ}{\mathtt{T}}
\newcommand{\TabQ}{\mathtt{Q}}
\newcommand{\TabValue}{\mathtt{V}}
\newcommand{\TabIV}{\mathtt{IV}}

\newcommand{\xstar}{x^*}
\newcommand{\cstar}{c^*}
\newcommand{\mstar}{m^*}
\newcommand{\vstar}{v^*}

\newcommand{\satisfies}{\textsf{ satisfies}}

\newcommand{\advice}{\alpha}
\newcommand{\advicem}{\alpha_\textrm{m}}
\newcommand{\adviceg}{\alpha_\textrm{g}}
\newcommand{\wext}{\calE^+}
%\newcommand{\ext}{\calE}
\newcommand{\extenh}{\ext^+}
\newcommand{\extA}{\calE_A}
\newcommand{\extB}{\calE_B}
\newcommand{\extC}{\calE_C}
\newcommand{\Signoracle}{\mathsf{Sign}}
\newcommand{\environ}{{Env}}
\newcommand{\Poracle}{P}
\newcommand{\Pobject}{P}
\newcommand{\Horacle}{\mathsf{H}}
\newcommand{\Ooracle}{\mathsf{O}}
\newcommand{\pOracle}{\textsf{P}}
\newcommand{\chalOracle}{\mathbb{D}}
\newcommand{\encOracle}{\mathbb{E}}
\newcommand{\decOracle}{\mathbb{D}}
\newcommand{\extOracle}{\textsf{Ex}}
\newcommand{\extOracleA}{\textsf{SimEx}}
\newcommand{\extOracleB}{{\extOracle}}
\newcommand{\extkea}{\calE_{\textrm{kea}}}
\newcommand{\Akea}{A_{\textrm{kea}}}
\newcommand{\reconstruct}{\textnormal{Rec}}
\newcommand{\recons}{\textnormal{Rec}}
\newcommand{\Ext}{\textnormal{Ext}}
\newcommand{\pra}{\textnormal{pra}}
\newcommand{\wpra}{\textnormal{wpra}}
\newcommand{\extone}{\textnormal{pra}}
\newcommand{\exttwo}{\textnormal{ext}}
\newcommand{\extthree}{\textnormal{ext3}}
\newcommand{\ExtOne}[1]{\Exp^{\extone}_{#1}}
%\newcommand{\eExtOne}[1]{\Exp^{e\textnormal{-ext}}_{#1}}
%\newcommand{\ExtOneStar}[1]{\Exp^{\textnormal{ext*}}_{#1}}
\newcommand{\WPAExp}[1]{\Exp^{\textnormal{wpra}}_{#1}}
\newcommand{\WPAExpone}[1]{\Exp^{\textnormal{wpra}\dash1}_{#1}}
\newcommand{\InvExp}[1]{\Exp^{\textnormal{inv}}_{#1}}
\newcommand{\PAExp}[1]{\Exp^{\extone}_{#1}}
\newcommand{\PAExpone}[1]{\Exp^{\extone\dash1}_{#1}}
\newcommand{\CRExp}[1]{\Exp^{\mathrm{cr}}_{#1}}
\newcommand{\ExtExp}[2]{\Exp^{\exttwo\dash#1}_{#2}}
\newcommand{\ExtTwo}[2]{\Exp^{\exttwo\dash#1}_{#2}}
\newcommand{\ExtThree}[1]{\Exp^{\extthree}_{#1}}
%\newcommand{\ExtLR}[2]{\Exp^{\textnormal{ext-ind-}#2}_{#1}}
\newcommand{\orac}{{(\cdot)}}
\newcommand{\oracs}{{(\cdots)}}
\newcommand{\exorac}{\ext}
\newcommand{\Rorac}{\$}
\newcommand{\XSet}{\calX}
\newcommand{\flag}{\textsf{flag}}
\newcommand{\csROR}{\textrm{ROR}}

\newcommand{\ComO}{\textbf{Com}}



\newcommand{\AdvOr}{\calA_\textrm{OR}}
\newcommand{\AdvCoins}{\calA_\textrm{\$\$}}
\newcommand{\coins}{\omega}
\newcommand{\Zp}{\mathbb{Z}_p}
\newcommand{\bbG}{\mathbb{G}}
\newcommand{\group}{\bbG}

\newcommand{\rOWF}{\textnormal{rOWF}}
\newcommand{\ePre}{\textnormal{ePre}}
\newcommand{\CR}{\textnormal{CR}}
\newcommand{\CRbf}{\textbf{CR}}
%\newcommand{\EXT}{\textnormal{EXTow}}
\newcommand{\EXTbf}{\textbf{EXT}}
\newcommand{\EXT}{\textnormal{EXT}}
\newcommand{\PROp}{\textnormal{PRO-Pr}}

\newcommand{\WPA}{\textnormal{WPrA}}
\newcommand{\WPAbf}{\textbf{WPrA}}

\newcommand{\vPA}{v\textnormal{-PrA}}
\newcommand{\onePA}{\textnormal{1-PrA}}
\newcommand{\onedashpa}{\mathrm{1\dash pra}}
\newcommand{\vdashpa}{\mathrm{v\dash pra}}
\newcommand{\onePAExp}[1]{\Exp^{\onedashpa}_{#1}}
\newcommand{\vPAExp}[1]{\Exp^{\vdashpa}_{#1}}

\newcommand{\PA}{\textnormal{PrA}}
\newcommand{\PAbf}{\textbf{PrA}}
\newcommand{\PAp}{\textnormal{PrA-Pr}}
\newcommand{\PApbf}{\textbf{PrA-Pr}}

\newcommand{\PAgame}{\textnormal{PrA}}

\newcommand{\EXTone}{\textnormal{EXT1}}
\newcommand{\EXTonebf}{\textbf{EXT1}}
%\newcommand{\EXTcbf}{\textbf{EXT1}}
\newcommand{\EXTlr}{\textnormal{EXT-IND}}
%\newcommand{\EXTp}{\textnormal{EXT-$\$$}}
\newcommand{\EXTtwo}{\textnormal{EXT2}}
\newcommand{\EXTtwobf}{\textbf{EXT2}}
%\newcommand{\eEXT}{e\textnormal{-EXT}}

\newcommand{\pro}{\textnormal{pro}}
\newcommand{\PRO}{\textnormal{PRO}}
\newcommand{\PRObf}{\textbf{PRO}}
\newcommand{\PUBRO}{\textnormal{pub-RO}}
\newcommand{\PUBROsc}{\textsc{pub-RO}}
\newcommand{\pubpro}{\textnormal{pub-pro}}
\newcommand{\PUBPRO}{\textnormal{pub-PRO}}
\newcommand{\PUBPROsc}{\textsc{pub-PRO}}
\newcommand{\PUBPRObf}{\textbf{pub-PRO}}

\newcommand{\PUBICF}{\textnormal{pub-GRO}}
\newcommand{\PUBPICF}{\textnormal{pub-GPRO}}
\newcommand{\PUBPICFbf}{\textbf{pub-GPRO}}
\newcommand{\pubgpro}{\textnormal{pub-gpro}}
\newcommand{\icf}{f}

\newcommand{\PROG}{\textnormal{PROG}}
\newcommand{\PROGbf}{\textbf{PROG}}


\newcommand{\SimF}{\textnormal{Sim-$F$}}
\newcommand{\Choosef}{\textnormal{Choose-$f$}}
\newcommand{\ChooseE}{\textnormal{Choose-$E$}}
\newcommand{\ChooseD}{\textnormal{Choose-$D$}}
\newcommand{\RngSet}{\mathcal{R}}
\newcommand{\DomSet}{\mathcal{D}}

\newcommand{\eval}{\mathit{eval}}
\newcommand{\geval}{\mathit{geval}}
\newcommand{\reveal}{\mathit{reveal}}

\newcommand{\cheatflag}{\mathsf{flag}}
\newcommand{\notcheatflag}{\overline{\mathsf{flag}}}

\newcommand{\IV}{IV}
\newcommand{\sfpad}{\textsf{sfpad}}
\newcommand{\pfpad}{\textsf{pfpad}}
\newcommand{\strippad}{\textsf{unpad}}
\newcommand{\SMD}{\textnormal{SMD}}
\newcommand{\SMDb}{\textbf{SMD}}
\newcommand{\EXTP}{\textnormal{EXT-Pr}}
\newcommand{\EXTPbf}{\textbf{EXT-Pr}}
\newcommand{\EXTPone}{\textnormal{EXTow-Pr}}
\newcommand{\EXTPonebf}{\textbf{EXTow-Pr}}
\newcommand{\EXTPtwo}{\textnormal{EXT1-Pr}}
\newcommand{\EXTPtwobf}{\textbf{EXT1-Pr}}



\newcommand{\AdvLeg}{\calA}
\newcommand{\AdvSingle}{\calA_1}
\newcommand{\ADmodel}{\textrm{AD}}
\newcommand{\ORmodel}{\textrm{OR}}
\newcommand{\Cmodel}{\textrm{\$\$}}
\newcommand{\AdvAdvice}{\calA_{\ADmodel}}
\newcommand{\ExtSingle}{\calE_{\textrm{SI}}}

\newcommand{\simu}{\calS}

\newcommand{\PreIm}{\textsf{PreIm}}

\newcommand{\IC}[1]{\mathsf{IC}_{#1}}

\newcommand{\RF}[1]{\mathsf{RF}_{#1}}
\newcommand{\finv}{f^{-1}}
\newcommand{\TDPgen}{\calF}

\newcommand{\RSAkem}{\mathsf{KEM}}

\newcommand{\intspy}{\mathcal{I}}

\newcommand{\RO}{\textnormal{RO}}

\newcommand{\FDH}{\textsf{FDH}}
\newcommand{\Kg}{\textsf{Kg}}
\newcommand{\Sign}{\textsf{Sign}}
\newcommand{\Ver}{\textsf{Ver}}

\newcommand{\dropin}[1]{\mathsf{#1}}
\newcommand{\Func}{\mathrm{Func}}
\newcommand{\Perm}{\mathrm{Perm}}

\newcommand{\Ddom}{D^+}
\newcommand{\state}{st}

\newcommand{\extfails}{\mathsf{\ext\ fails}}
\newcommand{\oracleP}{P}

\newcommand{\Coll}{\mathsf{Coll}}
\newcommand{\Einv}{D}
\newcommand{\cbar}{\overline{c}}
\newcommand{\mbar}{\overline{m}}
\newcommand{\ybar}{\overline{y}}

\newcommand{\cnext}{w}

\newcommand{\calSA}{{\cal S}_A}
\newcommand{\calSB}{{\cal S}_B}

\newcommand{\calOA}{\calO'}
\newcommand{\MMO}{\textsf{MMO}}
\newcommand{\MMOguarded}{\overline{\textsf{MMO}}}
\newcommand{\DM}{\textsf{DM}}
\newcommand{\DMrest}{\overline{\textsf{DM}}}
\newcommand{\DMguarded}{\overline{\textsf{DM}}}
\newcommand{\ellmax}{\ell_{max}}

\newcommand{\adv}{\textnormal{adv}}
\newcommand{\priv}{\textnormal{priv}}
\newcommand{\pub}{\textnormal{pub}}
\newcommand{\hguarded}{\bar{h}}

\newcommand{\pubro}{\calF}
\newcommand{\ro}{\calF}
\newcommand{\rostr}{\calF}
\newcommand{\idcipher}{\calC}

\newcommand{\eqnand}{\hspace{1em}\textnormal{ and }\hspace{1em}}

\newcommand{\buf}{\mathrm{Buf}}
\newcommand{\bufpub}{\mathrm{PubBuf}}
\newcommand{\bufpriv}{\mathrm{PrivBuf}}

\newcommand{\numfont}[1]{{\footnotesize \texttt{#1}}\,}
\newcommand{\numfontbf}[1]{{\footnotesize \textbf{\texttt{#1}}}\,}
\newcommand{\Finish}{\textnormal{Finish()}}

\newcommand{\cpre}{C^{\mathsc{pre}}}
\newcommand{\cpreinv}{C^{\mathsc{-pre}}}
\newcommand{\cpost}{C^{\mathsc{post}}}
\newcommand{\caux}{C^{\mathsc{aux}}}
\newcommand{\cpostinv}{C^{\mathsc{-post}}}
\newcommand{\cauxinv}{C^{\mathsc{-aux}}}

\newcommand{\given}{\; | \;}
\newcommand{\indiff}{\textrm{indiff}}
\newcommand{\ExpIndiffCons}[1]{\Exp^{\indiff\dash1}_{#1}}
\newcommand{\ExpIndiffSim}[1]{\Exp^{\indiff\dash0}_{#1}}

% ========================================================================


\newcommand{\eve}{\advA}
\newcommand{\keyse}{\mathcal{K_{\mathrm{se}}}}
\newcommand{\keyma}{\mathcal{K_{\mathrm{ma}}}}

\newcommand{\then}{\,;\,}
\newcommand{\andthen}{\thinspace : \thinspace}
\newcommand{\bitsorc}{\$}
\newcommand{\newmsg}{\mathrm{new\mbox{-}msg}}

%\newcommand{\botse}{\bot_e}
%\newcommand{\botma}{\bot_m}
%\newcommand{\botcode}{\bot_c}
\newcommand{\botse}{\bot}
\newcommand{\botma}{\bot}
\newcommand{\botcode}{\bot}

\newcommand{\pforge}{\mathrm{pforge}}
\newcommand{\cforge}{\mathrm{cforge}}
\newcommand{\invalid}{\mathrm{invalid}}
\newcommand{\valid}{f}
\newcommand{\dectest}{\mathsf{Test}}
\newcommand{\testproc}{\procfont{Test}}
\newcommand{\lrorc}{\mathsf{LR}}

\newcommand{\MEE}{\mathsf{MEE}}
\newcommand{\MEEK}{{K}}
\newcommand{\ma}{\mathrm{ma}}
\newcommand{\se}{\mathrm{se}}
\newcommand{\MEEKma}{{K_{\ma}}}
\newcommand{\MEEKse}{{K_{\se}}}
\newcommand{\TLSEN}{\textsf{TLScode}}
\newcommand{\PadTLS}{\textsf{PadTLS}}
\newcommand{\UnpadTLS}{\textsf{UnpadTLS}}
\newcommand{\TLSencode}{\textsf{TLSencode}}
\newcommand{\TLSdecode}{\textsf{TLSdecode}}


\newcommand{\meetlscbc}{\mathsf{MEE\dash TLS\dash CBC}}
\newcommand{\meetlsctr}{\mathsf{MEE\dash TLS\dash CTR}}
\newcommand{\meegencbc}{\mathsf{MEE\dash GEN\dash CBC}}
\newcommand{\meegenctr}{\mathsf{MEE\dash GEN\dash CTR}}

\newcommand{\prfkeyspace}{\calK_{\ma}}
\newcommand{\prfkey}{K}
\newcommand{\PaddingScheme}{\textsf{PadS}}

\newcommand{\collevent}{\mathsc{DColl}}

\newcommand{\ith}{{i^{\mathrm{th}}}}
\newcommand{\jth}{{j^{\mathrm{th}}}}
\newcommand{\newXflag}{{\mathsf{NewX}}}
\newcommand{\repeatXflag}{{\mathsf{RepeatX}}}
\newcommand{\notrepeatXflag}{{\mathsf{NoRepeatX}}}

\newcommand{\bnm}{\begin{newmath}}
\newcommand{\enm}{\end{newmath}}
\newcommand{\bne}{\begin{newequation}}
\newcommand{\ene}{\end{newequation}}
\newcommand{\sets}{\textnormal{ sets }}
\newcommand{\CTR}{\textnormal{CTR}}
\newcommand{\CBC}{\textnormal{CBC}}
\newcommand{\PCBC}{\textnormal{P-CBC}}
\newcommand{\tilC}{\tilde{C}}
\newcommand{\gamesfontsize}{\scriptsize}

\newcommand{\mgood}{\mathsf{Mgood}}
\newcommand{\tagused}{\mathsf{TagUsed}}
\newcommand{\mblock}{\mathsf{Mblock}}
\newcommand{\newcblock}{\mathsf{Cnew}}

\newcommand{\reqlen}{\ell}
\newcommand{\reqlenmax}{\reqlen_{\mathrm{max}}}
\newcommand{\padlen}{\psi}
\newcommand{\padlenmax}{\psi}
\newcommand{\padblen}{p}
\newcommand{\tagidx}{t}
\newcommand{\numpadblocks}{\gamma}
\newcommand{\header}{H}

\newcommand{\taglen}{\tau}
\newcommand{\lastbyte}{\textnormal{lastbyte}}
\newcommand{\inttobyte}{\textnormal{int2byte}}
\newcommand{\bytetoint}{\textnormal{byte2int}}

\newcommand{\bitsplit}{\textnormal{split}}
\newcommand{\last}{\textnormal{last}}
\newcommand{\first}{\textnormal{first}}
\newcommand{\chop}{\textnormal{chop}}

\newcommand{\modF}{\tilde{F}}
\newcommand{\phase}{\textsf{phase}}


\newcommand{\main}{{\procfont{main}}}
\newcommand{\mainn}[1]{{\underline{\main\ {#1}}:}}
\newcommand{\mainnv}[1]{\setstretch{1.1}{\underline{\main\ {#1}}:}\smallskip}
\newcommand{\bzo}{\bracketize{1}}
\newcommand{\bzt}{\bracketize{2}}
\newcommand{\koe}{k_1^e}
\newcommand{\kie}[1]{k_{#1}^e}
\newcommand{\kim}[1]{k_{#1}^m}
\newcommand{\tbc}{\mathsf{TBC}}
\newcommand{\tbce}{\widetilde{E}}
\newcommand{\tbcd}{\widetilde{D}}
\newcommand{\kte}{k_2^e}
\newcommand{\kom}{k_1^m}
\newcommand{\ktm}{k_2^m}
\newcommand{\eecl}{t}
\newcommand{\headerspace}{\calH}
\newcommand{\reqlenspace}{\calL}

\newcommand{\spacecomma}{\;,\;}
\newcommand{\calZ}{\mathcal{Z}}
\newcommand{\trunc}{\mathrm{Trunc}}
\newcommand{\truncleft}{\overline{\mathrm{Trunc}}}

\newcommand{\Pad}{\mathrm{Pad}}
\newcommand{\PadH}{\mathrm{PadH}}
\newcommand{\PadM}{\mathrm{PadM}}
%%Alternate padding

\newcommand{\AltPaddingScheme}{\textsf{AltPadD}}

\newcommand{\AltPad}{\mathrm{AltPad}}
\newcommand{\AltPadH}{\mathrm{AltPadH}}
\newcommand{\AltPadM}{\mathrm{AltPadM}}
\newcommand{\AltPadSuffix}{\mathrm{AltPadSuf}}
%%

\newcommand{\Unpad}{\mathrm{Unpad}}
\newcommand{\UnpadH}{\mathrm{UnpadH}}
\newcommand{\UnpadM}{\mathrm{UnpadM}}
\newcommand{\Parse}{\mathrm{Parse}}
\newcommand{\PadSuffix}{\mathrm{PadSuf}}
\newcommand{\Padcheck}{\mathrm{PadCheck}}
\newcommand{\tagbit}{\mathtt{t}}
\newcommand{\TabS}{\mathtt{S}}
\newcommand{\ellc}{{\ell_c}}
\newcommand{\ellj}{{\ell_j}}
\newcommand{\teec}{{t_c}}
\newcommand{\teej}{{t_j}}
\newcommand{\essj}{{s_j}}
\newcommand{\essc}{{s_c}}
\newcommand{\tpad}{\tilde{P}}
\newcommand{\taglenn}{{n}}
\newcommand{\bmin}{{b_{\mathrm{min}}}}
\newcommand{\headerstar}{\header^*}
\newcommand{\gamenote}[1]{{\tiny #1}}
%\newcommand{\ellmax}{{\ell_{\mathrm{max}}}}

\newcommand{\highlight}[1]{\mbox{\colorbox{mygrey}{$#1$}}}
\newcommand{\calYbad}{\calB}%{\calY_{bad}}
\newcommand{\PadPrefix}{\textnormal{PadPrefix}}
\newcommand{\testcount}{\mathrm{cnt}}


\newcommand{\CN}{{CN}}
\newcommand{\CRan}{{R}}
\newcommand{\SN}{{SN}}
\newcommand{\orbit}{{orb}}


\newcommand{\KDF}{\mathsf{KDF}}

\newcommand{\skserver}{\sk_s}
\newcommand{\skclient}{\sk_c}
\newcommand{\Nonce}{N}
\newcommand{\Nclient}{N_c}
\newcommand{\Nserver}{N_s}
\newcommand{\csuites}{\mathcal{CS}}
\newcommand{\certclient}{\cert_c}
\newcommand{\certserver}{\cert_s}


\newcommand{\servername}{\mathrm{SN}}

\newcommand{\confserver}{\mathrm{conf}_s}
\newcommand{\confclient}{\mathrm{conf}_c}
\newcommand{\confsn}{\mathrm{conf}_{\servername}}
\newcommand{\bracketize}[1]{\lbrack #1 \rbrack}
\newcommand{\bktz}[1]{\bracketize{#1}}
\newcommand{\dbktz}[2]{\bracketize{#1}\bracketize{#2}}
\newcommand{\sid}{\mathrm{sid}}
\newcommand{\kex}{\mathrm{kex}}
\newcommand{\kexguess}{\mathrm{kex}^*}
\newcommand{\hello}{\mathrm{hel}}
\newcommand{\fin}{\mathrm{fin}}
\newcommand{\chc}{\mathrm{chc}}
\newcommand{\ext}{\mathrm{ext}}
\newcommand{\appdata}{\mathrm{app data}}
\newcommand{\encrypted}[1]{\{#1\}}
\newcommand{\initencrypted}[1]{\{#1\}^{\mathrm{init}}}
\newcommand{\signed}[1]{\llbracket#1\rrbracket}
%\newcommand{\optional}[1]{\textcolor{blue}{[} #1 \textcolor{blue}{]}}
\newcommand{\optional}[1]{\textcolor{Gray}{ #1 }}
\newcommand{\sidclient}{\sid_c}
\newcommand{\sidserver}{\sid_s}
\newcommand{\verclient}{\ver_c}
\newcommand{\verserver}{\ver_s}
\newcommand{\kexclient}{\kex_c}
\newcommand{\kexclientguess}{\kexguess_c}
\newcommand{\kexserver}{\kex_s}
\newcommand{\helloclient}{\hello_c}
\newcommand{\helloserver}{\hello_s}
\newcommand{\finclient}{\fin_c}
\newcommand{\finserver}{\fin_s}
\newcommand{\chcclient}{\chc_c}
\newcommand{\chcserver}{\chc_s}
\newcommand{\extclient}{\ext_c}
\newcommand{\extserver}{\ext_s}

\newcommand{\msk}{{ms}}



\newcommand{\HIDE}{\textnormal{HIDE}}
\newcommand{\FROB}{\textnormal{FROB}}
\newcommand{\CROB}{\textnormal{CROB}}
\newcommand{\BIND}{\textnormal{r-BIND}}
\newcommand{\SRBIND}{\textnormal{sr-BIND}} %%% strong binding
\newcommand{\TRBIND}{\textnormal{tr-BIND}} %%% targeted receiver binding
\newcommand{\RBIND}{\BIND}
\newcommand{\VFROB}{\textnormal{eFROB}}
\newcommand{\SBIND}{\textnormal{s-BIND}}
\newcommand{\MRBIND}{\textnormal{mrBIND}}
\newcommand{\VBIND}{\textnormal{vBIND}}
\newcommand{\DBIND}{\textnormal{t-BIND}}
\newcommand{\EBIND}{\textnormal{EBIND}}
\newcommand{\MBIND}{\textnormal{mBIND}}
\newcommand{\TBIND}{\textnormal{TBIND}}
\newcommand{\CS}{{\textnormal{\textsf{CS}}}}
\newcommand{\Commit}{\textsf{Com}}
\newcommand{\Open}{\textsf{Open}}
\newcommand{\Verify}{\textsf{VerC}}
\newcommand{\openspace}{\keyspace_f}
\newcommand{\openingspace}{\keyspace_{\EC}}
\newcommand{\commitspace}{\calC}
\newcommand{\frankspace}{\calT}
\newcommand{\Efrankspace}{\calT}
\newcommand{\HMAC}{\textsf{HMAC}}
\newcommand{\FSE}{\textsf{FSE}}
\newcommand{\ftag}{T}
\newcommand{\fopen}{K_{f}}
\newcommand{\open}{K_{c}}
\newcommand{\kimg}{K_{\text{a}}}
\newcommand{\nimg}{N_{\text{a}}}
\newcommand{\gcmenc}{\mathsf{GCM\dash Enc}}
\newcommand{\gcmdec}{\mathsf{GCM\dash Dec}}
\newcommand{\fbid}{\textsf{id}}
\newcommand{\fbtag}{\mathsf{FBTag}}
\newcommand{\fbauthtag}{\textsf{t}_{\text{FB}}}
\newcommand{\imct}{C_{\text{a}}}
\newcommand{\imgd}{D_{\text{a}}}
\newcommand{\shatfs}{\mathsf{SHA\dash 256}}
\newcommand{\collidegcm}{\mathsf{Collide\dash GCM}}
\newcommand{\AD}{\text{AD}}
\newcommand{\mlen}{\text{mlen}}
\newcommand{\adlen}{\text{adlen}}
\newcommand{\junk}{\mathsf{M}_{\text{j}}}
\newcommand{\CE}{\textnormal{{\textsf{CE}}}}
\newcommand{\CEH}{\textsf{CEH}}
\newcommand{\CKg}{{\textsf{Kg}}}
\newcommand{\CEnc}{{\textsf{Enc}}}
\newcommand{\CDec}{{\textsf{Dec}}}
\newcommand{\CVer}{{\textsf{Ver}}}
\newcommand{\cmpr}{\mathsf{f}}
\newcommand{\modu}{\,\mathrm{mod}\,}
\newcommand{\PadScheme}{\mathbb{P}}
\newcommand{\UpdatePad}{\mathrm{UpdatePad}}
\newcommand{\FinalizePad}{\mathrm{FinalizePad}}
\newcommand{\kprf}{K_{\text{prf}}}
\newcommand{\attch}{M_{\text{a}}}
\newcommand{\attchct}{C_{\text{a}}}
\newcommand{\attchab}{\attch^{\text{ab}}}
\newcommand{\AESCTREnc}{\text{CTR-Enc}}
\newcommand{\AESCTRDec}{\text{CTR-Dec}}
\newcommand{\AMerge}{\text{Att-Merge}}
\newcommand{\gpad}{P}
\newcommand{\kimgab}{K_{1}}
\newcommand{\kjunk}{K_{2}}
\newcommand{\abus}{_{1}}
\newcommand{\juhk}{_{2}}
\newcommand{\accum}{\mathsf{acc}}
\newcommand{\blen}{\mathsf{blen}}
\newcommand{\ghash}{\mathsf{GHASH}}
\newcommand{\sshare}{\mathsf{SShare}}

% randomized symmetric encryption defs
\newcommand{\SEenc}{\textsf{Enc}}
\newcommand{\SEencsim}{\textsf{EncSim}}
\newcommand{\RORreal}{\textnormal{REAL}}
\newcommand{\RORrand}{\textnormal{RAND}}


%newcommand{\ctxtlen}{\ell}
\newcommand{\EC}{\textnormal{{\textsf{EC}}}}
\newcommand{\EKg}{{\textsf{EKg}}}
\newcommand{\EEnc}{{\textsf{EC}}}
\newcommand{\EDec}{{\textsf{DO}}}
\newcommand{\EVer}{{\textsf{EVer}}}


\newcommand{\MDE}{\HFC}
\newcommand{\MDKg}{{\textsf{\HFC Kg}}}
\newcommand{\MDEnc}{{\textsf{\HFC Enc}}}
\newcommand{\MDDec}{{\textsf{\HFC Dec}}}
\newcommand{\MDVer}{{\textsf{\HFC Ver}}}

\newcommand{\SPE}{\textnormal{{\textsf{SPE}}}}
\newcommand{\SPKg}{{\textsf{SPKg}}}
\newcommand{\SPEnc}{{\textsf{SPEnc}}}
\newcommand{\SPDec}{{\textsf{SPDec}}}
\newcommand{\SPVer}{{\textsf{SPVer}}}

\newcommand{\HFC}{\textsf{HFC}}
\newcommand{\EtM}{\mathsf{EtM}}
\newcommand{\MtE}{\mathsf{MtE}}
\newcommand{\GCM}{\mathsf{GCM}}

\newcommand{\ctxt}{C}
\newcommand{\ctxtA}{C}
\newcommand{\ctxtB}{C_B}
\newcommand{\ctxtAE}{C_{\textnormal{\textsf{AE}}}}
\newcommand{\EctxtA}{\ctxtAEC}
\newcommand{\EctxtB}{\ctxtBEC}

\newcommand{\ctxtlen}{\textsf{clen}}

\newcommand{\CEbad}{\textnormal{\textsf{CEBad}}}
\newcommand{\CEncBad}{\textsf{EncBad}}
\newcommand{\CDecBad}{\textsf{DecBad}}
\newcommand{\CVerBad}{\textsf{VerBad}}

\newcommand{\CSbad}{\textnormal{\textsf{CSBad}}}
\newcommand{\CommitBad}{\textsf{ComBad}}
\newcommand{\VerifyBad}{\textsf{VerBad}}



\newcommand{\FBfrank}{\textsf{FB}}
\newcommand{\FBenc}{\textsf{FBEnc}}
\newcommand{\FBdec}{\textsf{FBDec}}
\newcommand{\FBver}{\textsf{FBVer}}

\newcommand{\AEADror}{\textnormal{ROR}} % nae = nonce-based auth enc
\newcommand{\AEADreal}{\textnormal{REAL}}
\newcommand{\AEADrand}{\textnormal{RAND}}
\newcommand{\AEADctxt}{\textnormal{\CTXT}}
\newcommand{\COREC}{\textnormal{COR}} % nae = nonce-based auth enc
\newcommand{\SCOREC}{\textnormal{S-COR}} % nae = nonce-based auth enc

\newcommand{\MORORctxt}{\textnormal{MO-ROR-C}} % nae = nonce-based auth enc
\newcommand{\MOREALctxt}{\textnormal{MO-REAL-C}} % nae = nonce-based auth enc
\newcommand{\MORANDctxt}{\textnormal{MO-RAND-C}} % nae = nonce-based auth enc
\newcommand{\SORORctxt}{\textnormal{ROR-C}} % nae = nonce-based auth enc
\newcommand{\SOREALctxt}{\textnormal{SO-REAL-C}} % nae = nonce-based auth enc
\newcommand{\SORANDctxt}{\textnormal{SO-RAND-C}} % nae = nonce-based auth enc
\newcommand{\MOROR}{\textnormal{MO-ROR}} % nae = nonce-based auth enc
\newcommand{\MOREAL}{\textnormal{MO-REAL}} % nae = nonce-based auth enc
\newcommand{\MORAND}{\textnormal{MO-RAND}} % nae = nonce-based auth enc
\newcommand{\SOROR}{\textnormal{ROR}} % nae = nonce-based auth enc
\newcommand{\SOREAL}{\textnormal{REAL}} % nae = nonce-based auth enc
\newcommand{\SORAND}{\textnormal{RAND}} % nae = nonce-based auth enc
%\newcommand{\REALce}{\textnormal{REAL}} % nae = nonce-based auth enc
%\newcommand{\IDEALce}{\textnormal{RAND}} % nae = nonce-based auth enc

\newcommand{\NMOROR}{\textnormal{MO-nROR}} % nae = nonce-based auth enc
\newcommand{\NMOREAL}{\textnormal{MO-nREAL}} % nae = nonce-based auth enc
\newcommand{\NMORAND}{\textnormal{MO-nRAND}} % nae = nonce-based auth enc

\newcommand{\NMORORctxt}{\textnormal{MO-nROR-C}} % nae = nonce-based auth enc
\newcommand{\NMOREALctxt}{\textnormal{MO-nREAL-C}} % nae = nonce-based auth enc
\newcommand{\NMORANDctxt}{\textnormal{MO-nRAND-C}} % nae = nonce-based auth enc

\newcommand{\NSOREAL}{\textnormal{nREAL}} % nae = nonce-based auth enc
\newcommand{\NSORAND}{\textnormal{nRAND}} % nae = nonce-based auth enc
\newcommand{\NSOCTXT}{\textnormal{nCTXT}} % nae = nonce-based auth enc

\newcommand{\EtE}{\textnormal{EtE}} % nae = nonce-based auth enc

\newcommand{\CtE}{\textnormal{CtE1}}
\newcommand{\CtEEnc}{\textnormal{CtE1-Enc}}
\newcommand{\CtEDec}{\textnormal{CtE1-Dec}}
\newcommand{\CtEVer}{\textnormal{CtE1-Ver}}

\newcommand{\FCtE}{\textnormal{CtE2}}
\newcommand{\FCtEEnc}{\textnormal{CtE2-Enc}}
\newcommand{\FCtEDec}{\textnormal{CtE2-Dec}}
\newcommand{\FCtEVer}{\textnormal{CtE2-Ver}}

\newcommand{\NCE}{\textnormal{nCE}}
\newcommand{\NCEKg}{\textnormal{Kg}}
\newcommand{\NCEEnc}{\textnormal{Enc}}
\newcommand{\NCEDec}{\textnormal{Dec}}
\newcommand{\NCEVer}{\textnormal{Ver}}

\newcommand{\FastNCE}{\textnormal{CEP}}
\newcommand{\FastKg}{\textnormal{CEP-Kg}}
\newcommand{\FastEnc}{\textnormal{CEP-Enc}}
\newcommand{\FastDec}{\textnormal{CEP-Dec}}
\newcommand{\FastVer}{\textnormal{CEP-Ver}}
\newcommand{\LCP}{\text{LCP}}
\newcommand{\OCB}{\textnormal{OCB}}
\newcommand{\OCBKg}{\textnormal{OCB-Kg}}
\newcommand{\OCBEnc}{\textnormal{OCB-Enc}}
\newcommand{\OCBDec}{\textnormal{OCB-Dec}}
\newcommand{\OCBVer}{\textnormal{OCB-Ver}}
\newcommand{\querya}{a}
\newcommand{\BC}{\textnormal{BC}}
\newcommand{\BCKg}{\textnormal{BC-Kg}}
\newcommand{\BCEnc}{\textnormal{BC-Enc}}
\newcommand{\BCDec}{\textnormal{BC-Dec}}
\newcommand{\BCVer}{\textnormal{BC-Ver}}

\newcommand{\tweakCipher}{\tilde{\calE}}
\newcommand{\tweakE}{\tilde{E}}
\newcommand{\tweakD}{\tilde{D}}

\newcommand{\tlk}{\tilde{K}}
\newcommand{\GSub}{\textnormal{G}}
\newcommand{\twc}{\textnormal{TC}}
\newcommand{\tweakspace}{\mathcal{T}}
\newcommand{\CSCTXT}{\textnormal{NM}} % nae = nonce-based auth enc
\newcommand{\OCommit}{\textbf{Com}} % nae = nonce-based auth enc
\newcommand{\OChalVerify}{\textbf{ChalVerC}} % nae = nonce-based auth enc

\newcommand{\MOCTXT}{\textnormal{MO-CTXT}} % nae = nonce-based auth enc
\newcommand{\SOCTXT}{\textnormal{CTXT}} % nae = nonce-based auth enc
\newcommand{\NMOCTXT}{\textnormal{MO-nCTXT}} % nae = nonce-based auth enc
\newcommand{\REALnae}{\textnormal{REAL}} % nae = nonce-based auth enc
\newcommand{\IDEALnae}{\textnormal{RAND}} % nae = nonce-based auth enc

\newcommand{\OEnc}{\textbf{Enc}} % nae = nonce-based auth enc
\newcommand{\ODec}{\textbf{Dec}} % nae = nonce-based auth enc
\newcommand{\OChalEnc}{\textbf{ChalEnc}} % nae = nonce-based auth enc
\newcommand{\OChalDec}{\textbf{ChalDec}} % nae = nonce-based auth enc

\newcommand{\TagAlg}{\textsf{tag}} % nae = nonce-based auth enc
\newcommand{\VerAlg}{\textsf{ver}} % nae = nonce-based auth enc

\newcommand{\OTag}{\textbf{Tag}} % nae = nonce-based auth enc
\newcommand{\OVer}{\textbf{Ver}} % nae = nonce-based auth enc
\newcommand{\OKey}{\textbf{KeyCom}} % nae = nonce-based auth enc

\newcommand{\regF}{F}
\newcommand{\crF}{F^{\mathrm{cr}}}

\newcommand{\Valid}{\mathbb{V}}
\newcommand{\Zos}{B}
\newcommand{\badtruezo}{\bad_{01}\gets\true}
\newcommand{\badtrueot}{\bad_{12}\gets\true}
\newcommand{\badzo}{\bad_{01}}
\newcommand{\badot}{\bad_{12}}
\newcommand{\prim}{f}
\newcommand{\func}{f}
\newcommand{\triphash}{\mathcal{H}}
\newcommand{\LP}{LP_{231}}
\newcommand{\LPPRF}{LP^{\text{prf}}_{231}}
\newcommand{\chain}{V}
\newcommand{\two}{\textbf{2}}
\newcommand{\perm}{\pi}
\newcommand{\TC}{TC}
\newcommand{\moENC}{\textnormal{\textsf{mo\textnormal{-}\CEnc}}}
\newcommand{\moDEC}{\textnormal{\textsf{mo\textnormal{-}\CDec}}}
\newcommand{\moVER}{\textnormal{\textsf{mo\textnormal{-}\CVer}}}
\newcommand{\moCE}{\textnormal{\textsf{mo\textnormal{-}\CE}}}
\newcommand{\moKG}{\textnormal{\textsf{mo\textnormal{-}\CKg}}}
\newcommand{\aeadlen}{\textsf{aead.len}}
\newcommand{\tagT}{T}
\newcommand{\encrypt}{encryptment }%%%Macros for names of encryptment algorithsm
\newcommand{\opening}{opening }
\newcommand{\ctxtAEC}{C_{\EC}}
\newcommand{\ctxtBEC}{B_{\EC}}
\newcommand{\KEC}{K_{\EC}}
\newcommand{\queryflag}{\textsf{query\textnormal{-}made}}
\newcommand{\kblen}{\kappa}
\newcommand{\gstate}{S}
\newcommand{\padded}{P}
\newcommand{\squeeze}{X}
\newcommand{\taglength}{t}

\newcommand{\otCTXT}{\text{otCTXT}}
\newcommand{\headlen}{\ell_H}
\newcommand{\msglen}{\ell_M}
\newcommand{\padindex}{n}
\newcommand{\supp}{\textnormal{Supp}}

\newcommand{\runtime}{t}
\newcommand{\numqueries}{q}


%%% Ciphers
\newcommand{\cipher}{\calE}
\newcommand{\cipherK}{\calK}
\newcommand{\cipherE}{E}
\newcommand{\cipherD}{D}
\newcommand{\keyspace}{\calK}
\newcommand{\msgspace}{\calM}
\newcommand{\ctxtspace}{\calC}
\newcommand{\msg}{M}
\newcommand{\ct}{C}
\newcommand{\randfn}{\rho}

\newcommand{\blockciphers}{\textnormal{BC}}
\newcommand{\ic}{E}
\newcommand{\icInv}{D}
\newcommand{\Domain}{\texttt{Dom}}
\newcommand{\Range}{\texttt{Ran}}

\newcommand{\TKR}{\textnormal{TKR}} % security game for key recovery
\newcommand{\Fn}{\textnormal{Fn}}
\newcommand{\FnInv}{\textnormal{FnInv}}
\newcommand{\FnSim}{\textnormal{FnSim}}
\newcommand{\AdvTKR}[2]{\Adv^{\mathrm{tkr}}_{#1}(#2)}

\newcommand{\eks}{\textnormal{eks}}

\newcommand{\KR}{\textnormal{KR}} % security game for key recovery
\newcommand{\AdvKR}[2]{\Adv^{\mathrm{kr}}_{#1}(#2)}

\newcommand{\OTIND}{\textnormal{otIND}} % security game for key recovery
\newcommand{\AdvOTIND}[2]{\Adv^{\mathrm{ot\dash ind}}_{#1}(#2)}


\newcommand{\PRP}{\textnormal{PRP}}
\newcommand{\AdvPRP}[2]{\Adv^{\mathrm{prp}}_{#1}(#2)}

\newcommand{\TPRP}{\textnormal{TPRP}}
\newcommand{\AdvTPRP}[2]{\Adv^{\mathrm{tprp}}_{#1}(#2)}
\newcommand{\tweakpi}{\tilde{\pi}}

\newcommand{\STPRP}{\textnormal{STPRP}}
\newcommand{\AdvSTPRP}[2]{\Adv^{\mathrm{\pm tprp}}_{#1}(#2)}



\newcommand{\SPRP}{\textnormal{SPRP}}
\newcommand{\AdvSPRP}[2]{\Adv^{\mathrm{\pm prp}}_{#1}(#2)}



% \newcommand{\inv}{^{-1}}
% \newcommand{\Adv}{\mathbf{Adv}}
% \newcommand{\keyspace}{\calK}
\newcommand{\PRFReal}{\text{PRF-Real}}
\newcommand{\PRFIdeal}{\text{PRF-Ideal}}
\newcommand{\RFRange}{\calR}
\newcommand{\RFun}{\mathbf{R}}
\newcommand{\Fun}{\mathsf{Fn}}
\newcommand{\SPRPIdeal}{\text{SPRP-Ideal}}
\newcommand{\SPRPReal}{\text{SPRP-Real}}
\newcommand{\PRPIdeal}{\text{PRP-Ideal}}
\newcommand{\PRPReal}{\text{PRP-Real}}
\newcommand{\TSPRPIdeal}{\text{TSPRP-Ideal}}
\newcommand{\TSPRPReal}{\text{TSPRP-Real}}
\newcommand{\TPRPIdeal}{\text{TPRP-Ideal}}
\newcommand{\TPRPReal}{\text{TPRP-Real}}
\newcommand{\AdvTSPRP}[2]{\AdvSTPRP{#1}{#2}}
%\newcommand{\AdvTPRP}[2]{\Adv^{\mathrm{tprp}}_{#1}(#2)}
\newcommand{\RPerm}{\Pi}
\newcommand{\RPermInv}{\Pi\inv}
%\newcommand{\tbce}{\widetilde{E}}
%\newcommand{\tbcd}{\widetilde{D}}
%\newcommand{\tbc}{\text{TBC}}
%\newcommand{\tweakspace}{\calT}
\newcommand{\noncespace}{\calN}
%\newcommand{\ctxtspace}{\calC}
\newcommand{\gf}[1]{\text{GF}(#1)}
%\newcommand{\SE}{\mathsf{SE}}
\newcommand{\xe}{\mathsf{XE}}
\newcommand{\lrw}{\mathsf{LRW}}
\newcommand{\xeenc}{\xe.\enc}
\newcommand{\xedec}{\xe.\dec}
\newcommand{\Perms}{\text{Perms}}
\newcommand{\Funs}{\text{Funs}}
\newcommand{\ocbcore}{\mathsf{OCB\dash Core}}
\newcommand{\ocbcoreenc}{\mathsf{OCB\dash Core}.\enc}
\newcommand{\ocbcoredec}{\mathsf{OCB\dash Core}.\dec}
\newcommand{\clen}{\ctxtlen}
\newcommand{\rorcpareal}{\text{RoR-CPA-Real}}
\newcommand{\rorcpaideal}{\text{RoR-CPA-Ideal}}
\newcommand{\integers}{\mathbb{Z}}

\newcommand{\PRF}{\textnormal{PRF}}
\newcommand{\AdvPRF}[2]{\Adv^{\mathrm{prf}}_{#1}(#2)}


\newcommand{\Feistel}{\cipher}
\newcommand{\Maj}{\textnormal{Maj}}

\newcommand{\MRUMA}{\textnormal{MR-UMA}}
\newcommand{\MRUMAideal}{\textnormal{MR-UMA-IDEAL}}
\newcommand{\msgsampler}{\calP}
\newcommand{\mdist}{p_m}
\newcommand{\AdvMRUMA}[2]{\Adv^{\mathrm{mr\dash uma}}_{#1}(#2)}
\newcommand{\hatE}{\hat{E}}

\newcommand{\REAL}{\textnormal{REAL}}
\newcommand{\RAND}{\textnormal{RAND}}
\newcommand{\ROR}{\textnormal{ROR}}
\newcommand{\AdvROR}[2]{\Adv^{\mathrm{ror}}_{#1}(#2)}
\newcommand{\AdvRAND}[2]{\Adv^{\mathrm{\INDRAND}}_{#1}(#2)}
\newcommand{\RORCCA}{\textnormal{ROR-CCA}}
\newcommand{\AdvRORCCA}[2]{\Adv^{\mathrm{ror\dash cca}}_{#1}(#2)}

% \newcommand{\INDSIM}{\textnormal{INDSIM}}
\newcommand{\AdvINDSIM}[2]{\Adv^{\mathrm{ind\dash sim}}_{#1}(#2)}


\newcommand{\EncOracle}{\textnormal{Enc}}
\newcommand{\DecOracle}{\textnormal{Dec}}
\newcommand{\EncSim}{\textnormal{EncSim}}
\newcommand{\DecSim}{\textnormal{DecSim}}
\newcommand{\Time}{\textsf{T}}

\newcommand{\tildeC}{\tilde{C}}
\newcommand{\EM}{\textnormal{EM}}
\newcommand{\Sbox}{\textnormal{Sbox}}
\newcommand{\myInd}{\hspace*{1em}}

\newcommand{\MA}{\mathsf{MA}}
\newcommand{\mtag}{\mathsf{tag}}
\newcommand{\ver}{\mathsf{ver}}

\newcommand{\msgset}{\textsf{MsgSet}}
\newcommand{\pairset}{\textsf{PairSet}}
\newcommand{\TagOracle}{\textnormal{Tag}}
\newcommand{\VerOracle}{\textnormal{Ver}}
\newcommand{\TagSim}{\textnormal{TagSim}}

\newcommand{\cAU}{\textnormal{cAU}}
\newcommand{\AdvcAU}[2]{\Adv^{\mathrm{cau}}_{#1}(#2)}

\newcommand{\specialcell}[2][c]{%
  \begin{tabular}[#1]{@{}l@{}}#2\end{tabular}}

\newcommand{\OWF}{\textnormal{OWF}}

\newcommand{\SPR}{\textnormal{SPR}}
\newcommand{\AdvSPR}[2]{\Adv^{\mathrm{spr}}_{#1}(#2)}

\newcommand{\PWR}{\textnormal{PWR}}
\newcommand{\AdvPWR}[2]{\Adv^{\mathrm{pwr}}_{#1}(#2)}
\newcommand{\pw}{pw}
\newcommand{\salt}{sa}

\newcommand{\Hinfty}{H^{\infty}}
\newcommand{\Hshan}{H}

\newcommand{\simoracle}{\textsf{Sim}}
\newcommand{\foracle}{\textsf{f}}
\newcommand{\INDIFF}{\textnormal{Indiff}}
\newcommand{\AdvINDIFF}[2]{\Adv^{\mathrm{indiff}}_{#1}(#2)}


\newcommand{\RKAPRF}{\textnormal{RKA-PRF}}
\newcommand{\AdvRKAPRF}[2]{\Adv^{\mathrm{rka\dash prf}}_{#1}(#2)}


\newcommand{\RSAk}{\textnormal{RSA}\dash k}
\newcommand{\AdvOWFRSA}[2]{\Adv^{\mathrm{owf}}_{#1}(#2)}
%%% Local Variables:
%%% mode: latex
%%% TeX-master: "main"
%%% End:



\addbibresource{notes.bib}


\title{\textbf{Designing Secure Cryptography}}

\author{
  Cornell CS 6831
}


\ifnum\submission=1
\date{}
\fi



\begin{document}

\maketitle


\section*{Notes}

\tnote{A couple notes for myself, will remove later.}
\begin{itemize}
  \item Fix a pseudocode language, including randomness, and an associated model
    of computation. Treat memory usage as well as run times of algorithms. We
    will most often treat memory as subservient to run-time, bounding the former
    by the latter. But special treatments are needed.
    The goal is to have a simplified abstraction, but which
    provides good predictions of running code.  Probably look at ``Careful with
    Composition'' paper as starting point.

 \item Algorithms are simply pseudocode that take input and produce an output.
    Define what it means for algorithm to be runnable. Define adversaries
    as algorithms and require they be runnable. Comment on how this interacts
    with (and rules out) non-uniform reductions, other such things.

  \item  Algorithms can be parameterized, but this is just notational sugar. The result
    must still be runnable. They may have oracles, and this defines some
    interface.  Give conventions on resource usage of algorithms: number of
    oracle queries, run time (with oracle queries unit cost), etc. Discuss how
    runnable algorithms can then be composed.

  \item Associate probability space to any algorithm, be pedantic here since we
    may want to prove some basic stuff via direct manipulation of probability
    space (i.e., coin-counting arguments).


  \item Introduce security definitions as special algorithms called games. The
    game is just another algorithm parameterized by some adversary, scheme, etc.
    Thus typically what we define as a game is actually a family of algorithms,
    one for each instantiation of things. (How much notion of a template do we
    need? Is this confusing?) It is used to measure adversarial advantage, a
    which is a measure of success for an adversary.

  \item Discuss the viewpoint underlying all the above. Security models are
    coarse abstractions of real cryptographic systems. They must be coarse for
    us to do certain kinds of analysis. This means that what we prove in our
    formalizations do not typically apply to real systems. But they are a good
    heuristic: schemes for which we have good analyses tend to provide better
    security. We will therefore judge the value of models by their utility in
    helping us build cryptographic schemes that resist attackers in practice.
    We will include examples and discussions reinforcing this viewpoint along
    the way. (One example to treat right away is asymptotics. Could be good
    heuristic, but is often quite poor, and gives little advice about how to set security
    parameters meaningfully.)

  \item Discuss assumptions. These are formalized as games. What separates them
    from security definitions is contextual. They are ``lower level'' and
    less related to the goals of cryptographic protocols.

  \item Reductions are runnable algorithms.

  \item Using reductions to set security parameters. Maybe interleave some of
    this discussion with first couple of chapters
\end{itemize}


\newpage


\subsection{Introduction}

We have seen two main asymmetric primitive in the last lectures - RSA and Discrete Log Problem. In RSA problem, we work in ${Z_N}^*$ where $N=pq$ and p,q are prime number. The main assumption is that, given N and it's factor p and q, we can have a pair of public key e and secret key d such that $X^e mod N$ doesn't reveal any information about d. The Discrete Log Problem works in ${Z_p}^*$ for a prime P and the assumption is given $g^x mod p$, where g is a generator, we can't recover x. In both of the above primitives, we need to increase the group size considerably to provide better security guarantee. eg to provide 256-bit security, the group size needs to be 15360 bit size. The increase in group size leads to increase in  bandwidth, storage and computational time. This lead to the development of cryptographic primitives based on Elliptic Curve, which had group size just twice larger than the security parameter. Elliptic Curve are now the go-to state of the art practice and driven by the fact that RSA and Discrete Log Problem has become slower and slower over the years.

Elliptic Curves are discrete log-based system. They use a new kind of group defined relative to a finite field. Finite field allows basic operations like Addition, Multiplication, Subtraction and Division. Thus, finite fields have two operation and everything has inverse under both operations. An example of finite field group is integer modulo p which is denoted by $F_p$ or $GF(p)$. Elliptic Curves will use curves over $F_p$. 

\newtheorem{df}{Definition}
\begin{df}
Elliptic Curves are defined as by the set of x,y points in $F_p$ defined by the equation 
\bnm
E={(x,y) \mid y^2 = x^3+ax+b\mod p}
\enm 
where a,b are fixed values also from $F_p$, plus one special point called Point of Infinity $\cal O$
\end{df}

It is also required $E$ to be non-singular. In other words,
the partial derivatives shouldn't vanish simultaneously for any point
on $E$. This is equivalent to requiring that the
equation $x^{3}+Ax+B=0$ has no multiple roots. Thus, solving the above condition leads to the following lemma.

\begin{lemma}
 E is non singular if and only if $4A^{3} + 27B^{2}$ is not equal to zero
\end{lemma}

The \figref{fig:plot} shows the Eliptic Curve plot for the curve with A = -5 and B = 5. The graph is symmetric over the x-axis. It should be noted that graph consists of real points on other hand finite field consist of some points on this graph. For example few points on this graph are (1,1) and (1,-1) are points of $F_5$ of this graph.


\begin{figure}[!h]
\begin{adjustbox}
{center}
\begin{tikzpicture}
	
        \begin{axis}[
            xmin=-4,
            xmax=4,
            ymin=-4,
            ymax=4,
            xlabel={$x$},
            ylabel={$y$},
            scale only axis,
            axis lines=middle,
            % set the minimum value to the minimum x value
            % which in this case is $-\sqrt[3]{7}$
             domain=-2.7427:3,     % <-- works for pdfLaTeX and LuaLaTeX
%            domain=-1.91293118:3,   % <-- would also work for LuaLaTeX
            samples=200,
            smooth,
            % to avoid that the "plot node" is clipped (partially)
            clip=false,
            % use same unit vectors on the axis
            axis equal image=true,
        ]
            \addplot [line width=2pt,black] {sqrt(x^3-5*x+5)}
                node[right] {$y^2=x^3-5*x+5$};
            \addplot [line width=2pt,black] {-sqrt(x^3-5*x+5)};
        \end{axis}
    \end{tikzpicture}
    \end{adjustbox}
      \caption{Plot for $y^2=x^3-5*x+5$}
        \label{fig:plot}

\end{figure}

\subsection{Elliptic Curve Group Operation.} A Group is a set G along with a binary operation associated with this group. This operation should have certain set of properties. It should be closed and associative. Also, the group must contain an identity element and also have an inverse within the group for each element. The abelian groups also follow commutative property. 

\begin{figure}
\includegraphics{ecc/ec_group_operations.pdf}
\caption{Geometric Representation of Elliptic Curve Group Operation}
 \label{fig:oper}
\end{figure}

All the points on any elliptic curve form an abelian group. The rule associated with this group can be explained using \figref{fig:oper}. It is assumed that every vertical line passes through the neutral element or point of infinity $ \cal O$ as shown in the figure. Inverse element of a point P, $P^{-1}$ is the point of intersection of vertical line through P and the curve. The group operation for Elliptical Curve is called Point Addition and denoted by $P+Q$, where P and Q are points on the curve. Let $\diamond$ operator on points P and Q on the curve represent the intersection point of the line joining P and Q and the curve. Then the point addition can be represented as $\cal O \diamond (P \diamond Q)$. In other words, it represents the  intersection of the vertical line through $P \diamond Q$ and the curve. The doubling of P, i.e P+P, is represented by similarly by taking the intersection of tangent of P and the curve and drawing a vertical line through that point to get it's intersection with the curve as shown in figure \figref{fig:oper}.

\paragraph{Algebraic Representation.} We can represent the group operation algebraically as well for finite fields. The slope for the line between point P(x1,y1) and Q(x2,y2) is $D = \frac{y2-y1}{x2-x1} \mod p$. Thus the equation for the line will be  $y = D(x-x1) + y1 mod p$. To find the intersection of the line with the curve, we will substitute the value of y to get ${D(x-x1) + y1}^2 = x^3 + ax + b \mod p$. The solution of this equation are x1 , x2 and $x3 = D^2-x1-x2 \mod p$, which gives $y3 = D(x3-x1) + y1 \mod p$. Thus, the pseudo code to get the Point Addition between two points P and Q is given in \ref{algo}. The pseudo code also handles the corner cases where the P or Q is $\cal O$, then the result will the corresponding other point. In case the points are symmetric, the point addition will return $\cal O$. Finally in case of point doubling, the slope with the slope of tangent at that point.

\makeatletter
\def\BState{\State\hskip-\ALG@thistlm}
\makeatother

\begin{algorithm}
\caption{PointAddition Algorithm with input P(x1,y1) and Q(x2,y2) }\label{algo}
\begin{algorithmic}[1]
\Procedure{PointAddition}{}
\If {$P = \cal O$}
return Q
\EndIf
\If {$Q = \cal O$}
return P
\EndIf
\If {$x1 = x2$ and $y1 = -y2$}
return $\cal O$
\EndIf
\If{$x1 = x2$ and $y1 = y2$}
\State $D \gets \frac{3*{x1}^2+a}{2*y1} \mod p$
\Else 
\State $D \gets \frac{y2-y1}{x2-x1} \mod p$
\EndIf
\State $x3 \gets D^2-x1-x2 \mod p$
\State $y3 \gets D(x3-x1) + y1 \mod p$
\State return (x3,-y3)
\EndProcedure
\end{algorithmic}
\end{algorithm}

Interestingly, the point addition is a abelian group operation as it satisfies all four properties for group operation plus commutative property.
\begin{itemize}
\item Closure Property. P+Q lies on the curve for all P and Q, as discussed earlier.
\item Associative Property. (P+Q)+R = P+(Q+R)
\item Inverse Property. $P + P^{-1} = \cal O $ where $P^{-1} = (-x,y)$ for P=(x,y). Also ${\cal O}^{-1} = \cal O$. Since line passing through P and $P^{-1}$ is vertical line, it passes through $\cal O$.
\item Identity Property. P+$\cal O$ = P, as the vertical line intersects the inverse point $P^{-1}$, and vertical line through $P^{-1}$ intersect $P$. Thus point of infinity is identity for the group.
\item Commutative Property. P+Q = Q +P, as the line passing through P and Q will be same in both case.
\end{itemize}

\paragraph{Scalar Multiplication.} Scalar multiplication $nP$ is equivalent to adding P to itself n times. This is analogous to exponentiation in $Z_p$ and can be computed by point double and add algorithm similar to square and multiply in case of integers. We can also pick generator P that defines cyclic subgroup of E such that $0P, 1P, 2P, 3P \dots qP$ all lies on the curve. It should be the case the subgroups are large.

\paragraph{Building Elliptic Curve Groups.} Earlier people use to find suitable curves by a deterministic algorithm that picks a large prime p and pick values a and b to define a curve. Then check using Schoof algorithm the size of group if it is prime for security reasons. Finally pick a point P to be generator. This process was very slow. Recently, these curves and generator P have been standardized instead. Some examples are NIST curves, Curve25519 etc. 








\newpage
%%%%%%%%%%%%%%%%%%%%%%%%%%%%%%%%%%%%%%%%%%%%%%%%%%%%%%%%%%%%%%%%%%%%%%%%%%%%%%%%
\section{Preliminaries and Notation}
\label{sec:notation}

We collect here some notation that we will use throughout these notes. 
\tnote{If you introduce new notation in writing up a lecture, please consider
adding it here.} 


\paragraph{Basics.}
We most often denote sets by calligraphic, capital letters, such as $\calX$,
$\calY$, $\calZ$. A discrete probability distribution is a set $\Omega$, the
event space, together with a
function~$p\Colon\Omega\rightarrow[0,1]$ for which $\sum_{\omega\in\Omega}
p_\Omega(\omega) = 1$.
We write $p_\Omega$ when we need to disambiguate the event space, but
otherwise simply $p$ when $\Omega$ is clear from context. We will also,
following convention, often write $\Pr[\omega]$ instead of $p(\omega)$.
The support of $p$ is defined to be the set $\supp(p) = \{\omega \;|\; \omega
\in \Omega \land p(\omega) > 0\}$, 
i.e., the set of all points in $\Omega$ that have non-zero probability.  
As a slight abuse of notation, we can write 
$\Pr[\calS] = p(\calS) = \sum_{\omega\in\calS} p(\omega)$ for any set $\calS \subseteq \Omega$. 

A random variable $X$ is a map $Y\Colon\Omega\rightarrow\calX$ from some event
space $\Omega$ with associated probability distribution $p_\Omega$ over a  set
$\Omega$. So for some other set $\calS$ we let $\Pr[X\in\calS] = \Pr[\{\omega
\;|\; X(\omega) \in \calS\}]$ to be the probability that the random variable
takes on a value in $\calS$, over the probability distribution $p_\Omega$.

We use the language of probability as the foundation for formalizing
cryptographic algorithms, security, and more. Interestingly the probability
spaces involved get complicated quickly, and a common problem is that they end 
ambiguous. We will therefore rely on a crutch that has proved quite useful for
communicating, and reasoning about, probability distributions.


\paragraph{Pseudocode games.} We fix some pseudocode language to describe
security models, cryptographic algorithms, and more. Roughly we will follow the
notational tradition established by Bellare and Rogaway in the
2000s~\autocite{bellare2006security}, but using slightly different syntax/symantecs that are
based most closely on a treatment from~\autocite{ristenpart2011careful}.  Code-based
games are convenient for more precisely defining probability spaces, which are
what we use to model security and correctness goals, algorithms, and more.  

%We follow~\cite{BR06} 
%with some differences.
We will use procedures, variables, and typical
programming statements (such as operators, loops, procedure calls, etc.).
Typical statements are shown in \figref{fig:statements}.
We do not provide a formal specification of the programming language,
see~\cite[Appendix B]{bellare2006security} for an example of doing so. 
%Games include
%procedures, variables, and typical programming 
%statements (operators, loops, etc.). 
We will rely on some conventions to help make sense of games.
Types should be understable
from context. The names of syntactic objects (procedures, variables, etc.)
must be distinct.  Variables are implicitly 
initialized to default values:  integer variables are set
to 0, arrays are everywhere~$\bot$, etc. Here $\bot$ is a distinguished symbol
by tradition used to denote an error in the cryptographic literature.
By distinguished we mean that it assumed to not be used for any other reason,
not appearing in alphabets over which strings are taken, etc. We will often use $x \getsr \calX$ to denote that variable $x$ is assigned a value that is sampled uniformly at random from set $\calX$. 


\begin{wrapfigure}{r}{3in}
\gamesfontsize
\begin{tabular}{lp{2in}}
\toprule
$x \gets y$ & Assignment\\
$x \getsr \calX$ & Uniform sampling from a set\\
$z \getsr P(x,y)$ & Call a randomized procedure\\
$z \gets P(x,y)$ & Call a deterministic procedure\\
Ret $x$ & Return from a procedure\\
$z \gets x \concat y$ & String concatenation\\
$(x,y) \getparse{n} z$ & Parse string $z$ s.t.~$|x| = n$\\
\bottomrule
\end{tabular}
\caption{Some statements used in our games.}
\label{fig:statements}
\end{wrapfigure}


A \textit{procedure} is a sequence of statements together 
with zero or more variable inputs and zero or more 
outputs. %
%Variables are by default local, meaning they can only be 
%used within a single procedure, but they retain their state 
%between calls to the procedure. %
An \emph{unspecified procedure} is a procedure whose pseudocode, inputs, and
outputs are understood from context.  We will see some examples of
unspecified procedures, the most frequent in our security games being the 
\emph{adversary} which is often left unspecified. 
%
%
A call to a procedure requires providing it with inputs, running its sequence of
statements, and returning its output. We will interchangeably use the term call
and \emph{query} for procedure invocation.
A procedure~$P$ can itself query other
procedures. The set of procedures $Q_1,\ldots,Q_k$ called by a procedure are
statically fixed, and we require that there are no type mismatches in inputs and
outputs. 

Say that the code of~$P$ expects to be able to call~$k$ distinct procedures.
We will write $P^{Q_1,Q_2,\ldots,Q_k}$ to denote that these calls are
handled by $Q_1,Q_2,\ldots,Q_k$ and implicitly assume (for all $i\in[k]$) that there are no
syntactic mismatches between the calls that~$P$ makes to~$Q_i$ and the inputs
of~$Q_i$, as well as between the return values of~$Q_i$
and the return values expected by~$P$.   
%We stress that~$P$ does not
%call~$Q_i$ by name, but rather calls to a procedure that is
%instantiated by~$Q_i$.

We assume that all procedures eventually halt, regardless of randomness used, 
returning their outputs, at which point execution returns to the 
calling procedure.
Two procedures~$P_1$ and~$P_2$ are said to 
\emph{export the same interface} 
if their inputs and outputs have the same number and types. 

Variables will be local by default, meaning they can only be used within a
single procedure. Variables are static, meaning that they retain their 
state between calls to the procedure. It will be convenient to 
allow sharing of variables at times, for which we use a 
\textit{collection of procedures}. This is a set 
of one or more procedures whose variables have scope covering all of the
procedures. We will denote a collection of procedures using 
a common prefix ending with a period, so for example~$(P.x,P.y,\ldots)$,
Sometimes we will use the term interfaces for the specific prefixes of the a
collection of procedures~$P$. 

A \emph{main procedure} is a special procedure that takes no inputs and has some
output.  We will mark it by \main{} (though below
we'll see some syntactic sugar that provides greater brevity).  No procedure may
call~\main{}, it can access all the variables of other specified procedures
(though not unspecified procedures). 

A game consists of a main procedure together with a set of zero or more
specified procedures. We write $\G$ for a game. A game may also make use of
unspecified procedures (such as adversaries), which we enumerate as
superscripts, e.g., $\G^{P_1,P_2,\ldots,P_k}$. In most games used as security
definitions, one (or more) of the unspecified procedures will be called the
adversary, most often denoted by~$\advA$. For a given instantiation of the
unspecified procedures, one can run a game: execute its
statements starting with the designated \main{} procedure, and ultimately
outputing whatever \main{} returns.  

\paragraph{Run times and random variables.} Games can make random choices, due
to the supported statements for sampling according to a distribution. We can
associate to games a model of computation, which specifies how much running each
(type of) statement costs in terms of time, memory, or both.   A typical model
is to assign to each statement the same abstract unit cost, and the run time
then becomes the number of statements executed in the course of the game. 
When procedures are called, we attribute the unit cost of the call statement to
the caller and the remaining cost of executing the procedure's statements to the
callee. 

By default we will require that games terminate in some finite number of
steps~$\runtime$, and clarify explicitly when this does not hold. The number of
queries made by the main procedure, or any other procedure for that matter
including unspecified ones, is
therefore also upper bound by $\runtime$. We  may often limit the number of
queries made by an adversary to some maximum number $\numqueries \le \runtime$. 

Given these finiteness conditions, we similarly know that there is a finite
limit on the number of random samples made in a game. Since we restricted to 
random sampling from finite sets, we have that for any game $\G$ 
there is an event space~$\Omega_\G$ and an associated probability distribution
defining the output of the game $\G$. Given our restrictions, $\Omega_\G$ is a
set of possible values, the cross-product of all the random sampling procedures
within $\G$. We sometimes refer to $\Omega_\G$ as the \emph{coins} of the game.
For some fixed unspecified procedures
$P_1,\ldots,P_k$ we denote the event that executing the game
$\G^{P_1,\ldots,P_k}$ outputs a particular value $y$ by
``$\G^{P_1,\ldots,P_k}\Rightarrow y$'' and the associated probability over
$\Omega_\G$ is denoted $\Pr[\G^{P_1,\ldots,P_k}\Rightarrow y]$. When $y$ is clear
from context we will omit it, writing instead $\Pr[\G^{P_1,\ldots,P_k}]$.  
For example, we will often have games output a boolean and then
$y$ will most often be the value \true.

We can similarly associate to any variable within a game an event within
$\Omega_\G$. These can also be equivalently considered to be random variables on
domain $\Omega_G$. Our convention will be to overload notation, and define the
event that a variable $X$ in a game $\G$ takes on a certain value $y$ as simply
``$X = y$'' with associated probability $\Pr[X = y]$ over the coins of
$\Omega_\G$. 


\paragraph{Runnable games.}  We want to emphasize a point, which is that games
are by our conventions above runnable. That means that, once any unspecified
procedures are fixed, you could write a program in a conventional programming
language, and actually run the game on a real computer. Obviously like with all
abstract algorithms, the actual run times will vary, but assuming relative
efficiency the game will complete.  

It relatively frequently arises in proofs that one deviates from runnable games.
A common example is logic of the following form. Let $\advA$ be an adversary,
and consider a game $\G^\advA$. Remember we implicitly assume that $\advA$ is
compatible with $\G$, and we have that $\G^\advA$ is runnable.  Now consider the
set of all adversaries compatible with $\advA$, and let $\advA_{\max}$ be the
member of this set that maximizes $\Pr[\G^\advA \Rightarrow \true]$. But now
$\G^{\advA_{\max}}$ is no longer runnable, because $\advA_{\max}$ is not
concretely specified. While mathematically $\advA_{\max}$ is well-defined, there
is no clear way to write down its code, even given the code defining~$\advA$.
We will try whenever possible to avoid such arguments, as they have various
subtle implications that are, in general, not great for making 
clear claims about security. 






%\newpage
%%%%%%%%%%%%%%%%%%%%%%%%%%%%%%%%%%%%%%%%%%%%%%%%%%%%%%%%%%%%%%%%%%%%%%%%%%%%%%%%%
\section{Ciphers and Initial Security Notions}
\label{sec:se}

\paragraph{Ciphers.}
We start by defining a cipher. A cipher $\cipher = (\cipherE,\cipherD)$ is
defined by a a pair of deterministic algorithms $\cipherE$ and $\cipherD$.  To
any cipher $\cipher$ we associate sets called the key space $\keyspace$, message
space $\msgspace$, and ciphertext space $\ctxtspace$. We do not surface in the
notation for a cipher these sets, and will require that the association be clear
from context.

The algorithms are two-input. Enciphering takes a key $K \in \keyspace$ and
message $M \in \msgspace$, and outputs a ciphertext $C \in \ctxtspace$. Because
$\cipherE$ is deterministic, we can equally formalize it as a map 
$\cipherE\Colon\keyspace\times\msgspace\rightarrow\ctxtspace$. For a given key
$K$ we let $\cipherE_K\Colon\msgspace\rightarrow\ctxtspace$ be defined by
$\cipherE_K(M) = \cipherE(K,M)$ for all $M \in \msgspace$.  Deciphering takes a key $K\in \keyspace$ and ciphertext
$C \in \ctxtspace$ and outputs a message $M \in \msgspace$. Again, we can view
it as a map $\cipherD\Colon\keyspace\times\ctxtspace\rightarrow\msgspace$. 

Both $\cipherE$ and $\cipherD$ must be efficiently computable for all $K \in
\keyspace$.  (We have not defined efficiently computable, and use the term here
informally.) We require that a cipher be correct, meaning that for
$\cipherD_K(\cipherE_K(M)) = M$ for all $M$. 

\paragraph{Security notions.} What do we intuitively expect of a cipher?
Minimally:
\begin{itemize}
\item Secret key should remain secret
\item Message should remain secret
\end{itemize}
Let's try to formalize these notions.
\begin{itemize}
\item TKR security
\item KR security
\item (ot-)IND security
\end{itemize}


\begin{figure}[p]
\fpage{.45}{
\underline{$\TKR^\advA_\cipher$}\\[1pt]
$K \getsr \keyspace$\\
$K^* \getsr \advA^\Fn$\\
Ret $(K = K^*)$\medskip

\underline{$\Fn(M)$}\\
$C \gets \cipherE_K(M)$\\
Ret $C$
}
\end{figure}

We let $\TKR_\cipher$-advantage of a $\TKR_\cipher$-adversary $\advA$ be defined by 
\bnm
  \AdvTKR{\cipher}{\advA} = \Prob{\TKR^\advA_\cipherE \Rightarrow\true}  \;.
\enm

\begin{figure}[p]
\fpage{.45}{
\underline{$\KR^\advA_\cipher$}\\[1pt]
$K \getsr \keyspace$\\
$K^* \getsr \advA^\Fn$\\
$\win \gets \true$\\ 
For $M \in \calX$:\\
\ind If $\cipherE_{K^*}(M) \ne \cipherE_{K}(M)$ then\\
\ind\ind $\win \gets \false$\\
Ret $\win$\medskip

\underline{$\Fn(M)$}\\
$\calX \gets \calX \cup \{M\}$\\
$C \gets \cipherE_K(M)$\\
Ret $C$
}
\end{figure}

We let $\KR_\cipher$-advantage of a $\KR_\cipher$-adversary $\advA$ be defined by 
\bnm
  \AdvKR{\cipher}{\advA} = \Prob{\KR^\advA_\cipherE \Rightarrow\true}  \;.
\enm


\begin{figure}[p]
\fpage{.45}{
\underline{$\OTIND^\advA_\cipher$}\\[1pt]
$K \getsr \keyspace$\\
$b \getsr \bits$\\
$b' \getsr \advA^\Fn$\\
Ret $(b = b')$\medskip

\underline{$\Fn(M_0,M_1)$}\\
$C \gets \cipherE_K(M_b)$\\
Ret $C$
}
\end{figure}

We let $\OTIND_\cipher$-advantage of a $\OTIND_\cipher$-adversary $\advA$ be defined by 
\bnm
  \AdvOTIND{\cipher}{\advA} = 2\cdotsm\Prob{\OTIND^\advA_\cipherE \Rightarrow\true} - 1  \;.
\enm

\bigskip
\bigskip


\begin{itemize}
\item First example: Our simple OTP cipher is not $\TKR$ secure? Go over example: $\advA$
queries once on arbitrary message, recovers $K$ by computing $M \oplus C$. This
is guaranteed to succeed because $M \oplus C$ uniquely defines $K$. What does
this mean? Isn't OTP considered secure? Shannon said so!
%
\item Second example: Give toy cipher $\cipher$ for which $\TKR_\cipher$ has
$\AdvTKR{\cipher}{\advA} = 0$ for any adversary $\advA$. What is it?
$\cipherE(K,M) = M$. It is correct 
%
\item Third example: Exhaustive key search attack against generic
cipher. Emphasize that lower-bounding the efficacy of this is not possible in
general. Why? Consider toy identity cipher! 
%
\item Discuss the KR definition. Rules out the
identity map as being relevant. Lower-bounding security is 
%
\item Shannon's perfect secrecy (one-time left-or-right indistinguishability). 
\end{itemize}

\begin{theorem}
Let $\cipher$ be a cipher. For any $\TKR_\cipher$-adversary $\advA$, we give a
$\KR_\cipher$-adversary $\advB$ such that 
  $\AdvKR{\cipher}{\advA} = \AdvTKR{\cipher}{\advB}$.
\end{theorem}

In our theorem statements including reductions,  we need to interpret the words
``we give a''. We will focus on concrete, specified reductions. That means that
the adversary $\advB$ not only exists, but is fully specified  --- minus the
details of $\advA$ ---  within the proof.  In particular, if you give someone
$\advA$ then $\advB$ becomes runnable.  Runnable reductions are generally
speaking easier to interpret when it comes to implied security guarantees. They
even allow us to use the human-ignorance model~\cite{rogaway2006formalizing}
which, roughly, states that a reduction even to a mathematically easy assumption
can still be meaningful. (We will revisit this particular issue with an example
in the context of collision resistance.)
An example of a non-runnable $\advB$ would be one that includes some constant
value that we know exists, but don't know an exact value for. This comes up in
various arguments, and can cause problems in interpreting the reduction in terms
of concrete security.  This issue is subtle and we will revisit it.

The takeaway here being that one interprets ``we give a'' to mean runnable
adversaries that are specified fully in the proof. (Or when brevity is at stake,
specified to a leave of detail that the average reader could specify it in
detail easily.)  When we deviate from this convention we should remark on it.




\begin{theorem}
Let $\cipher$ be the OTP cipher. Then for any single-query
$\OTIND_\cipher$-adversary $\advA$ it holds that $\AdvOTIND{\cipher}{\advA} = 0$. \end{theorem}


\begin{theorem}
Let $\cipher$ be a cipher defined 
over $(\keyspace,\msgspace,\ctxtspace)$ such that for any $\OTIND_\cipherE$-adversary 
$\advA$ it holds that $\AdvOTIND{\cipher}{\advA} = 0$. Then $|\keyspace| \ge
|\msgspace|$. 
\end{theorem}


\begin{figure}
\fpage{.45}{
\underline{$\advA_{\textrm{eks}}$}\\[1pt]
$C \gets \Fn(M)$\\
For $K \in \calK$ do:\\
\ind If $C = \cipherE(K,M)$ then\\
\ind\ind Return $K$\\
Return $\bot$
}
\end{figure}

\paragraph{PRPs and PRFs.}

\begin{itemize}
\item Basic cipher definitions, key recovery attacks
\item PRP and PRF definitions
\item PRP/PRF switching lemma
\begin{itemize}
  \item Incorrect conditioning argument 
  \item Correct game-playing argument
\end{itemize}
\item PRP from PRF constructions: Feistel networks
\item Luby-Rackoff proof
\end{itemize}


\begin{figure}
\fpage{.15}{
\underline{$\PRP_{\cipher}^\advA$}\\
$b \getsr \bits$\\
$K \getsr \keyspace$\\
$\pi \getsr \Perm(n)$\\
$b' \getsr \advA^\Fn$\\
Return $(b = b')$\medskip

\underline{$\Fn(M)$}\\
If $b = 1$ then\\
\ind Return $\cipherE_K(M)$\\
Return $\pi(M)$
}
\end{figure}

\bnm
\AdvPRP{\cipher}{\advA} = 2\cdotsm\Prob{\PRP^\advA_\cipher\Rightarrow\true}- 1
\enm

\begin{figure}
\hfpages{.25}{
\underline{$\PRP1_{\cipher}^\advA$}\\
$K \getsr \keyspace$\\
$b' \getsr \advA^\Fn$\\
Return $b'$\medskip

\underline{$\Fn(M)$}\\
Return $\cipherE_K(M)$\\
}{
\underline{$\PRP0_{\cipher}^\advA$}\\
$\pi \getsr \Perm(n)$\\
$b' \getsr \advA^\Fn$\\
Return $b'$\medskip

\underline{$\Fn(M)$}\\
Return $\pi(M)$\\
}


\end{figure}


\begin{figure}
\hfpages{.25}{
\underline{$\PRF1_{\cipher}^\advA$}\\
$K \getsr \keyspace$\\
$b' \getsr \advA^\Fn$\\
Return $b'$\medskip

\underline{$\Fn(M)$}\\
Return $\cipherE_K(M)$\\
}{
\underline{$\PRF0_{\cipher}^\advA$}\\
$\rho \getsr \Func(n,n)$\\
$b' \getsr \advA^\Fn$\\
Return $b'$\medskip

\underline{$\Fn(M)$}\\
Return $\rho(M)$\\
}


\end{figure}



\begin{lemma} Let $\cipher$ be a cipher with ciphertext space $\bits^n$. 
Let $\advA$ be an adversary making at most $q$ queries. Then
\bnm
  \left| \Prob{\PRF0_\cipher^\advA\Rightarrow 1} 
      - \Prob{\PRP0_\cipher^\advA\Rightarrow1} \right| \le \frac{q^2}{2^n}  \;.
\enm
\end{lemma}

\begin{align*}
&\left| \Prob{\PRP0_\cipher^\advA\Rightarrow 1} 
      - \Prob{\PRF0_\cipher^\advA\Rightarrow1} \right| \\ 
     &\myInd\myInd\myInd =  \left|\Prob{\G0} - \Prob{\PRF0_\cipher^\advA\Rightarrow1} \right|  \\
     &\myInd\myInd\myInd  =  \left|\Prob{\G1} - \Prob{\PRF0_\cipher^\advA\Rightarrow1} \right|  \\
     &\myInd\myInd\myInd  \le \left|\Prob{\G2} + \Prob{\bad_2} - \Prob{\PRF0_\cipher^\advA\Rightarrow1} \right|\\
     &\myInd\myInd\myInd  = \Prob{\bad_2}\\
     &\myInd\myInd\myInd  \le \frac{q^2}{2^n}\\
\end{align*}

\begin{lemma} Let $\G$, $\Hgame$ be games that are identical-until-bad and $y$ be any
value. Then
\bnm
  \big| \Prob{\G\Rightarrow y} 
      - \Prob{\Hgame\Rightarrow y} \big| \le \Prob{\Hgame\setsbad} = \Prob{\G\setsbad}  \;.
\enm
\end{lemma}


\begin{figure}
\hfpagess{.20}{.20}{
\underline{$\G0$}\\[2pt]
$b' \getsr \advA^\Fn$\\
Return $b'$\medskip

\underline{$\Fn(M)$}\\
If $\TabF[M] = \bot$ then\\
\ind $\TabF[M] \getsr \bits^n \setminus \TabF$\\
Return $\TabF[M]$
}{
\underline{\fbox{$\G1$} \;\;\; $\G2$}\\[2pt]
$b' \getsr \advA^\Fn$\\
Return $b'$\medskip

\underline{$\Fn(M)$}\\
$C \getsr \bits^n$\\
If $C \in \TabF$ then\\
\ind $\badtrue$\\
\ind \fbox{$C \getsr \bits^n \setminus \TabF$}\\
$\TabF[M] \gets C$\\
Return $\TabF[M]$
}


\end{figure}



\begin{figure}
\hfpagessss{.20}{.20}{.20}{.20}{
\underline{$\G0$}\\[2pt]
$K \getsr \bits^k$\\
$b' \getsr \advA^\Fn$\\
Return $b'$\medskip

\underline{$\Fn(LR)$}\\
$L_1 \gets R$\\
$R_1 \gets L \oplus F_K(\langle 1\rangle \concat R)$\\
$L_2 \gets R_1$\\
$R_2 \gets L_1 \oplus F_K(\langle 2 \rangle \concat R_1)$\\
$L_3 \gets R_2$\\
$R_3 \gets L_2 \oplus F_K(\langle 3 \rangle \concat R_2)$\\
Return $L_3 \concat R_3$
}{
\underline{$\G1$}\\[2pt]
$\rho \getsr \Func(2n,n)$\\
$b' \getsr \advA^\Fn$\\
Return $b'$\medskip

\underline{$\Fn(LR)$}\\
$L_1 \gets R$\\
$R_1 \gets L \oplus \rho(\langle 1\rangle \concat R)$\\
$L_2 \gets R_1$\\
$R_2 \gets L_1 \oplus \rho(\langle 2 \rangle \concat R_1)$\\
$L_3 \gets R_2$\\
$R_3 \gets L_2 \oplus \rho(\langle 3 \rangle \concat R_2)$\\
Return $L_3 \concat R_3$
}{
\underline{$\fbox{\G2}$\;\;\;\G3}\\[2pt]
$b' \getsr \advA^\Fn$\\
Return $b'$\medskip

\underline{$\Fn(LR)$}\\
$L_1 \gets R$\\
If $\TabF[1,R] = \bot$ then\\
\ind $\TabF[1,R] \getsr \bits^n$\\
$R_1 \gets L \oplus \TabF[1,R]$\\
$L_2 \gets R_1$\\
$X_2 \getsr \bits^n$\\
If $\TabF[2,R_1] \ne \bot$ then\\
\ind $\badtrue$\\
\ind \fbox{$X_2 \gets \TabF[2,R_1]$}\\
$\TabF[2,R_1] \gets X_2$\\
$R_2 \gets L_1 \oplus X_2$\\
$L_3 \gets R_2$\\
$X_3 \getsr \bits^n$\\
If $\TabF[3,R_2] \ne \bot$ then\\
\ind $\badtrue$\\
\ind \fbox{$X_3 \gets \TabF[2,R_2]$}\\
$\TabF[3,R_2] \gets X_3$\\
$R_3 \gets L_2 \oplus X_3$\\
Return $L_3 \concat R_3$
}{
\underline{$\G4$}\\[2pt]
$b' \getsr \advA^\Fn$\\
Return $b'$\medskip

\underline{$\Fn(LR)$}\\
$L_1 \gets R$\\
If $\TabF[1,R] = \bot$ then\\
\ind $\TabF[1,R] \getsr \bits^n$\\
$R_1 \gets L \oplus \TabF[1,R]$\\
$L_2 \gets R_1$\\
If $\TabF[2,R_1] \ne \bot$ then\\
\ind $\badtrue$\\
$\TabF[2,R_1] \gets 1$\\
$R_2 \getsr \bits^n$\\
$L_3 \gets R_2$\\
%$X_3 \getsr \bits^n$\\
If $\TabF[3,R_2] \ne \bot$ then\\
\ind $\badtrue$\\
$\TabF[3,R_2] \gets 1$\\
$R_3 \getsr \bits^n$\\
Return $L_3 \concat R_3$
}
\end{figure}

\begin{figure}
\fpage{.25}{
\underline{$\advB^\Fn$}\\[2pt]
$K \getsr \bits^k$\\
$b' \getsr \advA^\FnSim$\\
Return $b'$\medskip

\underline{$\FnSim(LR)$}\\
$L_1 \gets R$\\
$R_1 \gets L \oplus \Fn(\langle 1\rangle \concat R)$\\
$L_2 \gets R_1$\\
$R_2 \gets L_1 \oplus \Fn(\langle 2 \rangle \concat R_1)$\\
$L_3 \gets R_2$\\
$R_3 \gets L_2 \oplus \Fn(\langle 3 \rangle \concat R_2)$\\
Return $L_3 \concat R_3$
}
\end{figure}

\newpage

\begin{theorem}
Let $\Feistel$ be the 3-round Feistel cipher using round function 
$F\Colon\bits^k\times\bits^n\rightarrow \bits^n$. For any
$\PRP_\cipher$-adversary $\advA$ making at most $q$ queries 
we give an $\PRF_F$-adversary $\advB$ making at most $3q$ queries such that
\bnm
  \AdvPRP{\Feistel}{\advA} \le \AdvPRF{F}{\advB} + \frac{2q^2}{2^n} +
  \frac{q^2}{2^{2n}} \;.
\enm
\end{theorem}



\begin{align*}
\AdvPRP{\cipher}{\advA} 
    &= \left|\Prob{\PRP1^\advA_\cipher} - \Prob{\PRP0^\advA_\cipher}\right|\\
    &= \left|\Prob{\G0} - \Prob{\PRP0^\advA_\cipher}\right|\\
    &\le \left|\Prob{\G1} + \AdvPRF{F}{\advB} - \Prob{\PRP0^\advA_\cipher}\right|\\
    &=   \left|\Prob{\G2} + \AdvPRF{F}{\advB} - \Prob{\PRP0^\advA_\cipher}\right|\\
    &\le \left|\Prob{\G3} + \Prob{\bad_3} + \AdvPRF{F}{\advB} - \Prob{\PRP0^\advA_\cipher}\right|\\
    &= \left|\Prob{\G4} + \Prob{\bad_4} + \AdvPRF{F}{\advB} - \Prob{\PRP0^\advA_\cipher}\right|\\
    &\le \left|\Prob{\PRP0^\advA_\cipher} + \frac{q^2}{2^{2n}} + \Prob{\bad_4} + \AdvPRF{F}{\advB} - \Prob{\PRP0^\advA_\cipher}\right|\\
    &= \frac{q^2}{2^{2n}} + \Prob{\bad_4} + \AdvPRF{F}{\advB}\\
    &\le \frac{q^2}{2^{2n}} + \frac{2q^2}{2^n} + \AdvPRF{F}{\advB}\\
\end{align*}


\newpage
%%%%%%%%%%%%%%%%%%%%%%%%%%%%%%%%%%%%%%%%%%%%%%%%%%%%%%%%%%%%%%%%%%%%%%%%%%%%%%%%
\section{Ciphers and Initial Security Notions}
\label{sec:se}

\paragraph{Ciphers.}
We start by defining a cipher. A cipher $\cipher = (\cipherE,\cipherD)$ is
defined by a pair of deterministic algorithms $\cipherE$ and $\cipherD$.  To
any cipher $\cipher$ we associate sets called the key space $\keyspace$, message
space $\msgspace$, and ciphertext space $\ctxtspace$. We do not surface these sets in the
notation for a cipher, and we will require that the association be clear
from context.

Each algorithm has two inputs. Enciphering takes a key $K \in \keyspace$ and
message $M \in \msgspace$, and outputs a ciphertext $C \in \ctxtspace$. Because
$\cipherE$ is deterministic, we can equally formalize it as a map 
$\cipherE\Colon\keyspace\times\msgspace\rightarrow\ctxtspace$. For a given key
$K$ we let $\cipherE_K\Colon\msgspace\rightarrow\ctxtspace$ be defined by
$\cipherE_K(M) = \cipherE(K,M)$ for all $M \in \msgspace$.  Deciphering takes a key $K\in \keyspace$ and ciphertext
$C \in \ctxtspace$ and outputs a message $M \in \msgspace$. Again, we can view
it as a map $\cipherD\Colon\keyspace\times\ctxtspace\rightarrow\msgspace$. 

Both $\cipherE$ and $\cipherD$ must be efficiently computable for all $K \in
\keyspace$.  (We have not defined efficiently computable and use the term here
informally.) We require that a cipher be correct, meaning that
$\forall M \in \msgspace, \forall K \in \keyspace$, it holds that $\cipherD_K(\cipherE_K(M)) = M$.

\begin{example} One simple example of a cipher is the \textbf{one-time pad} (OTP). Let $\keyspace = \msgspace=\ctxtspace=\{0,1\}^n$ for some $n \in \N$. Then for any key $K\getsr\keyspace$ and for any message $M \in \msgspace$, we define the OTP as follows:
\begin{equation*}
\cipherE_K(M) = M \oplus K \qquad
\cipherD_K(C) = C \oplus K
\end{equation*}
Claude Shannon proved that the OTP is perfectly secure for one-time use in 1949 \cite{shannon1949communication}. Intuitively, we expect a cipher to be \textbf{perfectly secure} if a ciphertext provides no information about the message.
Formally, a cipher is perfectly secure if for any $M_1, M_2 \in \msgspace$, any $K \in \keyspace$, and any $C \in \ctxtspace$, 
\begin{equation*}
\Prob{K \getsr \keyspace: \cipherE_{K}(M_1) = C} = \Prob{K \getsr \keyspace: \cipherE_{K}(M_2) = C}.
\end{equation*}
Notice that although the OTP is perfectly secure, it is not practical since the key must be as large as the message. 
\end{example}

\paragraph{Security notions.} What do we intuitively expect of a cipher?
Minimally:
\begin{itemize}
\item The secret key should remain secret.
\item The message should remain secret.
\end{itemize}

\begin{wrapfigure}[13]{r}{0pt}
	\fpage{.15}{
		\underline{$\TKR^\advA_\cipher$}\\[1pt]
		$K \getsr \keyspace$\\
		$K^* \getsr \advA^\Fn$\\
		Return $(K = K^*)$\medskip
		
		\underline{$\Fn(M)$}\\
		$C \gets \cipherE_K(M)$\\
		Return $C$
	}
	\caption{The target key recovery game.}
	\label{fig:tkr}
\end{wrapfigure}

Let's try to formalize these ideas. We will start with a security notion called \textbf{target key recovery security} (TKR).
As the name suggests, the goal of the adversary is to recover the challenge key given a chosen plaintext attack, meaning the adversary can choose which messages for which it will receive the corresponding ciphertexts. In the $\TKR_\cipher$ game, the adversary makes the queries for these messages to what is called the $\Fn$ oracle. On input $M$, the $\Fn$ oracle simply returns the corresponding ciphertext $C \gets \cipherE_K(M)$, thereby hiding information about key $K$ from the adversary. 
 The game pseudocode is provided in \figref{fig:tkr}. We let $\TKR_\cipher$-advantage of a $\TKR_\cipher$-adversary $\advA$ be defined by 
\bnm
\AdvTKR{\cipher}{\advA} = \Prob{\TKR^\advA_\cipherE \Rightarrow\true}  \;.
\enm

We must now ask ourselves how well TKR captures the security of a cipher. Let us first try to analyze the security of the OTP using this definition. Notice that the OTP actually fails to provide TKR security, which we show with the following adversary.
\begin{center}
	\fpage{.15}{
		\underline{\textbf{adversary }$\advA$}\\[2pt]
		$K \gets \Fn(0^n)$ \\
		Return $K$
	}
\end{center}	

$\advA$ simply queries for $0^n$, which returns $0^n \oplus K = K$, thereby recovering the challenge key with $\AdvTKR{\cipher}{\advA} = 1$. This then means that the OTP is actually insecure according to the TKR security definition. But as we noted earlier, the OTP is considered perfectly secret! Our definition then fails to capture the goal of perfect secrecy. \pfreadernote{say something about implications, 2-time security}

Now consider the identity cipher $\cipherE_{K}(M) = M$ for $\keyspace=\{0,1\}^k$. Since the cipher simply returns the message, no information about the key is included in the ciphertext. The best an adversary can do is return a random key, which has probability $2^{-k}$ of being the correct target key.
Then for any adversary $\advA$, it holds that $\AdvTKR{\cipher}{\advA} = 2^{-k}$, meaning the identity cipher is ``secure''. However, the identity cipher clearly does not provide message confidentiality, and thus our TKR notion says nothing about this property.

\begin{wrapfigure}[17]{r}{0pt}
	\fpage{.25}{
		\underline{$\KR^\advA_\cipher$}\\[1pt]
		$\win \gets \false$\\
		$K \getsr \keyspace$\\
		$K^* \getsr \advA^\Fn$\\ 
		For $M \in \calX$:\\
		\ind If $\cipherE_{K^*}(M) \ne \cipherE_{K}(M)$ then\\
		\ind\ind $\win \gets \false$\\
		Return $\win$\medskip
		
		\underline{$\Fn(M)$}\\
		$\win \gets \true$ \\
		$\calX \gets \calX \cup \{M\}$\\
		$C \gets \cipherE_K(M)$\\
		Return $C$
	}
	\caption{The key recovery game.}
	\label{fig:kr}
\end{wrapfigure}

Furthermore, this security notion is ``unfair'' to an adversary, since there can be many keys that are \textit{consistent} on a query transcript. We then look at a different notion called \textbf{key recovery security} (KR). Under this definition, if an adversary outputs a key that is consistent with the query transcript, then it wins. The game pseudocode is provided in \figref{fig:kr}. Notice that if an adversary makes no queries to the $\Fn$ oracle, then $\KR_\cipherE$ will return $\false$.
We let $\KR_\cipher$-advantage of a $\KR_\cipher$-adversary $\advA$ be defined by 
\bnm
\AdvKR{\cipher}{\advA} = \Prob{\KR^\advA_\cipherE \Rightarrow\true}  \;.
\enm

How does KR compare to TKR? We now look at how to formally compare security definitions to gain an understanding of the relationship between TKR and KR. 

\paragraph{Comparing security definitions.} To show that some definition DEF1 does not imply another definition DEF2, we can show a \textit{counter-example}. This requires producing a scheme such that we can show that no (reasonable) DEF1-adversary has a good advantage. We then give a DEF2-adversary that does maintain a good DEF2 advantage (perhaps under some assumption, such as the existence of a DEF2-secure scheme). 

Conversely, to show that DEF1 does imply DEF2, we can show a \textit{reduction}. This requires converting a DEF2-adversary $\advA$ into a DEF1-adversary $\advB$ such that $\advB$'s DEF1 advantage upper bounds $\advA$'s DEF2 advantage. Moreover, the resources, such as number of queries and running time, of $\advB$ should be close to that of $\advA$.

\begin{example}
TKR $\not \Rightarrow$ KR \\
To show this, we need a counter-example, and in this case we can use the identity cipher $\cipherE_{K}(M) = M$ for $\keyspace=\{0,1\}^k$ and $\msgspace=\{0,1\}^n$. 
We have already discussed that any adversary cannot get a TKR advantage greater than $2^{-k}$. Next we must provide a KR-adversary that achieves a ``good'' advantage. We construct a KR-adversary $\advA_{KR}$ that simply returns $0^k$. Since $\forall M\in\msgspace, \forall K, K^* \in \keyspace, \cipherE_{K}(M) = \cipherE_{K^*}(M) = M$, $\advA_{KR}$ achieves advantage 1. Thus, we have shown that TKR $\not \Rightarrow$ KR. 
\end{example}

We now ask whether KR $\Rightarrow$ TKR and prove this with the following theorem. 

\begin{theorem}
	\label{thm-tkr-kr}
	Let $\cipher$ be a cipher with message space $\msgspace$. For any $\TKR_\cipher$-adversary $\advA$, we give a
	$\KR_\cipher$-adversary $\advB$ such that 
	$\AdvTKR{\cipher}{\advA} = \AdvKR{\cipher}{\advB}$, where the runtime and query usage of $\advB$ is the same as that of $\advA$. 
\end{theorem}

%[\thref{thm-tkr-kr}]

\begin{proof}
	We are given an adversary $\advA$ that wins the $\TKR_\cipher$ game with advantage $\AdvTKR{\cipher}{\advA}$, meaning that $\advA$ will return the target key chosen by the game with probability $\AdvTKR{\cipher}{\advA}$. We now want to construct an adversary $\advB$ that wins the $\KR_\cipher$ game and do so with the following.
	\begin{center}
	\fpage{.23}{
		\underline{\textbf{adversary } $\advB^\Fn$} \\
		$K^* \getsr \advA^\Fn$ \\
		If $\advA$ made no queries then \\
		\ind $M \getsr \msgspace$ ; $C \gets \Fn(M)$ \\
		Return $K^*$	
	}
	\end{center}

	$\advB$ is given access to its oracle $\Fn$, and it runs $\advA$ using this same oracle. Notice that $\advB$ can simply provide $\advA$ its own oracle because the distribution of outputs for oracle $\Fn$ in game $\TKR_\cipher$ and for oracle $\Fn$ in game $\KR_\cipher$ are equivalent. This is known as an \textit{elementary wrapper}: a reduction that runs an adversary once and simulates an oracle in a simple and efficient way \cite{BonehShoupBook}.
	
	Now $\advA$ returns the precise target key chosen by the game, so $\advB$ returns this key because the target key will always be consistent with all queries. In the case that $\advA$ makes no queries to oracle $\Fn$, $\advB$ would automatically lose the game, so it makes a query to $\Fn$ with a randomly chosen message. $\AdvKR{\cipher}{\advB}$ is then the probability that the key returned by $\advA$ is the target key, which is $\AdvTKR{\cipher}{\advA}$.
\end{proof}


In our theorem statements including reductions,  we need to interpret the words
``we give a''. We will focus on concrete, specified reductions. That means that
the adversary $\advB$ not only exists, but is fully specified  --- minus the
details of $\advA$ ---  within the proof.  In particular, if you give someone
$\advA$ then $\advB$ becomes a runnable adversary.  Runnable reductions are generally
speaking easier to interpret when it comes to implied security guarantees. They
even allow us to use the human-ignorance model~\cite{rogaway2006formalizing}
which, roughly, states that a reduction even to a mathematically easy assumption
can still be meaningful. (We will revisit this particular issue with an example
in the context of collision resistance.)
An example of a non-runnable $\advB$ would be one that includes some constant
value that we know exists, but don't know an exact value for. This comes up in
various arguments, and can cause problems in interpreting the reduction in terms
of concrete security.  This issue is subtle and we will revisit it.

The takeaway here is that one interprets ``we give a'' to mean runnable
adversaries that are specified fully in the proof. (Or when brevity is at stake,
specified to a level of detail that the average reader could specify it in
detail easily.)  When we deviate from this convention we should remark on it.



\paragraph{Exhaustive key search.} We now ask whether we can lower-bound (T)KR security in general. We do this by providing a \textit{generic} attack, or an attack that works against any cipher. One such generic attack is the exhaustive key search attack. The pseudocode is shown in \figref{fig:eks}. At a high level, this attack simply chooses a message at random, queries the $\Fn$ oracle to get the corresponding ciphertext, and then brute-force searches through the entire keyspace until it finds the key that outputs the correct ciphertext.  

\begin{wrapfigure}[7]{l}{0pt}
	\fpage{.22}{
		\underline{$\advA^\eks_\Fn$}\\[1pt]
		$M \getsr \msgspace$ \\
		$C \gets \Fn(M)$ \\
		For $K^* \in \keyspace$ do: \\
		\ind If $C = \cipherE(K^*, M)$ then \\
		\ind\ind Return $K^*$ \\
		Return $\bot$
	}
	\caption{The exhaustive key search attack.}
	\label{fig:eks}
\end{wrapfigure} 

We know that $\AdvKR{\cipher}{\advA_\eks} = 1$ since a consistent key is guaranteed to exist. However, notice that finding $\AdvTKR{\cipher}{\advA_\eks}$ is trickier: $\advA_\eks$ might return a consistent key that is not necessarily the target key. For instance, this attack would not work on the identity map cipher since every key is consistent. For ``real'' ciphers, we expect this to be close to 1. 
The worst-case running time for this attack is $|\keyspace|$, while the expected running time is $|\keyspace|/2$.

\paragraph{Computational security.} Computational security presents a large paradigm shift from previous notions. It focuses on computationally-bound adversaries. For instance, an exhaustive key search attack is clearly not computationally efficient for large key spaces and is thus not considered a threat in practice. We measure computational costs by assuming abstract unit costs of (most) operations. This is course-grained but useful for our purposes. 

We have shown previously that TKR is not a good security notion, but we now ask whether KR is an improved definition. In particular, the identity cipher has been shown to be secure under TKR security, yet it is insecure under KR security, which is clearly an improvement. However, KR does not imply message confidentiality: it is a necessary but not sufficient goal. We now move on to very different notions of security for ciphers.

\paragraph{PRF security.} A standard goal for cipher security is security in the sense of pseudorandom permutations and/or pseudorandom functions. For simplicity, we will focus on \textbf{block ciphers}, which for keyspace $\keyspace=\bits^k$ and message space $\msgspace=\bits^n$ are defined by $\cipherE : \bits^k \times \{0,1\}^n \to \bits^n$. Let $\Perm(n)$ be the set of all permutations on $n$ bits. Notice that since $|\bits^n| = 2^n$, then $|\Perm(n)| = 2^n!$. Let $\Func(n,n)$ be the set of all functions from $\bits^n \to \bits^n$. Note that $|\Func(n,n)| = (2^n)^{2^n}$.

We now define a \textbf{pseudorandom function} (PRF) as a function that is indistinguishable from a random function (RF). At a high level this means that the input-output behavior of some block cipher $\cipherE_K$ ``looks like'' the input-output behavior of a random function assuming key $K$ is kept secret. There are two games defined for PRF security: PRF1 and PRF0. The pseudocode for both is provided in \figref{fig:prf}. In PRF1, the adversary has access to an $\Fn$ oracle that returns the output from the block cipher $\cipherE_K$. However, in PRF0 the adversary instead receives the output from a random function $\rho$ when it queries the $\Fn$ oracle. The adversary $\advA$ does not know in which game it is playing and must query the $\Fn$ oracle to distinguish between $\cipherE_K$ and $\rho$. Adversary $\advA$ returns a bit signifying which game it believes it is in. The PRF advantage for $\advA$ is defined as 
\begin{equation*}
\AdvPRF{\cipher}{\advA} = \left| \Prob{\PRF1_\cipher^\advA\Rightarrow 1} 
- \Prob{\PRF0_\cipher^\advA\Rightarrow1} \right|.
\end{equation*}

The adversary $\advA$ wins if the probability that $\advA$ outputs 1 in game $\PRF1_\cipher^\advA$ is far greater than the probability that it outputs 1 in game $\PRF0_\cipher^\advA$. In particular, notice that if $\advA$ simply always output 1, $\AdvPRF{\cipher}{\advA}$ would be 0 as expected, since $\advA$ did not successfully distinguish $\cipher$ from a random function. 

\begin{figure}
	\centering
	\hfpages{.15}{
		\underline{$\PRF1_{\cipher}^\advA$}\\
		$K \getsr \keyspace$\\
		$b' \getsr \advA^\Fn$\\
		Return $b'$\medskip
		
		\underline{$\Fn(M)$}\\
		Return $\cipherE_K(M)$
	}{
		\underline{$\PRF0_{\cipher}^\advA$}\\
		$\rho \getsr \Func(n,n)$\\
		$b' \getsr \advA^\Fn$\\
		Return $b'$\medskip
		
		\underline{$\Fn(M)$}\\
		Return $\rho(M)$
	}	
\caption{The PRF security games.}
\label{fig:prf}
\end{figure} 

Just as we provided a generic attack for TKR security using the exhaustive key search attack, is there a generic distinguishing attack for any cipher? One interesting observation is that for a given key, a block cipher $\cipherE_K$ is a permutation, meaning that two different inputs cannot produce the same output. (If this were not the case, decryption would be impossible.) However, a random function simply chooses outputs at random, so it is entirely possible for two different inputs to produce the same output. The probability of choosing $q$ values at random from $\{0,1\}^n$ and for two of these values to be the same is approximately $\frac{q^2}{2^n}$. This is colloquially known as the \textbf{birthday paradox}, since it implies that the number of people expected to produce two individuals with the same birthday is far fewer than what one might expect. 

If $\advA$ were to query its $\Fn$ oracle enough times, eventually the probability that a repeat value is produced would be large enough and $\advA$ could then check to see if such a repeat value exists. If no repeat occurs, then $\advA$ can assume it is in game $\PRF1_{\cipher}$; otherwise, $\advA$ must be in game $\PRF0_{\cipher}$. This attack is called the \textbf{birthday attack}. The pseudocode for this adversary is defined below.
 
\begin{center}
\fpage{.35}{
	\underline{\textbf{adversary} $\advA^\Fn_{\text{bday}}$} \\
	Let $M_1, M_2, \cdots, M_q \gets \{0,1\}^n$ be distinct \\
	For $i=1, \cdots, q$ do $C_i \gets \Fn(M_i)$ \\
	If $C_1, \cdots, C_q$ are all distinct then return 1 \\
	Else return 0	
}
\end{center}

In addition to creating $q$ messages and querying $\Fn$ $q$ times, $\advA_{\text{bday}}$ must check to see if there are any duplicates in its answered queries. Overall, $\advA_{\text{bday}}$ will then have running time of $\calO(q)$.
To bound the advantage of $\advA_{\text{bday}}$, first notice that in game $\PRF1_{\cipher}$, $\Fn$ outputs the value from $\cipher$, so all output values will be distinct. Thus, $\advA_{\text{bday}}$ will always return 1 and $\Prob{\PRF1_\cipher\Rightarrow 1}=1$. The probability that $\advA_{\text{bday}}$ returns 1 in game $\PRF0_{\cipher}$ is trickier to bound. To do so, we first define $C(N,q)$ as the probability that in the event of choosing $q$ values uniformly at random from set $\{1,\cdots,N\}$, not all of the values chosen are distinct. In game $\PRF0_{\cipher}$, $\Fn$ returns the output from a random function and since $M_1, \cdots, M_q$ are all distinct, then $C_1, \cdots, C_q$ are independently distributed, random values from $\bits$. The probability of all these values being distinct is the probability that there does not exist a collision, which is $1-C(N,q)$, where $N=2^n$. $C(N,q)$ is the birthday bound, so it can be lower-bounded by $\frac{q^2}{2^n}$. 
The PRF-advantage of $\advA^\Fn_{\text{bday}}$ is then defined as 
\begin{align*}
\AdvPRF{\cipher}{\advA_\text{bday}} &= \left| \Prob{\PRF1_\cipher^{\advA_\text{bday}}\Rightarrow 1} 
- \Prob{\PRF0_\cipher^{\advA_\text{bday}}\Rightarrow1} \right| \\
&= 1 - (1 - C(N,q)) \\ 
&= C(N,q) \\ 
&\geq \frac{q^2}{2^n}.
\end{align*}

In order for $\advA_{\text{bday}}$ to have a high probability of succeeding, we expect $q \approx 2^{n/2}$. This means $\advA_{\text{bday}}$ will also have running time of about $2^{n/2}$. For large values of $n$, this becomes an impractical attack. 


\paragraph{PRP security.} We define a \textbf{pseudorandom permutation} (PRP) as a function that is indistinguishable from a random permutation (RP). The games for PRP security are provided in \figref{fig:prp}. These games work similarly to the PRF games but now utilize a random permutation in $\PRP0_\cipher^\advA$ rather than a random function. The PRP advantage for $\advA$ is defined as 
\begin{equation*}
\AdvPRP{\cipher}{\advA} = \left| \Prob{\PRP1_\cipher^\advA\Rightarrow 1} 
- \Prob{\PRP0_\cipher^\advA\Rightarrow1} \right|.
\end{equation*}

\begin{figure}
	\centering
	\hfpages{.15}{
		\underline{$\PRP1_{\cipher}^\advA$}\\
		$K \getsr \keyspace$\\
		$b' \getsr \advA^\Fn$\\
		Return $b'$\medskip
		
		\underline{$\Fn(M)$}\\
		Return $\cipherE_K(M)$
	}{
		\underline{$\PRP0_{\cipher}^\advA$}\\
		$\pi \getsr \Perm(n)$\\
		$b' \getsr \advA^\Fn$\\
		Return $b'$\medskip
		
		\underline{$\Fn(M)$}\\
		Return $\pi(M)$
	}
\caption{The PRP security games.}
\label{fig:prp}	
\end{figure}

Considering that these notions are similar, can we relate them to each other? Intuitively, there is no difference between a random function and a random permutation when observing only a few input-output pairs, as long as the ciphertext space $n$ is sufficiently large. We formalize this intuition with the following lemma. 

\begin{lem}[PRF-PRP Switching Lemma]
	\label{switching-lem}
	Let $\cipher$ be a cipher with ciphertext space $\bits^n$. 
	Let $\advA$ be an adversary making at most $q$ queries. Then
	\bnm
	\left| \Prob{\PRF0_\cipher^\advA\Rightarrow 1} 
	- \Prob{\PRP0_\cipher^\advA\Rightarrow1} \right| \le \frac{q^2}{2^n}  \;.
	\enm
\end{lem}

The intuition for the following proof is that if you have oracle access to a random function or a random permutation, then you need to make enough queries to witness a collision, as determined by the birthday bound. 

One's first instinct might be to bound the difference using a conditioning argument. For instance, let $\mathsf{Dist}$ be the event that in game $\PRF0_{\cipher}^\advA$ all values returned from oracle $\Fn$ are distinct. Then one might say that $\Prob{\PRP1_\cipher^\advA\Rightarrow 1} =   \Prob{\PRP0_\cipher^\advA\Rightarrow1 | \mathsf{Dist}}$. However, this is incorrect and in fact $\Prob{\PRP1_\cipher^\advA\Rightarrow 1} \neq   \Prob{\PRP0_\cipher^\advA\Rightarrow1 | \mathsf{Dist}}$. Refer to \cite{bellare2006multi} for further technical details. 

To correctly prove this, we will instead use a game-playing argument. We first provide the following definitions and lemma. 

\begin{definition}
	A \textbf{flag} is a variable in a pseudocode game that is set upon the occurrence of some event in the game. 
\end{definition}

Typically, games will utilize a flag called \textit{bad} which is set to $\true$.  

\begin{definition}
	Games $\G1$ and $\G2$ are \textbf{identical-until-bad} if they both contain a flag $\bad$  and their code differs only in statements following the setting of $\bad$ to $\true$. 
\end{definition} 

\begin{lem}[Fundamental Lemma of game playing \cite{bellare2006security}]
	Let $\G$, $\Hgame$ be games that are identical-until-bad and let $y$ be any
	value. Then
	\bnm
	\big| \Prob{\G\Rightarrow y} 
	- \Prob{\Hgame\Rightarrow y} \big| \le \Prob{\Hgame\setsbad} = \Prob{\G\setsbad}  \;.
	\enm
\end{lem}

Stated another way, this lemma tells us about the advantage an adversary gains in distinguishing a pair of identical-until-bad games, $\G$ and $\Hgame$. Since $\G$ and $\Hgame$ only differ upon the setting of flag $\bad$, then intuitively we can see that the advantage of an adversary to distinguish between these games must be at most the probability that $\bad$ is set during its execution.

The overall technique we will implement here (and generally in game-playing arguments) is to create a chain of games that are identical-until-bad. We can then invoke the Fundamental Lemma of game playing to upper-bound the adversary's advantage by the probability that $\bad$ gets set in either game. We can slowly modify the games in ways that change the probability of $\bad$ being set until we reach some terminal game where we can utilize conventional probabilistic methods to bound the probability of setting $\bad$. Note that the following proof introduces a $\bad$ flag that can be immediately bounded by traditional means without modification, although future proofs we will see will require further transformations. 

\begin{proof}[Proof of \lemref{switching-lem}]
	 We define the games in \figref{fig:switching}. Notice that the output from game $\G0$ has an identical distribution to that of game $\PRP0_\cipher^\advA$. The only difference between them is that game $\PRP0_\cipher^\advA$ chooses a random permutation and returns the output from that, while game $\G0$ chooses unique values at random as $\advA$ makes queries to $\Fn$. Then $\Prob{\PRP0_\cipher^\advA\Rightarrow1} = \Prob{\G0}$. Game $\G1$ includes the boxed statement and also has an identical output distribution to that of game $\G0$. It simply chooses an output value at random, and if it detects a repeat it then chooses another unique value. In the case of a repeat, it also sets the $\bad$ flag to $\true$. We then have that $\Prob{\G1} = \Prob{\G0}$. We next transition to game $\G2$ and notice that $\G1$ and $\G2$ are \textbf{identical-until-bad}. The Fundamental Lemma of game playing then states that $\Prob{\G1} \leq \Prob{\G2} + \Prob{\bad \text{ set to } \true}$. Game $\G2$ returns values chosen at random and allows for repeat values, so it has an identical output distribution to $\PRF0_\cipher^\advA$. This means $\Prob{\PRF0_\cipher^\advA\Rightarrow1} = \Prob{\G2}$. Finally, the probability that $\bad$ is set to $\true$ is the probability that a random value is chosen by $\Fn$ such that it is not distinct, which is bounded by the birthday bound. We then have the following:
	 
	 \begin{figure}
	 	\centering
	 	\hfpagess{.20}{.20}{
	 		\underline{$\G0$}\\[2pt]
	 		$b' \getsr \advA^\Fn$\\
	 		Return $b'$\medskip
	 		
	 		\underline{$\Fn(M)$}\\
	 		If $\TabF[M] = \bot$ then\\
	 		\ind $\TabF[M] \getsr \bits^n \setminus \TabF$\\
	 		Return $\TabF[M]$
	 	}{
	 		\underline{\fbox{$\G1$} \;\;\; $\G2$}\\[2pt]
	 		$b' \getsr \advA^\Fn$\\
	 		Return $b'$\medskip
	 		
	 		\underline{$\Fn(M)$}\\
	 		$C \getsr \bits^n$\\
	 		If $C \in \TabF$ then\\
	 		\ind $\badtrue$\\
	 		\ind \fbox{$C \getsr \bits^n \setminus \TabF$}\\
	 		$\TabF[M] \gets C$\\
	 		Return $\TabF[M]$
	 	}
	 	\caption{The games for the proof of the PRF-PRP Switching Lemma (\lemref{switching-lem}).}
	 	\label{fig:switching}	
	 \end{figure} 
	 
	 \begin{align*}
	 \left| \Prob{\PRP0_\cipher^\advA\Rightarrow 1} 
	 - \Prob{\PRF0_\cipher^\advA\Rightarrow1} \right|  
	 &=  \left|\Prob{\G0} - \Prob{\PRF0_\cipher^\advA\Rightarrow1} \right|  \\
	 &=  \left|\Prob{\G1} - \Prob{\PRF0_\cipher^\advA\Rightarrow1} \right|  \\
	 &\le \left|\Prob{\G2} + \Prob{\bad \text{ set to } \true} - \Prob{\PRF0_\cipher^\advA\Rightarrow1} \right|\\
	 &= \Prob{\bad \text{ set to } \true}\\
	 &\le \frac{q^2}{2^n} \\
	 \end{align*} 
\end{proof}
\newpage
\section{PRF and PRP Security}
\label{sec:prf}

A standard goal for cipher security is security in the sense of pseudorandom permutations and/or pseudorandom functions. For simplicity, we will focus on \textbf{block ciphers}, which for keyspace $\keyspace=\bits^k$ and message space $\msgspace=\bits^n$ are defined by $\cipherE : \bits^k \times \{0,1\}^n \to \bits^n$. Let $\Perm(n)$ be the set of all permutations on $n$ bits. Notice that since $|\bits^n| = 2^n$, then $|\Perm(n)| = 2^n!$. Let $\Func(n,n)$ be the set of all functions from $\bits^n \to \bits^n$. Note that $|\Func(n,n)| = (2^n)^{2^n}$.

\subsection{PRF security}

We now define a \textbf{pseudorandom function} (PRF) as a function that is indistinguishable from a random function (RF). At a high level this means that the input-output behavior of some block cipher $\cipherE_K$ ``looks like'' the input-output behavior of a random function assuming key $K$ is kept secret. There are two games defined for PRF security: PRF1 and PRF0. The pseudocode for both is provided in \figref{fig:prf}. In PRF1, the adversary has access to an $\Fn$ oracle that returns the output from the block cipher $\cipherE_K$. However, in PRF0 the adversary instead receives the output from a random function $\rho$ when it queries the $\Fn$ oracle. The adversary $\advA$ does not know in which game it is playing and must query the $\Fn$ oracle to distinguish between $\cipherE_K$ and $\rho$. Adversary $\advA$ returns a bit signifying which game it believes it is in. The PRF advantage for $\advA$ is defined as 
\begin{equation*}
\AdvPRF{\cipher}{\advA} = \left| \Prob{\PRF1_\cipher^\advA\Rightarrow 1} 
- \Prob{\PRF0_\cipher^\advA\Rightarrow1} \right|.
\end{equation*}

The adversary $\advA$ wins if the probability that $\advA$ outputs 1 in game $\PRF1_\cipher^\advA$ is far greater than the probability that it outputs 1 in game $\PRF0_\cipher^\advA$. In particular, notice that if $\advA$ simply always output 1, $\AdvPRF{\cipher}{\advA}$ would be 0 as expected, since $\advA$ did not successfully distinguish $\cipher$ from a random function. 

\begin{figure}
	\centering
	\hfpages{.15}{
		\underline{$\PRF1_{\cipher}^\advA$}\\
		$K \getsr \keyspace$\\
		$b' \getsr \advA^\Fn$\\
		Return $b'$\medskip
		
		\underline{$\Fn(M)$}\\
		Return $\cipherE_K(M)$
	}{
		\underline{$\PRF0_{\cipher}^\advA$}\\
		$\rho \getsr \Func(n,n)$\\
		$b' \getsr \advA^\Fn$\\
		Return $b'$\medskip
		
		\underline{$\Fn(M)$}\\
		Return $\rho(M)$
	}	
\caption{The PRF security games.}
\label{fig:prf}
\end{figure} 

Just as we provided a generic attack for TKR security using the exhaustive key search attack, is there a generic distinguishing attack for any cipher? One interesting observation is that for a given key, a block cipher $\cipherE_K$ is a permutation, meaning that two different inputs cannot produce the same output. (If this were not the case, decryption would be impossible.) However, a random function simply chooses outputs at random, so it is entirely possible for two different inputs to produce the same output. The probability of choosing $q$ values at random from $\{0,1\}^n$ and for two of these values to be the same is approximately $\frac{q^2}{2^n}$. This is colloquially known as the \textbf{birthday paradox}, since it implies that the number of people expected to produce two individuals with the same birthday is far fewer than what one might expect. 

If $\advA$ were to query its $\Fn$ oracle enough times, eventually the probability that a repeat value is produced would be large enough and $\advA$ could then check to see if such a repeat value exists. If no repeat occurs, then $\advA$ can assume it is in game $\PRF1_{\cipher}$; otherwise, $\advA$ must be in game $\PRF0_{\cipher}$. This attack is called the \textbf{birthday attack}. The pseudocode for this adversary is defined below.
 
\begin{center}
\fpage{.35}{
	\underline{\textbf{adversary} $\advA^\Fn_{\text{bday}}$} \\
	Let $M_1, M_2, \cdots, M_q \gets \{0,1\}^n$ be distinct \\
	For $i=1, \cdots, q$ do $C_i \gets \Fn(M_i)$ \\
	If $C_1, \cdots, C_q$ are all distinct then return 1 \\
	Else return 0	
}
\end{center}

In addition to creating $q$ messages and querying $\Fn$ $q$ times, $\advA_{\text{bday}}$ must check to see if there are any duplicates in its answered queries. Overall, $\advA_{\text{bday}}$ will then have running time of $\calO(q)$.
To bound the advantage of $\advA_{\text{bday}}$, first notice that in game $\PRF1_{\cipher}$, $\Fn$ outputs the value from $\cipher$, so all output values will be distinct. Thus, $\advA_{\text{bday}}$ will always return 1 and $\Prob{\PRF1_\cipher\Rightarrow 1}=1$. The probability that $\advA_{\text{bday}}$ returns 1 in game $\PRF0_{\cipher}$ is trickier to bound. To do so, we first define $C(N,q)$ as the probability that in the event of choosing $q$ values uniformly at random from set $\{1,\cdots,N\}$, not all of the values chosen are distinct. In game $\PRF0_{\cipher}$, $\Fn$ returns the output from a random function and since $M_1, \cdots, M_q$ are all distinct, then $C_1, \cdots, C_q$ are independently distributed, random values from $\bits$. The probability of all these values being distinct is the probability that there does not exist a collision, which is $1-C(N,q)$, where $N=2^n$. $C(N,q)$ is the birthday bound, so it can be lower-bounded by $\frac{q^2}{2^n}$. 
The PRF-advantage of $\advA^\Fn_{\text{bday}}$ is then defined as 
\begin{align*}
\AdvPRF{\cipher}{\advA_\text{bday}} &= \left| \Prob{\PRF1_\cipher^{\advA_\text{bday}}\Rightarrow 1} 
- \Prob{\PRF0_\cipher^{\advA_\text{bday}}\Rightarrow1} \right| \\
&= 1 - (1 - C(N,q)) \\ 
&= C(N,q) \\ 
&\geq \frac{q^2}{2^n}.
\end{align*}

In order for $\advA_{\text{bday}}$ to have a high probability of succeeding, we expect $q \approx 2^{n/2}$. This means $\advA_{\text{bday}}$ will also have running time of about $2^{n/2}$. For large values of $n$, this becomes an impractical attack. 


\subsection{PRP security} 
We define a \textbf{pseudorandom permutation} (PRP) as a function that is indistinguishable from a random permutation (RP). The games for PRP security are provided in \figref{fig:prp}. These games work similarly to the PRF games but now utilize a random permutation in $\PRP0_\cipher^\advA$ rather than a random function. The PRP advantage for $\advA$ is defined as 
\begin{equation*}
\AdvPRP{\cipher}{\advA} = \left| \Prob{\PRP1_\cipher^\advA\Rightarrow 1} 
- \Prob{\PRP0_\cipher^\advA\Rightarrow1} \right|.
\end{equation*}

\begin{figure}
	\centering
	\hfpages{.15}{
		\underline{$\PRP1_{\cipher}^\advA$}\\
		$K \getsr \keyspace$\\
		$b' \getsr \advA^\Fn$\\
		Return $b'$\medskip
		
		\underline{$\Fn(M)$}\\
		Return $\cipherE_K(M)$
	}{
		\underline{$\PRP0_{\cipher}^\advA$}\\
		$\pi \getsr \Perm(n)$\\
		$b' \getsr \advA^\Fn$\\
		Return $b'$\medskip
		
		\underline{$\Fn(M)$}\\
		Return $\pi(M)$
	}
\caption{The PRP security games.}
\label{fig:prp}	
\end{figure}

Considering that these notions are similar, can we relate them to each other? Intuitively, there is no difference between a random function and a random permutation when observing only a few input-output pairs, as long as the ciphertext space $n$ is sufficiently large. We formalize this intuition with the following lemma. 

\begin{lem}[PRF-PRP Switching Lemma]
	\label{switching-lem}
	Let $\cipher$ be a cipher with ciphertext space $\bits^n$. 
	Let $\advA$ be an adversary making at most $q$ queries. Then
	\bnm
	\left| \Prob{\PRF0_\cipher^\advA\Rightarrow 1} 
	- \Prob{\PRP0_\cipher^\advA\Rightarrow1} \right| \le \frac{q^2}{2^n}  \;.
	\enm
\end{lem}

The intuition for the following proof is that if you have oracle access to a random function or a random permutation, then you need to make enough queries to witness a collision, as determined by the birthday bound. 

One's first instinct might be to bound the difference using a conditioning argument. For instance, let $\mathsf{Dist}$ be the event that in game $\PRF0_{\cipher}^\advA$ all values returned from oracle $\Fn$ are distinct. Then one might say that $\Prob{\PRP1_\cipher^\advA\Rightarrow 1} =   \Prob{\PRP0_\cipher^\advA\Rightarrow1 | \mathsf{Dist}}$. However, this is incorrect and in fact $\Prob{\PRP1_\cipher^\advA\Rightarrow 1} \neq   \Prob{\PRP0_\cipher^\advA\Rightarrow1 | \mathsf{Dist}}$. Refer to \cite{bellare2006multi} for further technical details. 

To correctly prove this, we will instead use a game-playing argument. We first provide the following definitions and lemma. 

\begin{definition}
	A \textbf{flag} is a variable in a pseudocode game that is set upon the occurrence of some event in the game. 
\end{definition}

Typically, games will utilize a flag called \textit{bad} which is set to $\true$.  

\begin{definition}
	Games $\G1$ and $\G2$ are \textbf{identical-until-bad} if they both contain a flag $\bad$  and their code differs only in statements following the setting of $\bad$ to $\true$. 
\end{definition} 

\begin{lem}[Fundamental Lemma of game playing \cite{bellare2006security}]
	Let $\G$, $\Hgame$ be games that are identical-until-bad and let $y$ be any
	value. Then
	\bnm
	\big| \Prob{\G\Rightarrow y} 
	- \Prob{\Hgame\Rightarrow y} \big| \le \Prob{\Hgame\setsbad} = \Prob{\G\setsbad}  \;.
	\enm
\end{lem}

Stated another way, this lemma tells us about the advantage an adversary gains in distinguishing a pair of identical-until-bad games, $\G$ and $\Hgame$. Since $\G$ and $\Hgame$ only differ upon the setting of flag $\bad$, then intuitively we can see that the advantage of an adversary to distinguish between these games must be at most the probability that $\bad$ is set during its execution.

The overall technique we will implement here (and generally in game-playing arguments) is to create a chain of games that are identical-until-bad. We can then invoke the Fundamental Lemma of game playing to upper-bound the adversary's advantage by the probability that $\bad$ gets set in either game. We can slowly modify the games in ways that change the probability of $\bad$ being set until we reach some terminal game where we can utilize conventional probabilistic methods to bound the probability of setting $\bad$. Note that the following proof introduces a $\bad$ flag that can be immediately bounded by traditional means without modification, although future proofs we will see will require further transformations. 

\begin{proof}[Proof of \lemref{switching-lem}]
	 We define the games in \figref{fig:switching}. Notice that the output from game $\G0$ has an identical distribution to that of game $\PRP0_\cipher^\advA$. The only difference between them is that game $\PRP0_\cipher^\advA$ chooses a random permutation and returns the output from that, while game $\G0$ chooses unique values at random as $\advA$ makes queries to $\Fn$. Then $\Prob{\PRP0_\cipher^\advA\Rightarrow1} = \Prob{\G0}$. Game $\G1$ includes the boxed statement and also has an identical output distribution to that of game $\G0$. It simply chooses an output value at random, and if it detects a repeat it then chooses another unique value. In the case of a repeat, it also sets the $\bad$ flag to $\true$. We then have that $\Prob{\G1} = \Prob{\G0}$. We next transition to game $\G2$ and notice that $\G1$ and $\G2$ are \textbf{identical-until-bad}. The Fundamental Lemma of game playing then states that $\Prob{\G1} \leq \Prob{\G2} + \Prob{\bad \text{ set to } \true}$. Game $\G2$ returns values chosen at random and allows for repeat values, so it has an identical output distribution to $\PRF0_\cipher^\advA$. This means $\Prob{\PRF0_\cipher^\advA\Rightarrow1} = \Prob{\G2}$. Finally, the probability that $\bad$ is set to $\true$ is the probability that a random value is chosen by $\Fn$ such that it is not distinct, which is bounded by the birthday bound. We then have the following:
	 
	 \begin{figure}
	 	\centering
	 	\hfpagess{.20}{.20}{
	 		\underline{$\G0$}\\[2pt]
	 		$b' \getsr \advA^\Fn$\\
	 		Return $b'$\medskip
	 		
	 		\underline{$\Fn(M)$}\\
	 		If $\TabF[M] = \bot$ then\\
	 		\ind $\TabF[M] \getsr \bits^n \setminus \TabF$\\
	 		Return $\TabF[M]$
	 	}{
	 		\underline{\fbox{$\G1$} \;\;\; $\G2$}\\[2pt]
	 		$b' \getsr \advA^\Fn$\\
	 		Return $b'$\medskip
	 		
	 		\underline{$\Fn(M)$}\\
	 		$C \getsr \bits^n$\\
	 		If $C \in \TabF$ then\\
	 		\ind $\badtrue$\\
	 		\ind \fbox{$C \getsr \bits^n \setminus \TabF$}\\
	 		$\TabF[M] \gets C$\\
	 		Return $\TabF[M]$
	 	}
	 	\caption{The games for the proof of the PRF-PRP Switching Lemma (\lemref{switching-lem}).}
	 	\label{fig:switching}	
	 \end{figure} 
	 
	 \begin{align*}
	 \left| \Prob{\PRP0_\cipher^\advA\Rightarrow 1} 
	 - \Prob{\PRF0_\cipher^\advA\Rightarrow1} \right|  
	 &=  \left|\Prob{\G0} - \Prob{\PRF0_\cipher^\advA\Rightarrow1} \right|  \\
	 &=  \left|\Prob{\G1} - \Prob{\PRF0_\cipher^\advA\Rightarrow1} \right|  \\
	 &\le \left|\Prob{\G2} + \Prob{\bad \text{ set to } \true} - \Prob{\PRF0_\cipher^\advA\Rightarrow1} \right|\\
	 &= \Prob{\bad \text{ set to } \true}\\
	 &\le \frac{q^2}{2^n} \\
	 \end{align*} 
\end{proof}
%!TEX root = ../main.tex
%%%%%%%%%%%%%%%%%%%%%%%%%%%%%%%%%%%%%%%%%%%%%%%%%%%%%%%%%%%%%%%%%%%%%%%%%%%%%%%%
\tikzset{XOR/.style={thick,
					draw,
					circle,
					append after command={
					        [shorten >=\pgflinewidth, shorten <=\pgflinewidth,]
					        (\tikzlastnode.north) edge[thick] (\tikzlastnode.south)
					        (\tikzlastnode.east) edge[thick] (\tikzlastnode.west)
        				},
    				},
}

%!TEX root = ../main.tex
%%%%%%%%%%%%%%%%%%%%%%%%%%%%%%%%%%%%%%%%%%%%%%%%%%%%%%%%%%%%%%%%%%%%%%%%%%%%%%%%
\subsection{Sponge Construction}

\begin{tikzpicture}[scale=0.45]

\tikzset{SpongePerm/.style=rounded corners=4pt,};
%\tikzset{edge/.style=-latex new, arrow head=8pt, thick};
%\tikzset{edgee/.style=latex new-latex new, arrow head=8pt, thick};
\tikzset{edge/.style=->, thick};
\tikzset{edgee/.style=<->, thick};

\path (12.5, -1) node {absorbing phase};
\path (21.5, -1) node {squeezing phase};

%%
\begin{scope}[xshift=0cm]
  \draw[thick] (0,0) rectangle node {$0$} ++(1,3);
  \draw[thick] (0,3) rectangle node {$0$} ++(1,7);

  \node[XOR,thick] (xm0) at (1+1.5,8) {};
  \draw[edge,thick] (1,8) -- (xm0);
  \draw[edge,thick] (1,2) -- ++(3,0);
  \draw[edge,thick] (1+1.5,10.5) node[above] {\large $m_{0}$} -- (xm0);
  \draw[edge,thick] (xm0) -- ++(1.5,0);

	\draw[edgee,anchor=east] (-1,0) -- node[left] {$c$ bits} ++(0,3);
	\draw[edgee,anchor=east] (-1,3) -- node[left] {$r$ bits} ++(0,7);

\end{scope}

\foreach \z in {1,2} {
  \begin{scope}[xshift=\z*4cm]
    \draw[SpongePerm]
    		(0,0) rectangle node {\large$f$} ++(1,10);
  \end{scope}
}

\foreach \z in {4,11} {
  \begin{scope}[xshift=\z*1cm]
    \node[XOR,thick] (xm\z) at (1+1.5,8) {};
    \draw[edge,thick] (1,8) -- (xm\z);
    \draw[edge,thick] (1,2) -- ++(3,0);
    \draw[edge,thick] (xm\z) -- ++(1.5,0);
  \end{scope}
}
\draw[edge,thick] (5+1.5,10.5) node[above] {\large $m_{1}$} -- (xm4);
\draw[edge,thick] (12+1.5,10.5) node[above] {\large $m_{n-1}$} -- (xm11);

\begin{scope}[xshift=8cm]
  \draw[edge,thick] (1,2) -- ++(1.5,0) node[right] {$\dots$};
  \draw[edge,thick] (1,8) -- ++(1.5,0) node[right] {$\dots$};
\end{scope}

%%
\begin{scope}[xshift=15cm]
  \draw[SpongePerm]
  		(0,0) rectangle node {\large$f$} ++(1,10);

  \draw[thick] (1,2) -- ++(1,0);
  \draw[thick] (1,8) -- ++(1,0);
  \draw[dashed,thick] (2,-2) -- ++(0,14);
\end{scope}
%%

\foreach \z in {0,1} {
  \begin{scope}[xshift=\z*4cm+17cm]
    \draw[edge,thick] (0,2) -- ++(3,0);
    \draw[edge,thick] (0,8) -- ++(3,0);
    \draw[edge,thick] (0+1.5,8) -- ++(0,2.5) node[above] {\large $z_{\z}$};
    \draw[SpongePerm] (3,0) rectangle node {\large$f$} ++(1,10);
  \end{scope}
}

\begin{scope}[xshift=25cm]
  \draw[edge,thick] (0,2) -- ++(1.5,0) node[right] {$\dots$};
  \draw[edge,thick] (0,8) -- ++(1.5,0) node[right] {$\dots$};
  \draw[edge,thick] (3,2) -- ++(1.5,0);
  \draw[edge,thick] (3,8) -- ++(1.5,0);
  \draw[SpongePerm] (4.5,0) rectangle node {\large$f$} ++(1,10);
  \draw[edge,thick] (5.5,8) -- ++(1.5,0) -- ++(0,2.5) node[above] {\large $z_{m-1}$};
\end{scope}
\end{tikzpicture}

\newpage
%!TEX root = ../main.tex
%%%%%%%%%%%%%%%%%%%%%%%%%%%%%%%%%%%%%%%%%%%%%%%%%%%%%%%%%%%%%%%%%%%%%%%%%%%%%%%%
\section{Further Security Properties for Hash Functions}

Hash functions are used to accomplish a wide array tasks in cryptography that require properties stronger than, weaker than, and orthogonal to just collision resistance.
In this section, we will examine a few of these alternate security goals for hash functions and their relations to each other.

\subsection{Preimage and Second Preimage Resistance}

Preimage resistance, often known as ``one-wayness'' of a one-way function, means that given an output $Y$ of a function $\hash$, it is hard to invert $\hash$ and find an element of the domain space that evaluates to $Y$, i.e., it is hard to find any preimage of $Y$.
Second preimage resistance means that given a domain element and function $\hash$, it is hard to find a second domain element that evaluates to the same output $Y$ for both domain elements, i.e., it is hard to find a second preimage of $Y$ even if already given one preimage.
Pseudocode for preimage resistance and second preimage resistance, $\OWF$ and $\SPR$, respectively, is given in Figure~\ref{fig:preimage-resistance}.
The advantage of an adversary against these games is defined as:
\bnm
\AdvOWF{\hash}{\advA} = \Prob{\OWF^\advA_{\hash}\Rightarrow\true}\hspace{2em}\text{and}\hspace{2em}\AdvSPR{\hash}{\advA} = \Prob{\SPR^\advA_{\hash}\Rightarrow\true}\;.
\enm

\begin{wrapfigure}{r}{3.5in}
  \centering
\hfpagess{.2}{.3}{
\underline{$\OWF^\advA_{\hash}$}\\[1pt]
$M \getsr \msgspace$\\
$Y \gets \hash(M)$\\
$M' \getsr \advA(Y)$\\
Return $(\hash(M') = Y)$
}{
\underline{$\SPR^\advA_{\hash}$}\\[1pt]
$M \getsr \msgspace$\\
$Y \gets \hash(M)$\\
$M' \getsr \advA(M)$\\
Return $(\hash(M') = \hash(M)) \land (M \ne M')$
}
\caption{
Security games for Preimage resistance (left) and second preimage resistance (right).
}
\label{fig:preimage-resistance}
\end{wrapfigure}

These definitions, along with collision resistance, are certainly closely related.
We can think about how they compare to each other.
For example, we might conjecture $\CR\Rightarrow\SPR$.
Thinking about the contrapositive, if one can find a second preimage of an output $Y$ than those two preimages constitute a collision.
It turns out that we can formally show the following relations: $\CR\Rightarrow\SPR\Rightarrow\OWF$;
and furthermore, there exist separations for the opposite direction: $\OWF\not\Rightarrow\SPR\not\Rightarrow\CR$.
Formally showing these implications and separations is left as an exercise.
Note that these results hold specifically for the definitions of preimage resistance, second preimage resistance, and collision resistance given in this book.
While they are perhaps the most common such definitions, there exist several nuanced alternate definitions for these notions in the literature that do not satisfy the above implications.
Rogaway and Shrimpton~\cite{RogawayShrimpton04} explore the landscape of these definitions.

%!TEX root = ../main.tex
%%%%%%%%%%%%%%%%%%%%%%%%%%%%%%%%%%%%%%%%%%%%%%%%%%%%%%%%%%%%%%%%%%%%%%%%%%%%%%%%
\subsection{Hash-based PRF and the Random Oracle Model}

\begin{wrapfigure}{r}{2in}
  \centering
\begin{tikzpicture}[scale=0.4]
	\node[draw,trapezium,trapezium left angle=70,trapezium right angle=70,minimum height=1.0cm,thick,shift={(1.15,0)},rotate=-90] (hfxn)
	{\begin{sideways}\Large $\hash$ \end{sideways}};
	\draw[->,thick] (0,0) node[left] {$\prfkey \concat \msg$} -- (hfxn);
	\draw[->,thick] (hfxn) -- ++(3,0) node[right] {$Y$};
\end{tikzpicture}
\caption{Simple hash-based PRF.}
\label{fig:hash-based-prf}
\end{wrapfigure}

Another property that is often desired of hash functions is that of a PRF.
Consider a simple hash-based PRF construction, $\prf:\keyspace\times\msgspace\rightarrow\bits^{n}$, constructed from a hash function $\hash:\bits^{*}\rightarrow\bits^{n}$ such that $\prf_{\prfkey}(\msg) = \hash(\prfkey\concat\msg)$.

Now we can consider how to prove the PRF security of $\prf$.
One approach would be to instantiate $\hash$ with a specific hash function construction, e.g. SHA-2 or SHA-3, and prove security with respect to the building blocks of those constructions, e.g. PRF security of the compression function or PRP security of the permutation.
However, this would not be a modular approach, as a new proof would be required for each hash function.
Instead, we will use a similar approach to as when we proved the PRF security of the Davies-Meyer compression function.
\scribenote{Add reference}
There we introduced an idealized model for the block cipher, called the ideal cipher model.
Here, we will introduce another idealized model known as the \emph{random oracle model} (ROM), in which we, as the name suggests, model the hash function as a random oracle.

\begin{wrapfigure}{r}{2in}
  \centering
\hfpages{.15}{
\underline{$\PRF1_{\prf,\Horacle}^\advA$}\vspace{0.5em}\\
$K \getsr \keyspace$\\
$b' \getsr \advA^{\Fn,\Horacle}$\\
Return $b'$\medskip

\underline{$\Fn(M)$}\\
Return $F^\Horacle_K(M)$\medskip

\underline{$\Horacle(X)$}\\
If $\TabH[X] = \bot$ then\\
\myInd $\TabH[X] \getsr \bits^n$\\
Return $\TabH[X]$

}{
\underline{$\PRF0_{F,\Horacle}^\advA$}\vspace{0.3em}\\
$\rho \getsr \Func(\msgspace,n)$\\
$b' \getsr \advA^{\Fn,\Horacle}$\\
Return $b'$\medskip

\underline{$\Fn(M)$}\\
Return $\rho(M)$\\

\underline{$\Horacle(X)$}\\
If $\TabH[X] = \bot$ then\\
\myInd $\TabH[X] \getsr \bits^n$\\
Return $\TabH[X]$
}
\caption{PRF security games in the random oracle model.}
\label{fig:prf-ro}
\end{wrapfigure}

Consider the modified PRF security games shown in Figure~\ref{fig:prf-ro}.
In the ROM, a new oracle $\Horacle$ is added to the security game.
Calling the hash function is done by sending queries to $\Horacle$.
The random oracle $\Horacle$ is instantiated lazily by maintaining a table $\TabH$, randomly sampling and storing in $\TabH$ return values for new queries.
We prove the following theorem about the security of the hash-based PRF in the random oracle model:

\begin{theorem}
Let  $\Horacle\Colon\msgspace\rightarrow\bits^n$ be modeled as a random oracle
and let $F^\Horacle\Colon\keyspace\times\msgspace\rightarrow\bits^n$ be the
hash-based PRF defined as $F^\Horacle(K,M) = \Horacle(K \concat M)$.
Then for any $\PRF_{F,\Horacle}$-adversary $\advA$ making at most $q$ queries
to $\Horacle$ it holds that
\bnm
  \AdvPRF{F,\Horacle}{\advA} \le \frac{q}{|\keyspace|} \;.
\enm
\end{theorem}


\begin{wrapfigure}{r}{3.8in}
  \centering
\hfpagesss{.15}{.18}{.15}{
\underline{$\G0$}\vspace{0.5em}\\
$\prfkey \getsr \keyspace$\\
$b' \getsr \advA^{\Fn,\Horacle}$\\
Return $b'$\medskip

\underline{$\Fn(\msg)$}\\
Return $\Horacle(\prfkey\concat\msg)$\medskip

\underline{$\Horacle(X)$}\\
If $\TabH[X] = \bot$ then\\
\myInd $\TabH[X] \getsr \bits^n$\\
Return $\TabH[X]$

}{
\underline{$\fbox{\G1}$\;\;$\G2$}\vspace{0.5em}\\
$\prfkey \getsr \keyspace$\\
$b' \getsr \advA^{\Fn,\Horacle}$\\
Return $b'$\medskip

\underline{$\Fn(\msg)$}\\
If $\TabT[\prfkey\concat\msg] = \bot$ then\\
\myInd $\TabT[\prfkey\concat\msg]\getsr \bits^{n}$\\
Return $\TabT[\prfkey\concat\msg]$\medskip

\underline{$\Horacle(X)$}\\
If $\TabT[X]\neq\bot$ then\\
\myInd $\badtrue$\\
\myInd \fbox{$\TabH[X]\gets\TabT[X]$}\\
If $\TabH[X] = \bot$ then\\
\myInd $\TabH[X] \getsr \bits^n$\\
Return $\TabH[X]$

}{
\underline{$\G3$}\vspace{0.3em}\\
$\rho \getsr \Func(\msgspace,n)$\\
$b' \getsr \advA^{\Fn,\Horacle}$\\
Return $b'$\medskip

\underline{$\Fn(M)$}\\
Return $\rho(M)$\\

\underline{$\Horacle(X)$}\\
If $\TabH[X] = \bot$ then\\
\myInd $\TabH[X] \getsr \bits^n$\\
Return $\TabH[X]$
}
\caption{PRF security games in the random oracle model.}
\label{fig:hash-prf-games}
\end{wrapfigure}

\emph{Proof.}
Consider the set of games given in Figure~\ref{fig:hash-prf-games}.
Game \G0 is identical to $\PRF1_{\prf,\Horacle}$ with pseudocode for the PRF oracle $\Fn$ filled in to represent the hash-based PRF.
In Game \G1, the PRF oracle does not call the random oracle, but instead maintains a separate table $\TabT$ to respond to queries with random values.
If the random oracle $\Horacle$ is called with a key\dash message pair that is stored in $\TabT$, a $\bad$ flag is set and the random value sampled in $\Fn$ is copied over and $\Horacle$ responds in a consistent manner.
Since the value is copied over from $\TabT$ to $\TabH$, \G1 is identical to \G0.

Game \G2 is identical until $\bad$ to \G1.
Instead of copying the value from $\TabT$ to $\TabH$, a new random value is sampled.
By the fundamental lemma of game playing, we bound the difference in advantage between \G1 and \G2 by the probability the $\bad$ flag is set.
To set the $\bad$ flag, $\advA$ must make a query to $\Horacle$ that begins with PRF key $\prfkey$.
Since only random values are returned, $\advA$ has no knowledge of $\prfkey$, so on any given query, the probability the $\bad$ flag is set is $1/\abs{\keyspace}$.
Thus, over $q$ queries, using the union bound we have
\begin{align*}
	\absv*{\Prob{\G1\Rightarrow 1} - \Prob{\G2\Rightarrow1}} &\le \Prob{\G1 \setsbad}\\
  &\le \frac{q}{\abs{\keyspace}}.
\end{align*}

Lastly, \G2 is identical to \G3 is identical to $\PRF0_{\prf,\Horacle}$.
Since the values $\TabH$ is not updated with the values from $\TabT$ both $\Fn$ and $\Horacle$ act as independent random functions.\hfill$\dqed$

So we've proved security of the hash-based PRF security when the hash function is modeled by a random oracle.
However, how do we gain confidence that the random oracle model fits for specific hash functions?
In other words, if we prove a protocol secure with respect to the random oracle model, is it safe to instantiate the random oracle with a hash function such as SHA-2 or SHA-3?
In the case of the ideal cipher model, we gain confidence through use of cryptanalysis of specific block ciphers.

\begin{wrapfigure}{r}{2.9in}
\begin{tikzpicture}[scale=0.4]

	\begin{scope}[]
		\node [draw,trapezium,trapezium left angle=50,trapezium right angle=90,minimum height=0.75cm,thick,shift={(1.15,0.3)},rotate=-90]
		{\begin{sideways}\Large$f$\end{sideways}};
		\draw[->,thick] ++(0.5,+4) node[above] {$\prfkey$} -- ++(0,-1.5) -- ++(1.4,0);
		\draw[->,thick] ++(0,0.5) node[left] {$\IV$} -- ++(1.9,0);
	\end{scope}

	\begin{scope}[shift={(3.8,0)}]
		\node [draw,trapezium,trapezium left angle=50,trapezium right angle=90,minimum height=0.75cm,thick,shift={(1.35,0.3)},rotate=-90] (centerbox)
		{\begin{sideways}\Large$f$\end{sideways}};
		\draw[->,thick] ++(0.9,+4) node[above] {$\msg_1$} -- ++(0,-1.5) -- ++(1.4,0);
		\draw[->,thick] ++(0,0.5) -- ++(2.4,0);
	\end{scope}

	\begin{scope}[shift={(8.2,0)}]
		\node [draw,trapezium,trapezium left angle=50,trapezium right angle=90,minimum height=0.75cm,thick,shift={(1.35,0.3)},rotate=-90]
		{\begin{sideways}\Large$f$\end{sideways}};
		\draw[->,thick] ++(0.9,+4) node[above] {$\qquad\msg_2 \concat 10^r \concat \bm{\langle} \left| \msg \right| \bm{\rangle}$} -- ++(0,-1.5) -- ++(1.4,0);
		\draw[->,thick] ++(0,0.5) -- ++(2.4,0);
		\draw[->,thick] ++(4.4,0.5) -- ++(2,0) node[right] {$Y$};
	\end{scope}

	\begin{scope}[shift={(7, -9)}]
		\node [draw,trapezium,trapezium left angle=50,trapezium right angle=90,minimum height=0.75cm,thick,shift={(1.15,0.3)},rotate=-90]
		{\begin{sideways}\Large$f$\end{sideways}};
		\draw[->,thick] ++(0.5,+4) node[above] {$\qquad\msg^* \concat 10^{r'} \concat \bm{\langle} \left| \msg' \right| \bm{\rangle}$} -- ++(0,-1.5) -- ++(1.4,0);
		\draw[->,thick] ++(4.2,0.5) -- ++(2,0) node[right] {$Y'$};
	\end{scope}

	\draw[->,thick] (16,0.5) -- ++(1,0) -- ++(0, -3) -- ++(-15, 0) -- ++(0, -6) -- ++(6.9,0);
	\path (2, -11) node[anchor=west] {$\msg = \msg_1\concat\msg_2$};
	\path (1.8, -12.5) node[anchor=west] {$\msg' = \msg_1\concat\msg_2\concat 10^r\concat\bm{\langle}\left|\msg\right|\bm{\rangle}\concat\msg^*$};

\end{tikzpicture}
\caption{Length extension attack when instantiating the proposed hash-based PRF construction with a Merkle-Damg\aa rd hash function.}
\label{fig:length-extension-md}
\end{wrapfigure}

In fact, this can be a problem in practice, as real hash functions often do not act as random oracles, i.e., there exist attacks that take advantage of the structure of the hash function.
Take the example of instantiating the hash-based PRF that we just proved secure in the random oracle model with the Merkle-Damg\aa rd function.
This construction is distinguishable from a PRF through a straightforward length extension attack.
The attack is depicted in Figure~\ref{fig:length-extension-md}.
The output of the Merkle\dash Damg\aa rd hash function on one message can be chained into computing the output for a longer message without knowledge of the secret key.

There are variants of the basic Merkle\dash Damg\aa rd hash function that can be used to instantiate the hash-based construction, but this leads us to a larger question.
If not all hash functions are properly modeled by the random oracle model, how do we proceed?
As mentioned before, a non-modular approach would be to prove protocols without the random oracle model.
Another approach is to build hash functions that are better modeled by the random oracle model.
This is a notion known as indifferentiability and we will look at more closely later in the section.
One of the strengths of the sponge construction from Figure~\ref{fig:sponge} is that it meets this indifferentiability property.

%!TEX root = ../main.tex
%%%%%%%%%%%%%%%%%%%%%%%%%%%%%%%%%%%%%%%%%%%%%%%%%%%%%%%%%%%%%%%%%%%%%%%%%%%%%%%%

\section*{Exercises}

\begin{enumerate}[label=\textbf{Exercise \thesection.\arabic*}, wide=0pt]
  \item Show the reductions $\CR\Rightarrow\SPR$ and $\SPR\Rightarrow\OWF$.
  \item Show the separations $\OWF\not\Rightarrow\SPR$ and $\SPR\not\Rightarrow\CR$.
\end{enumerate}


\newpage
%%%%%%%%%%%%%%%%%%%%%%%%%%%%%%%%%%%%%%%%%%%%%%%%%%%%%%%%%%%%%%%%%%%%%%%%%%%%%%%%
\section{Block Cipher Design and Cryptanalysis}
\label{sec:cryptanalysis}

We've seen some theoretical ways to build block ciphers, namely Feistel
constructions using round functions that are secure as PRFs. There are other
constructions as well, such as Evan-Mansour \tnote{Should probably include
somewhere}. But this is reductive, since it's unclear how to build the PRFs
themselves. One could imagine trying to use actual random functions, but this is
intractable for large block sizes --- the secret key in this case being a random
table requring $n2^n$ bits. 

In practice block ciphers are built using a bag of design principles that have been built
up over the last 60 or so years, in response to new techniques for
cryptanalysis.

\tnote{Perhaps discuss really trivially broken ciphers, such as ones that are
completely linear functions, or partially linear functions}


\paragraph{Linear cryptanalysis.} \tnote{Discuss Matsui's paper, the high level
intuition and how it can be used to break DES}  

\tnote{Also shown in Boneh-Shoup book}


\bne
\label{eq:linear-approx}
  \Prob{M[S_m] \oplus C[S_c] = K[S_k]} = \frac{1}{2} + \epsilon
\ene

\begin{theorem}
Let $\cipher$ be a cipher such that \eqref{eq:linear-approx} holds with
$\epsilon > 0$, and let $K \in \calK$. Let $M_1,\ldots,M_q$ be sampled uniformly
from $\bits^n$ and let $C_i = E_K(M_i)$ for $i \in \{1,\ldots,q\}$. Then 
\bnm
  \Prob{K[S_k] = \Maj\left(\{M_i[S_m]\oplus C_i[S_c]\}_{i=1}^q\right)} \ge 1 - e^{-q\epsilon^2/2} \;.
\enm
\end{theorem}


\begin{theorem}[Chernoff bounds]
Let $X = \sum_{i=1}^n X_i$, all $X_i$ independent and where $X_i = 1$ with probability $p_i$ and $X_i = 0$
with probability $1-p_i$. Let $\mu = \Ex[X]$. Then  
\begin{align}
  \Prob{X \ge (1 + \delta)\mu} &\le e^{-\frac{\delta^2}{2+\delta}\mu}\\
  \Prob{X \le (1 + \delta)\mu} &\le e^{-\frac{\delta^2}{2}\mu}
\end{align}
\end{theorem}


\begin{lemma}
Let $X_i$ for $1 \le i \le n$ be indendent random variables with probabilities
$p_i$ of being one and $1-p_i$ of being zero. Then  
\bnm
  \Prob{X_1\oplus \cdots \oplus X_n = 0} = \frac{1}{2} + 2^{n-1}\prod_{i=1}^n
  \left(p_i - \frac{1}{2}\right) \;.
\enm
\end{lemma}


\bnm
  \AdvPRF{\EM,P}{\advA} \le \frac{2\cdot q_1 \cdot q_2}{2^n}
\enm

\newpage
%%%%%%%%%%%%%%%%%%%%%%%%%%%%%%%%%%%%%%%%%%%%%%%%%%%%%%%%%%%%%%%%%%%%%%%%%%%%%%%%
\section{Deterministic Encryption and Frequency Analysis}
\label{sec:freqanalysis}

We have seen how cryptanalysts design block ciphers for practical use, such as AES and DES, with the hope that enough study of such ciphers will allow us to create stronger ciphers before these are broken.
 Such ciphers are used both as a building block towards randomized and authenticating encryption but also as applications themselves.

\paragraph{Length preserving within encryption.} 
Length preserving encryption (LPE) is simply encryption where a message and its ciphertext have the same length. The need for such encryption originated with hard disks. If private information is stored on some unencrypted physical disk, stealing the information is just a matter of stealing the disk itself. However, if the encryption is expanding then we could use much more (valuable) disk space. If we can use length preserving encryption on the disk layer, than deleting the secret key effectively erases the hard drive. There are many implementations of LPE including ECB-Mask-ECB (EME) \cite{Halevi2004EME}, CBC-Mask-CBC (CMC) \cite{Halevi2003CMC}, and linear-Transformation; linear-Transformation (TET) \cite{Halevi2007TET}.

\paragraph{Format preserving encryption.}
Format preserving encryption (FPE) is a superset of LPE. In FPE, we may want a ciphertext to be the same length as the message, but we may also want more features of the message to be preserved. This notion has existed for decades informally \cite{commerce1981FPE} \cite{Brightwell1997FPE}, but the motivation for the first more formal presentation of FPE was so that credit card companies could transition databases that held credit card numbers in plaintext into encrypted databases \cite{Rogaway2002FPE}. Software using the numbers assumed a certain syntactic format, so it was useful to have the encryption preserve such a format in addition to length.

It is natural that encryption for sensitive databases would be deterministic. As the main function of databases is to actually use the data, it would be ideal if database queries were supported on the encrypted database. For example, it would be useful for the encrypted database to support equality search (find all rows where the name is ``Alice"), range queries (find all rows where the age is between 20 and 30), and to return sorted lists (return the records ordered by salary). Whether or not the database supports such operations depends on the method of encryption. 

\paragraph{Proxy-Based encrypted database} One simple approach is proxy-based. That is, there is some proxy between the client and the server. The client only sees plaintext and the server holds the entire database but with every entry encrypted. The client sends its queries to the proxy which encrypts information from the client entry by entry. The proxy then gives the server the same query with all the data encrypted. The server returns back encrypted data which is decrypted by the proxy and sent back to the client. 

In this case deterministic encryption is useful. Given, say, a search query for all rows with Name=``Alice", the proxy can simply ask the server for all rows where the value in the corresponding column is the encryption of ``Alice" and return to the client the decryption of the rows returned by the server. Because encryption is deterministic, ``Alice" encrypts to the same value every time, so the proxy only needs to search the one value.

This approach, however has certain drawbacks in that it reveals information about plaintexts. The server can see every time a plaintext is encrypted more than once with the same key.
Thus the server has access to the plaintext frequency information of the database. As each column uses one key throughout the column, the server can see when plaintexts are repeated in a column.
%In real databases, columns hold names, short numbers, or genders so plaintexts are often repeated.
This will allow for what we will call a \textit{frequency analysis attack}.
    
One possible defense might be to use a different key for every entry so that messages do not encrypt to the same ciphertext every time. However, the proxy must then encrypt every entry in a column in order to do a search query, which  is detrimental to the efficiency of the proxy. However, solutions which try to hide frequency information with less computational overhead, such as frequency smoothing \cite{Lacharité2018FSE}, have been proposed.

\paragraph{Frequency analysis attacks.}
    Why is the knowledge of a repeated plaintext sensitive information? Suppose the most common name is ``Bob", and in the US 20\% of people are named ``Bob". If I have a database which is sampled fairly uniformly among people in the US and 20\% of the ciphertexts in the name column are $0x011010$, I can be fairly confident that the decryption of this name is in fact ``Bob". This is already more information than we would like to share with an untrusted server, but if we can easily recover a high percentage of names, and we can in a similar fashion recover a high percentage of information in other columns, we reveal information throughout specific rows and possibly of specific users.
    
    More generally, a frequency analysis attack uses the adversary's knowledge of the distribution of messages to match ciphertexts to plaintexts. That is, given a list of ciphertexts which were encrypted from messages sampled from a known distribution, the adversary attempts to correctly guess which message encrypts to which ciphertext. Consider the following game $\MRUMA$:
    
\begin{figure}[H]
\centering
\fpage{.22}{
		\underline{$\MRUMA^{\advA}_{\cipher,\mdist,q}$}\\[1pt]
		$K \getsr \keyspace$\\
    For $i = 1$ to $q$\\
    \myInd $M_i \get{\mdist} \msgspace$\\
    \myInd $C_i \gets E_K(M_i)$\\
		$\hatE \getsr \advA(C_1,\ldots,C_q)$\\
    Ret $\forall_{i=1}^q \left(\hatE(M_i) = C_i\right)$
	}
\end{figure}

Here we assume that the adversary $\advA$ has access to the probability distribution $\mdist$, and this distribution can be sampled efficiently. The notation $\get{\mdist}$ means to sample according to $\mdist$. Thus the adversary wins if it can find which message produced which ciphertext and return the resulting map (encryption scheme).

The advantage that an adversary would like to maximize is the following:
\bnm
   \AdvMRUMA{\cipher,\mdist,q}{\calA} = \Prob{\MRUMA^\advA_{\cipher,\mdist,q}\Rightarrow\true}
\enm

We claim that, for a ``good" encryption scheme, the best adversary for $\MRUMA$ simply matches the $i$-th most common ciphertext with the $i$-th most probable message. Formally, let the adversary $\advA^*$ be the following:

%Such databases need to support certain database operations, such as search  ideally while encrypted.
\begin{figure}[H]
 \centering
\fpage{.40}{

		\underline{$\advA^*(C_1,\ldots,C_q)$}\\[1pt]
    Let $c$ be number of unique ciphertexts in $C_1,\ldots,C_q$\\
    Let $\tildeC_1,\ldots,\tildeC_c$ be unique ciphertexts\\
    Let $N_{\tildeC_i}$ be number of occurences of $\tildeC_i$\\
    $\hatE \gets \argmax_f \prod_{i=1}^{c} \mdist\left(f^{-1}(\tildeC_i)\right)^{N_{\tildeC_i}}$\\
    Ret $\hatE$
	}  
\end{figure}

	
 Note that this adversary literally picks the most likely map. For every map of messages to ciphertexts, $\advA^*$ chooses the one that was most likely to result in $\tildeC_i$ occurring $N_{\tildeC_i}$ times. Because multiplication is an increasing function, this happens to be the map which matches the most common ciphertexts with the most probable messages.
 
Theorem \ref{freq-opt} shows that any adversary which does significantly better that $\advA^*$ ``breaks" the cipher, in the sense of distinguishing it from a PRP.
\begin{theorem}[Optimality Frequency Analysis Attack]
	\label{freq-opt}
Let $\cipher$ be a cipher, $\mdist$ a message distribution, and $q>0$. Let
$\advA^*$ be the frequency analysis $\MRUMA_{\cipher,q}$-adversary and $\advA$
be some $\MRUMA_{\cipher,q}$-adversary. Then we give a
$\PRP_\cipher$-adversary $\advB$ such that
\bnm
  \AdvMRUMA{\cipher,\mdist,q}{\advA} \le 
        \AdvMRUMA{\cipher,\mdist,q}{\advA^*} + \AdvPRP{\cipher}{\advB}
\enm
$\advB$ makes $q$ queries and runs in time 
$\Time(\advA) + 2q+q\cdotsm\Time(\mdist)$.
\end{theorem}

 \begin{proof}[\thref{freq-opt}]We begin by considering the following game which is the same as $\MRUMA$ except instead of a cipher, we use a random permutation\footnote{We use a permutation to avoid considering noise in the frequencies caused by function collisions.}.

\begin{figure}[H]
\centering
\fpage{.22}{
		\underline{$\G1$}\\[1pt]
		$\pi \getsr \Perm(\msgspace)$\\
     For $i = 1$ to $q$\\
    \myInd $M_i \get{\mdist} \msgspace$\\
    \myInd $C_i \gets \pi(M_i)$\\
    $\hatE \getsr \advA(C_1,\ldots,C_q)$\\
    Ret $\forall_{i=1}^q \left(\hatE(M_i) = C_i\right)$
	}
\end{figure}

% Let the advantage for this game be the following:
% $$\Adv^{\textrm{G1}}_{\mdist,q}(\advA)=\Prob{\textrm{G1}^\advA_{\mdist,q}\Rightarrow\true}.$$

A random permutation can be considered a perfect cipher in that it is distinguishable from a PRP with probability 0 because it is one. If the theorem holds then certainly so must the following lemma.

\begin{lemma}
\label{freqsidelem}
Let $\mdist$ be a message distribution, and $q>0$. For any G1-adversary $\advA$, $\Adv^{\textrm{G1}}_{\mdist,q}(\advA)\leq\Adv^{\textrm{G1}}_{\mdist,q}(\advA^*)$.
\end{lemma}

\begin{proof}[\lemref{freqsidelem}]

The lemma asserts that $A^*$ is the adversary with the highest advantage for G1.  This is because we are looking to maximize 
$\Prob{\textrm{G1}^\advA_{\mdist,q}\Rightarrow\true}$
and so we want $\advA$ to output the function $\hatE$ which maximizes $\Prob{\forall_{i=1}^q \left(\hatE(M_i) = C_i\right)}$. In other words, the best we can do is 

$$\hatE=\argmax_f \textrm{ } \Prob{\forall_{i=1}^q \left(f(M_i) = C_i\right)}.$$

This probability is taken over all messages and permutations, so we have:

$$\hatE= \argmax_f \sum_{\pi\in\Perm(\msgspace)}\CondProb{\pi=\rho}{\rho\getsr\Perm(\msgspace)} \sum_{m_1, \ldots,m_q\in \msgspace} \mdist(m_1)\ldots\mdist(m_q)\Prob{\forall_{i=1}^q (f(m_i)=\pi(m_i))}.$$

Because $\CondProb{\pi=\rho}{\rho\getsr\Perm(\msgspace)} $ is a positive constant, we can pull it out of the sum and then the $\argmax$ to get

$$\hatE=\argmax_f \sum_{\pi\in\Perm(\msgspace)}\sum_{m_1, \ldots,m_q\in \msgspace} \mdist(m_1)\ldots\mdist(m_q)\Prob{\forall_{i=1}^q (f(m_i)=\pi(m_i))}.$$

 Now, if $\advA$ is deterministic, then $\Prob{\forall_{i=1}^q (f(m_i)=\pi(m_i))}$ is 1 or 0 because $\advA$ outputs the same $f$ whenever $\pi$ and $m_1,\ldots,m_q$ are the same. In this sense, $\Prob{\forall_{i=1}^q (f(m_i)=\pi(m_i))}$ is the same as 1 if $\forall_{i=1}^q f(m_i)=\pi(m_i)$ and 0 otherwise. Thus we have

$$\hatE=\argmax_f \sum_{m_1, \ldots,m_q\in \msgspace }\sum_{\substack{\pi\in\Perm \\ \forall_{i=1}^q f(m_i)=\pi(m_i)}} \mdist(m_1)\ldots\mdist(m_q).$$

Becuase $\pi$ is a permutation and thus a bijection, we can replace $m_i$ in our bounds with $c_i$ such that $\pi(m_i)=c_i$ to get 

$$\hatE=\argmax_f \sum_{c_1, \ldots,c_q\in \msgspace }\sum_{\substack{\pi\in\Perm \\\forall_{i=1}^q f(\pi^{-1}(c_i))=c_i}} \mdist(\pi^{-1}(c_1))\ldots\mdist(\pi^{-1}(c_q)).$$

We can disregard $f$ which is not 1-1 or onto because for all such $f$,  $\forall_{i=1}^q f(m_i)=\pi(m_i)$ will not hold. This is useful so that $f^{-1}$ is well defined. This means the condition $\forall_{i=1}^q f(\pi^{-1}(m_i))=c_i$ is equivalent to $\forall_{i=1}^q \pi^{-1}(m_i)=f^{-1}(c_i)$. Now we have

\begin{align*}
   \hatE&=\argmax_f \sum_{c_1, \ldots,c_q\in \msgspace }\sum_{\substack{\pi\in\Perm \\\forall_{i=1}^q \pi^{-1}(c_i)=f^{-1}(c_i)}} \mdist(f^{-1}(c_1))\ldots\mdist(f^{-1}(c_q))\\
    &=\argmax_f \textrm{ }  \mdist(f^{-1}(c_1))\ldots\mdist(f^{-1}(c_q)).
\end{align*}


The last equality holds because we can mazimize a sum by maximizing the summand. Thus the best function we can hope for is the one returned by $\advA^*$, so long as the adversary must be deterministic.

If $\advA$ can be nondeterministic, it still cannot beat $\advA^*$. This is because the probability of success is now over all sets of random coins that $\advA^*$ receives. The coins that it receives are based only on $q$, so if for any $q$, the probability of success for $\advA$ is greater than the probability of success of $\advA^*$, then there is one set of random coins for which the probability of success for $\advA$ is greater than the probability of success of $\advA^*$. This implies that for each $q$, there is a deterministic machine which beats $\advA^*$, which we have proven in the previous lemma is false. Thus $\advA^*$ is actually the best deterministic or nondeterministic machine for G1.
\end{proof}

To prove the theorem, we will define $\advB$ based on $\advA$. Remember that because $\advB$ is a $\PRP_\cipher$-adversary, it is given access to an oracle and is successful if it can decide if the oracle is a random permutation or not.
\begin{figure}[H]
\centering
\fpage{.22}{
		\underline{$\advB^\textrm{Fn}$}\\[1pt]
	%$\rho \getsr \Perm(\msgspace)$\\
    For $i = 1$ to $q$\\
    \myInd $M_i \get{\mdist} \msgspace$\\
    \myInd $C_i \gets \textrm{Fn}(M_i)$\\
    If $\exists i, j$, $M_i\neq M_j$, $C_i=C_j$\\ \myInd return 0\\
    $\hatE \getsr \advA(C_1,\ldots,C_q)$\\
    If $\forall_{i=1}^q \left(\hatE(M_i) = C_i\right)$\\ \myInd return 1\\ else\\ \myInd return 0
	}
\end{figure}

Here, $\textrm{F}_0$ is a random That is, $\advB$ returns 1 if and only if $\advA$ ``wins" its $\MRUMA$ game. This means that $\Prob{\PRP1_\cipher^\advA\Rightarrow 1}= \AdvMRUMA{\cipher,\mdist,q}{\advA}$ and
$\Prob{\PRP0_\cipher^\advA\Rightarrow1}=\Adv^{\textrm{G1}}_{\cipher,\mdist,q}(\advA)$
In our analysis of G1, we have shown that $\Adv^{\textrm{G1}}_{\cipher,\mdist,q}(\advA)\leq\Adv^{\textrm{G1}}_{\cipher,\mdist,q}(\advA^*)$. Thus we have

\begin{align*}
    \AdvPRP{\cipher}{\advB}&=\left| \AdvMRUMA{\cipher,\mdist,q}{\advA}-\Adv^{\textrm{G1}}_{\mdist,q}(\advA) \right|\\
    \AdvMRUMA{\cipher,\mdist,q}{\advA}&\leq\AdvPRP{\cipher}{\advB}+\Adv^{\textrm{G1}}_{\mdist,q}(\advA)\\
    \AdvMRUMA{\cipher,\mdist,q}{\advA}&\leq\AdvPRP{\cipher}{\advB}+\Adv^{\textrm{G1}}_{\mdist,q}(\advA^*).
\end{align*}

However, because of the nature of $\advA^*$, it will be correct with the same probability given the same messages no matter what encryption function is used, so long as the ciphertexts of different messages are different. If the ciphertexts of different messages are the same, we automatically return 0, as the function is clearly not a random permutation. This only increases the advantage of $\advB$. Thus 
$$ \AdvMRUMA{\cipher,\mdist,q}{\advA} \le 
        \AdvMRUMA{\cipher,\mdist,q}{\advA^*} + \AdvPRP{\cipher}{\advB}.$$

\end{proof}

\paragraph{Frequency analysis in practice}
In actuality, the assumptions we have made for \thref{freq-opt} do not necessarily hold. For example the adversary likely does not have access to the exact probability distribution of any column in any database. Thus it is unclear whether this attack should work in practice. However, there have been simulated case studies which have shown the effectiveness of $\advA^*$ despite practical issues.

Generally, these case studies use machine learning techniques on some representative dataset, such as healthcare data or even a breach dataset, to create a message probability distribution $\hat p_m$ which estimates $\mdist$. Then they run $\advA^*$ with $\hat p_m$ many times on some independent dataset to find its the success rate.

The success of such attacks has so far depended on factors such as the size and uniformity of the message space. If a message space is small and very nonuniform then the attack should work better. For example, if the message is 0 with 99\% probability and 1 with 1\% probability, it should be obvious with enough samples which ciphertext decrypts to which bit. The closer the probabilities move to 50\% each, or the more messages in the message space, the less clear it will be.

In actual experiments, this is exactly what happened. In a hospital dataset in which most columns had few possibilities and were nonuniform, 100\% of the deterministically encrypted values were recovered \cite{Bindschaelder2018tao}. Similarly, \cite{Naveed2015inference} saw that attributes like sex and whether a patient died during their stay were much more easily recoverable than age or admission month, which have more values and are more uniform.

\paragraph{Problem 5.1 (From Boneh-Shoup book \cite{BonehShoupBook})} Suppose we are given a block cipher $(E,D)$ operating
on domain $X$ . We want a block cipher $(E', D')$ that operates on a smaller domain $X'\subseteq X$ . Define $(E', D')$ as follows:

\begin{align*}
   E'(k, x) := \myInd &y \leftarrow E(k, x)\\
    &\textrm{while } y \not\in X' \textrm{ do}: y \leftarrow E(k, y)\\
    &\textrm{output }  y 
\end{align*}

$D'(k, y)$ is defined analogously, applying $D(k, ·)$ until the result falls in $X'$. Clearly $(E', D')$ are defined on domain $X'$.

\begin{enumerate}
    \item With $t := |X|/|X'|$, how many evaluations of $E$ are needed in expectation to evaluate $E'(k, x)$ as a function of $t$? You answer shows that when $t$ is small (e.g., $t \leq 2$) evaluating $E'(k, x)$ can be done efficiently.
    
    \item Show that if $(E,D)$ is a secure block cipher with domain $X$ then $(E, D')$ is a secure block cipher with domain $X'$. Try proving security by induction on $|X |-|X'|$.
\end{enumerate}

\newpage
%%%%%%%%%%%%%%%%%%%%%%%%%%%%%%%%%%%%%%%%%%%%%%%%%%%%%%%%%%%%%%%%%%%%%%%%%%%%%%%%
\section{Tweakable Ciphers}
\label{sec:freqanalysis}



\begin{figure}
\fpage{.22}{
		\underline{$\MRUMA^{\advA}_{\cipher,\mdist,q}$}\\[1pt]
		$K \getsr \keyspace$\\
    For $i = 1$ to $q$\\
    \myInd $M_i \get{\mdist} \msgspace$\\
    \myInd $C_i \gets E_K(M_i)$\\
		$\hatE \getsr \advA(C_1,\ldots,C_q)$\\
    Ret $\forall_{i=1}^q \left(\hatE(M_i) = C_i\right)$
	}
\end{figure}

\begin{figure}
\fpage{.22}{
		\underline{$\G1$}\\[1pt]
		$\rho \getsr \Func(\msgspace)$\\
     For $i = 1$ to $q$\\
    \myInd $M_i \get{\mdist} \msgspace$\\
    \myInd $C_i \gets \rho(M_i)$\\
    $\hatE \getsr \advA(C_1,\ldots,C_q)$\\
    Ret $\forall_{i=1}^q \left(\hatE(M_i) = C_i\right)$
	}
\end{figure}

\fpage{.40}{
		\underline{$\advA^*(C_1,\ldots,C_q)$}\\[1pt]
    Let $c$ be number of unique ciphertexts in $C_1,\ldots,C_q$\\
    Let $\tildeC_1,\ldots,\tildeC_c$ be unique ciphertexts\\
    Let $N_{\tildeC_i}$ be number of occurences of $\tildeC_i$\\
    $\hatE \gets \argmax_f \prod_{i=1}^{c} \mdist\left(f^{-1}(\tildeC_i)\right)^{N_{\tildeC_i}}$\\
    Ret $\hatE$
	}


\bnm
   \AdvMRUMA{\cipher,\mdist,q}{\calA} = \Prob{\MRUMA^\advA_{\cipher,\mdist,q}\Rightarrow\true}
\enm


\begin{theorem}
Let $\cipher$ be a cipher, $\mdist$ a message distribution, and $q>0$. Let
$\advA^*$ be the frequency analysis $\MRUMA_{\cipher,q}$-adversary and $\advA$
be some $\MRUMA_{\cipher,q}$-adversary. Then we give a
$\PRF_\cipher$-adversary $\advB$ such that
\bnm
  \AdvMRUMA{\cipher,\mdist,q}{\advA} \le 
        \AdvMRUMA{\cipher,\mdist,q}{\advA^*} + \AdvPRF{\cipher}{\advB}
\enm
$\advB$ makes $q$ queries and runs in time 
$\Time(\advA) + 2q+q\cdotsm\Time(\mdist)$.
\end{theorem}


\begin{align*}
  &\argmax_f \CondProb{C_1,\ldots,C_q}{f} \\
  &\myInd\myInd  = \argmax_f \prod_{i=1}^q \CondProb{C_i}{M_1 = f^{-1}(C_1),\ldots,M_q = f^{-1}(C_q)}\\
  &\myInd\myInd = \argmax_f \prod_{i=1}^{c} \mdist(f^{-1}(\tilde{C}_i))^{N_{\tilde{C}_i}}
\end{align*}


\bnm
\tweakCipher\Colon\keyspace\times\tweakspace\times\msgspace \rightarrow \msgspace
\enm

Let $\Perm(\tweakspace,\msgspace$ be set of all functions
$\tweakspace\times\msgspace\rightarrow\msgspace$ for which 

\hfpages{.15}{
		\underline{$\TPRP1_{\tweakCipher}^\advA$}\\
		$K \getsr \keyspace$\\
		$b' \getsr \advA^\Fn$\\
		Return $b'$\medskip
		
		\underline{$\Fn(T,M)$}\\[1pt]
		Return $\tweakE_K(T,M)$
	}{
		\underline{$\TPRP0_{\tweakCipher}^\advA$}\\
		$\tweakpi \getsr \Perm(\tweakspace,\msgspace)$\\
		$b' \getsr \advA^\Fn$\\
		Return $b'$\medskip
		
		\underline{$\Fn(T,M)$}\\[1pt]
		Return $\tweakpi(T,M)$
	}

\bnm
\AdvTPRP{\tweakCipher}{\advA} = \left|\Prob{\TPRP1^\advA\Rightarrow1} - \Prob{\TPRP0^\advA\Rightarrow1} \right|
\enm

\fpage{.25}{
\underline{$\REAL^\advA_{\SE}$}\\[1pt]
$K \getsr \kg$\\
$b' \getsr \advA^{\EncOracle}$\\
Ret $b'$\medskip

\underline{$\EncOracle(M)$}\\
$C \getsr \enc_K(M)$\\
Ret $C$
}


\fpage{.25}{
\underline{$\RAND^\advA_{\SE}$}\\[1pt]
$b' \getsr \advA^{\EncOracle}$\\
Ret $b'$\medskip

\underline{$\EncOracle(M)$}\\
$C \getsr \bits^{\ctxtlen(M)}$\\
Ret $C$
}


\bnm
\AdvROR{\SE}{\advA} = \left|\Prob{\REAL_{\SE}^\advA\Rightarrow 1} - \Prob{\REAL_{\SE}^\advA\Rightarrow1} \right| 
\enm




\newpage
%%%%%%%%%%%%%%%%%%%%%%%%%%%%%%%%%%%%%%%%%%%%%%%%%%%%%%%%%%%%%%%%%%%%%%%%%%%%%%%%
\section{Randomized and Nonce-Respecting Symmetric Encryption}
\label{sec:symenc}

Given some symmetric encryption scheme, in what sense is it secure?  To answer
this question in a formal manner, we give several security conditions below.

\subsection{Security Conditions}

\paragraph{Real-or-Random Indistinguishability (\INDRAND or \ROR)}

\fpage{.25}{
  \underline{$\RORreal^{\advA}_{\SEscheme}$}\\[1pt]
  $K \getsr \kg$ \\
  $b' \getsr \advA^{\CEnc}$ \\
  return $b'$ \\ \\
  \underline{$\SEenc(M)$} \\
  $C \getsr \enc_{K}(M)$ \\
  return $C$
}
\fpage{.25}{
  \underline{$\RORrand^{\advA}_{\SEscheme}$}\\[1pt]
  $b' \getsr \advA^{\SEenc}$ \\
  return $b'$ \\ \\ \\
  \underline{$\SEenc(M)$} \\
  $C \getsr \{0,1\}^{\ctxtlen(\absv{M})}$ \\
  return $C$
}

Intuitively, this condition captures the notion that ciphertexts ``look'' like
random bits under chosen-plaintext attacks. In the figure above, two
games are defined: in $\RORreal$, the oracle for the adversary actually
encrypts the adversary's chosen plaintexts, while in $\RORrand$ the oracle
just generates a random bitstring. The advantage for the adversary $\advA$ given
a symmetric encryption scheme $\SEscheme$ is

\bnm
\AdvROR{\SE}{\advA} =
\absv{\Prob{\RORreal_{\SE}^{\advA} \Rightarrow 1} - 
      \Prob{\RORrand_{\SE}^{\advA} \Rightarrow 1}}
\enm

Thus the adversary has high advantage when it can reliably distinguish the real
or random worlds.

\paragraph{Indistinguishability under Chosen-Plaintext Attack (\INDCPA)}


\fpage{.25}{
\underline{$\INDCPA^\advA_{\SE}$}\\[1pt]
$K \getsr \kg$\\
$b \getsr \bits$\\
$b' \getsr \advA^{\EncOracle}$\\
Ret $b = b'$\medskip

\underline{$\EncOracle(M_0,M_1)$}\\
If $|M_0| \ne |M_1|$ then\\
\myInd Ret $\bot$\\
$C \getsr \enc_K(M_b)$\\
Ret $C$
}

This condition captures the notion that, given a ciphertex, an adversary can't
infer which of two chosen generated the ciphertext under encryption with a
randomly sampled key. This means that ciphertexts don't leak information about
their messages. In the figure above, a secret challenge bit is drawn randomly,
and the adversary's oracle uses it to determine which one of two plaintexts
chosen by the adversary is encrypted and returned.  Because the adversary can
correctly guess the challenge bit with probability $1/2$ just by randomly
drawing from $\{0,1\}$, the advantage for adversary $\advA$ given symmetric
encryption scheme $\SEscheme$  is scaled as follows:

\bnm
\AdvINDCPA{\SE}{\advA} = 2\cdotsm\Prob{\INDCPA_\SE^\advA\Rightarrow\true} - 1
\enm

Thus the adversary has high advantage if it can reliably guess the challenge
bit.

Alternatively, if we define two games $\INDCPA1$ and $\INDCPA0$ where the
challenge bit is set to 1 and 0 respectively, 

\bnm
\AdvINDCPA{\SE}{\advA} = 
    \left|\Prob{\INDCPA1_{\SE}^\advA\Rightarrow 1} - \Prob{\INDCPA0_{\SE}^\advA\Rightarrow1} \right| 
\enm

\paragraph{Simulation-based security (\INDSIM)}

\fpage{.25}{
\underline{$\INDSIM1^\advA_{\SE}$}\\[1pt]
$K \getsr \kg$\\
$b' \getsr \advA^{\EncOracle}$\\
Ret $b'$\medskip

\underline{$\EncOracle(M)$}\\
$C \getsr \enc_K(M)$\\
Ret $C$
}

\fpage{.25}{
\underline{$\INDSIM0^{\advA,\simu}_{\SE}$}\\[1pt]
$b' \getsr \advA^{\EncOracle}$\\
Ret $b'$\medskip

\underline{$\EncOracle(M)$}\\
$C \getsr \simu(|M|)$\\
Ret $C$
}

Finally, this condition captures the notion that ``having ciphertext is as good
as not having ciphertext'' --- more specifically, ciphertexts of chosen
plaintexts of random keys are indistinguishable from the output of a simulator
that only knows about the length of the plaintext. In the figure above, two
games are defined: in $\INDSIM1$, the oracle for the adversary actually
encrypts the adversary's chosen plaintexts, while in $\INDSIM0$ the oracle
returns the output of simulator $\simu$. The advantage for the adversary
$\advA$ given symmetric encryption scheme $\SEscheme$ is

\bnm
\AdvINDSIM{\SE,\simu}{\advA} = 
    \left|\Prob{\INDSIM1_{\SE}^\advA\Rightarrow 1} - \Prob{\INDSIM0_{\SE,\simu}^\advA\Rightarrow1} \right| 
\enm

Thus the adversary has high advantage if it can distinguish between
actual ciphertexts and simulator output.

\subsection{Reductions and Separations}
\label{sec:randomenc-reduct}

We have discussed three security conditions: \INDRAND, \INDCPA, and \INDSIM.
What is the relationship between these? It turns out that \INDRAND is
strictly stronger than \INDCPA and \INDSIM, and that \INDCPA and \INDSIM are
equivalent. We give a series of reductions and counterexamples to prove this.

\subsubsection*{$\INDRAND \Rightarrow \INDCPA$}

\begin{theorem}
Let $\SE$ be a symmetric encryption scheme. Let $\advA$ be any
$\INDCPA_\SE$-adversary making at most $q$ queries. 
We give an $\ROR_\SE$-adversary $\advB$ such that
\bnm
  \AdvINDCPA{\SE}{\advA} \le 2\cdotsm\AdvROR{\SE}{\advB}
\enm
Adversary $\advB$ makes at most $q$ 
queries and runs in time that of $\advA$.
\end{theorem}

% commenting this out in case tom wants to put it back in
%
% \tnote{We'll possibly introduce the following at some point, but didn't need it
% here yet.}
% We sometimes use $\bigO$ notation to hide small values that can be derived from
% proofs, but don't matter to the interpretation of the theorem. If we were to do
% an asymptotic treatment, this would correspond to hiding constants, hence the
% abuse of notation.  Thus in above theorem we would replace $q+3$ with
% $\bigO(q)$. 

\paragraph{Proof.}

Construct $\advB$ as follows: take $\advA$ and provide it an $\SEencsim$
oracle with the same signature as $\advA$'s $\SEenc$ oracle (i.e., it takes two
messages $M_0, M_1$ as input and returns some ciphertext) that sends to
$\advB$'s oracle one of two messages according to some challenge bit $b$.

\fpage{.20}{
\underline{$\advB^{\Enc}$}\\[1pt]
$b \getsr \bits$\\
$b' \getsr \advA^\EncSim$\\
If $(b = b')$ then Ret 1\\
Ret 0\medskip

\underline{$\EncSim(M_0,M_1)$}\\
Return $\Enc(M_b)$
}

Notice that that $\advB$ playing the $\RORreal$ game is equivalent to 
$\advA$ playing the $\INDCPA$ game, such that

\bnm
\Prob{\RORreal_{\SEscheme}^{\advB} \Rightarrow 1} =
\Prob{\INDCPA_{\SEscheme}^{\advA} \Rightarrow \true}
\enm

Also notice that for $\advB$ playing the $\RORrand$ game, its oracle returns
random bitstrings instead of actual ciphertext such that $\advA$, used by
$\advB$, gets no information about the scheme whatsoever --- the bitstrings
are generated independently of the value of $b$. This implies that $\advA$
is making a random guess (``flipping a coin'') regarding the value of $b$,
thus $\Prob{\RORrand_{\SEscheme}^{\advB} \Rightarrow 1} = 1/2$.

With these two facts and some elementary algebra, we get

\begin{align*}
\AdvROR{\SE}{\advB} 
    &= \left|\Prob{\ROR1_\SE^\advB\Rightarrow 1} -
                                \Prob{\ROR0_\SE^\advB\Rightarrow 1}\right|\\
    &= \left|\Prob{\INDCPA_\SE^\advA\Rightarrow\true} - \frac{1}{2}\right|\\
    &= \left|\frac{1}{2} +
    \frac{1}{2}\cdot\AdvINDCPA{\SE}{\advA} - \frac{1}{2}\right|\\
    &= \frac{1}{2}\cdot\AdvINDCPA{\SE}{\advA}
\end{align*}

as needed. $\blacksquare$

\subsubsection*{$\INDRAND \Rightarrow \INDSIM$}

\begin{theorem*}
Let $\SE$ be a symmetric encryption scheme. We give a simulator $\simu$ such
that for any  an
$\INDSIM_\SE$-adversary $\advA$, we can give 
a $\ROR_\SE$-adversary $\advB$ such that
\bnm
  \AdvINDSIM{\SE,\simu}{\advA} \le \AdvROR{\SE}{\advB} \;.
\enm
Adversary $\advB$ runs in time that of $\advA$ and makes the same number of
queries. Simulator $\simu$ requires just two operations.
\end{theorem*}

\paragraph{Proof.}

Construct $\simu$ and $\advB$ as follows:

\fpage{.15}{
\underline{$\simu(\ell)$}\\
$C \getsr \bits^{\ctxtlen(\ell)}$\\
Ret $C$
}

\fpage{.15}{
\underline{$\advB^\Enc$}\\
$b' \getsr \advA^{\EncSim}$\\
Ret $b'$\medskip

\underline{$\EncSim(M)$}\\
Return $\Enc(M)$
}

The intuition is that the simulator draws random bits much like the
$\RORrand$ game and thus $\advA$  essentially is playing the
$\RORreal$ and $\RORrand$ games respectively for
$\INDSIM1$ and $\INDSIM0$, such that, given $\advB$ is just an elementary
wrapper over $\advA$,
$\Prob{\RORreal_{\SEscheme}^{\advB}} = \Prob{\INDSIM1_{\SEscheme}^{\advA}}$
and
$\Prob{\RORrand_{\SEscheme}^{\advB}} = \Prob{\INDSIM0_{\SEscheme}^{\advA}}$.
Thus

\begin{align*}
  \AdvROR{\SEscheme}{\advB} &=
  \absv{\Prob{\RORreal^{\advB}_{\SEscheme}} - \Prob{\RORrand^{\advB}_{\SEscheme}}} \\
  &= \absv{\Prob{\INDSIM1^{\advA}_{\SEscheme}} - \Prob{\INDSIM0^{\advA}_{\SEscheme}}} \\
  &= \AdvINDSIM{\SEscheme}{\advA} \\
  &\geq \AdvINDSIM{\SEscheme}{\advA}
\end{align*}

as needed. $\blacksquare$

\subsubsection*{$\INDCPA \not\Rightarrow \INDRAND$}

Intuitively, a scheme with ciphertexts indistinguishable from each other
doesn't necessarily have ciphertexts indistinguishable from random bits.
We show a separation by constructing a scheme $\overline{\SEscheme}$ and
$\INDRAND$-adversary $\advA$ such that for any $\INDCPA$-adversary $\advB$,

\bnm
\AdvROR{\overline{\SEscheme}}{\advA} \geq
\AdvINDCPA{\overline{\SEscheme}}{\advB}
\enm

\paragraph{Proof.}

Pick a scheme $\SEscheme = (\textbf{kg},\textbf{enc},\textbf{dec})$
such that for some $0 < p < 1$, $\adv_{\SEscheme}^{\INDCPA}(\advB) \leq p$
for any $\INDCPA$-adversary $\advB$.
Construct scheme
$\overline{\SEscheme} = (\textbf{kg},\overline{\textbf{enc}},\overline{\textbf{dec}})$
by padding ciphertexts generated by $\SEscheme$ with 0s. The width of the
padding, given by $n$, is calculated from $p$: specifically, $n$ must be picked
such that $n \geq f(p)$ for function $f$ defined below.
$\overline{\textbf{enc}}$ is defined below as well.
$\overline{\textbf{dec}}$ strips the padding from the ciphertext before
decrypting with $\textbf{dec}$.

\fpage{.25}{
  \underline{$\overline{\enc}_K(M)$}\\[1pt]
  $C \getsr \enc_K(M)$ \\
  return $0^n \parallel C$
}

Notice that the padding is the same for all ciphertexts, so
$\INDCPA$-adversaries does not learn any additional information from the
padding. This implies that for all $\INDCPA$-adversaries $\advB$,

\bnm
\AdvINDCPA{\overline{\SEscheme}}{\advB} = \AdvINDCPA{\SEscheme}{\advB}
\enm

Next we construct a $\INDRAND$-adversary $\advA$ that uses its $\Enc$ oracle to
encrypt a random bitstring and then checks whether the first $n$ bits of the
ciphertext is 0s; if it does, $\advA$ guesses it is in the real world.

\fpage{.25}{
  \underline{$\advA^{\SEenc}$}\\[1pt]
  $M \getsr \{0,1\}$ \\
  $C \gets \SEenc(M)$ \\
  if $C[1..n] = 0^n$ then \\
  \ind return $1$ \\
  else \\
  \ind return $0$
}

By construction of $\overline{\SEscheme}$, if $\advA$ is playing the
$\RORreal$-game then it always guesses it is in the real world.
If $\advA$ is playing the $\RORrand$-game, with probability $1 / 2^n$
(the probability of drawing a random bitstring whose first $n$ bits is all 0s)
it guesses that it is in the real world. Thus 

\bnm
\adv_{\overline{\SEscheme}}^{\INDRAND}(\advA) = 1 - \frac{1}{2^n}
\enm

Recall that $n$ is picked such that $n \geq f(n)$, where intuitively $f$ is a
lower bound on the width of padding required so that $\advA$ obtains high enough
advantage over any $\advB$. Define $f$ as

\bnm
f(p) \geq \log_2(1 - p)
\enm

We arrive at the definition of $f$ by solving for the following inequality:

\begin{align*}
  p &\leq 1 - 1 / 2^{f(p)} \\
  1 / 2^{f(p)} &\leq 1 - p \\
  \log_2(1 / 2^{f(p)}) &\leq \log_2(1 - p) \\
  - f(p) &\leq \log_2(1 - p) \\
  f(p) &\geq \log_2(1 - p)
\end{align*}

Putting it all together, for any $\INDCPA$-adversary $\advB$ we have

\bnm
  \AdvINDCPA{\overline{\SEscheme}}{\INDCPA}{\advB}
  = \AdvINDCPA{\SEscheme}{\advB} \\
  \leq p \\
  \leq 1 - 1 / 2^{f(p)} \\
  \leq 1 - 1 / 2^n \\
  = \AdvROR{\overline{\SEscheme}}{\advA}
\enm

as needed. $\blacksquare$

\subsubsection*{$\INDSIM \not\Rightarrow \INDRAND$}

This separation immediately follows from composing the reduction
$\INDCPA \Rightarrow \INDSIM$ (below) and separation
$\INDCPA \not\Rightarrow \INDRAND$.

\subsubsection*{$\INDCPA \Rightarrow \INDSIM$}

Let $\SEscheme$ be a symmetric encryption scheme. There is
a simulator $\simu$ such that for all $\INDSIM$-adversaries $\advA$ there is
a $\INDCPA$-adversary $\advB$ such that 

\bnm
\AdvINDSIM{\SEscheme}{\advA} \leq \AdvINDCPA{\SEscheme}{\advB}
\enm

\paragraph{Proof.}

We construct $\advB$ and $\simu$ as shown below.

\fpage{.25}{
  \underline{$\advB^{\Enc}$}\\[1pt]
  $a \gets \advA^{\EncSim}$\\
  return $a$\\
  \\
  \underline{$\EncSim(M)$}\\[1pt]
  $r \getsr \{0,1\}^{\absv{M}}$\\
  return $\Enc(r, M)$
}
\fpage{.25}{
  \underline{$\simu(\ell)$}\\[1pt]
  $r \getsr \{0,1\}^{\clen(M)}$\\
  return $r$
}


The intuition is that $\Enc$ chooses to encrypt either the plaintext chosen
by $\advA$ or a random bitstring. The former corresponds to the $\INDSIM1$
game while the latter corresponds to the $\INDSIM0$ game. The correspondence
for the former follows immediately because it is just encrypting the chosen
plaintext. The correspondence for the latter follows because the ciphertext
for a random bitstring can be treated as a random bitstring itself;
it has no information about the plaintext chosen by $\advA$ since it is
randomly and independently generated from the chosen plaintext.

To arrive at the advantage of $\advB$, we first calculate the probability
that it will win the $\INDCPA$ game, where $b$ is the secret challenge bit:

\begin{align*}
  \Prob{\INDCPA^{\advB}_{\SEscheme} \Rightarrow \ctrue}
  &= \Prob{b = 0 \wedge a = 0} + \Prob{b = 1 \wedge a = 1} \\
  &= \Prob{b = 0}\Prob{a = 0 \mid b = 0} + \Prob{b = 1}\Prob{a = 1 \mid b = 1} \\
  &= 1/2(1 - \Prob{\INDSIM0^{\advA}_{\SEscheme} \Rightarrow 1})
     + 1/2(\Prob{\INDSIM1^{\advA}_{\SEscheme} \Rightarrow 1}) \\
  &= 1/2(1 - \Prob{\INDSIM0^{\advA}_{\SEscheme} \Rightarrow 1}
          + \Prob{\INDSIM1^{\advA}_{\SEscheme} \Rightarrow 1})
\end{align*}

Thus

\begin{align*}
  \AdvINDCPA{\SEscheme}{\advB} &=
    2 \Prob{\INDCPA^{\advB}_{\SEscheme} \Rightarrow \ctrue} - 1 \\
  &= 2(1/2(1 - \Prob{\INDSIM0^{\advA}_{\SEscheme} \Rightarrow 1}
          + \Prob{\INDSIM1^{\advA}_{\SEscheme} \Rightarrow 1})) -1 \\
  &= \Prob{\INDSIM1^{\advA}_{\SEscheme} \Rightarrow 1})) -
     \Prob{\INDSIM0^{\advA}_{\SEscheme} \Rightarrow 1} \\
  &= \AdvINDSIM{\SEscheme}{\advA} \\
  &\geq \AdvINDSIM{\SEscheme}{\advA} \\
\end{align*}

as needed. $\blacksquare$


\subsubsection*{$\INDSIM \Rightarrow \INDCPA$}

Let $\SEscheme$ be a symmetric encryption scheme. For any $\INDCPA$-adversary
$\advA$, we construct a $\INDSIM$-adversary $\advB$ such that

\bnm
\AdvINDCPA{\SEscheme}{\advA} \leq 2 \AdvINDSIM{\SEscheme}{\advB}
\enm

\paragraph{Proof.}

\fpage{.25}{
  \underline{$\simu(\ell)$}\\[1pt]
  $C \gets \{0,1\}^{\ctxtlen(\ell)}$ \\
  return $C$
}
\fpage{.25}{
  \underline{$\advB^{\SEenc}$}\\[1pt]
  $b \getsr \{0,1\}$ \\
  $b' \getsr \advA^{\SEencsim}$ \\
  return $b = b'$ \\ \\
  \underline{$\SEencsim(M_0, M_1)$}\\[1pt]
  return $\SEenc(M_b)$
}

This reduction is very similar to $\INDRAND \Rightarrow \INDCPA$.  The
constructed $\INDSIM$-adversary $\advB$ is essentially the same as the
constructed adversary for that reduction. Since $\advB$ playing the $\INDSIM1$
game is the same as $\advA$ playing the $\RORreal$ game,
$\Prob{\INDSIM1_{\SE}^{\advB} \Rightarrow 1} = \Prob{\RORreal_{\SE}^{\advA} \Rightarrow 1}$.
Notice that when $\advB$ plays the $\RORrand$ game its oracle returns random
bitstrings instead of actual ciphertext such that $\advA$, used by $\advB$,
gets no information about the scheme whatsoever.  Thus
$\Prob{\INDSIM0_{\SEscheme}^{\advB} \Rightarrow 1} = 1/2$.

Then

\begin{align*}
  \AdvINDSIM{\SEscheme}{\advB} &=
    \absv{\Prob{\INDSIM1^{\advB}_{\SEscheme} \Rightarrow 1} -
          \Prob{\INDSIM0^{\advB}_{\SEscheme} \Rightarrow 1}} \\
  &= \absv{\Prob{\INDCPA^{\advA}_{\SEscheme} \Rightarrow \true} - 1 / 2} \\
  &= \absv{1 / 2 + 1 / 2 \cdot \AdvINDCPA{\SEscheme}{\advA} - 1 / 2} \\
  &= 1 / 2 \cdot \AdvINDCPA{\SEscheme}{\advA} \\
  &\geq 1 / 2 \cdot \AdvINDCPA{\SEscheme}{\advA}
\end{align*}

as needed. $\blacksquare$

\subsection{Example: IND\$ for CTR mode}

\fpage{.25}{
\underline{$\Enc(K,M)$}\\
$\IV \getsr \bits^n$\\
$M_1,\ldots,M_m \getparse{n} M$\\
For $i = 1$ to $m$ do\\
\myInd $C_i \gets E_K(\IV \oplus \langle i\rangle) \oplus M_i$\\
Ret $\IV \concat C_1 \concat\cdots\concat C_m$\medskip

\underline{$\Dec(K,C)$}\\
If $|C| \le n$ then Ret $\bot$\\
$\IV,C_1,\ldots,C_m \getparse{n} C$\\
For $i = 1$ to $m$ do\\
\myInd $C_i \gets E_K(\IV \oplus \langle i\rangle) \oplus C_i$\\
Ret $M_1 \concat\cdots\concat M_m$
}

Notation $X_1,\ldots,X_m \getparse{n} X$ takes a string $X$ and partitions it
into as many full $n$-bit blocks as possible, and lets $X_m$ be remaining bits,
thus $|X_m| = |X| \bmod n$. Also recall that we assume that  $X \oplus Y$ for
$|X| > |Y|$ 
first truncates $X$ to $|Y|$ bits, and returns the exclusive or of that
truncated string with $Y$. We similarly define the operation when $|X| < |Y|$.

Given a block cipher $\varepsilon$, we can construct the $\CTR$-mode
symmetric encryption scheme as seen above. We then show that
this scheme is $\INDRAND$ secure.

\begin{theorem*}
Let $\CTR$ be the CTR-mode symmetric encryption scheme built using blockcipher
$\cipherE\Colon\bits^k\times\bits^n\rightarrow\bits^n$. Let $\advA$ be an
$\ROR_\CTR$-adversary making at most~$q$ queries each totaling at most~$\sigma$
blocks. Then we give an $\PRF_\cipher$-adversary $\advB$ such
that
\bnm
  \AdvROR{\CTR}{\advA} \le \AdvPRF{\cipher}{\advB} + \frac{2\sigma q^2}{2^n}
\enm
Adversary~$\advB$ runs in time that of $\advA$ plus $\bigO(q\sigma)$ and makes at most
$\sigma$ queries.
\end{theorem*}

\paragraph{Proof.}

The proof is a game-hopping argument, as seen below.

\hfpagessss{.23}{.23}{.23}{.23}{
\underline{$\G0$}\\
$K \getsr \bits^k$\\
$b' \getsr \advA^\EncOracle$\\
Ret $b'$\medskip

\underline{$\EncOracle(M)$}\\
$\IV \getsr \bits^n$\\
$M_1,\ldots,M_m \getparse{n} M$\\
For $i = 1$ to $m$ do\\
\myInd $C_i \gets E_K(\IV \oplus \langle i\rangle) \oplus M_i$\\
Ret $\IV \concat C_1 \concat\cdots\concat C_m$
}{
\underline{$\G1$}\\
$\rho \getsr \Func(n,n)$\\
$b' \getsr \advA^\EncOracle$\\
Ret $b'$\medskip

\underline{$\EncOracle(M)$}\\
$\IV \getsr \bits^n$\\
$M_1,\ldots,M_m \getparse{n} M$\\
For $i = 1$ to $m$ do\\
\myInd $C_i \gets \rho(\IV \oplus \langle i\rangle) \oplus M_i$\\
Ret $\IV \concat C_1 \concat\cdots\concat C_m$
}{
\underline{\fbox{$\G2$} \;\;\; $\G3$}\\
$b' \getsr \advA^\EncOracle$\\
Ret $b'$\medskip

\underline{$\EncOracle(M)$}\\
$\IV \getsr \bits^n$\\
$M_1,\ldots,M_m \getparse{n} M$\\
For $i = 1$ to $m$ do\\
\myInd $P_i \getsr \bits^n$\\
\myInd If $\TabT[\IV\oplus \langle i \rangle] \ne \bot$ then\\
\myInd\myInd $\bad\gets\true$\\
\myInd\myInd \fbox{$C_i \gets \TabT[\IV \oplus \langle i\rangle]$}\\
\myInd $\TabT[\IV \oplus \langle i\rangle] \gets P_i $\\
\myInd $C_i \gets P_i \oplus M_i$\\
Ret $\IV \concat C_1 \concat\cdots\concat C_m$
}{
\underline{$\G4$}\\
$b' \getsr \advA^\EncOracle$\\
Ret $b'$\medskip

\underline{$\EncOracle(M)$}\\
$\IV \getsr \bits^n$\\
$M_1,\ldots,M_m \getparse{n} M$\\
For $i = 1$ to $m$ do\\
\myInd If $\TabT[\IV\oplus \langle i \rangle] \ne \bot$ then\\
\myInd\myInd $\bad\gets\true$\\
\myInd $\TabT[\IV \oplus \langle i\rangle] \gets 1 $\\
\myInd $C_i \getsr \bits^n$\\
Ret $\IV \concat C_1 \concat\cdots\concat C_m$
}

First, notice that $G0$ game is the same as the $\RORreal$ game for $\INDRAND$
security, and thus

\bnm
\Prob{\RORreal^{\advA}_{\CTR} \Rightarrow 1} = \Prob{G0 \Rightarrow 1}
\enm

Next, notice that $G0$ corresponds to $\textup{PRF1}$ and $G1$ corresponds to
$\textup{PRF0}$, in the sense that in $G0$ the message block is XORed with the
encryption of the IV and the block index, while in $G1$ the message block is
XORed with the output of a random function with the IV and block index as input.
It is then trivial to construct a $\PRF$-adversary $\advB$ whose advantage 
is

\begin{align*}
  \absv{\Prob{G0 \Rightarrow 1} - \Prob{G1 \Rightarrow 1}} = \AdvPRF{\varepsilon}{\advB} \\
  \Prob{G0 \Rightarrow 1} \leq \Prob{G1 \Rightarrow 1} + \AdvPRF{\varepsilon}{\advB}
\end{align*}

Next, since in $G2$ the value in $\TabT$ is re-used in the event of a collision
and is randomly selected at each sampled point, it essentially acts a random
function, which implies
$\Prob{G1 \Rightarrow 1} = \Prob{G2 \Rightarrow 1}$.

Since $G2$ and $G3$ are identical-until-bad, by the fundamental lemma of
game playing we have

\bnm
\Prob{G2 \Rightarrow 1} \leq \Prob{G3 \Rightarrow 1} + \Prob{\bad_3}
\enm

The only difference between $G3$ and $G4$ is that $G3$ derives ciphertext
using one-time pads and $G4$ derives ciphertext from random strings.
The distribution of ciphertexts for one-time pads is uniform, thus making
it equal to the distribution of random strings. This implies that the adversary
should not be able to distinguish between the two, so
$\Prob{G3 \Rightarrow 1} = \Prob{G4 \Rightarrow 1}$.
Furthermore, notice that $\bad$ is set in the same conditions as $G3$ and $G4$,
so $\Prob{\bad_3} = \Prob{\bad_4}$. Then

\bnm
\Prob{G3 \Rightarrow 1} + \Prob{\bad_3} = \Prob{G4 \Rightarrow 1} + \Prob{\bad_4}
\enm

Next, since $G4$ assigns the ciphertext to random bitstrings,

\bnm
\Prob{G4 \Rightarrow 1} = \Prob{\RORrand^{\advA}_{\CTR} \Rightarrow 1}
\enm

To analyze the setting of $\bad_4$ we, first, observe that the choices of $\IV$
are independent of the adversary's queries. For query $i$, let $m_i$ be the
length in blocks of that query and $\IV_i$ be the value $\IV$ chosen.
Let $X_{i,j} = \IV_i \oplus \langle j\rangle$ for $1 \le j \le m_i$. Then we are
asking whether there are any collisions among 
\begin{align*}
    &\IV_1 , \IV_1 \oplus \langle 1\rangle, \ldots , \IV_1 \oplus \langle m_1\rangle \\
    &\IV_2 , \IV_2 \oplus \langle 1\rangle, \ldots , \IV_2 \oplus \langle m_2\rangle \\
    &\phantom{\IV_2 , \IV_2 \oplus \langle 1\rangle, } \vdots\\
    &\IV_q , \IV_q \oplus \langle 1\rangle, \ldots , \IV_q \oplus \langle m_q\rangle \\
\end{align*}
If $m_i < 2^n-1$ for all queries then we can ignore wraparound effects, and so
no collisions can occur on each row. Now consider any two pairs of rows $1 < i
< j$. We can consider $\IV_i$ to be fixed, and so a collision across rows
occurs if $\IV_j \in \{\IV_i + \alpha \;:\; -\sigma +1 \le \alpha \le
\sigma-1\}$. This occurs with probability $(2\sigma-1) / 2^n$. Since there are
at most $q^2$ such pairs, we have that the total probability of a collision is
at most $2q^2\sigma / 2^n$. See Lemma 5.18 of Bellare-Rogaway for more details.

Putting all of these together, we arrive at

\begin{align*}
  \AdvROR{\CTR}{\advA} 
    &= \left| \Prob{\REAL_\CTR^\advA} - \Prob{\RAND_\CTR^\advA}\right|\\
    &= \left| \Prob{\G0} - \Prob{\RAND_\CTR^\advA}\right|\\
    &\le \left| \Prob{\G1} + \AdvPRF{\cipher}{\advB} - \Prob{\RAND_\CTR^\advA}\right|\\
    &= \left| \Prob{\G2} + \AdvPRF{\cipher}{\advB} - \Prob{\RAND_\CTR^\advA}\right|\\
    &\le \left| \Prob{\G3} + \Prob{\bad_3} \AdvPRF{\cipher}{\advB} - \Prob{\RAND_\CTR^\advA}\right|\\
    &= \left| \Prob{\G4} + \Prob{\bad_4} + \AdvPRF{\cipher}{\advB} - \Prob{\RAND_\CTR^\advA}\right|\\
    &= \Prob{\bad_4} + \AdvPRF{\cipher}{\advB}\\
    &\le \AdvPRF{\cipher}{\advB} + \frac{2q\sigma^2}{2^n}
\end{align*}

as needed. $\blacksquare$


%\begin{align*}
%  \Prob{\bad_4} = \Prob{\wedge_i X_i \in T_i  \lor X_2 \in T_i \cdots \lor X_{
%\end{align*}


\newpage
%%%%%%%%%%%%%%%%%%%%%%%%%%%%%%%%%%%%%%%%%%%%%%%%%%%%%%%%%%%%%%%%%%%%%%%%%%%%%%%%
\section{Authenticated Encryption}
\label{sec:authenc}

\tnote{We'll possibly introduce the following at some point, but didn't need it
here yet.}
We sometimes use $\bigO$ notation to hide small values that can be derived from
proofs, but don't matter to the interpretation of the theorem. If we were to do
an asymptotic treatment, this would correspond to hiding constants, hence the
abuse of notation.  Thus in above theorem we would replace $q+3$ with
$\bigO(q)$. \\


\fpage{.20}{
\underline{$\advB^{\Enc}$}\\[1pt]
$b \getsr \bits$\\
$b' \getsr \advA^\EncSim$\\
If $(b = b')$ then Ret 1\\
Ret 0\medskip

\underline{$\EncSim(M_0,M_1)$}\\
Return $\Enc(M_b)$
}

\begin{align*}
\AdvROR{\SE}{\advB} 
    &= \left|\Prob{\ROR1_\SE^\advB\Rightarrow 1} -
                                \Prob{\ROR0_\SE^\advB\Rightarrow 1}\right|\\
    &= \left|\Prob{\INDCPA_\SE^\advA\Rightarrow\true} - \frac{1}{2}\right|\\
    &= \left|\frac{1}{2} +
    \frac{1}{2}\cdot\AdvINDCPA{\SE}{\advA} - \frac{1}{2}\right|\\
    &= \frac{1}{2}\cdot\AdvINDCPA{\SE}{\advA}
\end{align*}



\subsection{Active attacks on unauthenticated encryption schemes.}
We left off in the previous chapter understanding how to build chosen plaintext attack (CPA) secure encryption for randomized encryption schemes (i.e., schemes with key generation and an encryption algorithm that take a secret and make a random nonce or IV for every message to which the scheme is applied). In particular, we saw this with real or random (ROR) CPA games:

\begin{theorem}
Let $\SE$ be a symmetric encryption scheme. Let $\advA$ be any
$\INDCPA_\SE$-adversary making at most $q$ queries. 
We give an $\ROR_\SE$-adversary $\advB$ such that
\bnm
  \AdvINDCPA{\SE}{\advA} \le 2\cdotsm\AdvROR{\SE}{\advB}
\enm
Adversary $\advB$ makes at most $q$ 
queries and runs in time that of $\advA$.
\label{theorem:ror-cpa}
\end{theorem}

This is one of many ways to define CPA encryption.
%Recall IND-\$. 
The IND-CPA (i.e., Left-or-Right indistinguishability) game implies we cannot learn even a single bit about the underlying plaintext. We also saw IND\$ which is widely used, generally because it is easier to work with and strictly stronger. IND-SIM is not as widely used as it was introduced mainly to demonstrate simulator-based definitions. We will see that IND-CPA is not nearly strong enough for practical use, however, but that it is useful conceptually and as a technical building block. However, none of these CPA settings are strong enough in the face of active attacks, in which ciphertexts are \emph{mauled}. As we will see soon, active attacks to ciphertexts raise {\bf confidentiality} and {\bf integrity} issues with our current CPA settings.

Let's remind ourselves of the simplest CPA-secure scheme, CTR mode. 
CTR mode chooses a random IV and uses its block cipher as a one-time pad. It pads enough such that you can XOR the needed amount of bits with the message. We can prove CTR is secure in the ROR game, as long as the IV is never chosen in a way that it collides with the input to $E_k$. To minimize the risk of collision as CTR mode is used for many messages with the same key (as the birthday bound will become close to one), the key should be refreshed at regular intervals.
What is an active attack against CTR mode? Malleability attacks: we can change the ciphertext such that it decrypts correctly but to a different message that was not originally written. It is trivial to maul messages given a ciphertext even without the secret key $k$ because if you XOR in any fixed bitstring pattern to a ciphertext, decryption is performed the result will be the original message XOR'ed with that pattern. As an example, to flip the first bit of a message encrypted with CTR mode, flip the first bit of the ciphertext before it is submitted for decryption, and the result will see decryption performed correctly, but with the original message's first bit flipped in the result. There is nothing to protect CTR mode against message modification. We continue to lay the case for why active attacks on the encryption modes we have seen so far can be devastating in practice, through an example.

\begin{example}[Session handling and login requests to web services.] One important application of encryption on the web is in encrypting cookie values. When a login is performed to a website, for instance, Facebook, there is first an HTTP GET request which then sets an anonymous cookie in the client's browser. This allows the server to link one request to a subsequent one. 
When you click submit on a website's login button, this will initiate a POST request over HTTPS using the anonymous cookie (the channel to the server is encrypted over HTTPS, but the remote server still needs to see your password). If the authentication protocol of the server is done well, submitting login credentials to the server would results in the server returning a new cookie to identify you as authenticated. 
As one concrete example, there were serious problems with Facebook's authentication protocol that were not fixed until 2011. 
The length of a user's password could leak, despite HTTPS.
Even worse, users' session ID cookies were being sent in plaintext. You could hijack other accounts by sniffing some local network, extract it and then use it in a connection request to Facebook without needing to login. There was even a tool called Firesheep used to do this (a Firefox browser extension).
This example demonstrates that a bad way to do session IDs is to outsource (as the server) security-relevant state about the client in the cookie and store it on the client side. Storing the data on the client in plaintext would go against the security model of an untrusted client, so servers often encrypt the cookies. But bad libraries in the past have encrypted cookies with CTR mode. Sometimes permissions levels are included in that encrypted cookie, and use the cookie to check permissions levels on every subseuqent request to the server. So a malicious client could trivially XOR bits in the right spots (if you know the scheme) and could get arbitrary permissions. The fundamental misunderstanding here is that encryption modes like CTR don't provide message integrity.
\end{example}

CBC mode has trivial ``malleability'' (mauling) issues, too. Recall that in CBC mode, a random IV is chosen and is XORed with the first block of the message, then run through the cipher, and so on. %take the first ciphertext block to XOR with the next block message %is IND-\$ secure. 
How could we change bits in the first block of the message sent to the server? %Refer to \ref{sec:symenc}.
If we change any of the C0 bits, M1 will be modified even though it can still decrypt correctly. You could modify bits of C2, as well, but bits of the input M2 would be randomized.% --------------------

The encryption modes we have discussed so far (CTR, CBC) do not provide \emph{integrity}. We need a notion of \textbf{authenticated encryption} in order to attain this property. 
As mentioned before, there are also confidentiality risks with these modes. In CBC, it is possible to use malleability tricks and clever queries to recover all of the plaintext. \scribenote{Full attack as a homework problem?}
CBC mode handles messages with length a multiple of $n$ bits. We use \emph{padding} to make it work for arbitrary length messages. Padding checks often give rise to {\bf padding oracle attacks}.
A simple case of this problem: padding by one byte. When we decrypt, we decrypt with CBC mode and check if the padding byte is what it is supposed to be (zero) that the padding rules defined. If you give an adversary some ciphertext under encryption $E_k$, we show that you can recover one byte of plaintext data if given access to the decryption oracle. 
The decryption oracle in this case is as follows:

\vspace{0.2cm}\fpage{.30}{
Dec(K, C'):\\
M[1]' $\Vert$ M[2]' $\Vert$ P' = CBC-Dec(K,C')\\
If P' $\neq$ 0x00 then \\
\myInd   Return error\\
Else\\
\myInd    Return ok
}\\

If we give some adversary the ciphertexts C0,C1,C2 they can recover one byte of information about the plaintext by interacting with the decryption oracle. If the adversary submits C0,C1,C2 unmodified to the oracle, then it will return `ok'. But if we flip the last bit in the C1 block, then the decryption will return `error' because during decryption, when processing C2 and going backwards, the string when XORed with C1 gives a low byte of zero, but now that we've flipped C1 on the low bit, the P value is now 1 instead of 0. How do we learn a byte about the message? We could flip all of the bits until reaching the beginning of the padding.
%If we flip a bit in the second to least significant byte , we'd get back `ok'. 
The adversary could replace C2 with C1, and C1 with C0. C0 XORed with M1 is M1. 
%
We need to get the oracle to operate on confidential message bits. If we move the ciphertext blocks around, we can get the oracle to decrypt values that are dependent on the unknown message. In particular if we move C1 to C2's position, then the recipient thinks this is the padding byte, the low order byte, but it is actually message bits, which the oracle will tell us.
So more concretely, the adversary could send R,C0,C1 instead of C0,C1,C2 to the decryption oracle. When we decrypt, we'd get an error unless M1 low byte is $0^n$. But if we are unlucky, we could search over all byte values to keep trying to get an `ok' when the low byte of M1 = $i$. C0 XOR'ed $i$ XOR'ed the message byte will equal zero, so we'd know that the message byte equals $i$. This gives us a way to break confidentiality using an active mauling attack. 

CBC is actually much more vulnerable than in this example. But to understand why, we need to understand PKCS\#7 padding, which is often used in CBC mode in practice:
PCKCS\#7-Pad($M$)=$M||P||\dots||P$. In such a padding setting, there are $P$ repetitions of bytes encoding number of bytes padded. For a block length of 16 bytes, we never need more than 16 bytes of padding (10 10 \dots 10).

Possible paddings:\\
\vspace{0.2cm}\fpage{.30}{
01\\
02 02\\
03 03 03\\
04 04 04 04\\
\dots\\
FF FF FF FF \dots FF (could go up to 256).
}\\

Why would we want more padding than necessary to get to a multiple of the block length (i.e., the block offset)? This allows us to try to obfuscate the lengths of messages. This only prevents leaking some lower order bits about messages. But in general on the Web, this type of padding does not work very well. But what happens if the message ends in a padding sequence? How do we recover the original message from the padded message? Look at the last byte recovered from CBC mode and remove that much (the remaining prefix is the actual message).

CBC decryption with PKCS\#7 padding pseudocode (strict padding checks):\\
\vspace{0.2cm}\fpage{.30}{
Dec(K,C)\\
M[1] $\Vert$ ... $\Vert$ M[m] = CBC-Dec(K,C)\\
P = RemoveLastbyte(M[m])\\
while i < int(P):\\
\myInd    P' = RemoveLastByte(M[m])\\
\myInd    If P' != P then Return error\\
\myInd    i++\\
Return ok
}\\
``ok'' or ``error'' in this setting can be stand-ins for some other behavior:
\begin{enumerate}
    \item Passing data to application layer (web server)
    \item Returning other error code (if padding fails)
\end{enumerate}

Let's attack CBC mode with PKCS\#7 padding. Because of the way we have defined padding, we can recover entire messages using a variant of the type of padding oracle attack we saw previously. Now we can trick the algorithm into thinking aribtrary bytes in the message block are padding bytes.
We have a padding decryption oracle as seen above in pseudocode. If we (the adversary) submit R,C0,C1, then as before we will get an ``error'' from the padding decryption oracle. But if we continue to XOR C0 from 1 to $i$, like before we will eventually get an `ok' from the oracle. Most likely, this corresponds to finding a relationship between $i$ and the actual message byte underneath C0 such that it looks like the padding byte P=01.
The low byte M1 = i XOR 0. \scribenote{Homework problem: Why?} 
Let X[1] = D(K, C1). Then C0[16] XOR X[1][16] = M[1][16], then C0[16] XOR $i$ XOR X[1][16] = 01, so M[1][16] XOR $i$ = 01.

That was the most likely valid padding. But it could also be the case that M[1][16] XOR $i$ = 02 and it just so happened that the next byte over to the left also came out 02 during decryption. Submit another query to rule this out: flip the bits of the 15th byte in the decrypted value. XOR in $i$ to C0's low byte, then go over one and flip a byte. Would get an ``error'' if it had been 02, since that's not a valid padding.
So we know M[1][16] and we can find the offset to XOR into the low byte of C0 to ensure that the first padding byte during decryption is 02, then we can do a search for the second byte of message underneath C1 because it will be looking for 02 and ``error'' until it finds an $i$ that decrypts correctly. 

As we have seen, our ability to provide message confidentiality is compromised in the face of active attacks on CBC mode. Could we change the decryption implementation to not use strict padding checks? We need some padding checks to tell server what to return to application layer (otherwise the end of the variable-length message won't be known). For correctness the server must look at the first byte. Some implementations can make padding oracle attacks harder to do, but it often makes decryption less efficient. What servers typically do is to not return errors, or obfuscate when padding errors occured versus a correct decryption, but this is really hard to do because of timing side channels.

Long litany of CCA attacks on CBC: Vaudenay (2001; 10's of chosen ciphertexts, recovers message bits from a cipertext. Called ``padding oracle attack''), Canvel et al. (2003; shows how to use Vaudenay's ideas against TLS), Degabriele,
Paterson, (2006; breaks IPsec encryption-only mode), Albrecht et al. (2009; plaintext recovery against SSH), Duong, Rizzo (2011; breaking ASP.net encryption), Jager, Somorovsky (2011; XML encryption standard), Duong, Rizzo (2011; ``beast'' attacks against TLS), AlFardan, Paterson (2012; attack against DTLS), AlFardan, Paterson (2013; lucky 13 attack against DTLS and TLS), Albrecht, Paterson (2016; lucky microseconds against Amazon's s2n library).

% -------------------
\subsection{Authenticated encryption security.}

Our encryption algorithms don't have confidentiality or integrity guarantees in the face of active attacks. In order to remedy this, we need new schemes.
We will show a first variant of this as a way to build authenticated encryption security, which builds off of our ROR CPA notion.
In that notion, we had ROR for an encryption oracle versus a random bits oracle.
Now in this setting, we add a decryption oracle to which you can submit ciphertexts:\\

\hfpagess{.15}{.15}{
\underline{$\RORCCA1^\advA_{\SE}$}\\[1pt]
$K \getsr \kg$\\
$b' \getsr \advA^{\EncOracle,\DecOracle}$\\
Ret $b'$\medskip

\underline{$\EncOracle(M)$}\\
$C \getsr \enc_K(M)$\\
$\calC \gets \calC \cup \{C\}$\\
Ret $C$\medskip

\underline{$\DecOracle(C)$}\\
If $C \in \calC$ then \\
\myInd Ret $\bot$\\
Ret $\dec_K(C)$
}{
\underline{$\RORCCA0^\advA_{\SE}$}\\[1pt]
$b' \getsr \advA^{\EncOracle,\DecOracle}$\\
Ret $b'$\medskip

\underline{$\EncOracle(M)$}\\
$C \getsr \bits^{\ctxtlen(|M|)}$\\
Ret $C$\medskip

\underline{$\DecOracle(C)$}\\
Ret $\bot$
}\\


In the real world, the decryption oracle just checks if a ciphertext was returned legitimately by the encryption oracle. But in our game, if it was then we ignore that case. Otherwise we decrypt and give the message. Why must there be a restriction in our first variant that the attacker can't send to the decryption oracle that you get from encryption? Then you'd have to add more logic about what ciphertexts are returned (via a table of decryptions). If not in table, return bottom. These definitions can be shown equivalent pretty straightforwardly. Intuitively, the adversary needn't submit a message they encrypt to the descryption oracle, since they already know the original message. In the ideal world, the decryption always outputs an error symbol. Intuitively, this means that there should be no way to produce a valid ciphertext such that it decrypts to anything other than $\bot$. And lastly, we define advantage in the usual way:\\
$\AdvRORCCA{\SE}{\advA} = \left|\Prob{\RORCCA1_\SE^\advA\Rightarrow1} - \Prob{\RORCCA0_\SE^\advA\Rightarrow1}\right|$\\

CBC mode will \emph{not} be secure under this notion. Why? As an aside, we can ``decompose'' ROR-CCA security into two separate notions: (1) ROR security, which we saw earlier, and (2) Cipertext integrity (CTXT), in which the adversary, given the encryption oracle, cannot construct \emph{any} new ciphertext that decrypts to some message. So ROR + CTXT <=> ROR-CCA.

Our new notion of ciphertext integrity just focuses on the ability to build encryption that doesn't decrypt to $\bot$. We show the CTXT game as follows:\\

\fpage{.15}{
\underline{$\CTXT^\advA_{\SE}$}\\[1pt]
$K \getsr \kg$\\
$\win \gets \false$\\
$\advA^{\EncOracle,\DecOracle}$\\
Ret $\win$\medskip

\underline{$\EncOracle(M)$}\\
$C \getsr \enc_K(M)$\\
$\calC \gets \calC \cup \{C\}$\\
Ret $C$\medskip

\underline{$\DecOracle(C)$}\\
If $C \in \calC$ then \\
\myInd Ret $\bot$\\
$M \gets \dec_K(C)$\\
If $M \ne \bot$ then \\
\myInd $\win\gets\true$\\
Ret $M$
}

\bnm
\AdvCTXT{\SE}{\advA} = \Prob{\CTXT_\SE^\advA\Rightarrow\true}\\
\enm

We have an encryption and decryption oracle. The goal of an adversary is to try to get $\DecOracle$ to return something other than $\bot$ to a cipertext that wasn't returned by the encryption oracle. So it turns out that ROR confidentiality and CTXT together imply ROR-CCA definition. We can show that for any adversary $\advA$ against the ROR-CCA security of any scheme, then we can build adversaries against ROR security, or another adversary against CTXT security such that the advantage of ROR-CCA is upper bounded by the sum of the advantages of the ROR-CPA and CTXT security, with some constants. Informally, a scheme that is ROR-CPA secure, and is CTXT secure, then it is ROR+CTXT secure. \scribenote{Homework problem: showing ROR implied by ROR-CCA, CTXT implied by ROR-CCA.} Note that in general, ROR is considered to mean ROR-CCA in practice, but we have used it to mean ROR-CPA in the lecture notes so far.

%We can define a variant of ROR-CCA which in this game could return
Let's see about building ROR-CCA using the CPA core OCB mode.
Recall the tweakable block ciphers (OCB-CCA) scheme. Running it to differentiate between different applications of the block cipher, with random nonces to create per-message randomness. The intuition being that if N never collides across different messages being used, then there are fresh tweaks being used, and no repetition within message blocks given the counter. But this is not CTXT secure. The decryption algorithm is as follows:
Reject if $|C|\neq 4n$; Decrypt via inverse of tweakable blockcipher; Return resulting $M1||M2||M3$. It is easy to build a ciphertext that decrypts validly against this. A CTXT attack would be: any ciphertext string of size $4n$, like $0^{4n}$ and it would decrypt. No error in any case unless it's the wrong length. Intuitively, we need redundancy in encryption to ensure the decryption oracle can reject good versus bad ciphertexts. One approach could be: let's assume 1.5n bit messages, then encode $0^n / 2 || M1$, $0^n / 2 || M2, 0^n / 2 || M3$. Then encrypt. So this makes the messages smaller blocks and then pad out the blocks with zeroes for each independent block. Then during decryption we need to check the zeroes; if so, we allow decryption, and if not, output $\bot$. Is this secure? Yes, up until bounds. If you query a random C0,C1,C2,C3 to the decryption oracle without getting any encryptions, what is the probability that it would decrypt properly? It would be roughly $1/ 2^{3n/2}$, which is the probability of three random $n$ over 2 bitstrings equalling all zeroes. Using this, one could make an attack to forge a decryption in $q / 2^{3n/2}$. Then query until you get a valid padding. Better than the birthday bound, but not very efficient because per-block of encryption, this approach is only processing half as many bits of plaintext.
%Other idea: add XORed random number in M1 (3:06pm). Might not work.
The idea that the people who came up with OCB realized was that the could add another block of the tweakable block cipher application, a simple redundancy check which XORs all the message blocks and get a value called the tag. Then decryption recomputes the tag and check that it's equal to the tag block's redundancy value. The tweak on a tag value needs to be different from other tweaks. If it does so, then this suffices to provide integrity. This approach is very efficient, as it only requires running the block cipher one more time. This can all be done in one pass. Rather than encrypting in one, and then authenticating. This is the core idea underlying OCB mode.
\newpage
%%%%%%%%%%%%%%%%%%%%%%%%%%%%%%%%%%%%%%%%%%%%%%%%%%%%%%%%%%%%%%%%%%%%%%%%%%%%%%%%
\section{Message Authentication}
\label{sec:msgauth}


A message authentication scheme $\MA = (\kg,\mtag,\ver)$ is a triple of
algorithms. Key generation is randomized and outputs a key. Message
authentication $\mac$ takes as input a key and message an outputs a tag.
Verification takes as input a key, a message, and a tag, and outputs a bit. 
\tnote{Fill in correctness conditions}



\fpage{.2}{
\underline{$\UFCMA_\MA^\advA$}\\[1pt]
$K \getsr \kg$\\
$(M^*,T^*) \getsr \advA^{\TagOracle}$\\
If $M^* \in \msgset$ then\\
\myInd Ret $\false$\\
Ret $\ver(K,M^*,T^*)$\medskip

\underline{$\TagOracle(M)$}\\
$\msgset \gets \msgset \cup \{M\}$\\
Ret $\mtag(K,M)$
}


\bnm
  \AdvUFCMA{\MA}{\advA} = \Prob{\UFCMA_\MA^\advA\Rightarrow 1} 
\enm


\fpage{.25}{
\underline{$\SUFCMA_\MA^\advA$}\\[1pt]
$K \getsr \kg$\\
$\win\gets\false$\\
$(M^*,T^*) \getsr \advA^{\TagOracle,\VerOracle}$\\
Ret $\win$\medskip

\underline{$\TagOracle(M)$}\\
$T \gets \mtag(K,M)$\\
$\pairset \gets \pairset \cup \{(M,T)\}$\\
Ret $T$\medskip 

\underline{$\VerOracle(M,T)$}\\
If $(M,T) \in \pairset$ then \\
\myInd Ret $\bot$\\
$b \gets \ver(K,M,T)$\\
If $b = 1$ then $\win\gets\true$\\
Ret $b$
}


\bnm
  \AdvUFCMA{\MA}{\advA} = \Prob{\UFCMA_\MA^\advA\Rightarrow 1} 
\enm

\begin{theorem*}
Let $F\Colon\bits^k\times\msgspace\rightarrow\bits^n$. Then for any
$\UFCMA_F$-adversary $\advA$ making $q$ queries, we give a $\PRF_F$-adversary $\advB$ such that
\bnm
  \AdvUFCMA{F}{\advA} \le \AdvPRF{F}{\advB} + \frac{1}{2^n} \;.
\enm
Adversary $\advB$ runs in time that of $\advA$ and makes $q+1$ queries.
\end{theorem*}

\bnm
F\Colon\bits^k\times\msgspace\rightarrow\bits^n
\enm


\fpage{.20}{
\underline{$\advA^{\Tag}$}\\[1pt]
$T_1 \gets \TagOracle(0^n)$\\
$M = 0^n\concat T_1$\\
Ret $(M,T_1)$
}


A keyed function $H\Colon\bits^k\times\msgspace\rightarrow\bits^n$ is
an $\epsilon$-almost universal (AU) hash function  if for all $M \ne M'$
\bnm
  \Prob{H_K(M) = H_K(M')} \le \epsilon
\enm
where the probability is over choice of $K$.

A computational AU hash function is one for 
\fpage{.10}{
\underline{$\cAU_H^\advA$}\\[1pt]
$M,M' \getsr \advA$\\
$K \getsr \bits^k$\\
If $M = M'$ then Ret $\false$\\
Ret $H_K(M) = H_K(M')$
}


\bnm
  \AdvcAU{H}{\advA} = \Prob{\cAU^\advA_H\Rightarrow\true}
\enm


\begin{theorem*}
Let $E\Colon\bits^k\times\bits^n\rightarrow\bits^n$ and $H$ be the CBC-MAC. 
Then for any $\cAU_H$-adversary $\advA$ outputting messages of length at most
$\sigma$ blocks, we give a $\PRF_E$-adversary $\advB$
such that
\bnm
  \AdvcAU{H}{\advA} \le \AdvPRF{E}{\advB} + \frac{\sigma^2}{2^n} \;.
\enm
Adversary $\advB$ runs in time that of $\advA$ and makes at most $2\sigma$
queries.
\end{theorem*}


\begin{theorem*}
Let $\SE$ be an SE scheme and $\MA$ be a MAC and $\EtM$ be the Encrypt-then-Mac
scheme built from them.  Let $\advA$ be an $\RORCCA_\EtM$-adversary making at most
$q$ queries. Then we give a $\ROR_\SE$-adversary $\advB_{\textrm{se}}$ 
and a $\UFCMA_\MA$-adversary  $\advB_\textrm{mac}$ 
such that
\bnm
  \AdvRORCCA{\EtM}{\advA} \le \AdvROR{\SE}{\advB_\textrm{se}} + 2\cdotsm\AdvUFCMA{\MA}{\advB_{\textrm{mac}}} \;.
\enm
Adversaries $\advB_\textrm{se}$ and $\advB_{\textrm{mac}}$ each run in time that of 
$\advA$ plus small overhead and each make at most $q$ queries.
\end{theorem*}

\newpage
\section{Authenticated Encryption from SE and MACs}
\label{sec:genericcomp}
Let $\SE=(\SE.\kg,\SE.\enc,\SE.\dec)$ be a symmetric encryption scheme and $\MA=(\MA.\kg,\mtag,\ver)$ be a message authentication scheme. In this section, we combine these two primitives in different ways to construct an authenticated encryption scheme $\SE'=(\kg,\enc,\dec)$, and we describe the security guarantees of each construction. 
\begin{figure}[h]
	\centering
	\fpage{.12}{
		\underline{$\kg$}\\[1pt]
		$K_1\getsr \SE.\kg$\\
		$K_2\getsr \MA.\kg$\\
		Ret $(K_1,K_2)$
	}
	\fpage{.16}{
		\underline{$\enc(K,M)$}\\[1pt]
		$(K_1,K_2)\gets K$\\
		$C\gets \SE.\enc(K_1,M)$\\
		$T\gets \mtag(K_2,M)$\\
		Ret $(C,T)$
	}
	\fpage{.18}{
		\underline{$\dec(K,(C,T))$}\\[1pt]
		$(K_1,K_2)\gets K$\\
		$M\gets \SE.\dec(K_1,C)$\\
		$T'\gets \mtag(K_2,M)$\\
		If $T'\neq T$ then Ret $\bot$\\
		Ret $M$
	}
	\caption{The Encrypt-and-Mac composition. The ciphertext and tag are both computed on the original message.}
\end{figure}
\begin{figure}[h]
	\centering
	\fpage{.12}{
		\underline{$\kg$}\\[1pt]
		$K_1\getsr \SE.\kg$\\
		$K_2\getsr \MA.\kg$\\
		Ret $(K_1,K_2)$
	}
	\fpage{.2}{
		\underline{$\enc(K,M)$}\\[1pt]
		$(K_1,K_2)\gets K$\\
		$T\gets \mtag(K_2,M)$\\
		$C\gets \SE.\enc(K_1,M\concat T)$\\
		Ret $C$
	}
	\fpage{.22}{
		\underline{$\dec(K,C)$}\\[1pt]
		$(K_1,K_2)\gets K$\\
		$M\concat T\gets \SE.\dec(K_1,C)$\\
		\scribenote{fix the concat notation}\\
		$T'\gets \mtag(K_2,M)$\\
		If $T'\neq T$ then Ret $\bot$\\
		Ret $M$
	}
	\caption{The Mac-then-Encrypt composition. The tag is computed on the original message, then the message and the tag are encrypted together.}
\end{figure}
\begin{figure}[h]
	\centering
	\fpage{.12}{
		\underline{$\kg$}\\[1pt]
		$K_1\getsr \SE.\kg$\\
		$K_2\getsr \MA.\kg$\\
		Ret $(K_1,K_2)$
	}
	\fpage{.16}{
		\underline{$\enc(K,M)$}\\[1pt]
		$(K_1,K_2)\gets K$\\
		$C\gets \SE.\enc(K_1,M)$\\
		$T\gets \mtag(K_2,C)$\\
		Ret $(C,T)$
	}
	\fpage{.18}{
		\underline{$\dec(K,(C,T))$}\\[1pt]
		$(K_1,K_2)\gets K$\\
		$M\gets \SE.\dec(K_1,C)$\\
		$T'\gets \mtag(K_2,C)$\\
		If $T'\neq T$ then Ret $\bot$\\
		Ret $M$
	}
	\caption{The Encrypt-then-Mac composition. The ciphertext is computed on the original message, then the tag is computed on the ciphertext.}
\end{figure}

If we assume that $\SE$ is $\ROR$ secure and $\MA$ is $\UFCMA$ secure, then we can say that the Encrypt-then-Mac composition of them is $\RORCCA$ secure, which we prove in Theorem~\ref{thm:rorcca-etm}. However, we cannot say the same for the Encrypt-and-Mac or Mac-then-Encrypt compositions in general. \scribenote{Give counter examples for EaM and MtE?}

\begin{theorem}
Let $\SE$ be an SE scheme and $\MA$ be a MAC and $\EtM$ be the Encrypt-then-Mac
scheme built from them.  Let $\advA$ be an $\RORCCA_\EtM$-adversary making at most
$q$ queries. Then we give a $\ROR_\SE$-adversary $\advB_{\textrm{se}}$ 
and a $\UFCMA_\MA$-adversary  $\advB_\textrm{mac}$ 
such that
\bnm
  \AdvRORCCA{\EtM}{\advA} \le \AdvROR{\SE}{\advB_\textrm{se}} + 2\cdotsm\AdvUFCMA{\MA}{\advB_{\textrm{mac}}} \;.
\enm
Adversaries $\advB_\textrm{se}$ and $\advB_{\textrm{mac}}$ each run in time that of 
$\advA$ plus small overhead and each make at most $q$ queries.
\label{thm:rorcca-etm}
\end{theorem}
\begin{proof}
	\scribenote{TODO}
\end{proof}
\newpage
%%%%%%%%%%%%%%%%%%%%%%%%%%%%%%%%%%%%%%%%%%%%%%%%%%%%%%%%%%%%%%%%%%%%%%%%%%%%%%%%
\section{Authenticated Encryption Variants}
\label{sec:authenc}

An authenticated encryption with associated data (AEAD) scheme is a triple of
algorithms $\AEAD = (\kg,\enc,\dec)$. It is the same as an SE scheme, except
that encryption and decryption both take an additional input, called the
associated data. Sometimes the associated data is referred to as the header.

\hfpagess{.15}{.15}{
\underline{$\RORCCA1^\advA_{\AEAD}$}\\[1pt]
$K \getsr \kg$\\
$b' \getsr \advA^{\EncOracle,\DecOracle}$\\
Ret $b'$\medskip

\underline{$\EncOracle(H,M)$}\\
$C \getsr \enc_K(H,M)$\\
$\calC \gets \calC \cup \{(H,C)\}$\\
Ret $C$\medskip

\underline{$\DecOracle(H,C)$}\\
If $(H,C) \in \calC$ then \\
\myInd Ret $\bot$\\
Ret $\dec_K(H,C)$
}{
\underline{$\RORCCA0^\advA_{\AEAD}$}\\[1pt]
$b' \getsr \advA^{\EncOracle,\DecOracle}$\\
Ret $b'$\medskip

\underline{$\EncOracle(H,M)$}\\
$C \getsr \bits^{\ctxtlen(|M|)}$\\
Ret $C$\medskip

\underline{$\DecOracle(H,C)$}\\
Ret $\bot$
}


\bnm
\AdvRORCCA{\SE}{\advA} = \left|\Prob{\RORCCA1_\SE^\advA\Rightarrow1}
                                    -\Prob{\RORCCA0_\SE^\advA\Rightarrow1}\right|
\enm


\fpage{.15}{
\underline{$\CTXT^\advA_{\SE}$}\\[1pt]
$K \getsr \kg$\\
$\win \gets \false$\\
$\advA^{\EncOracle,\DecOracle}$\\
Ret $\win$\medskip

\underline{$\EncOracle(M)$}\\
$C \getsr \enc_K(M)$\\
$\calC \gets \calC \cup \{C\}$\\
Ret $C$\medskip

\underline{$\DecOracle(C)$}\\
If $C \in \calC$ then \\
\myInd Ret $\bot$\\
$M \gets \dec_K(C)$\\
If $M \ne \bot$ then \\
\myInd $\win\gets\true$\\
Ret $M$
}

\bnm
\AdvCTXT{\SE}{\advA} = \Prob{\CTXT_\SE^\advA\Rightarrow\true}\\
\enm


\begin{theorem}
Let $\SE$ be a symmetric encryption scheme. Let $\advA$ be any
$\INDCPA_\SE$-adversary making at most $q$ queries. 
We give an $\ROR_\SE$-adversary $\advB$ such that
\bnm
  \AdvINDCPA{\SE}{\advA} \le 2\cdotsm\AdvROR{\SE}{\advB}
\enm
Adversary $\advB$ makes at most $q$ 
queries and runs in time that of $\advA$.
\end{theorem}

\tnote{We'll possibly introduce the following at some point, but didn't need it
here yet.}
We sometimes use $\bigO$ notation to hide small values that can be derived from
proofs, but don't matter to the interpretation of the theorem. If we were to do
an asymptotic treatment, this would correspond to hiding constants, hence the
abuse of notation.  Thus in above theorem we would replace $q+3$ with
$\bigO(q)$. 


\fpage{.20}{
\underline{$\advB^{\Enc}$}\\[1pt]
$b \getsr \bits$\\
$b' \getsr \advA^\EncSim$\\
If $(b = b')$ then Ret 1\\
Ret 0\medskip

\underline{$\EncSim(M_0,M_1)$}\\
Return $\Enc(M_b)$
}

\begin{align*}
\AdvROR{\SE}{\advB} 
    &= \left|\Prob{\ROR1_\SE^\advB\Rightarrow 1} -
                                \Prob{\ROR0_\SE^\advB\Rightarrow 1}\right|\\
    &= \left|\Prob{\INDCPA_\SE^\advA\Rightarrow\true} - \frac{1}{2}\right|\\
    &= \left|\frac{1}{2} +
    \frac{1}{2}\cdot\AdvINDCPA{\SE}{\advA} - \frac{1}{2}\right|\\
    &= \frac{1}{2}\cdot\AdvINDCPA{\SE}{\advA}
\end{align*}



\tnote{Let's add this strong tweakable block cipher stuff to the chapter on
tweakable block ciphers. The scribe for Feb 18 can write it up, but we'll merge
it into previous chapter later.}


\hfpages{.15}{
		\underline{$\STPRP1_{\tweakCipher}^\advA$}\\
		$K \getsr \keyspace$\\
		$b' \getsr \advA^{\Fn,\FnInv}$\\
		Return $b'$\medskip
		
		\underline{$\Fn(T,M)$}\\[1pt]
		Return $\tweakE_K(T,M)$\medskip
  
   \underline{$\FnInv(T,C)$}\\[1pt]
		Return $\tweakD_K(T,C)$

	}{
		\underline{$\STPRP0_{\tweakCipher}^\advA$}\\
		$\tweakpi \getsr \Perm(\tweakspace,\msgspace)$\\
		$b' \getsr \advA^{\Fn,\FnInv}$\\
		Return $b'$\medskip
		
		\underline{$\Fn(T,M)$}\\[1pt]
		Return $\tweakpi(T,M)$\medskip
  
   \underline{$\FnInv(T,C)$}\\[1pt]
		Return $\tweakpi^{-1}_K(T,C)$
	}

\bnm
\AdvSTPRP{\tweakCipher}{\advA} = \left|\Prob{\STPRP1^\advA\Rightarrow1} -
                                    \Prob{\STPRP0^\advA\Rightarrow1} \right|
\enm



\newpage
%%%%%%%%%%%%%%%%%%%%%%%%%%%%%%%%%%%%%%%%%%%%%%%%%%%%%%%%%%%%%%%%%%%%%%%%%%%%%%%%
\section{Authenticated Encryption in Practice and Theory}
\label{sec:authencpractice}

So, now that we have seen and analyzed authenticated encryption schemes, we turn to the practical deployment of AE schemes and the trade-offs that 
arise therein.  We have seen that CPA-only encryption proves insufficient due to the active attacks we've explored, including padding oracle attacks 
and attacks that violate integrity, and that authenticated encryption provides much stronger properties concerning the validity of ciphertexts that 
addresses this gap.  We've also seen the OCB construction for achieving authenticated encryption.

We now focus on more generic constructions that attempt to achieve the same goals of ROR+CTXT or ROR-CCA security.  A natural strategy for achieving 
authenicated encryption is to combine symmetric message authentication and encryption primitives (for integrity and confidentiality, respectively).  
For example, it is possible to combine a CPA-secure encryption scheme with a UF-CMA MAC to achieve what appears to be authenticated encryption.  
Several initial proposals for doing this are sumarized in Figure~\ref{fig:cpamac}, and were originally explored by Bellare in~\cite{Bellare2000}.


\subsection{``Stackoverflow Errrors"}

Figure~\ref{fig:reusescheme} shows the Wikipedia diagram of a generic composition achieving ``authenticated encryption" (what we will call the 
AE-WIKI) on the left.  Unfortunately, AE-WIKI does not imply any of our previous security notions, even when appropriate primitives are chosen.  For 
one, AE-WIKI makes reference to a ``hash function", which is shown to have a key as input.  The specifics of this composition are not clear: hash 
functions generally are not a keyed primitive, and the exact security requirements for a hash function to result in a secure construction remain 
unclear.

Worse still, the key is shown as being reused in this diagram.  This could lead to serious security issues in many constructions; a notion we explore 
next.

\subsection{Against Key Reuse}

\begin{figure}[h]
\centering
\fpage{.35}{\scalebox{0.5}{\includegraphics{aeinpractice/aewiki}}}
\fpage{.5}{
\scalebox{0.6}{\Huge
% Graphic for TeX using PGF
% Title: /home/phil/keyreuse.dia
% Creator: Dia v0.97+git
% CreationDate: Fri Mar  8 17:01:06 2019
% For: phil
% \usepackage{tikz}
% The following commands are not supported in PSTricks at present
% We define them conditionally, so when they are implemented,
% this pgf file will use them.
\ifx\du\undefined
  \newlength{\du}
\fi
\setlength{\du}{15\unitlength}
\begin{tikzpicture}[even odd rule]
\pgftransformxscale{1.000000}
\pgftransformyscale{-1.000000}
\definecolor{dialinecolor}{rgb}{0.000000, 0.000000, 0.000000}
\pgfsetstrokecolor{dialinecolor}
\pgfsetstrokeopacity{1.000000}
\definecolor{diafillcolor}{rgb}{1.000000, 1.000000, 1.000000}
\pgfsetfillcolor{diafillcolor}
\pgfsetfillopacity{1.000000}
\pgfsetlinewidth{0.100000\du}
\pgfsetdash{}{0pt}
\pgfsetmiterjoin
{\pgfsetcornersarced{\pgfpoint{0.000000\du}{0.000000\du}}\definecolor{diafillcolor}{rgb}{1.000000, 1.000000, 1.000000}
\pgfsetfillcolor{diafillcolor}
\pgfsetfillopacity{1.000000}
\fill (0.464466\du,12.116117\du)--(0.464466\du,15.566117\du)--(5.464466\du,15.566117\du)--(5.464466\du,12.116117\du)--cycle;
}{\pgfsetcornersarced{\pgfpoint{0.000000\du}{0.000000\du}}\definecolor{dialinecolor}{rgb}{0.000000, 0.000000, 0.000000}
\pgfsetstrokecolor{dialinecolor}
\pgfsetstrokeopacity{1.000000}
\draw (0.464466\du,12.116117\du)--(0.464466\du,15.566117\du)--(5.464466\du,15.566117\du)--(5.464466\du,12.116117\du)--cycle;
}% setfont left to latex
\definecolor{dialinecolor}{rgb}{0.000000, 0.000000, 0.000000}
\pgfsetstrokecolor{dialinecolor}
\pgfsetstrokeopacity{1.000000}
\definecolor{diafillcolor}{rgb}{0.000000, 0.000000, 0.000000}
\pgfsetfillcolor{diafillcolor}
\pgfsetfillopacity{1.000000}
\node[anchor=base,inner sep=0pt, outer sep=0pt,color=dialinecolor] at (2.964466\du,14.185561\du){$E_K$};
% setfont left to latex
\definecolor{dialinecolor}{rgb}{0.000000, 0.000000, 0.000000}
\pgfsetstrokecolor{dialinecolor}
\pgfsetstrokeopacity{1.000000}
\definecolor{diafillcolor}{rgb}{0.000000, 0.000000, 0.000000}
\pgfsetfillcolor{diafillcolor}
\pgfsetfillopacity{1.000000}
\node[anchor=base west,inner sep=0pt,outer sep=0pt,color=dialinecolor] at (2.464466\du,3.116117\du){IV};
\pgfsetlinewidth{0.100000\du}
\pgfsetdash{}{0pt}
\pgfsetmiterjoin
{\pgfsetcornersarced{\pgfpoint{0.000000\du}{0.000000\du}}\definecolor{diafillcolor}{rgb}{1.000000, 1.000000, 1.000000}
\pgfsetfillcolor{diafillcolor}
\pgfsetfillopacity{1.000000}
\fill (7.464466\du,4.116117\du)--(7.464466\du,7.566117\du)--(12.464466\du,7.566117\du)--(12.464466\du,4.116117\du)--cycle;
}{\pgfsetcornersarced{\pgfpoint{0.000000\du}{0.000000\du}}\definecolor{dialinecolor}{rgb}{0.000000, 0.000000, 0.000000}
\pgfsetstrokecolor{dialinecolor}
\pgfsetstrokeopacity{1.000000}
\draw (7.464466\du,4.116117\du)--(7.464466\du,7.566117\du)--(12.464466\du,7.566117\du)--(12.464466\du,4.116117\du)--cycle;
}% setfont left to latex
\definecolor{dialinecolor}{rgb}{0.000000, 0.000000, 0.000000}
\pgfsetstrokecolor{dialinecolor}
\pgfsetstrokeopacity{1.000000}
\definecolor{diafillcolor}{rgb}{0.000000, 0.000000, 0.000000}
\pgfsetfillcolor{diafillcolor}
\pgfsetfillopacity{1.000000}
\node[anchor=base,inner sep=0pt, outer sep=0pt,color=dialinecolor] at (9.964466\du,6.185561\du){$E_K$};
\pgfsetlinewidth{0.100000\du}
\pgfsetdash{}{0pt}
\pgfsetmiterjoin
{\pgfsetcornersarced{\pgfpoint{0.000000\du}{0.000000\du}}\definecolor{diafillcolor}{rgb}{1.000000, 1.000000, 1.000000}
\pgfsetfillcolor{diafillcolor}
\pgfsetfillopacity{1.000000}
\fill (7.464466\du,12.116117\du)--(7.464466\du,15.566117\du)--(12.464466\du,15.566117\du)--(12.464466\du,12.116117\du)--cycle;
}{\pgfsetcornersarced{\pgfpoint{0.000000\du}{0.000000\du}}\definecolor{dialinecolor}{rgb}{0.000000, 0.000000, 0.000000}
\pgfsetstrokecolor{dialinecolor}
\pgfsetstrokeopacity{1.000000}
\draw (7.464466\du,12.116117\du)--(7.464466\du,15.566117\du)--(12.464466\du,15.566117\du)--(12.464466\du,12.116117\du)--cycle;
}% setfont left to latex
\definecolor{dialinecolor}{rgb}{0.000000, 0.000000, 0.000000}
\pgfsetstrokecolor{dialinecolor}
\pgfsetstrokeopacity{1.000000}
\definecolor{diafillcolor}{rgb}{0.000000, 0.000000, 0.000000}
\pgfsetfillcolor{diafillcolor}
\pgfsetfillopacity{1.000000}
\node[anchor=base,inner sep=0pt, outer sep=0pt,color=dialinecolor] at (9.964466\du,14.185561\du){$E_K$};
\pgfsetlinewidth{0.100000\du}
\pgfsetdash{}{0pt}
\pgfsetmiterjoin
{\pgfsetcornersarced{\pgfpoint{0.000000\du}{0.000000\du}}\definecolor{diafillcolor}{rgb}{1.000000, 1.000000, 1.000000}
\pgfsetfillcolor{diafillcolor}
\pgfsetfillopacity{1.000000}
\fill (14.464466\du,4.116117\du)--(14.464466\du,7.566117\du)--(19.464466\du,7.566117\du)--(19.464466\du,4.116117\du)--cycle;
}{\pgfsetcornersarced{\pgfpoint{0.000000\du}{0.000000\du}}\definecolor{dialinecolor}{rgb}{0.000000, 0.000000, 0.000000}
\pgfsetstrokecolor{dialinecolor}
\pgfsetstrokeopacity{1.000000}
\draw (14.464466\du,4.116117\du)--(14.464466\du,7.566117\du)--(19.464466\du,7.566117\du)--(19.464466\du,4.116117\du)--cycle;
}% setfont left to latex
\definecolor{dialinecolor}{rgb}{0.000000, 0.000000, 0.000000}
\pgfsetstrokecolor{dialinecolor}
\pgfsetstrokeopacity{1.000000}
\definecolor{diafillcolor}{rgb}{0.000000, 0.000000, 0.000000}
\pgfsetfillcolor{diafillcolor}
\pgfsetfillopacity{1.000000}
\node[anchor=base,inner sep=0pt, outer sep=0pt,color=dialinecolor] at (16.964466\du,6.185561\du){$E_K$};
\pgfsetlinewidth{0.100000\du}
\pgfsetdash{}{0pt}
\pgfsetmiterjoin
{\pgfsetcornersarced{\pgfpoint{0.000000\du}{0.000000\du}}\definecolor{diafillcolor}{rgb}{1.000000, 1.000000, 1.000000}
\pgfsetfillcolor{diafillcolor}
\pgfsetfillopacity{1.000000}
\fill (14.464466\du,12.116117\du)--(14.464466\du,15.566117\du)--(19.464466\du,15.566117\du)--(19.464466\du,12.116117\du)--cycle;
}{\pgfsetcornersarced{\pgfpoint{0.000000\du}{0.000000\du}}\definecolor{dialinecolor}{rgb}{0.000000, 0.000000, 0.000000}
\pgfsetstrokecolor{dialinecolor}
\pgfsetstrokeopacity{1.000000}
\draw (14.464466\du,12.116117\du)--(14.464466\du,15.566117\du)--(19.464466\du,15.566117\du)--(19.464466\du,12.116117\du)--cycle;
}% setfont left to latex
\definecolor{dialinecolor}{rgb}{0.000000, 0.000000, 0.000000}
\pgfsetstrokecolor{dialinecolor}
\pgfsetstrokeopacity{1.000000}
\definecolor{diafillcolor}{rgb}{0.000000, 0.000000, 0.000000}
\pgfsetfillcolor{diafillcolor}
\pgfsetfillopacity{1.000000}
\node[anchor=base,inner sep=0pt, outer sep=0pt,color=dialinecolor] at (16.964466\du,14.185561\du){$E_K$};
% setfont left to latex
\definecolor{dialinecolor}{rgb}{0.000000, 0.000000, 0.000000}
\pgfsetstrokecolor{dialinecolor}
\pgfsetstrokeopacity{1.000000}
\definecolor{diafillcolor}{rgb}{0.000000, 0.000000, 0.000000}
\pgfsetfillcolor{diafillcolor}
\pgfsetfillopacity{1.000000}
\node[anchor=base west,inner sep=0pt,outer sep=0pt,color=dialinecolor] at (2.464466\du,9.349417\du){$C_0$};
% setfont left to latex
\definecolor{dialinecolor}{rgb}{0.000000, 0.000000, 0.000000}
\pgfsetstrokecolor{dialinecolor}
\pgfsetstrokeopacity{1.000000}
\definecolor{diafillcolor}{rgb}{0.000000, 0.000000, 0.000000}
\pgfsetfillcolor{diafillcolor}
\pgfsetfillopacity{1.000000}
\node[anchor=base west,inner sep=0pt,outer sep=0pt,color=dialinecolor] at (9.464466\du,9.349417\du){$C_1$};
% setfont left to latex
\definecolor{dialinecolor}{rgb}{0.000000, 0.000000, 0.000000}
\pgfsetstrokecolor{dialinecolor}
\pgfsetstrokeopacity{1.000000}
\definecolor{diafillcolor}{rgb}{0.000000, 0.000000, 0.000000}
\pgfsetfillcolor{diafillcolor}
\pgfsetfillopacity{1.000000}
\node[anchor=base west,inner sep=0pt,outer sep=0pt,color=dialinecolor] at (16.464466\du,9.349417\du){$C_2$};
% setfont left to latex
\definecolor{dialinecolor}{rgb}{0.000000, 0.000000, 0.000000}
\pgfsetstrokecolor{dialinecolor}
\pgfsetstrokeopacity{1.000000}
\definecolor{diafillcolor}{rgb}{0.000000, 0.000000, 0.000000}
\pgfsetfillcolor{diafillcolor}
\pgfsetfillopacity{1.000000}
\node[anchor=base west,inner sep=0pt,outer sep=0pt,color=dialinecolor] at (21.464466\du,9.349417\du){T};
\pgfsetlinewidth{0.100000\du}
\pgfsetdash{}{0pt}
\pgfsetbuttcap
{
\definecolor{diafillcolor}{rgb}{0.000000, 0.000000, 0.000000}
\pgfsetfillcolor{diafillcolor}
\pgfsetfillopacity{1.000000}
% was here!!!
\pgfsetarrowsend{stealth}
\definecolor{dialinecolor}{rgb}{0.000000, 0.000000, 0.000000}
\pgfsetstrokecolor{dialinecolor}
\pgfsetstrokeopacity{1.000000}
\draw (2.947796\du,3.666117\du)--(2.931136\du,8.116117\du);
}
\pgfsetlinewidth{0.100000\du}
\pgfsetdash{}{0pt}
\pgfsetbuttcap
{
\definecolor{diafillcolor}{rgb}{0.000000, 0.000000, 0.000000}
\pgfsetfillcolor{diafillcolor}
\pgfsetfillopacity{1.000000}
% was here!!!
\pgfsetarrowsend{stealth}
\definecolor{dialinecolor}{rgb}{0.000000, 0.000000, 0.000000}
\pgfsetstrokecolor{dialinecolor}
\pgfsetstrokeopacity{1.000000}
\draw (9.964466\du,7.566117\du)--(9.864466\du,8.566117\du);
}
\pgfsetlinewidth{0.100000\du}
\pgfsetdash{}{0pt}
\pgfsetbuttcap
{
\definecolor{diafillcolor}{rgb}{0.000000, 0.000000, 0.000000}
\pgfsetfillcolor{diafillcolor}
\pgfsetfillopacity{1.000000}
% was here!!!
\pgfsetarrowsend{stealth}
\definecolor{dialinecolor}{rgb}{0.000000, 0.000000, 0.000000}
\pgfsetstrokecolor{dialinecolor}
\pgfsetstrokeopacity{1.000000}
\draw (16.964466\du,7.566117\du)--(16.914466\du,8.566117\du);
}
\pgfsetlinewidth{0.100000\du}
\pgfsetdash{}{0pt}
\pgfsetbuttcap
{
\definecolor{diafillcolor}{rgb}{0.000000, 0.000000, 0.000000}
\pgfsetfillcolor{diafillcolor}
\pgfsetfillopacity{1.000000}
% was here!!!
\pgfsetarrowsend{stealth}
\definecolor{dialinecolor}{rgb}{0.000000, 0.000000, 0.000000}
\pgfsetstrokecolor{dialinecolor}
\pgfsetstrokeopacity{1.000000}
\draw (2.972796\du,9.499417\du)--(2.964466\du,12.116117\du);
}
\pgfsetlinewidth{0.100000\du}
\pgfsetdash{}{0pt}
\pgfsetbuttcap
\pgfsetmiterjoin
\pgfsetlinewidth{0.100000\du}
\pgfsetbuttcap
\pgfsetmiterjoin
\pgfsetdash{}{0pt}
\definecolor{diafillcolor}{rgb}{1.000000, 1.000000, 1.000000}
\pgfsetfillcolor{diafillcolor}
\pgfsetfillopacity{1.000000}
\pgfpathellipse{\pgfpoint{10.026166\du}{10.801217\du}}{\pgfpoint{0.575000\du}{0\du}}{\pgfpoint{0\du}{0.600000\du}}
\pgfusepath{fill}
\definecolor{dialinecolor}{rgb}{0.000000, 0.000000, 0.000000}
\pgfsetstrokecolor{dialinecolor}
\pgfsetstrokeopacity{1.000000}
\pgfpathellipse{\pgfpoint{10.026166\du}{10.801217\du}}{\pgfpoint{0.575000\du}{0\du}}{\pgfpoint{0\du}{0.600000\du}}
\pgfusepath{stroke}
\pgfsetbuttcap
\pgfsetmiterjoin
\pgfsetdash{}{0pt}
\definecolor{dialinecolor}{rgb}{0.000000, 0.000000, 0.000000}
\pgfsetstrokecolor{dialinecolor}
\pgfsetstrokeopacity{1.000000}
\draw (10.026166\du,10.201217\du)--(10.026166\du,11.401217\du);
\pgfsetbuttcap
\pgfsetmiterjoin
\pgfsetdash{}{0pt}
\definecolor{dialinecolor}{rgb}{0.000000, 0.000000, 0.000000}
\pgfsetstrokecolor{dialinecolor}
\pgfsetstrokeopacity{1.000000}
\draw (9.451166\du,10.801217\du)--(10.601166\du,10.801217\du);
\pgfsetlinewidth{0.100000\du}
\pgfsetdash{}{0pt}
\pgfsetbuttcap
{
\definecolor{diafillcolor}{rgb}{0.000000, 0.000000, 0.000000}
\pgfsetfillcolor{diafillcolor}
\pgfsetfillopacity{1.000000}
% was here!!!
\pgfsetarrowsend{stealth}
\definecolor{dialinecolor}{rgb}{0.000000, 0.000000, 0.000000}
\pgfsetstrokecolor{dialinecolor}
\pgfsetstrokeopacity{1.000000}
\draw (10.034466\du,9.584517\du)--(10.026166\du,10.201217\du);
}
\pgfsetlinewidth{0.100000\du}
\pgfsetdash{}{0pt}
\pgfsetbuttcap
{
\definecolor{diafillcolor}{rgb}{0.000000, 0.000000, 0.000000}
\pgfsetfillcolor{diafillcolor}
\pgfsetfillopacity{1.000000}
% was here!!!
\pgfsetarrowsend{stealth}
\definecolor{dialinecolor}{rgb}{0.000000, 0.000000, 0.000000}
\pgfsetstrokecolor{dialinecolor}
\pgfsetstrokeopacity{1.000000}
\draw (10.026166\du,11.401217\du)--(9.964466\du,12.116117\du);
}
\pgfsetlinewidth{0.100000\du}
\pgfsetdash{}{0pt}
\pgfsetbuttcap
\pgfsetmiterjoin
\pgfsetlinewidth{0.100000\du}
\pgfsetbuttcap
\pgfsetmiterjoin
\pgfsetdash{}{0pt}
\definecolor{diafillcolor}{rgb}{1.000000, 1.000000, 1.000000}
\pgfsetfillcolor{diafillcolor}
\pgfsetfillopacity{1.000000}
\pgfpathellipse{\pgfpoint{16.992766\du}{10.867917\du}}{\pgfpoint{0.575000\du}{0\du}}{\pgfpoint{0\du}{0.600000\du}}
\pgfusepath{fill}
\definecolor{dialinecolor}{rgb}{0.000000, 0.000000, 0.000000}
\pgfsetstrokecolor{dialinecolor}
\pgfsetstrokeopacity{1.000000}
\pgfpathellipse{\pgfpoint{16.992766\du}{10.867917\du}}{\pgfpoint{0.575000\du}{0\du}}{\pgfpoint{0\du}{0.600000\du}}
\pgfusepath{stroke}
\pgfsetbuttcap
\pgfsetmiterjoin
\pgfsetdash{}{0pt}
\definecolor{dialinecolor}{rgb}{0.000000, 0.000000, 0.000000}
\pgfsetstrokecolor{dialinecolor}
\pgfsetstrokeopacity{1.000000}
\draw (16.992766\du,10.267917\du)--(16.992766\du,11.467917\du);
\pgfsetbuttcap
\pgfsetmiterjoin
\pgfsetdash{}{0pt}
\definecolor{dialinecolor}{rgb}{0.000000, 0.000000, 0.000000}
\pgfsetstrokecolor{dialinecolor}
\pgfsetstrokeopacity{1.000000}
\draw (16.417766\du,10.867917\du)--(17.567766\du,10.867917\du);
\pgfsetlinewidth{0.100000\du}
\pgfsetdash{}{0pt}
\pgfsetbuttcap
{
\definecolor{diafillcolor}{rgb}{0.000000, 0.000000, 0.000000}
\pgfsetfillcolor{diafillcolor}
\pgfsetfillopacity{1.000000}
% was here!!!
\pgfsetarrowsend{stealth}
\definecolor{dialinecolor}{rgb}{0.000000, 0.000000, 0.000000}
\pgfsetstrokecolor{dialinecolor}
\pgfsetstrokeopacity{1.000000}
\draw (17.001166\du,9.651217\du)--(16.992766\du,10.267917\du);
}
\pgfsetlinewidth{0.100000\du}
\pgfsetdash{}{0pt}
\pgfsetbuttcap
{
\definecolor{diafillcolor}{rgb}{0.000000, 0.000000, 0.000000}
\pgfsetfillcolor{diafillcolor}
\pgfsetfillopacity{1.000000}
% was here!!!
\pgfsetarrowsend{stealth}
\definecolor{dialinecolor}{rgb}{0.000000, 0.000000, 0.000000}
\pgfsetstrokecolor{dialinecolor}
\pgfsetstrokeopacity{1.000000}
\draw (16.992766\du,11.467917\du)--(16.964466\du,12.116117\du);
}
\pgfsetlinewidth{0.100000\du}
\pgfsetdash{}{0pt}
\pgfsetmiterjoin
\pgfsetbuttcap
{
\definecolor{diafillcolor}{rgb}{0.000000, 0.000000, 0.000000}
\pgfsetfillcolor{diafillcolor}
\pgfsetfillopacity{1.000000}
% was here!!!
\pgfsetarrowsend{stealth}
{\pgfsetcornersarced{\pgfpoint{0.000000\du}{0.000000\du}}\definecolor{dialinecolor}{rgb}{0.000000, 0.000000, 0.000000}
\pgfsetstrokecolor{dialinecolor}
\pgfsetstrokeopacity{1.000000}
\draw (2.964466\du,15.566117\du)--(2.964466\du,16.616117\du)--(6.182816\du,16.616117\du)--(6.182816\du,10.801217\du)--(9.451166\du,10.801217\du);
}}
\pgfsetlinewidth{0.100000\du}
\pgfsetdash{}{0pt}
\pgfsetbuttcap
{
\definecolor{diafillcolor}{rgb}{0.000000, 0.000000, 0.000000}
\pgfsetfillcolor{diafillcolor}
\pgfsetfillopacity{1.000000}
% was here!!!
\pgfsetarrowsend{stealth}
\definecolor{dialinecolor}{rgb}{0.000000, 0.000000, 0.000000}
\pgfsetstrokecolor{dialinecolor}
\pgfsetstrokeopacity{1.000000}
\draw (3.747796\du,2.699447\du)--(9.451166\du,2.791217\du);
}
\pgfsetlinewidth{0.100000\du}
\pgfsetdash{}{0pt}
\pgfsetmiterjoin
\pgfsetbuttcap
{
\definecolor{diafillcolor}{rgb}{0.000000, 0.000000, 0.000000}
\pgfsetfillcolor{diafillcolor}
\pgfsetfillopacity{1.000000}
% was here!!!
\pgfsetarrowsend{stealth}
{\pgfsetcornersarced{\pgfpoint{0.000000\du}{0.000000\du}}\definecolor{dialinecolor}{rgb}{0.000000, 0.000000, 0.000000}
\pgfsetstrokecolor{dialinecolor}
\pgfsetstrokeopacity{1.000000}
\draw (9.964466\du,15.566117\du)--(9.964466\du,16.616117\du)--(13.166116\du,16.616117\du)--(13.166116\du,10.867917\du)--(16.417766\du,10.867917\du);
}}
\pgfsetlinewidth{0.100000\du}
\pgfsetdash{}{0pt}
\pgfsetmiterjoin
\pgfsetbuttcap
{
\definecolor{diafillcolor}{rgb}{0.000000, 0.000000, 0.000000}
\pgfsetfillcolor{diafillcolor}
\pgfsetfillopacity{1.000000}
% was here!!!
\pgfsetarrowsend{stealth}
{\pgfsetcornersarced{\pgfpoint{0.000000\du}{0.000000\du}}\definecolor{dialinecolor}{rgb}{0.000000, 0.000000, 0.000000}
\pgfsetstrokecolor{dialinecolor}
\pgfsetstrokeopacity{1.000000}
\draw (16.964466\du,15.566117\du)--(16.964466\du,16.616117\du)--(20.381166\du,16.616117\du)--(20.381166\du,8.982787\du)--(21.481166\du,8.982787\du);
}}
% setfont left to latex
\definecolor{dialinecolor}{rgb}{0.000000, 0.000000, 0.000000}
\pgfsetstrokecolor{dialinecolor}
\pgfsetstrokeopacity{1.000000}
\definecolor{diafillcolor}{rgb}{0.000000, 0.000000, 0.000000}
\pgfsetfillcolor{diafillcolor}
\pgfsetfillopacity{1.000000}
\node[anchor=base west,inner sep=0pt,outer sep=0pt,color=dialinecolor] at (9.464466\du,1.339447\du){$M_1$};
\pgfsetlinewidth{0.100000\du}
\pgfsetdash{}{0pt}
\pgfsetbuttcap
\pgfsetmiterjoin
\pgfsetlinewidth{0.100000\du}
\pgfsetbuttcap
\pgfsetmiterjoin
\pgfsetdash{}{0pt}
\definecolor{diafillcolor}{rgb}{1.000000, 1.000000, 1.000000}
\pgfsetfillcolor{diafillcolor}
\pgfsetfillopacity{1.000000}
\pgfpathellipse{\pgfpoint{10.026166\du}{2.791217\du}}{\pgfpoint{0.575000\du}{0\du}}{\pgfpoint{0\du}{0.600000\du}}
\pgfusepath{fill}
\definecolor{dialinecolor}{rgb}{0.000000, 0.000000, 0.000000}
\pgfsetstrokecolor{dialinecolor}
\pgfsetstrokeopacity{1.000000}
\pgfpathellipse{\pgfpoint{10.026166\du}{2.791217\du}}{\pgfpoint{0.575000\du}{0\du}}{\pgfpoint{0\du}{0.600000\du}}
\pgfusepath{stroke}
\pgfsetbuttcap
\pgfsetmiterjoin
\pgfsetdash{}{0pt}
\definecolor{dialinecolor}{rgb}{0.000000, 0.000000, 0.000000}
\pgfsetstrokecolor{dialinecolor}
\pgfsetstrokeopacity{1.000000}
\draw (10.026166\du,2.191217\du)--(10.026166\du,3.391217\du);
\pgfsetbuttcap
\pgfsetmiterjoin
\pgfsetdash{}{0pt}
\definecolor{dialinecolor}{rgb}{0.000000, 0.000000, 0.000000}
\pgfsetstrokecolor{dialinecolor}
\pgfsetstrokeopacity{1.000000}
\draw (9.451166\du,2.791217\du)--(10.601166\du,2.791217\du);
\pgfsetlinewidth{0.100000\du}
\pgfsetdash{}{0pt}
\pgfsetbuttcap
{
\definecolor{diafillcolor}{rgb}{0.000000, 0.000000, 0.000000}
\pgfsetfillcolor{diafillcolor}
\pgfsetfillopacity{1.000000}
% was here!!!
\pgfsetarrowsend{stealth}
\definecolor{dialinecolor}{rgb}{0.000000, 0.000000, 0.000000}
\pgfsetstrokecolor{dialinecolor}
\pgfsetstrokeopacity{1.000000}
\draw (10.034466\du,1.574547\du)--(10.026166\du,2.191217\du);
}
\pgfsetlinewidth{0.100000\du}
\pgfsetdash{}{0pt}
\pgfsetbuttcap
{
\definecolor{diafillcolor}{rgb}{0.000000, 0.000000, 0.000000}
\pgfsetfillcolor{diafillcolor}
\pgfsetfillopacity{1.000000}
% was here!!!
\pgfsetarrowsend{stealth}
\definecolor{dialinecolor}{rgb}{0.000000, 0.000000, 0.000000}
\pgfsetstrokecolor{dialinecolor}
\pgfsetstrokeopacity{1.000000}
\draw (10.026166\du,3.391217\du)--(9.964466\du,4.116117\du);
}
% setfont left to latex
\definecolor{dialinecolor}{rgb}{0.000000, 0.000000, 0.000000}
\pgfsetstrokecolor{dialinecolor}
\pgfsetstrokeopacity{1.000000}
\definecolor{diafillcolor}{rgb}{0.000000, 0.000000, 0.000000}
\pgfsetfillcolor{diafillcolor}
\pgfsetfillopacity{1.000000}
\node[anchor=base west,inner sep=0pt,outer sep=0pt,color=dialinecolor] at (16.434466\du,1.279447\du){$M_2$};
\pgfsetlinewidth{0.100000\du}
\pgfsetdash{}{0pt}
\pgfsetbuttcap
\pgfsetmiterjoin
\pgfsetlinewidth{0.100000\du}
\pgfsetbuttcap
\pgfsetmiterjoin
\pgfsetdash{}{0pt}
\definecolor{diafillcolor}{rgb}{1.000000, 1.000000, 1.000000}
\pgfsetfillcolor{diafillcolor}
\pgfsetfillopacity{1.000000}
\pgfpathellipse{\pgfpoint{16.996166\du}{2.731217\du}}{\pgfpoint{0.575000\du}{0\du}}{\pgfpoint{0\du}{0.600000\du}}
\pgfusepath{fill}
\definecolor{dialinecolor}{rgb}{0.000000, 0.000000, 0.000000}
\pgfsetstrokecolor{dialinecolor}
\pgfsetstrokeopacity{1.000000}
\pgfpathellipse{\pgfpoint{16.996166\du}{2.731217\du}}{\pgfpoint{0.575000\du}{0\du}}{\pgfpoint{0\du}{0.600000\du}}
\pgfusepath{stroke}
\pgfsetbuttcap
\pgfsetmiterjoin
\pgfsetdash{}{0pt}
\definecolor{dialinecolor}{rgb}{0.000000, 0.000000, 0.000000}
\pgfsetstrokecolor{dialinecolor}
\pgfsetstrokeopacity{1.000000}
\draw (16.996166\du,2.131217\du)--(16.996166\du,3.331217\du);
\pgfsetbuttcap
\pgfsetmiterjoin
\pgfsetdash{}{0pt}
\definecolor{dialinecolor}{rgb}{0.000000, 0.000000, 0.000000}
\pgfsetstrokecolor{dialinecolor}
\pgfsetstrokeopacity{1.000000}
\draw (16.421166\du,2.731217\du)--(17.571166\du,2.731217\du);
\pgfsetlinewidth{0.100000\du}
\pgfsetdash{}{0pt}
\pgfsetbuttcap
{
\definecolor{diafillcolor}{rgb}{0.000000, 0.000000, 0.000000}
\pgfsetfillcolor{diafillcolor}
\pgfsetfillopacity{1.000000}
% was here!!!
\pgfsetarrowsend{stealth}
\definecolor{dialinecolor}{rgb}{0.000000, 0.000000, 0.000000}
\pgfsetstrokecolor{dialinecolor}
\pgfsetstrokeopacity{1.000000}
\draw (17.004466\du,1.514547\du)--(16.996166\du,2.131217\du);
}
\pgfsetlinewidth{0.100000\du}
\pgfsetdash{}{0pt}
\pgfsetbuttcap
{
\definecolor{diafillcolor}{rgb}{0.000000, 0.000000, 0.000000}
\pgfsetfillcolor{diafillcolor}
\pgfsetfillopacity{1.000000}
% was here!!!
\pgfsetarrowsend{stealth}
\definecolor{dialinecolor}{rgb}{0.000000, 0.000000, 0.000000}
\pgfsetstrokecolor{dialinecolor}
\pgfsetstrokeopacity{1.000000}
\draw (16.996166\du,3.331217\du)--(16.964466\du,4.116117\du);
}
\pgfsetlinewidth{0.100000\du}
\pgfsetdash{}{0pt}
\pgfsetmiterjoin
\pgfsetbuttcap
{
\definecolor{diafillcolor}{rgb}{0.000000, 0.000000, 0.000000}
\pgfsetfillcolor{diafillcolor}
\pgfsetfillopacity{1.000000}
% was here!!!
\pgfsetarrowsend{stealth}
{\pgfsetcornersarced{\pgfpoint{0.000000\du}{0.000000\du}}\definecolor{dialinecolor}{rgb}{0.000000, 0.000000, 0.000000}
\pgfsetstrokecolor{dialinecolor}
\pgfsetstrokeopacity{1.000000}
\draw (11.068759\du,9.016117\du)--(13.744962\du,9.016117\du)--(13.744962\du,2.731217\du)--(16.421166\du,2.731217\du);
}}
\end{tikzpicture}
\normalsize}}
  \caption{One encryption scheme that reuses keys across MAC and encryption primitives (right), as suggested in this diagram from Wikipedia (left)}
\label{fig:reusescheme}
\end{figure}

Consider the right side of Figure~\ref{fig:reusescheme}.  This shows one example construction, composing CBC-Mode+CBC-MAC using an encrypt-then-MAC 
paradigm where the ciphertexts are MACd, as suggested by AE-WIKI.  We concretely instantiate the scheme with messages consisting of two blocks, but 
consider the generalization to $n$ blocks by the obvious changes to the above structure.

While CBC mode provides the security we require, an adversary can easily violate ROR+CTXT for this scheme.  Remember the definition of the CTXT game 
in the previous chapter; an adversary must forge a ciphertext not returned by the encryption oracle, that decrypts without throwing an error.

To do so, consider a CTXT adversary $A$, whose operation is illustrated in Figure~\ref{fig:reuseadversary}.  First, the adversary queries the 
encryption of $0^n$, as shown on the left, so $C_0C_1T \leftarrow Enc(0^n)$.  The adversarially controlled inputs are shown with a red box surrounding 
them.

Next, the adversary calls $R \leftarrow Dec(0^nTT)$.  We will now show that $R \neq \bot$ with probability $1-\frac{1}{2^n}$, and therefore $A$ wins 
the CTXT game with this probability.


\begin{figure}
\centering
\scalebox{0.6}{% Graphic for TeX using PGF
% Title: /home/phil/keyreuse3.dia
% Creator: Dia v0.97+git
% CreationDate: Fri Mar  8 13:42:53 2019
% For: phil
% \usepackage{tikz}
% The following commands are not supported in PSTricks at present
% We define them conditionally, so when they are implemented,
% this pgf file will use them.
\Huge
\ifx\du\undefined
  \newlength{\du}
\fi
\setlength{\du}{15\unitlength}
\begin{tikzpicture}[even odd rule]
\pgftransformxscale{1.000000}
\pgftransformyscale{-1.000000}
\definecolor{dialinecolor}{rgb}{0.000000, 0.000000, 0.000000}
\pgfsetstrokecolor{dialinecolor}
\pgfsetstrokeopacity{1.000000}
\definecolor{diafillcolor}{rgb}{1.000000, 1.000000, 1.000000}
\pgfsetfillcolor{diafillcolor}
\pgfsetfillopacity{1.000000}
\pgfsetlinewidth{0.100000\du}
\pgfsetdash{}{0pt}
\pgfsetmiterjoin
{\pgfsetcornersarced{\pgfpoint{0.000000\du}{0.000000\du}}\definecolor{diafillcolor}{rgb}{1.000000, 0.000000, 0.000000}
\pgfsetfillcolor{diafillcolor}
\pgfsetfillopacity{1.000000}
\fill (24.300000\du,9.500000\du)--(24.300000\du,11.950000\du)--(42.300000\du,11.950000\du)--(42.300000\du,9.500000\du)--cycle;
}{\pgfsetcornersarced{\pgfpoint{0.000000\du}{0.000000\du}}\definecolor{dialinecolor}{rgb}{0.000000, 0.000000, 0.000000}
\pgfsetstrokecolor{dialinecolor}
\pgfsetstrokeopacity{1.000000}
\draw (24.300000\du,9.500000\du)--(24.300000\du,11.950000\du)--(42.300000\du,11.950000\du)--(42.300000\du,9.500000\du)--cycle;
}% setfont left to latex
\definecolor{dialinecolor}{rgb}{0.000000, 0.000000, 0.000000}
\pgfsetstrokecolor{dialinecolor}
\pgfsetstrokeopacity{1.000000}
\definecolor{diafillcolor}{rgb}{0.000000, 0.000000, 0.000000}
\pgfsetfillcolor{diafillcolor}
\pgfsetfillopacity{1.000000}
\node[anchor=base,inner sep=0pt, outer sep=0pt,color=dialinecolor] at (33.300000\du,10.920000\du){};
\pgfsetlinewidth{0.100000\du}
\pgfsetdash{}{0pt}
\pgfsetmiterjoin
{\pgfsetcornersarced{\pgfpoint{0.000000\du}{0.000000\du}}\definecolor{diafillcolor}{rgb}{1.000000, 1.000000, 1.000000}
\pgfsetfillcolor{diafillcolor}
\pgfsetfillopacity{1.000000}
\fill (23.200000\du,13.733333\du)--(23.200000\du,17.183333\du)--(28.200000\du,17.183333\du)--(28.200000\du,13.733333\du)--cycle;
}{\pgfsetcornersarced{\pgfpoint{0.000000\du}{0.000000\du}}\definecolor{dialinecolor}{rgb}{0.000000, 0.000000, 0.000000}
\pgfsetstrokecolor{dialinecolor}
\pgfsetstrokeopacity{1.000000}
\draw (23.200000\du,13.733333\du)--(23.200000\du,17.183333\du)--(28.200000\du,17.183333\du)--(28.200000\du,13.733333\du)--cycle;
}% setfont left to latex
\definecolor{dialinecolor}{rgb}{0.000000, 0.000000, 0.000000}
\pgfsetstrokecolor{dialinecolor}
\pgfsetstrokeopacity{1.000000}
\definecolor{diafillcolor}{rgb}{0.000000, 0.000000, 0.000000}
\pgfsetfillcolor{diafillcolor}
\pgfsetfillopacity{1.000000}
\node[anchor=base,inner sep=0pt, outer sep=0pt,color=dialinecolor] at (25.700000\du,15.802778\du){$E_k$};
% setfont left to latex
\definecolor{dialinecolor}{rgb}{0.000000, 0.000000, 0.000000}
\pgfsetstrokecolor{dialinecolor}
\pgfsetstrokeopacity{1.000000}
\definecolor{diafillcolor}{rgb}{0.000000, 0.000000, 0.000000}
\pgfsetfillcolor{diafillcolor}
\pgfsetfillopacity{1.000000}
\node[anchor=base west,inner sep=0pt,outer sep=0pt,color=dialinecolor] at (25.200000\du,4.733333\du){IV};
\pgfsetlinewidth{0.100000\du}
\pgfsetdash{}{0pt}
\pgfsetmiterjoin
{\pgfsetcornersarced{\pgfpoint{0.000000\du}{0.000000\du}}\definecolor{diafillcolor}{rgb}{1.000000, 1.000000, 1.000000}
\pgfsetfillcolor{diafillcolor}
\pgfsetfillopacity{1.000000}
\fill (30.300000\du,5.733333\du)--(30.300000\du,9.183333\du)--(35.300000\du,9.183333\du)--(35.300000\du,5.733333\du)--cycle;
}{\pgfsetcornersarced{\pgfpoint{0.000000\du}{0.000000\du}}\definecolor{dialinecolor}{rgb}{0.000000, 0.000000, 0.000000}
\pgfsetstrokecolor{dialinecolor}
\pgfsetstrokeopacity{1.000000}
\draw (30.300000\du,5.733333\du)--(30.300000\du,9.183333\du)--(35.300000\du,9.183333\du)--(35.300000\du,5.733333\du)--cycle;
}% setfont left to latex
\definecolor{dialinecolor}{rgb}{0.000000, 0.000000, 0.000000}
\pgfsetstrokecolor{dialinecolor}
\pgfsetstrokeopacity{1.000000}
\definecolor{diafillcolor}{rgb}{0.000000, 0.000000, 0.000000}
\pgfsetfillcolor{diafillcolor}
\pgfsetfillopacity{1.000000}
\node[anchor=base,inner sep=0pt, outer sep=0pt,color=dialinecolor] at (32.800000\du,7.802778\du){$E_K$};
\pgfsetlinewidth{0.100000\du}
\pgfsetdash{}{0pt}
\pgfsetmiterjoin
{\pgfsetcornersarced{\pgfpoint{0.000000\du}{0.000000\du}}\definecolor{diafillcolor}{rgb}{1.000000, 1.000000, 1.000000}
\pgfsetfillcolor{diafillcolor}
\pgfsetfillopacity{1.000000}
\fill (30.300000\du,13.733333\du)--(30.300000\du,17.183333\du)--(35.300000\du,17.183333\du)--(35.300000\du,13.733333\du)--cycle;
}{\pgfsetcornersarced{\pgfpoint{0.000000\du}{0.000000\du}}\definecolor{dialinecolor}{rgb}{0.000000, 0.000000, 0.000000}
\pgfsetstrokecolor{dialinecolor}
\pgfsetstrokeopacity{1.000000}
\draw (30.300000\du,13.733333\du)--(30.300000\du,17.183333\du)--(35.300000\du,17.183333\du)--(35.300000\du,13.733333\du)--cycle;
}% setfont left to latex
\definecolor{dialinecolor}{rgb}{0.000000, 0.000000, 0.000000}
\pgfsetstrokecolor{dialinecolor}
\pgfsetstrokeopacity{1.000000}
\definecolor{diafillcolor}{rgb}{0.000000, 0.000000, 0.000000}
\pgfsetfillcolor{diafillcolor}
\pgfsetfillopacity{1.000000}
\node[anchor=base,inner sep=0pt, outer sep=0pt,color=dialinecolor] at (32.800000\du,15.802778\du){$E_k$};
% setfont left to latex
\definecolor{dialinecolor}{rgb}{0.000000, 0.000000, 0.000000}
\pgfsetstrokecolor{dialinecolor}
\pgfsetstrokeopacity{1.000000}
\definecolor{diafillcolor}{rgb}{0.000000, 0.000000, 0.000000}
\pgfsetfillcolor{diafillcolor}
\pgfsetfillopacity{1.000000}
\node[anchor=base west,inner sep=0pt,outer sep=0pt,color=dialinecolor] at (25.300000\du,10.966667\du){$0^n$};
% setfont left to latex
\definecolor{dialinecolor}{rgb}{0.000000, 0.000000, 0.000000}
\pgfsetstrokecolor{dialinecolor}
\pgfsetstrokeopacity{1.000000}
\definecolor{diafillcolor}{rgb}{0.000000, 0.000000, 0.000000}
\pgfsetfillcolor{diafillcolor}
\pgfsetfillopacity{1.000000}
\node[anchor=base west,inner sep=0pt,outer sep=0pt,color=dialinecolor] at (32.433333\du,11.033333\du){T};
% setfont left to latex
\definecolor{dialinecolor}{rgb}{0.000000, 0.000000, 0.000000}
\pgfsetstrokecolor{dialinecolor}
\pgfsetstrokeopacity{1.000000}
\definecolor{diafillcolor}{rgb}{0.000000, 0.000000, 0.000000}
\pgfsetfillcolor{diafillcolor}
\pgfsetfillopacity{1.000000}
\node[anchor=base west,inner sep=0pt,outer sep=0pt,color=dialinecolor] at (37.300000\du,10.966667\du){$E_K(0^n)$};
\pgfsetlinewidth{0.100000\du}
\pgfsetdash{}{0pt}
\pgfsetbuttcap
{
\definecolor{diafillcolor}{rgb}{0.000000, 0.000000, 0.000000}
\pgfsetfillcolor{diafillcolor}
\pgfsetfillopacity{1.000000}
% was here!!!
\pgfsetarrowsend{stealth}
\definecolor{dialinecolor}{rgb}{0.000000, 0.000000, 0.000000}
\pgfsetstrokecolor{dialinecolor}
\pgfsetstrokeopacity{1.000000}
\draw (25.683333\du,5.283333\du)--(25.666667\du,9.733333\du);
}
\pgfsetlinewidth{0.100000\du}
\pgfsetdash{}{0pt}
\pgfsetbuttcap
{
\definecolor{diafillcolor}{rgb}{0.000000, 0.000000, 0.000000}
\pgfsetfillcolor{diafillcolor}
\pgfsetfillopacity{1.000000}
% was here!!!
\pgfsetarrowsend{stealth}
\definecolor{dialinecolor}{rgb}{0.000000, 0.000000, 0.000000}
\pgfsetstrokecolor{dialinecolor}
\pgfsetstrokeopacity{1.000000}
\draw (32.800000\du,9.183333\du)--(32.750000\du,10.183333\du);
}
\pgfsetlinewidth{0.100000\du}
\pgfsetdash{}{0pt}
\pgfsetbuttcap
{
\definecolor{diafillcolor}{rgb}{0.000000, 0.000000, 0.000000}
\pgfsetfillcolor{diafillcolor}
\pgfsetfillopacity{1.000000}
% was here!!!
\pgfsetarrowsend{stealth}
\definecolor{dialinecolor}{rgb}{0.000000, 0.000000, 0.000000}
\pgfsetstrokecolor{dialinecolor}
\pgfsetstrokeopacity{1.000000}
\draw (25.708333\du,11.116667\du)--(25.700000\du,13.733333\du);
}
\pgfsetlinewidth{0.100000\du}
\pgfsetdash{}{0pt}
\pgfsetbuttcap
\pgfsetmiterjoin
\pgfsetlinewidth{0.100000\du}
\pgfsetbuttcap
\pgfsetmiterjoin
\pgfsetdash{}{0pt}
\definecolor{diafillcolor}{rgb}{1.000000, 1.000000, 1.000000}
\pgfsetfillcolor{diafillcolor}
\pgfsetfillopacity{1.000000}
\pgfpathellipse{\pgfpoint{32.828333\du}{12.485096\du}}{\pgfpoint{0.575000\du}{0\du}}{\pgfpoint{0\du}{0.600000\du}}
\pgfusepath{fill}
\definecolor{dialinecolor}{rgb}{0.000000, 0.000000, 0.000000}
\pgfsetstrokecolor{dialinecolor}
\pgfsetstrokeopacity{1.000000}
\pgfpathellipse{\pgfpoint{32.828333\du}{12.485096\du}}{\pgfpoint{0.575000\du}{0\du}}{\pgfpoint{0\du}{0.600000\du}}
\pgfusepath{stroke}
\pgfsetbuttcap
\pgfsetmiterjoin
\pgfsetdash{}{0pt}
\definecolor{dialinecolor}{rgb}{0.000000, 0.000000, 0.000000}
\pgfsetstrokecolor{dialinecolor}
\pgfsetstrokeopacity{1.000000}
\draw (32.828333\du,11.885096\du)--(32.828333\du,13.085096\du);
\pgfsetbuttcap
\pgfsetmiterjoin
\pgfsetdash{}{0pt}
\definecolor{dialinecolor}{rgb}{0.000000, 0.000000, 0.000000}
\pgfsetstrokecolor{dialinecolor}
\pgfsetstrokeopacity{1.000000}
\draw (32.253333\du,12.485096\du)--(33.403333\du,12.485096\du);
\pgfsetlinewidth{0.100000\du}
\pgfsetdash{}{0pt}
\pgfsetbuttcap
{
\definecolor{diafillcolor}{rgb}{0.000000, 0.000000, 0.000000}
\pgfsetfillcolor{diafillcolor}
\pgfsetfillopacity{1.000000}
% was here!!!
\pgfsetarrowsend{stealth}
\definecolor{dialinecolor}{rgb}{0.000000, 0.000000, 0.000000}
\pgfsetstrokecolor{dialinecolor}
\pgfsetstrokeopacity{1.000000}
\draw (32.836667\du,11.268430\du)--(32.828333\du,11.885096\du);
}
\pgfsetlinewidth{0.100000\du}
\pgfsetdash{}{0pt}
\pgfsetbuttcap
{
\definecolor{diafillcolor}{rgb}{0.000000, 0.000000, 0.000000}
\pgfsetfillcolor{diafillcolor}
\pgfsetfillopacity{1.000000}
% was here!!!
\pgfsetarrowsend{stealth}
\definecolor{dialinecolor}{rgb}{0.000000, 0.000000, 0.000000}
\pgfsetstrokecolor{dialinecolor}
\pgfsetstrokeopacity{1.000000}
\draw (32.828333\du,13.085096\du)--(32.800000\du,13.733333\du);
}
\pgfsetlinewidth{0.100000\du}
\pgfsetdash{}{0pt}
\pgfsetmiterjoin
\pgfsetbuttcap
{
\definecolor{diafillcolor}{rgb}{0.000000, 0.000000, 0.000000}
\pgfsetfillcolor{diafillcolor}
\pgfsetfillopacity{1.000000}
% was here!!!
\pgfsetarrowsend{stealth}
{\pgfsetcornersarced{\pgfpoint{0.000000\du}{0.000000\du}}\definecolor{dialinecolor}{rgb}{0.000000, 0.000000, 0.000000}
\pgfsetstrokecolor{dialinecolor}
\pgfsetstrokeopacity{1.000000}
\draw (25.700000\du,17.183333\du)--(25.700000\du,18.233333\du)--(28.951667\du,18.233333\du)--(28.951667\du,12.485096\du)--(32.253333\du,12.485096\du);
}}
\pgfsetlinewidth{0.100000\du}
\pgfsetdash{}{0pt}
\pgfsetbuttcap
{
\definecolor{diafillcolor}{rgb}{0.000000, 0.000000, 0.000000}
\pgfsetfillcolor{diafillcolor}
\pgfsetfillopacity{1.000000}
% was here!!!
\pgfsetarrowsend{stealth}
\definecolor{dialinecolor}{rgb}{0.000000, 0.000000, 0.000000}
\pgfsetstrokecolor{dialinecolor}
\pgfsetstrokeopacity{1.000000}
\draw (26.483333\du,4.316667\du)--(32.256667\du,4.348430\du);
}
\pgfsetlinewidth{0.100000\du}
\pgfsetdash{}{0pt}
\pgfsetmiterjoin
\pgfsetbuttcap
{
\definecolor{diafillcolor}{rgb}{0.000000, 0.000000, 0.000000}
\pgfsetfillcolor{diafillcolor}
\pgfsetfillopacity{1.000000}
% was here!!!
\pgfsetarrowsend{stealth}
{\pgfsetcornersarced{\pgfpoint{0.000000\du}{0.000000\du}}\definecolor{dialinecolor}{rgb}{0.000000, 0.000000, 0.000000}
\pgfsetstrokecolor{dialinecolor}
\pgfsetstrokeopacity{1.000000}
\draw (32.800000\du,17.183333\du)--(32.800000\du,18.233333\du)--(36.216667\du,18.233333\du)--(36.216667\du,10.600000\du)--(37.316667\du,10.600000\du);
}}
% setfont left to latex
\definecolor{dialinecolor}{rgb}{0.000000, 0.000000, 0.000000}
\pgfsetstrokecolor{dialinecolor}
\pgfsetstrokeopacity{1.000000}
\definecolor{diafillcolor}{rgb}{0.000000, 0.000000, 0.000000}
\pgfsetfillcolor{diafillcolor}
\pgfsetfillopacity{1.000000}
\node[anchor=base west,inner sep=0pt,outer sep=0pt,color=dialinecolor] at (32.070000\du,2.596667\du){$0^n$};
\pgfsetlinewidth{0.100000\du}
\pgfsetdash{}{0pt}
\pgfsetbuttcap
\pgfsetmiterjoin
\pgfsetlinewidth{0.100000\du}
\pgfsetbuttcap
\pgfsetmiterjoin
\pgfsetdash{}{0pt}
\definecolor{diafillcolor}{rgb}{1.000000, 1.000000, 1.000000}
\pgfsetfillcolor{diafillcolor}
\pgfsetfillopacity{1.000000}
\pgfpathellipse{\pgfpoint{32.831667\du}{4.348430\du}}{\pgfpoint{0.575000\du}{0\du}}{\pgfpoint{0\du}{0.600000\du}}
\pgfusepath{fill}
\definecolor{dialinecolor}{rgb}{0.000000, 0.000000, 0.000000}
\pgfsetstrokecolor{dialinecolor}
\pgfsetstrokeopacity{1.000000}
\pgfpathellipse{\pgfpoint{32.831667\du}{4.348430\du}}{\pgfpoint{0.575000\du}{0\du}}{\pgfpoint{0\du}{0.600000\du}}
\pgfusepath{stroke}
\pgfsetbuttcap
\pgfsetmiterjoin
\pgfsetdash{}{0pt}
\definecolor{dialinecolor}{rgb}{0.000000, 0.000000, 0.000000}
\pgfsetstrokecolor{dialinecolor}
\pgfsetstrokeopacity{1.000000}
\draw (32.831667\du,3.748430\du)--(32.831667\du,4.948430\du);
\pgfsetbuttcap
\pgfsetmiterjoin
\pgfsetdash{}{0pt}
\definecolor{dialinecolor}{rgb}{0.000000, 0.000000, 0.000000}
\pgfsetstrokecolor{dialinecolor}
\pgfsetstrokeopacity{1.000000}
\draw (32.256667\du,4.348430\du)--(33.406667\du,4.348430\du);
\pgfsetlinewidth{0.100000\du}
\pgfsetdash{}{0pt}
\pgfsetbuttcap
{
\definecolor{diafillcolor}{rgb}{0.000000, 0.000000, 0.000000}
\pgfsetfillcolor{diafillcolor}
\pgfsetfillopacity{1.000000}
% was here!!!
\pgfsetarrowsend{stealth}
\definecolor{dialinecolor}{rgb}{0.000000, 0.000000, 0.000000}
\pgfsetstrokecolor{dialinecolor}
\pgfsetstrokeopacity{1.000000}
\draw (32.840000\du,3.131763\du)--(32.831667\du,3.748430\du);
}
\pgfsetlinewidth{0.100000\du}
\pgfsetdash{}{0pt}
\pgfsetbuttcap
{
\definecolor{diafillcolor}{rgb}{0.000000, 0.000000, 0.000000}
\pgfsetfillcolor{diafillcolor}
\pgfsetfillopacity{1.000000}
% was here!!!
\pgfsetarrowsend{stealth}
\definecolor{dialinecolor}{rgb}{0.000000, 0.000000, 0.000000}
\pgfsetstrokecolor{dialinecolor}
\pgfsetstrokeopacity{1.000000}
\draw (32.831667\du,4.948430\du)--(32.800000\du,5.733333\du);
}
% setfont left to latex
\definecolor{dialinecolor}{rgb}{0.000000, 0.000000, 0.000000}
\pgfsetstrokecolor{dialinecolor}
\pgfsetstrokeopacity{1.000000}
\definecolor{diafillcolor}{rgb}{0.000000, 0.000000, 0.000000}
\pgfsetfillcolor{diafillcolor}
\pgfsetfillopacity{1.000000}
\node[anchor=base west,inner sep=0pt,outer sep=0pt,color=dialinecolor] at (32.841667\du,10.650000\du){};
% setfont left to latex
\definecolor{dialinecolor}{rgb}{0.000000, 0.000000, 0.000000}
\pgfsetstrokecolor{dialinecolor}
\pgfsetstrokeopacity{1.000000}
\definecolor{diafillcolor}{rgb}{0.000000, 0.000000, 0.000000}
\pgfsetfillcolor{diafillcolor}
\pgfsetfillopacity{1.000000}
\node[anchor=base west,inner sep=0pt,outer sep=0pt,color=dialinecolor] at (32.775000\du,10.850000\du){};
\pgfsetlinewidth{0.100000\du}
\pgfsetdash{}{0pt}
\pgfsetbuttcap
{
\definecolor{diafillcolor}{rgb}{0.000000, 0.000000, 0.000000}
\pgfsetfillcolor{diafillcolor}
\pgfsetfillopacity{1.000000}
% was here!!!
\definecolor{dialinecolor}{rgb}{0.000000, 0.000000, 0.000000}
\pgfsetstrokecolor{dialinecolor}
\pgfsetstrokeopacity{1.000000}
\draw (21.233333\du,0.066667\du)--(21.233333\du,20.066667\du);
}
% setfont left to latex
\definecolor{dialinecolor}{rgb}{0.000000, 0.000000, 0.000000}
\pgfsetstrokecolor{dialinecolor}
\pgfsetstrokeopacity{1.000000}
\definecolor{diafillcolor}{rgb}{0.000000, 0.000000, 0.000000}
\pgfsetfillcolor{diafillcolor}
\pgfsetfillopacity{1.000000}
\node[anchor=base west,inner sep=0pt,outer sep=0pt,color=dialinecolor] at (34.360100\du,2.366667\du){};
\pgfsetlinewidth{0.100000\du}
\pgfsetdash{}{0pt}
\pgfsetmiterjoin
{\pgfsetcornersarced{\pgfpoint{0.000000\du}{0.000000\du}}\definecolor{diafillcolor}{rgb}{1.000000, 0.000000, 0.000000}
\pgfsetfillcolor{diafillcolor}
\pgfsetfillopacity{1.000000}
\fill (7.533333\du,0.966667\du)--(7.533333\du,3.366667\du)--(13.450100\du,3.366667\du)--(13.450100\du,0.966667\du)--cycle;
}{\pgfsetcornersarced{\pgfpoint{0.000000\du}{0.000000\du}}\definecolor{dialinecolor}{rgb}{0.000000, 0.000000, 0.000000}
\pgfsetstrokecolor{dialinecolor}
\pgfsetstrokeopacity{1.000000}
\draw (7.533333\du,0.966667\du)--(7.533333\du,3.366667\du)--(13.450100\du,3.366667\du)--(13.450100\du,0.966667\du)--cycle;
}% setfont left to latex
\definecolor{dialinecolor}{rgb}{0.000000, 0.000000, 0.000000}
\pgfsetstrokecolor{dialinecolor}
\pgfsetstrokeopacity{1.000000}
\definecolor{diafillcolor}{rgb}{0.000000, 0.000000, 0.000000}
\pgfsetfillcolor{diafillcolor}
\pgfsetfillopacity{1.000000}
\node[anchor=base,inner sep=0pt, outer sep=0pt,color=dialinecolor] at (10.491717\du,2.361667\du){};
\pgfsetlinewidth{0.100000\du}
\pgfsetdash{}{0pt}
\pgfsetmiterjoin
{\pgfsetcornersarced{\pgfpoint{0.000000\du}{0.000000\du}}\definecolor{diafillcolor}{rgb}{1.000000, 1.000000, 1.000000}
\pgfsetfillcolor{diafillcolor}
\pgfsetfillopacity{1.000000}
\fill (0.400000\du,13.733333\du)--(0.400000\du,17.183333\du)--(5.400000\du,17.183333\du)--(5.400000\du,13.733333\du)--cycle;
}{\pgfsetcornersarced{\pgfpoint{0.000000\du}{0.000000\du}}\definecolor{dialinecolor}{rgb}{0.000000, 0.000000, 0.000000}
\pgfsetstrokecolor{dialinecolor}
\pgfsetstrokeopacity{1.000000}
\draw (0.400000\du,13.733333\du)--(0.400000\du,17.183333\du)--(5.400000\du,17.183333\du)--(5.400000\du,13.733333\du)--cycle;
}% setfont left to latex
\definecolor{dialinecolor}{rgb}{0.000000, 0.000000, 0.000000}
\pgfsetstrokecolor{dialinecolor}
\pgfsetstrokeopacity{1.000000}
\definecolor{diafillcolor}{rgb}{0.000000, 0.000000, 0.000000}
\pgfsetfillcolor{diafillcolor}
\pgfsetfillopacity{1.000000}
\node[anchor=base,inner sep=0pt, outer sep=0pt,color=dialinecolor] at (2.900000\du,15.802778\du){$E_K$};
% setfont left to latex
\definecolor{dialinecolor}{rgb}{0.000000, 0.000000, 0.000000}
\pgfsetstrokecolor{dialinecolor}
\pgfsetstrokeopacity{1.000000}
\definecolor{diafillcolor}{rgb}{0.000000, 0.000000, 0.000000}
\pgfsetfillcolor{diafillcolor}
\pgfsetfillopacity{1.000000}
\node[anchor=base west,inner sep=0pt,outer sep=0pt,color=dialinecolor] at (2.400000\du,4.733333\du){IV};
\pgfsetlinewidth{0.100000\du}
\pgfsetdash{}{0pt}
\pgfsetmiterjoin
{\pgfsetcornersarced{\pgfpoint{0.000000\du}{0.000000\du}}\definecolor{diafillcolor}{rgb}{1.000000, 1.000000, 1.000000}
\pgfsetfillcolor{diafillcolor}
\pgfsetfillopacity{1.000000}
\fill (7.500000\du,5.733333\du)--(7.500000\du,9.183333\du)--(12.500000\du,9.183333\du)--(12.500000\du,5.733333\du)--cycle;
}{\pgfsetcornersarced{\pgfpoint{0.000000\du}{0.000000\du}}\definecolor{dialinecolor}{rgb}{0.000000, 0.000000, 0.000000}
\pgfsetstrokecolor{dialinecolor}
\pgfsetstrokeopacity{1.000000}
\draw (7.500000\du,5.733333\du)--(7.500000\du,9.183333\du)--(12.500000\du,9.183333\du)--(12.500000\du,5.733333\du)--cycle;
}% setfont left to latex
\definecolor{dialinecolor}{rgb}{0.000000, 0.000000, 0.000000}
\pgfsetstrokecolor{dialinecolor}
\pgfsetstrokeopacity{1.000000}
\definecolor{diafillcolor}{rgb}{0.000000, 0.000000, 0.000000}
\pgfsetfillcolor{diafillcolor}
\pgfsetfillopacity{1.000000}
\node[anchor=base,inner sep=0pt, outer sep=0pt,color=dialinecolor] at (10.000000\du,7.802778\du){$E_K$};
\pgfsetlinewidth{0.100000\du}
\pgfsetdash{}{0pt}
\pgfsetmiterjoin
{\pgfsetcornersarced{\pgfpoint{0.000000\du}{0.000000\du}}\definecolor{diafillcolor}{rgb}{1.000000, 1.000000, 1.000000}
\pgfsetfillcolor{diafillcolor}
\pgfsetfillopacity{1.000000}
\fill (7.500000\du,13.733333\du)--(7.500000\du,17.183333\du)--(12.500000\du,17.183333\du)--(12.500000\du,13.733333\du)--cycle;
}{\pgfsetcornersarced{\pgfpoint{0.000000\du}{0.000000\du}}\definecolor{dialinecolor}{rgb}{0.000000, 0.000000, 0.000000}
\pgfsetstrokecolor{dialinecolor}
\pgfsetstrokeopacity{1.000000}
\draw (7.500000\du,13.733333\du)--(7.500000\du,17.183333\du)--(12.500000\du,17.183333\du)--(12.500000\du,13.733333\du)--cycle;
}% setfont left to latex
\definecolor{dialinecolor}{rgb}{0.000000, 0.000000, 0.000000}
\pgfsetstrokecolor{dialinecolor}
\pgfsetstrokeopacity{1.000000}
\definecolor{diafillcolor}{rgb}{0.000000, 0.000000, 0.000000}
\pgfsetfillcolor{diafillcolor}
\pgfsetfillopacity{1.000000}
\node[anchor=base,inner sep=0pt, outer sep=0pt,color=dialinecolor] at (10.000000\du,15.802778\du){$E_K$};
% setfont left to latex
\definecolor{dialinecolor}{rgb}{0.000000, 0.000000, 0.000000}
\pgfsetstrokecolor{dialinecolor}
\pgfsetstrokeopacity{1.000000}
\definecolor{diafillcolor}{rgb}{0.000000, 0.000000, 0.000000}
\pgfsetfillcolor{diafillcolor}
\pgfsetfillopacity{1.000000}
\node[anchor=base west,inner sep=0pt,outer sep=0pt,color=dialinecolor] at (2.400000\du,10.966667\du){$C_0$};
% setfont left to latex
\definecolor{dialinecolor}{rgb}{0.000000, 0.000000, 0.000000}
\pgfsetstrokecolor{dialinecolor}
\pgfsetstrokeopacity{1.000000}
\definecolor{diafillcolor}{rgb}{0.000000, 0.000000, 0.000000}
\pgfsetfillcolor{diafillcolor}
\pgfsetfillopacity{1.000000}
\node[anchor=base west,inner sep=0pt,outer sep=0pt,color=dialinecolor] at (9.500000\du,10.966667\du){$C_1$};
% setfont left to latex
\definecolor{dialinecolor}{rgb}{0.000000, 0.000000, 0.000000}
\pgfsetstrokecolor{dialinecolor}
\pgfsetstrokeopacity{1.000000}
\definecolor{diafillcolor}{rgb}{0.000000, 0.000000, 0.000000}
\pgfsetfillcolor{diafillcolor}
\pgfsetfillopacity{1.000000}
\node[anchor=base west,inner sep=0pt,outer sep=0pt,color=dialinecolor] at (14.500000\du,10.966667\du){\huge{$T=E_K(0^n)$}};
\pgfsetlinewidth{0.100000\du}
\pgfsetdash{}{0pt}
\pgfsetbuttcap
{
\definecolor{diafillcolor}{rgb}{0.000000, 0.000000, 0.000000}
\pgfsetfillcolor{diafillcolor}
\pgfsetfillopacity{1.000000}
% was here!!!
\pgfsetarrowsend{stealth}
\definecolor{dialinecolor}{rgb}{0.000000, 0.000000, 0.000000}
\pgfsetstrokecolor{dialinecolor}
\pgfsetstrokeopacity{1.000000}
\draw (2.883333\du,5.283333\du)--(2.866667\du,9.733333\du);
}
\pgfsetlinewidth{0.100000\du}
\pgfsetdash{}{0pt}
\pgfsetbuttcap
{
\definecolor{diafillcolor}{rgb}{0.000000, 0.000000, 0.000000}
\pgfsetfillcolor{diafillcolor}
\pgfsetfillopacity{1.000000}
% was here!!!
\pgfsetarrowsend{stealth}
\definecolor{dialinecolor}{rgb}{0.000000, 0.000000, 0.000000}
\pgfsetstrokecolor{dialinecolor}
\pgfsetstrokeopacity{1.000000}
\draw (10.000000\du,9.183333\du)--(9.950000\du,10.183333\du);
}
\pgfsetlinewidth{0.100000\du}
\pgfsetdash{}{0pt}
\pgfsetbuttcap
{
\definecolor{diafillcolor}{rgb}{0.000000, 0.000000, 0.000000}
\pgfsetfillcolor{diafillcolor}
\pgfsetfillopacity{1.000000}
% was here!!!
\pgfsetarrowsend{stealth}
\definecolor{dialinecolor}{rgb}{0.000000, 0.000000, 0.000000}
\pgfsetstrokecolor{dialinecolor}
\pgfsetstrokeopacity{1.000000}
\draw (2.908333\du,11.116667\du)--(2.900000\du,13.733333\du);
}
\pgfsetlinewidth{0.100000\du}
\pgfsetdash{}{0pt}
\pgfsetbuttcap
\pgfsetmiterjoin
\pgfsetlinewidth{0.100000\du}
\pgfsetbuttcap
\pgfsetmiterjoin
\pgfsetdash{}{0pt}
\definecolor{diafillcolor}{rgb}{1.000000, 1.000000, 1.000000}
\pgfsetfillcolor{diafillcolor}
\pgfsetfillopacity{1.000000}
\pgfpathellipse{\pgfpoint{10.028333\du}{12.485096\du}}{\pgfpoint{0.575000\du}{0\du}}{\pgfpoint{0\du}{0.600000\du}}
\pgfusepath{fill}
\definecolor{dialinecolor}{rgb}{0.000000, 0.000000, 0.000000}
\pgfsetstrokecolor{dialinecolor}
\pgfsetstrokeopacity{1.000000}
\pgfpathellipse{\pgfpoint{10.028333\du}{12.485096\du}}{\pgfpoint{0.575000\du}{0\du}}{\pgfpoint{0\du}{0.600000\du}}
\pgfusepath{stroke}
\pgfsetbuttcap
\pgfsetmiterjoin
\pgfsetdash{}{0pt}
\definecolor{dialinecolor}{rgb}{0.000000, 0.000000, 0.000000}
\pgfsetstrokecolor{dialinecolor}
\pgfsetstrokeopacity{1.000000}
\draw (10.028333\du,11.885096\du)--(10.028333\du,13.085096\du);
\pgfsetbuttcap
\pgfsetmiterjoin
\pgfsetdash{}{0pt}
\definecolor{dialinecolor}{rgb}{0.000000, 0.000000, 0.000000}
\pgfsetstrokecolor{dialinecolor}
\pgfsetstrokeopacity{1.000000}
\draw (9.453333\du,12.485096\du)--(10.603333\du,12.485096\du);
\pgfsetlinewidth{0.100000\du}
\pgfsetdash{}{0pt}
\pgfsetbuttcap
{
\definecolor{diafillcolor}{rgb}{0.000000, 0.000000, 0.000000}
\pgfsetfillcolor{diafillcolor}
\pgfsetfillopacity{1.000000}
% was here!!!
\pgfsetarrowsend{stealth}
\definecolor{dialinecolor}{rgb}{0.000000, 0.000000, 0.000000}
\pgfsetstrokecolor{dialinecolor}
\pgfsetstrokeopacity{1.000000}
\draw (10.036667\du,11.268430\du)--(10.028333\du,11.885096\du);
}
\pgfsetlinewidth{0.100000\du}
\pgfsetdash{}{0pt}
\pgfsetbuttcap
{
\definecolor{diafillcolor}{rgb}{0.000000, 0.000000, 0.000000}
\pgfsetfillcolor{diafillcolor}
\pgfsetfillopacity{1.000000}
% was here!!!
\pgfsetarrowsend{stealth}
\definecolor{dialinecolor}{rgb}{0.000000, 0.000000, 0.000000}
\pgfsetstrokecolor{dialinecolor}
\pgfsetstrokeopacity{1.000000}
\draw (10.028333\du,13.085096\du)--(10.000000\du,13.733333\du);
}
\pgfsetlinewidth{0.100000\du}
\pgfsetdash{}{0pt}
\pgfsetmiterjoin
\pgfsetbuttcap
{
\definecolor{diafillcolor}{rgb}{0.000000, 0.000000, 0.000000}
\pgfsetfillcolor{diafillcolor}
\pgfsetfillopacity{1.000000}
% was here!!!
\pgfsetarrowsend{stealth}
{\pgfsetcornersarced{\pgfpoint{0.000000\du}{0.000000\du}}\definecolor{dialinecolor}{rgb}{0.000000, 0.000000, 0.000000}
\pgfsetstrokecolor{dialinecolor}
\pgfsetstrokeopacity{1.000000}
\draw (2.900000\du,17.183333\du)--(2.900000\du,18.233333\du)--(6.151667\du,18.233333\du)--(6.151667\du,12.485096\du)--(9.453333\du,12.485096\du);
}}
\pgfsetlinewidth{0.100000\du}
\pgfsetdash{}{0pt}
\pgfsetbuttcap
{
\definecolor{diafillcolor}{rgb}{0.000000, 0.000000, 0.000000}
\pgfsetfillcolor{diafillcolor}
\pgfsetfillopacity{1.000000}
% was here!!!
\pgfsetarrowsend{stealth}
\definecolor{dialinecolor}{rgb}{0.000000, 0.000000, 0.000000}
\pgfsetstrokecolor{dialinecolor}
\pgfsetstrokeopacity{1.000000}
\draw (3.683333\du,4.316667\du)--(9.456667\du,4.348430\du);
}
\pgfsetlinewidth{0.100000\du}
\pgfsetdash{}{0pt}
\pgfsetmiterjoin
\pgfsetbuttcap
{
\definecolor{diafillcolor}{rgb}{0.000000, 0.000000, 0.000000}
\pgfsetfillcolor{diafillcolor}
\pgfsetfillopacity{1.000000}
% was here!!!
\pgfsetarrowsend{stealth}
{\pgfsetcornersarced{\pgfpoint{0.000000\du}{0.000000\du}}\definecolor{dialinecolor}{rgb}{0.000000, 0.000000, 0.000000}
\pgfsetstrokecolor{dialinecolor}
\pgfsetstrokeopacity{1.000000}
\draw (10.000000\du,17.183333\du)--(10.000000\du,18.233333\du)--(13.416667\du,18.233333\du)--(13.416667\du,10.600000\du)--(14.516667\du,10.600000\du);
}}
% setfont left to latex
\definecolor{dialinecolor}{rgb}{0.000000, 0.000000, 0.000000}
\pgfsetstrokecolor{dialinecolor}
\pgfsetstrokeopacity{1.000000}
\definecolor{diafillcolor}{rgb}{0.000000, 0.000000, 0.000000}
\pgfsetfillcolor{diafillcolor}
\pgfsetfillopacity{1.000000}
\node[anchor=base west,inner sep=0pt,outer sep=0pt,color=dialinecolor] at (9.270000\du,2.596667\du){$0^n$};
\pgfsetlinewidth{0.100000\du}
\pgfsetdash{}{0pt}
\pgfsetbuttcap
\pgfsetmiterjoin
\pgfsetlinewidth{0.100000\du}
\pgfsetbuttcap
\pgfsetmiterjoin
\pgfsetdash{}{0pt}
\definecolor{diafillcolor}{rgb}{1.000000, 1.000000, 1.000000}
\pgfsetfillcolor{diafillcolor}
\pgfsetfillopacity{1.000000}
\pgfpathellipse{\pgfpoint{10.031667\du}{4.348430\du}}{\pgfpoint{0.575000\du}{0\du}}{\pgfpoint{0\du}{0.600000\du}}
\pgfusepath{fill}
\definecolor{dialinecolor}{rgb}{0.000000, 0.000000, 0.000000}
\pgfsetstrokecolor{dialinecolor}
\pgfsetstrokeopacity{1.000000}
\pgfpathellipse{\pgfpoint{10.031667\du}{4.348430\du}}{\pgfpoint{0.575000\du}{0\du}}{\pgfpoint{0\du}{0.600000\du}}
\pgfusepath{stroke}
\pgfsetbuttcap
\pgfsetmiterjoin
\pgfsetdash{}{0pt}
\definecolor{dialinecolor}{rgb}{0.000000, 0.000000, 0.000000}
\pgfsetstrokecolor{dialinecolor}
\pgfsetstrokeopacity{1.000000}
\draw (10.031667\du,3.748430\du)--(10.031667\du,4.948430\du);
\pgfsetbuttcap
\pgfsetmiterjoin
\pgfsetdash{}{0pt}
\definecolor{dialinecolor}{rgb}{0.000000, 0.000000, 0.000000}
\pgfsetstrokecolor{dialinecolor}
\pgfsetstrokeopacity{1.000000}
\draw (9.456667\du,4.348430\du)--(10.606667\du,4.348430\du);
\pgfsetlinewidth{0.100000\du}
\pgfsetdash{}{0pt}
\pgfsetbuttcap
{
\definecolor{diafillcolor}{rgb}{0.000000, 0.000000, 0.000000}
\pgfsetfillcolor{diafillcolor}
\pgfsetfillopacity{1.000000}
% was here!!!
\pgfsetarrowsend{stealth}
\definecolor{dialinecolor}{rgb}{0.000000, 0.000000, 0.000000}
\pgfsetstrokecolor{dialinecolor}
\pgfsetstrokeopacity{1.000000}
\draw (10.040000\du,3.131763\du)--(10.031667\du,3.748430\du);
}
\pgfsetlinewidth{0.100000\du}
\pgfsetdash{}{0pt}
\pgfsetbuttcap
{
\definecolor{diafillcolor}{rgb}{0.000000, 0.000000, 0.000000}
\pgfsetfillcolor{diafillcolor}
\pgfsetfillopacity{1.000000}
% was here!!!
\pgfsetarrowsend{stealth}
\definecolor{dialinecolor}{rgb}{0.000000, 0.000000, 0.000000}
\pgfsetstrokecolor{dialinecolor}
\pgfsetstrokeopacity{1.000000}
\draw (10.031667\du,4.948430\du)--(10.000000\du,5.733333\du);
}
% setfont left to latex
\definecolor{dialinecolor}{rgb}{0.000000, 0.000000, 0.000000}
\pgfsetstrokecolor{dialinecolor}
\pgfsetstrokeopacity{1.000000}
\definecolor{diafillcolor}{rgb}{0.000000, 0.000000, 0.000000}
\pgfsetfillcolor{diafillcolor}
\pgfsetfillopacity{1.000000}
\node[anchor=base west,inner sep=0pt,outer sep=0pt,color=dialinecolor] at (10.041667\du,10.650000\du){};
% setfont left to latex
\definecolor{dialinecolor}{rgb}{0.000000, 0.000000, 0.000000}
\pgfsetstrokecolor{dialinecolor}
\pgfsetstrokeopacity{1.000000}
\definecolor{diafillcolor}{rgb}{0.000000, 0.000000, 0.000000}
\pgfsetfillcolor{diafillcolor}
\pgfsetfillopacity{1.000000}
\node[anchor=base west,inner sep=0pt,outer sep=0pt,color=dialinecolor] at (9.975000\du,10.850000\du){};
% setfont left to latex
\definecolor{dialinecolor}{rgb}{0.000000, 0.000000, 0.000000}
\pgfsetstrokecolor{dialinecolor}
\pgfsetstrokeopacity{1.000000}
\definecolor{diafillcolor}{rgb}{0.000000, 0.000000, 0.000000}
\pgfsetfillcolor{diafillcolor}
\pgfsetfillopacity{1.000000}
\node[anchor=base west,inner sep=0pt,outer sep=0pt,color=dialinecolor] at (11.560100\du,2.366667\du){};
\end{tikzpicture}
\normalsize}
  \caption{The intuition behind the key reuse adversary for CTXT $A$}
\label{fig:reuseadversary}
\end{figure}

The intuition for the proof is shown in Figure~\ref{fig:reuseadversary}.  On the left side, this figure shows an adversary querying an encryption of 
$0^n$ with a random IV.  On the left side, this figure shows a valid encryption of the string $0^n$ under the IV $0^n$ yielding a desired ciphertext 
of $0^nTT$, which forms the adversarial input to the decryption procedure that wins the CTXT game.

To prove this, note that $C_1 = E_K(M_1 \oplus C_0) = E_K(0^n \oplus C_0) = E_K(C_0)$.

Also, $T=E_K(C_1 \oplus E_K(C_0))=E_K(C_1 \oplus C_1) = E_K(0^n)$ (by the above).  So, as indicated in the figure, $T=E_K(0^n)$.

Note that the above equalities, $C_1=E_K(C_0)$ and $T=E_K(0^n)$ hold in general when $M=0^n$.  The right side of the figure now sets $C_0=0^n$, which 
implies that $C_1=E_K(0^n)=T$.  So, the valid ciphertext for an encryption of $0^n$ under IV (input as $C_0$) $0^n$ is $0^nTT$, forming a valid 
ciphertext that will not return $\bot$ under decryption.

Thus, $A$ wins the CTXT game as long as this ciphertext was not previously returned by the encryption oracle.  Note that the only ciphertext returned 
by the encryption oracle for $A$ is $C_0C_1T$.  For this to be equal to the adverary's decryption query of $0^nTT$, $C_0$ must equal $T$.  But, $C_0$ 
is the IV that is uniformly chosen in the adversary's first call to the decryption oracle.  The probability that this random IV is equal to $0^n$ and 
therefore the decryption oracle outputs $\bot$ is $\frac{1}{2^n}$.

So, $A$ wins the CTXT game with probability $1-\frac{1}{2^n}$ in two oracle queries, proving a potential lack of security in key reuse and 
illustrating that extreme care is required when proving the security of composed constructions.

\subsection{Towards Proper Constructions}

These constructions show three natural possible such combinations, labeled $Enc$ and $Mac$.  We now explore the tradeoffs associated with each.  Note 
that the constructions we claim are \emph{insecure} specifically means that there exists some secure encryption and MAC algorithm which does not 
result in a secure composition; for more information and an extension of this result to weakly unforgeable MACs, see~\cite{Bellare2000}.  
Independently chosen $K_1$ and $K_2$ are naturally assumed for security.


\begin{figure}
\hspace{-10mm}\scalebox{0.6}{\huge
% Graphic for TeX using PGF
% Title: /home/phil/genericcomp.dia
% Creator: Dia v0.97+git
% CreationDate: Thu Mar  7 12:10:55 2019
% For: phil
% \usepackage{tikz}
% The following commands are not supported in PSTricks at present
% We define them conditionally, so when they are implemented,
% this pgf file will use them.
\ifx\du\undefined
  \newlength{\du}
\fi
\setlength{\du}{15\unitlength}
\begin{tikzpicture}[even odd rule]
\pgftransformxscale{1.000000}
\pgftransformyscale{-1.000000}
\definecolor{dialinecolor}{rgb}{0.000000, 0.000000, 0.000000}
\pgfsetstrokecolor{dialinecolor}
\pgfsetstrokeopacity{1.000000}
\definecolor{diafillcolor}{rgb}{1.000000, 1.000000, 1.000000}
\pgfsetfillcolor{diafillcolor}
\pgfsetfillopacity{1.000000}
\pgfsetlinewidth{0.100000\du}
\pgfsetdash{}{0pt}
\pgfsetmiterjoin
{\pgfsetcornersarced{\pgfpoint{0.000000\du}{0.000000\du}}\definecolor{diafillcolor}{rgb}{0.596078, 0.733333, 0.819608}
\pgfsetfillcolor{diafillcolor}
\pgfsetfillopacity{1.000000}
\fill (4.550000\du,6.850000\du)--(4.550000\du,9.361111\du)--(9.050000\du,9.361111\du)--(9.050000\du,6.850000\du)--cycle;
}{\pgfsetcornersarced{\pgfpoint{0.000000\du}{0.000000\du}}\definecolor{dialinecolor}{rgb}{0.000000, 0.000000, 0.000000}
\pgfsetstrokecolor{dialinecolor}
\pgfsetstrokeopacity{1.000000}
\draw (4.550000\du,6.850000\du)--(4.550000\du,9.361111\du)--(9.050000\du,9.361111\du)--(9.050000\du,6.850000\du)--cycle;
}% setfont left to latex
\definecolor{dialinecolor}{rgb}{0.000000, 0.000000, 0.000000}
\pgfsetstrokecolor{dialinecolor}
\pgfsetstrokeopacity{1.000000}
\definecolor{diafillcolor}{rgb}{0.000000, 0.000000, 0.000000}
\pgfsetfillcolor{diafillcolor}
\pgfsetfillopacity{1.000000}
\node[anchor=base,inner sep=0pt, outer sep=0pt,color=dialinecolor] at (6.800000\du,8.450000\du){Enc};
\pgfsetlinewidth{0.100000\du}
\pgfsetdash{}{0pt}
\pgfsetmiterjoin
{\pgfsetcornersarced{\pgfpoint{0.000000\du}{0.000000\du}}\definecolor{diafillcolor}{rgb}{0.737255, 0.819608, 0.596078}
\pgfsetfillcolor{diafillcolor}
\pgfsetfillopacity{1.000000}
\fill (12.005000\du,6.810000\du)--(12.005000\du,9.410000\du)--(16.455000\du,9.410000\du)--(16.455000\du,6.810000\du)--cycle;
}{\pgfsetcornersarced{\pgfpoint{0.000000\du}{0.000000\du}}\definecolor{dialinecolor}{rgb}{0.000000, 0.000000, 0.000000}
\pgfsetstrokecolor{dialinecolor}
\pgfsetstrokeopacity{1.000000}
\draw (12.005000\du,6.810000\du)--(12.005000\du,9.410000\du)--(16.455000\du,9.410000\du)--(16.455000\du,6.810000\du)--cycle;
}% setfont left to latex
\definecolor{dialinecolor}{rgb}{0.000000, 0.000000, 0.000000}
\pgfsetstrokecolor{dialinecolor}
\pgfsetstrokeopacity{1.000000}
\definecolor{diafillcolor}{rgb}{0.000000, 0.000000, 0.000000}
\pgfsetfillcolor{diafillcolor}
\pgfsetfillopacity{1.000000}
\node[anchor=base,inner sep=0pt, outer sep=0pt,color=dialinecolor] at (14.230000\du,8.454444\du){Mac};
% setfont left to latex
\definecolor{dialinecolor}{rgb}{0.200000, 0.321569, 0.168627}
\pgfsetstrokecolor{dialinecolor}
\pgfsetstrokeopacity{1.000000}
\definecolor{diafillcolor}{rgb}{0.200000, 0.321569, 0.168627}
\pgfsetfillcolor{diafillcolor}
\pgfsetfillopacity{1.000000}
\node[anchor=base west,inner sep=0pt,outer sep=0pt,color=dialinecolor] at (1.150000\du,8.500000\du){ $K_1$};
% setfont left to latex
\definecolor{dialinecolor}{rgb}{0.200000, 0.321569, 0.168627}
\pgfsetstrokecolor{dialinecolor}
\pgfsetstrokeopacity{1.000000}
\definecolor{diafillcolor}{rgb}{0.200000, 0.321569, 0.168627}
\pgfsetfillcolor{diafillcolor}
\pgfsetfillopacity{1.000000}
\node[anchor=base west,inner sep=0pt,outer sep=0pt,color=dialinecolor] at (5.955000\du,4.860000\du){ M};
% setfont left to latex
\definecolor{dialinecolor}{rgb}{0.200000, 0.321569, 0.168627}
\pgfsetstrokecolor{dialinecolor}
\pgfsetstrokeopacity{1.000000}
\definecolor{diafillcolor}{rgb}{0.200000, 0.321569, 0.168627}
\pgfsetfillcolor{diafillcolor}
\pgfsetfillopacity{1.000000}
\node[anchor=base west,inner sep=0pt,outer sep=0pt,color=dialinecolor] at (6.055000\du,11.910000\du){ C};
% setfont left to latex
\definecolor{dialinecolor}{rgb}{0.000000, 0.000000, 0.000000}
\pgfsetstrokecolor{dialinecolor}
\pgfsetstrokeopacity{1.000000}
\definecolor{diafillcolor}{rgb}{0.000000, 0.000000, 0.000000}
\pgfsetfillcolor{diafillcolor}
\pgfsetfillopacity{1.000000}
\node[anchor=base west,inner sep=0pt,outer sep=0pt,color=dialinecolor] at (7.150000\du,11.950000\du){};
% setfont left to latex
\definecolor{dialinecolor}{rgb}{0.200000, 0.321569, 0.168627}
\pgfsetstrokecolor{dialinecolor}
\pgfsetstrokeopacity{1.000000}
\definecolor{diafillcolor}{rgb}{0.200000, 0.321569, 0.168627}
\pgfsetfillcolor{diafillcolor}
\pgfsetfillopacity{1.000000}
\node[anchor=base west,inner sep=0pt,outer sep=0pt,color=dialinecolor] at (17.605000\du,8.410000\du){ $K_2$};
% setfont left to latex
\definecolor{dialinecolor}{rgb}{0.200000, 0.321569, 0.168627}
\pgfsetstrokecolor{dialinecolor}
\pgfsetstrokeopacity{1.000000}
\definecolor{diafillcolor}{rgb}{0.200000, 0.321569, 0.168627}
\pgfsetfillcolor{diafillcolor}
\pgfsetfillopacity{1.000000}
\node[anchor=base west,inner sep=0pt,outer sep=0pt,color=dialinecolor] at (13.555000\du,11.960000\du){ T};
\pgfsetlinewidth{0.100000\du}
\pgfsetdash{}{0pt}
\pgfsetbuttcap
{
\definecolor{diafillcolor}{rgb}{0.000000, 0.000000, 0.000000}
\pgfsetfillcolor{diafillcolor}
\pgfsetfillopacity{1.000000}
% was here!!!
\pgfsetarrowsend{stealth}
\definecolor{dialinecolor}{rgb}{0.000000, 0.000000, 0.000000}
\pgfsetstrokecolor{dialinecolor}
\pgfsetstrokeopacity{1.000000}
\draw (3.100000\du,8.126500\du)--(4.550000\du,8.105556\du);
}
% setfont left to latex
\definecolor{dialinecolor}{rgb}{0.000000, 0.000000, 0.000000}
\pgfsetstrokecolor{dialinecolor}
\pgfsetstrokeopacity{1.000000}
\definecolor{diafillcolor}{rgb}{0.000000, 0.000000, 0.000000}
\pgfsetfillcolor{diafillcolor}
\pgfsetfillopacity{1.000000}
\node[anchor=base west,inner sep=0pt,outer sep=0pt,color=dialinecolor] at (15.150000\du,-15.923500\du){};
\pgfsetlinewidth{0.100000\du}
\pgfsetdash{}{0pt}
\pgfsetbuttcap
{
\definecolor{diafillcolor}{rgb}{0.000000, 0.000000, 0.000000}
\pgfsetfillcolor{diafillcolor}
\pgfsetfillopacity{1.000000}
% was here!!!
\pgfsetarrowsend{stealth}
\definecolor{dialinecolor}{rgb}{0.000000, 0.000000, 0.000000}
\pgfsetstrokecolor{dialinecolor}
\pgfsetstrokeopacity{1.000000}
\draw (6.750000\du,5.026500\du)--(6.800000\du,6.850000\du);
}
\pgfsetlinewidth{0.100000\du}
\pgfsetdash{}{0pt}
\pgfsetbuttcap
{
\definecolor{diafillcolor}{rgb}{0.000000, 0.000000, 0.000000}
\pgfsetfillcolor{diafillcolor}
\pgfsetfillopacity{1.000000}
% was here!!!
\pgfsetarrowsend{stealth}
\definecolor{dialinecolor}{rgb}{0.000000, 0.000000, 0.000000}
\pgfsetstrokecolor{dialinecolor}
\pgfsetstrokeopacity{1.000000}
\draw (6.800000\du,9.410524\du)--(6.800000\du,10.876500\du);
}
\pgfsetlinewidth{0.100000\du}
\pgfsetdash{}{0pt}
\pgfsetmiterjoin
\pgfsetbuttcap
{
\definecolor{diafillcolor}{rgb}{0.000000, 0.000000, 0.000000}
\pgfsetfillcolor{diafillcolor}
\pgfsetfillopacity{1.000000}
% was here!!!
\pgfsetarrowsend{stealth}
{\pgfsetcornersarced{\pgfpoint{0.000000\du}{0.000000\du}}\definecolor{dialinecolor}{rgb}{0.000000, 0.000000, 0.000000}
\pgfsetstrokecolor{dialinecolor}
\pgfsetstrokeopacity{1.000000}
\draw (7.300000\du,11.426500\du)--(10.200000\du,11.426500\du)--(10.200000\du,5.760000\du)--(14.230000\du,5.760000\du)--(14.230000\du,6.810000\du);
}}
\pgfsetlinewidth{0.100000\du}
\pgfsetdash{}{0pt}
\pgfsetbuttcap
{
\definecolor{diafillcolor}{rgb}{0.000000, 0.000000, 0.000000}
\pgfsetfillcolor{diafillcolor}
\pgfsetfillopacity{1.000000}
% was here!!!
\pgfsetarrowsend{stealth}
\definecolor{dialinecolor}{rgb}{0.000000, 0.000000, 0.000000}
\pgfsetstrokecolor{dialinecolor}
\pgfsetstrokeopacity{1.000000}
\draw (14.230000\du,9.410000\du)--(14.266803\du,10.904411\du);
}
\pgfsetlinewidth{0.100000\du}
\pgfsetdash{}{0pt}
\pgfsetbuttcap
{
\definecolor{diafillcolor}{rgb}{0.000000, 0.000000, 0.000000}
\pgfsetfillcolor{diafillcolor}
\pgfsetfillopacity{1.000000}
% was here!!!
\pgfsetarrowsend{stealth}
\definecolor{dialinecolor}{rgb}{0.000000, 0.000000, 0.000000}
\pgfsetstrokecolor{dialinecolor}
\pgfsetstrokeopacity{1.000000}
\draw (17.750000\du,8.026500\du)--(16.455000\du,8.110000\du);
}
% setfont left to latex
\definecolor{dialinecolor}{rgb}{0.000000, 0.000000, 0.000000}
\pgfsetstrokecolor{dialinecolor}
\pgfsetstrokeopacity{1.000000}
\definecolor{diafillcolor}{rgb}{0.000000, 0.000000, 0.000000}
\pgfsetfillcolor{diafillcolor}
\pgfsetfillopacity{1.000000}
\node[anchor=base west,inner sep=0pt,outer sep=0pt,color=dialinecolor] at (2.050000\du,2.726500\du){(1) : Encrypt-then-MAC};
% setfont left to latex
\definecolor{dialinecolor}{rgb}{0.000000, 0.000000, 0.000000}
\pgfsetstrokecolor{dialinecolor}
\pgfsetstrokeopacity{1.000000}
\definecolor{diafillcolor}{rgb}{0.000000, 0.000000, 0.000000}
\pgfsetfillcolor{diafillcolor}
\pgfsetfillopacity{1.000000}
\node[anchor=base west,inner sep=0pt,outer sep=0pt,color=dialinecolor] at (22.005000\du,2.746500\du){(2) : MAC-then-Encrypt};
% setfont left to latex
\definecolor{dialinecolor}{rgb}{0.000000, 0.000000, 0.000000}
\pgfsetstrokecolor{dialinecolor}
\pgfsetstrokeopacity{1.000000}
\definecolor{diafillcolor}{rgb}{0.000000, 0.000000, 0.000000}
\pgfsetfillcolor{diafillcolor}
\pgfsetfillopacity{1.000000}
\node[anchor=base west,inner sep=0pt,outer sep=0pt,color=dialinecolor] at (42.005000\du,2.746500\du){(3) : Encrypt-and-MAC};
\pgfsetlinewidth{0.100000\du}
\pgfsetdash{}{0pt}
\pgfsetmiterjoin
{\pgfsetcornersarced{\pgfpoint{0.000000\du}{0.000000\du}}\definecolor{diafillcolor}{rgb}{0.596078, 0.733333, 0.819608}
\pgfsetfillcolor{diafillcolor}
\pgfsetfillopacity{1.000000}
\fill (24.405000\du,6.876500\du)--(24.405000\du,9.387611\du)--(28.905000\du,9.387611\du)--(28.905000\du,6.876500\du)--cycle;
}{\pgfsetcornersarced{\pgfpoint{0.000000\du}{0.000000\du}}\definecolor{dialinecolor}{rgb}{0.000000, 0.000000, 0.000000}
\pgfsetstrokecolor{dialinecolor}
\pgfsetstrokeopacity{1.000000}
\draw (24.405000\du,6.876500\du)--(24.405000\du,9.387611\du)--(28.905000\du,9.387611\du)--(28.905000\du,6.876500\du)--cycle;
}% setfont left to latex
\definecolor{dialinecolor}{rgb}{0.000000, 0.000000, 0.000000}
\pgfsetstrokecolor{dialinecolor}
\pgfsetstrokeopacity{1.000000}
\definecolor{diafillcolor}{rgb}{0.000000, 0.000000, 0.000000}
\pgfsetfillcolor{diafillcolor}
\pgfsetfillopacity{1.000000}
\node[anchor=base,inner sep=0pt, outer sep=0pt,color=dialinecolor] at (26.655000\du,8.476500\du){Enc};
\pgfsetlinewidth{0.100000\du}
\pgfsetdash{}{0pt}
\pgfsetmiterjoin
{\pgfsetcornersarced{\pgfpoint{0.000000\du}{0.000000\du}}\definecolor{diafillcolor}{rgb}{0.737255, 0.819608, 0.596078}
\pgfsetfillcolor{diafillcolor}
\pgfsetfillopacity{1.000000}
\fill (31.860000\du,6.836500\du)--(31.860000\du,9.436500\du)--(36.310000\du,9.436500\du)--(36.310000\du,6.836500\du)--cycle;
}{\pgfsetcornersarced{\pgfpoint{0.000000\du}{0.000000\du}}\definecolor{dialinecolor}{rgb}{0.000000, 0.000000, 0.000000}
\pgfsetstrokecolor{dialinecolor}
\pgfsetstrokeopacity{1.000000}
\draw (31.860000\du,6.836500\du)--(31.860000\du,9.436500\du)--(36.310000\du,9.436500\du)--(36.310000\du,6.836500\du)--cycle;
}% setfont left to latex
\definecolor{dialinecolor}{rgb}{0.000000, 0.000000, 0.000000}
\pgfsetstrokecolor{dialinecolor}
\pgfsetstrokeopacity{1.000000}
\definecolor{diafillcolor}{rgb}{0.000000, 0.000000, 0.000000}
\pgfsetfillcolor{diafillcolor}
\pgfsetfillopacity{1.000000}
\node[anchor=base,inner sep=0pt, outer sep=0pt,color=dialinecolor] at (34.085000\du,8.480944\du){Mac};
% setfont left to latex
\definecolor{dialinecolor}{rgb}{0.200000, 0.321569, 0.168627}
\pgfsetstrokecolor{dialinecolor}
\pgfsetstrokeopacity{1.000000}
\definecolor{diafillcolor}{rgb}{0.200000, 0.321569, 0.168627}
\pgfsetfillcolor{diafillcolor}
\pgfsetfillopacity{1.000000}
\node[anchor=base west,inner sep=0pt,outer sep=0pt,color=dialinecolor] at (21.005000\du,8.526500\du){ $K_1$};
% setfont left to latex
\definecolor{dialinecolor}{rgb}{0.200000, 0.321569, 0.168627}
\pgfsetstrokecolor{dialinecolor}
\pgfsetstrokeopacity{1.000000}
\definecolor{diafillcolor}{rgb}{0.200000, 0.321569, 0.168627}
\pgfsetfillcolor{diafillcolor}
\pgfsetfillopacity{1.000000}
\node[anchor=base west,inner sep=0pt,outer sep=0pt,color=dialinecolor] at (29.600000\du,4.500000\du){ M};
% setfont left to latex
\definecolor{dialinecolor}{rgb}{0.200000, 0.321569, 0.168627}
\pgfsetstrokecolor{dialinecolor}
\pgfsetstrokeopacity{1.000000}
\definecolor{diafillcolor}{rgb}{0.200000, 0.321569, 0.168627}
\pgfsetfillcolor{diafillcolor}
\pgfsetfillopacity{1.000000}
\node[anchor=base west,inner sep=0pt,outer sep=0pt,color=dialinecolor] at (25.910000\du,11.936500\du){ C};
% setfont left to latex
\definecolor{dialinecolor}{rgb}{0.000000, 0.000000, 0.000000}
\pgfsetstrokecolor{dialinecolor}
\pgfsetstrokeopacity{1.000000}
\definecolor{diafillcolor}{rgb}{0.000000, 0.000000, 0.000000}
\pgfsetfillcolor{diafillcolor}
\pgfsetfillopacity{1.000000}
\node[anchor=base west,inner sep=0pt,outer sep=0pt,color=dialinecolor] at (27.005000\du,11.976500\du){};
% setfont left to latex
\definecolor{dialinecolor}{rgb}{0.200000, 0.321569, 0.168627}
\pgfsetstrokecolor{dialinecolor}
\pgfsetstrokeopacity{1.000000}
\definecolor{diafillcolor}{rgb}{0.200000, 0.321569, 0.168627}
\pgfsetfillcolor{diafillcolor}
\pgfsetfillopacity{1.000000}
\node[anchor=base west,inner sep=0pt,outer sep=0pt,color=dialinecolor] at (37.460000\du,8.436500\du){ $K_2$};
\pgfsetlinewidth{0.100000\du}
\pgfsetdash{}{0pt}
\pgfsetbuttcap
{
\definecolor{diafillcolor}{rgb}{0.000000, 0.000000, 0.000000}
\pgfsetfillcolor{diafillcolor}
\pgfsetfillopacity{1.000000}
% was here!!!
\pgfsetarrowsend{stealth}
\definecolor{dialinecolor}{rgb}{0.000000, 0.000000, 0.000000}
\pgfsetstrokecolor{dialinecolor}
\pgfsetstrokeopacity{1.000000}
\draw (22.955000\du,8.153000\du)--(24.405000\du,8.132056\du);
}
\pgfsetlinewidth{0.100000\du}
\pgfsetdash{}{0pt}
\pgfsetbuttcap
{
\definecolor{diafillcolor}{rgb}{0.000000, 0.000000, 0.000000}
\pgfsetfillcolor{diafillcolor}
\pgfsetfillopacity{1.000000}
% was here!!!
\pgfsetarrowsend{stealth}
\definecolor{dialinecolor}{rgb}{0.000000, 0.000000, 0.000000}
\pgfsetstrokecolor{dialinecolor}
\pgfsetstrokeopacity{1.000000}
\draw (26.655000\du,9.437024\du)--(26.655000\du,10.903000\du);
}
\pgfsetlinewidth{0.100000\du}
\pgfsetdash{}{0pt}
\pgfsetbuttcap
{
\definecolor{diafillcolor}{rgb}{0.000000, 0.000000, 0.000000}
\pgfsetfillcolor{diafillcolor}
\pgfsetfillopacity{1.000000}
% was here!!!
\pgfsetarrowsend{stealth}
\definecolor{dialinecolor}{rgb}{0.000000, 0.000000, 0.000000}
\pgfsetstrokecolor{dialinecolor}
\pgfsetstrokeopacity{1.000000}
\draw (37.605000\du,8.053000\du)--(36.310000\du,8.136500\du);
}
\pgfsetlinewidth{0.100000\du}
\pgfsetdash{}{0pt}
\pgfsetbuttcap
{
\definecolor{diafillcolor}{rgb}{0.000000, 0.000000, 0.000000}
\pgfsetfillcolor{diafillcolor}
\pgfsetfillopacity{1.000000}
% was here!!!
\definecolor{dialinecolor}{rgb}{0.000000, 0.000000, 0.000000}
\pgfsetstrokecolor{dialinecolor}
\pgfsetstrokeopacity{1.000000}
\draw (20.000000\du,0.026500\du)--(20.000000\du,13.026500\du);
}
\pgfsetlinewidth{0.100000\du}
\pgfsetdash{}{0pt}
\pgfsetbuttcap
{
\definecolor{diafillcolor}{rgb}{0.000000, 0.000000, 0.000000}
\pgfsetfillcolor{diafillcolor}
\pgfsetfillopacity{1.000000}
% was here!!!
\definecolor{dialinecolor}{rgb}{0.000000, 0.000000, 0.000000}
\pgfsetstrokecolor{dialinecolor}
\pgfsetstrokeopacity{1.000000}
\draw (40.000000\du,0.000000\du)--(40.000000\du,13.000000\du);
}
\pgfsetlinewidth{0.100000\du}
\pgfsetdash{}{0pt}
\pgfsetmiterjoin
\pgfsetbuttcap
{
\definecolor{diafillcolor}{rgb}{0.000000, 0.000000, 0.000000}
\pgfsetfillcolor{diafillcolor}
\pgfsetfillopacity{1.000000}
% was here!!!
\pgfsetarrowsend{stealth}
{\pgfsetcornersarced{\pgfpoint{0.000000\du}{0.000000\du}}\definecolor{dialinecolor}{rgb}{0.000000, 0.000000, 0.000000}
\pgfsetstrokecolor{dialinecolor}
\pgfsetstrokeopacity{1.000000}
\draw (31.050000\du,4.076500\du)--(31.050000\du,4.126500\du)--(34.085000\du,4.126500\du)--(34.085000\du,6.836500\du);
}}
\pgfsetlinewidth{0.100000\du}
\pgfsetdash{}{0pt}
\pgfsetmiterjoin
{\pgfsetcornersarced{\pgfpoint{0.000000\du}{0.000000\du}}\definecolor{diafillcolor}{rgb}{0.596078, 0.733333, 0.819608}
\pgfsetfillcolor{diafillcolor}
\pgfsetfillopacity{1.000000}
\fill (44.010042\du,6.913000\du)--(44.010042\du,9.424111\du)--(48.510042\du,9.424111\du)--(48.510042\du,6.913000\du)--cycle;
}{\pgfsetcornersarced{\pgfpoint{0.000000\du}{0.000000\du}}\definecolor{dialinecolor}{rgb}{0.000000, 0.000000, 0.000000}
\pgfsetstrokecolor{dialinecolor}
\pgfsetstrokeopacity{1.000000}
\draw (44.010042\du,6.913000\du)--(44.010042\du,9.424111\du)--(48.510042\du,9.424111\du)--(48.510042\du,6.913000\du)--cycle;
}% setfont left to latex
\definecolor{dialinecolor}{rgb}{0.000000, 0.000000, 0.000000}
\pgfsetstrokecolor{dialinecolor}
\pgfsetstrokeopacity{1.000000}
\definecolor{diafillcolor}{rgb}{0.000000, 0.000000, 0.000000}
\pgfsetfillcolor{diafillcolor}
\pgfsetfillopacity{1.000000}
\node[anchor=base,inner sep=0pt, outer sep=0pt,color=dialinecolor] at (46.260042\du,8.513000\du){Enc};
\pgfsetlinewidth{0.100000\du}
\pgfsetdash{}{0pt}
\pgfsetmiterjoin
{\pgfsetcornersarced{\pgfpoint{0.000000\du}{0.000000\du}}\definecolor{diafillcolor}{rgb}{0.737255, 0.819608, 0.596078}
\pgfsetfillcolor{diafillcolor}
\pgfsetfillopacity{1.000000}
\fill (51.465042\du,6.873000\du)--(51.465042\du,9.473000\du)--(55.915042\du,9.473000\du)--(55.915042\du,6.873000\du)--cycle;
}{\pgfsetcornersarced{\pgfpoint{0.000000\du}{0.000000\du}}\definecolor{dialinecolor}{rgb}{0.000000, 0.000000, 0.000000}
\pgfsetstrokecolor{dialinecolor}
\pgfsetstrokeopacity{1.000000}
\draw (51.465042\du,6.873000\du)--(51.465042\du,9.473000\du)--(55.915042\du,9.473000\du)--(55.915042\du,6.873000\du)--cycle;
}% setfont left to latex
\definecolor{dialinecolor}{rgb}{0.000000, 0.000000, 0.000000}
\pgfsetstrokecolor{dialinecolor}
\pgfsetstrokeopacity{1.000000}
\definecolor{diafillcolor}{rgb}{0.000000, 0.000000, 0.000000}
\pgfsetfillcolor{diafillcolor}
\pgfsetfillopacity{1.000000}
\node[anchor=base,inner sep=0pt, outer sep=0pt,color=dialinecolor] at (53.690042\du,8.517444\du){Mac};
% setfont left to latex
\definecolor{dialinecolor}{rgb}{0.200000, 0.321569, 0.168627}
\pgfsetstrokecolor{dialinecolor}
\pgfsetstrokeopacity{1.000000}
\definecolor{diafillcolor}{rgb}{0.200000, 0.321569, 0.168627}
\pgfsetfillcolor{diafillcolor}
\pgfsetfillopacity{1.000000}
\node[anchor=base west,inner sep=0pt,outer sep=0pt,color=dialinecolor] at (40.610042\du,8.563000\du){ $K_1$};
% setfont left to latex
\definecolor{dialinecolor}{rgb}{0.200000, 0.321569, 0.168627}
\pgfsetstrokecolor{dialinecolor}
\pgfsetstrokeopacity{1.000000}
\definecolor{diafillcolor}{rgb}{0.200000, 0.321569, 0.168627}
\pgfsetfillcolor{diafillcolor}
\pgfsetfillopacity{1.000000}
\node[anchor=base west,inner sep=0pt,outer sep=0pt,color=dialinecolor] at (49.205042\du,5.136500\du){ M};
% setfont left to latex
\definecolor{dialinecolor}{rgb}{0.200000, 0.321569, 0.168627}
\pgfsetstrokecolor{dialinecolor}
\pgfsetstrokeopacity{1.000000}
\definecolor{diafillcolor}{rgb}{0.200000, 0.321569, 0.168627}
\pgfsetfillcolor{diafillcolor}
\pgfsetfillopacity{1.000000}
\node[anchor=base west,inner sep=0pt,outer sep=0pt,color=dialinecolor] at (45.515042\du,11.973000\du){ C};
% setfont left to latex
\definecolor{dialinecolor}{rgb}{0.000000, 0.000000, 0.000000}
\pgfsetstrokecolor{dialinecolor}
\pgfsetstrokeopacity{1.000000}
\definecolor{diafillcolor}{rgb}{0.000000, 0.000000, 0.000000}
\pgfsetfillcolor{diafillcolor}
\pgfsetfillopacity{1.000000}
\node[anchor=base west,inner sep=0pt,outer sep=0pt,color=dialinecolor] at (66.910042\du,11.613000\du){};
% setfont left to latex
\definecolor{dialinecolor}{rgb}{0.200000, 0.321569, 0.168627}
\pgfsetstrokecolor{dialinecolor}
\pgfsetstrokeopacity{1.000000}
\definecolor{diafillcolor}{rgb}{0.200000, 0.321569, 0.168627}
\pgfsetfillcolor{diafillcolor}
\pgfsetfillopacity{1.000000}
\node[anchor=base west,inner sep=0pt,outer sep=0pt,color=dialinecolor] at (57.065042\du,8.473000\du){ $K_2$};
% setfont left to latex
\definecolor{dialinecolor}{rgb}{0.200000, 0.321569, 0.168627}
\pgfsetstrokecolor{dialinecolor}
\pgfsetstrokeopacity{1.000000}
\definecolor{diafillcolor}{rgb}{0.200000, 0.321569, 0.168627}
\pgfsetfillcolor{diafillcolor}
\pgfsetfillopacity{1.000000}
\node[anchor=base west,inner sep=0pt,outer sep=0pt,color=dialinecolor] at (53.015042\du,12.023000\du){ T};
\pgfsetlinewidth{0.100000\du}
\pgfsetdash{}{0pt}
\pgfsetbuttcap
{
\definecolor{diafillcolor}{rgb}{0.000000, 0.000000, 0.000000}
\pgfsetfillcolor{diafillcolor}
\pgfsetfillopacity{1.000000}
% was here!!!
\pgfsetarrowsend{stealth}
\definecolor{dialinecolor}{rgb}{0.000000, 0.000000, 0.000000}
\pgfsetstrokecolor{dialinecolor}
\pgfsetstrokeopacity{1.000000}
\draw (42.560042\du,8.189500\du)--(44.010042\du,8.168556\du);
}
\pgfsetlinewidth{0.100000\du}
\pgfsetdash{}{0pt}
\pgfsetbuttcap
{
\definecolor{diafillcolor}{rgb}{0.000000, 0.000000, 0.000000}
\pgfsetfillcolor{diafillcolor}
\pgfsetfillopacity{1.000000}
% was here!!!
\pgfsetarrowsend{stealth}
\definecolor{dialinecolor}{rgb}{0.000000, 0.000000, 0.000000}
\pgfsetstrokecolor{dialinecolor}
\pgfsetstrokeopacity{1.000000}
\draw (46.260042\du,9.473524\du)--(46.260042\du,10.939500\du);
}
\pgfsetlinewidth{0.100000\du}
\pgfsetdash{}{0pt}
\pgfsetbuttcap
{
\definecolor{diafillcolor}{rgb}{0.000000, 0.000000, 0.000000}
\pgfsetfillcolor{diafillcolor}
\pgfsetfillopacity{1.000000}
% was here!!!
\pgfsetarrowsend{stealth}
\definecolor{dialinecolor}{rgb}{0.000000, 0.000000, 0.000000}
\pgfsetstrokecolor{dialinecolor}
\pgfsetstrokeopacity{1.000000}
\draw (53.690042\du,9.473000\du)--(53.726846\du,10.967411\du);
}
\pgfsetlinewidth{0.100000\du}
\pgfsetdash{}{0pt}
\pgfsetbuttcap
{
\definecolor{diafillcolor}{rgb}{0.000000, 0.000000, 0.000000}
\pgfsetfillcolor{diafillcolor}
\pgfsetfillopacity{1.000000}
% was here!!!
\pgfsetarrowsend{stealth}
\definecolor{dialinecolor}{rgb}{0.000000, 0.000000, 0.000000}
\pgfsetstrokecolor{dialinecolor}
\pgfsetstrokeopacity{1.000000}
\draw (57.210042\du,8.089500\du)--(55.915042\du,8.173000\du);
}
\pgfsetlinewidth{0.100000\du}
\pgfsetdash{}{0pt}
\pgfsetmiterjoin
\pgfsetbuttcap
{
\definecolor{diafillcolor}{rgb}{0.000000, 0.000000, 0.000000}
\pgfsetfillcolor{diafillcolor}
\pgfsetfillopacity{1.000000}
% was here!!!
\pgfsetarrowsend{stealth}
{\pgfsetcornersarced{\pgfpoint{0.000000\du}{0.000000\du}}\definecolor{dialinecolor}{rgb}{0.000000, 0.000000, 0.000000}
\pgfsetstrokecolor{dialinecolor}
\pgfsetstrokeopacity{1.000000}
\draw (49.305042\du,4.663000\du)--(49.305042\du,4.663000\du)--(46.260042\du,4.663000\du)--(46.260042\du,6.913000\du);
}}
\pgfsetlinewidth{0.100000\du}
\pgfsetdash{}{0pt}
\pgfsetmiterjoin
\pgfsetbuttcap
{
\definecolor{diafillcolor}{rgb}{0.000000, 0.000000, 0.000000}
\pgfsetfillcolor{diafillcolor}
\pgfsetfillopacity{1.000000}
% was here!!!
\pgfsetarrowsend{stealth}
{\pgfsetcornersarced{\pgfpoint{0.000000\du}{0.000000\du}}\definecolor{dialinecolor}{rgb}{0.000000, 0.000000, 0.000000}
\pgfsetstrokecolor{dialinecolor}
\pgfsetstrokeopacity{1.000000}
\draw (50.705042\du,4.663000\du)--(50.705042\du,4.713000\du)--(53.690042\du,4.713000\du)--(53.690042\du,6.873000\du);
}}
% setfont left to latex
\definecolor{dialinecolor}{rgb}{0.200000, 0.321569, 0.168627}
\pgfsetstrokecolor{dialinecolor}
\pgfsetstrokeopacity{1.000000}
\definecolor{diafillcolor}{rgb}{0.200000, 0.321569, 0.168627}
\pgfsetfillcolor{diafillcolor}
\pgfsetfillopacity{1.000000}
\node[anchor=base west,inner sep=0pt,outer sep=0pt,color=dialinecolor] at (24.655000\du,5.736500\du){$M || T$};
\pgfsetlinewidth{0.100000\du}
\pgfsetdash{}{0pt}
\pgfsetbuttcap
{
\definecolor{diafillcolor}{rgb}{0.000000, 0.000000, 0.000000}
\pgfsetfillcolor{diafillcolor}
\pgfsetfillopacity{1.000000}
% was here!!!
\pgfsetarrowsend{stealth}
\definecolor{dialinecolor}{rgb}{0.000000, 0.000000, 0.000000}
\pgfsetstrokecolor{dialinecolor}
\pgfsetstrokeopacity{1.000000}
\draw (26.700000\du,5.876500\du)--(26.655000\du,6.876500\du);
}
\pgfsetlinewidth{0.100000\du}
\pgfsetdash{}{0pt}
\pgfsetmiterjoin
\pgfsetbuttcap
{
\definecolor{diafillcolor}{rgb}{0.000000, 0.000000, 0.000000}
\pgfsetfillcolor{diafillcolor}
\pgfsetfillopacity{1.000000}
% was here!!!
\pgfsetarrowsend{stealth}
{\pgfsetcornersarced{\pgfpoint{0.000000\du}{0.000000\du}}\definecolor{dialinecolor}{rgb}{0.000000, 0.000000, 0.000000}
\pgfsetstrokecolor{dialinecolor}
\pgfsetstrokeopacity{1.000000}
\draw (29.650000\du,4.126500\du)--(23.700000\du,4.126500\du)--(23.700000\du,5.326500\du)--(24.700000\du,5.326500\du);
}}
\pgfsetlinewidth{0.100000\du}
\pgfsetdash{}{0pt}
\pgfsetmiterjoin
\pgfsetbuttcap
{
\definecolor{diafillcolor}{rgb}{0.000000, 0.000000, 0.000000}
\pgfsetfillcolor{diafillcolor}
\pgfsetfillopacity{1.000000}
% was here!!!
\pgfsetarrowsend{stealth}
{\pgfsetcornersarced{\pgfpoint{0.000000\du}{0.000000\du}}\definecolor{dialinecolor}{rgb}{0.000000, 0.000000, 0.000000}
\pgfsetstrokecolor{dialinecolor}
\pgfsetstrokeopacity{1.000000}
\draw (34.085000\du,9.436500\du)--(34.085000\du,10.376500\du)--(30.700000\du,10.376500\du)--(30.700000\du,5.376500\du)--(28.400000\du,5.376500\du);
}}
\end{tikzpicture}

\normalsize}
  \caption{Possible modes for generic compositions achieving AE}
\label{fig:cpamac}
\end{figure}

\textbf{Encrypt-then-MAC:} The first of our generic compositions, this composition is the most secure.  The MAC on the ciphertext $T$ prevents changes 
in $T$ from verifying as valid ciphertexts, ensuring reliability of the plaintext regardless of whether invalid ciphertexts can still decrypt 
correctly under $Enc$ and preventing chosen ciphertext attacks on $Enc$.  The correct decryption property combined with this provides integrity of the 
original message.  No additional information is leaked by $Mac$, which has as input only the output of $Enc$ and cannot allow for distinguishing of 
messages by CPA security.  When implemented correctly, this achieves ROR-CCA security.

\textbf{MAC-then-Encrypt:} This scheme has one small barrier to achieving our desired security notions.  Namely, since no integrity is provided to the 
ciphertext (as we have seen, some CPA-secure encryption schemes will fail to realize CTXT security), it could conceivably possible to craft a 
ciphertext which will decrypt to have a valid MAC, and the composition will not realize ROR+CTXT security.  This is generally not devastating in 
practice, as a secure MAC should make actually finding such a ciphertext unlikely for a given key.  TLS until version 1.2, for example, used this 
paradigm.

\textbf{MAC-and-Encrypt:} The least secure of our constructions, this scheme obviously provides plaintext integrity because of the direct MAC on the 
original message.  Unfortunately, message equality is leaked, as MACs are deterministic.  This obviously breaks ROR security.  Similar to 
MAC-then-Encrypt, this mode also does not MAC the ciphertext, which does not provide ciphertext integrity and could be of concern.


\subsection{In Practice}


\begin{figure}[h]
\centering
\scalebox{0.6}{\Huge
% Graphic for TeX using PGF
% Title: /home/phil/tls.dia
% Creator: Dia v0.97+git
% CreationDate: Fri Mar  8 17:47:30 2019
% For: phil
% \usepackage{tikz}
% The following commands are not supported in PSTricks at present
% We define them conditionally, so when they are implemented,
% this pgf file will use them.
\ifx\du\undefined
  \newlength{\du}
\fi
\setlength{\du}{15\unitlength}
\begin{tikzpicture}[even odd rule]
\pgftransformxscale{1.000000}
\pgftransformyscale{-1.000000}
\definecolor{dialinecolor}{rgb}{0.000000, 0.000000, 0.000000}
\pgfsetstrokecolor{dialinecolor}
\pgfsetstrokeopacity{1.000000}
\definecolor{diafillcolor}{rgb}{1.000000, 1.000000, 1.000000}
\pgfsetfillcolor{diafillcolor}
\pgfsetfillopacity{1.000000}
\pgfsetlinewidth{0.100000\du}
\pgfsetdash{}{0pt}
\pgfsetmiterjoin
{\pgfsetcornersarced{\pgfpoint{0.000000\du}{0.000000\du}}\definecolor{diafillcolor}{rgb}{1.000000, 1.000000, 1.000000}
\pgfsetfillcolor{diafillcolor}
\pgfsetfillopacity{1.000000}
\fill (3.000000\du,1.000000\du)--(3.000000\du,3.511111\du)--(7.250000\du,3.511111\du)--(7.250000\du,1.000000\du)--cycle;
}{\pgfsetcornersarced{\pgfpoint{0.000000\du}{0.000000\du}}\definecolor{dialinecolor}{rgb}{0.000000, 0.000000, 0.000000}
\pgfsetstrokecolor{dialinecolor}
\pgfsetstrokeopacity{1.000000}
\draw (3.000000\du,1.000000\du)--(3.000000\du,3.511111\du)--(7.250000\du,3.511111\du)--(7.250000\du,1.000000\du)--cycle;
}% setfont left to latex
\definecolor{dialinecolor}{rgb}{0.000000, 0.000000, 0.000000}
\pgfsetstrokecolor{dialinecolor}
\pgfsetstrokeopacity{1.000000}
\definecolor{diafillcolor}{rgb}{0.000000, 0.000000, 0.000000}
\pgfsetfillcolor{diafillcolor}
\pgfsetfillopacity{1.000000}
\node[anchor=base,inner sep=0pt, outer sep=0pt,color=dialinecolor] at (5.125000\du,2.600000\du){AD};
\pgfsetlinewidth{0.100000\du}
\pgfsetdash{}{0pt}
\pgfsetmiterjoin
{\pgfsetcornersarced{\pgfpoint{0.000000\du}{0.000000\du}}\definecolor{diafillcolor}{rgb}{0.784314, 0.960784, 0.772549}
\pgfsetfillcolor{diafillcolor}
\pgfsetfillopacity{1.000000}
\fill (7.000000\du,1.000000\du)--(7.000000\du,3.511111\du)--(21.045000\du,3.511111\du)--(21.045000\du,1.000000\du)--cycle;
}{\pgfsetcornersarced{\pgfpoint{0.000000\du}{0.000000\du}}\definecolor{dialinecolor}{rgb}{0.000000, 0.000000, 0.000000}
\pgfsetstrokecolor{dialinecolor}
\pgfsetstrokeopacity{1.000000}
\draw (7.000000\du,1.000000\du)--(7.000000\du,3.511111\du)--(21.045000\du,3.511111\du)--(21.045000\du,1.000000\du)--cycle;
}% setfont left to latex
\definecolor{dialinecolor}{rgb}{0.000000, 0.000000, 0.000000}
\pgfsetstrokecolor{dialinecolor}
\pgfsetstrokeopacity{1.000000}
\definecolor{diafillcolor}{rgb}{0.000000, 0.000000, 0.000000}
\pgfsetfillcolor{diafillcolor}
\pgfsetfillopacity{1.000000}
\node[anchor=base,inner sep=0pt, outer sep=0pt,color=dialinecolor] at (14.022500\du,2.600000\du){payload};
\pgfsetlinewidth{0.100000\du}
\pgfsetdash{}{0pt}
\pgfsetmiterjoin
{\pgfsetcornersarced{\pgfpoint{0.000000\du}{0.000000\du}}\definecolor{diafillcolor}{rgb}{0.384314, 0.905882, 0.470588}
\pgfsetfillcolor{diafillcolor}
\pgfsetfillopacity{1.000000}
\fill (9.000000\du,5.000000\du)--(9.000000\du,7.511111\du)--(15.000000\du,7.511111\du)--(15.000000\du,5.000000\du)--cycle;
}{\pgfsetcornersarced{\pgfpoint{0.000000\du}{0.000000\du}}\definecolor{dialinecolor}{rgb}{0.000000, 0.000000, 0.000000}
\pgfsetstrokecolor{dialinecolor}
\pgfsetstrokeopacity{1.000000}
\draw (9.000000\du,5.000000\du)--(9.000000\du,7.511111\du)--(15.000000\du,7.511111\du)--(15.000000\du,5.000000\du)--cycle;
}% setfont left to latex
\definecolor{dialinecolor}{rgb}{0.000000, 0.000000, 0.000000}
\pgfsetstrokecolor{dialinecolor}
\pgfsetstrokeopacity{1.000000}
\definecolor{diafillcolor}{rgb}{0.000000, 0.000000, 0.000000}
\pgfsetfillcolor{diafillcolor}
\pgfsetfillopacity{1.000000}
\node[anchor=base,inner sep=0pt, outer sep=0pt,color=dialinecolor] at (12.000000\du,6.600000\du){MAC};
\pgfsetlinewidth{0.100000\du}
\pgfsetdash{}{0pt}
\pgfsetmiterjoin
{\pgfsetcornersarced{\pgfpoint{0.000000\du}{0.000000\du}}\definecolor{diafillcolor}{rgb}{1.000000, 1.000000, 1.000000}
\pgfsetfillcolor{diafillcolor}
\pgfsetfillopacity{1.000000}
\fill (3.000000\du,17.000000\du)--(3.000000\du,19.511111\du)--(8.080000\du,19.511111\du)--(8.080000\du,17.000000\du)--cycle;
}{\pgfsetcornersarced{\pgfpoint{0.000000\du}{0.000000\du}}\definecolor{dialinecolor}{rgb}{0.000000, 0.000000, 0.000000}
\pgfsetstrokecolor{dialinecolor}
\pgfsetstrokeopacity{1.000000}
\draw (3.000000\du,17.000000\du)--(3.000000\du,19.511111\du)--(8.080000\du,19.511111\du)--(8.080000\du,17.000000\du)--cycle;
}% setfont left to latex
\definecolor{dialinecolor}{rgb}{0.000000, 0.000000, 0.000000}
\pgfsetstrokecolor{dialinecolor}
\pgfsetstrokeopacity{1.000000}
\definecolor{diafillcolor}{rgb}{0.000000, 0.000000, 0.000000}
\pgfsetfillcolor{diafillcolor}
\pgfsetfillopacity{1.000000}
\node[anchor=base,inner sep=0pt, outer sep=0pt,color=dialinecolor] at (5.540000\du,18.600000\du){header};
\pgfsetlinewidth{0.100000\du}
\pgfsetdash{}{0pt}
\pgfsetmiterjoin
{\pgfsetcornersarced{\pgfpoint{0.000000\du}{0.000000\du}}\definecolor{diafillcolor}{rgb}{1.000000, 1.000000, 1.000000}
\pgfsetfillcolor{diafillcolor}
\pgfsetfillopacity{1.000000}
\fill (8.000000\du,17.000000\du)--(8.000000\du,19.511111\du)--(22.045000\du,19.511111\du)--(22.045000\du,17.000000\du)--cycle;
}{\pgfsetcornersarced{\pgfpoint{0.000000\du}{0.000000\du}}\definecolor{dialinecolor}{rgb}{0.000000, 0.000000, 0.000000}
\pgfsetstrokecolor{dialinecolor}
\pgfsetstrokeopacity{1.000000}
\draw (8.000000\du,17.000000\du)--(8.000000\du,19.511111\du)--(22.045000\du,19.511111\du)--(22.045000\du,17.000000\du)--cycle;
}% setfont left to latex
\definecolor{dialinecolor}{rgb}{0.000000, 0.000000, 0.000000}
\pgfsetstrokecolor{dialinecolor}
\pgfsetstrokeopacity{1.000000}
\definecolor{diafillcolor}{rgb}{0.000000, 0.000000, 0.000000}
\pgfsetfillcolor{diafillcolor}
\pgfsetfillopacity{1.000000}
\node[anchor=base,inner sep=0pt, outer sep=0pt,color=dialinecolor] at (15.022500\du,18.600000\du){ciphertext};
\pgfsetlinewidth{0.100000\du}
\pgfsetdash{}{0pt}
\pgfsetmiterjoin
{\pgfsetcornersarced{\pgfpoint{0.000000\du}{0.000000\du}}\definecolor{diafillcolor}{rgb}{0.980392, 0.917647, 0.596078}
\pgfsetfillcolor{diafillcolor}
\pgfsetfillopacity{1.000000}
\fill (15.000000\du,9.000000\du)--(15.000000\du,11.511111\du)--(23.000000\du,11.511111\du)--(23.000000\du,9.000000\du)--cycle;
}{\pgfsetcornersarced{\pgfpoint{0.000000\du}{0.000000\du}}\definecolor{dialinecolor}{rgb}{0.000000, 0.000000, 0.000000}
\pgfsetstrokecolor{dialinecolor}
\pgfsetstrokeopacity{1.000000}
\draw (15.000000\du,9.000000\du)--(15.000000\du,11.511111\du)--(23.000000\du,11.511111\du)--(23.000000\du,9.000000\du)--cycle;
}% setfont left to latex
\definecolor{dialinecolor}{rgb}{0.000000, 0.000000, 0.000000}
\pgfsetstrokecolor{dialinecolor}
\pgfsetstrokeopacity{1.000000}
\definecolor{diafillcolor}{rgb}{0.000000, 0.000000, 0.000000}
\pgfsetfillcolor{diafillcolor}
\pgfsetfillopacity{1.000000}
\node[anchor=base,inner sep=0pt, outer sep=0pt,color=dialinecolor] at (19.000000\du,10.600000\du){MAC tag};
\pgfsetlinewidth{0.100000\du}
\pgfsetdash{}{0pt}
\pgfsetmiterjoin
{\pgfsetcornersarced{\pgfpoint{0.000000\du}{0.000000\du}}\definecolor{diafillcolor}{rgb}{0.784314, 0.960784, 0.772549}
\pgfsetfillcolor{diafillcolor}
\pgfsetfillopacity{1.000000}
\fill (1.000000\du,9.000000\du)--(1.000000\du,11.511111\du)--(15.045000\du,11.511111\du)--(15.045000\du,9.000000\du)--cycle;
}{\pgfsetcornersarced{\pgfpoint{0.000000\du}{0.000000\du}}\definecolor{dialinecolor}{rgb}{0.000000, 0.000000, 0.000000}
\pgfsetstrokecolor{dialinecolor}
\pgfsetstrokeopacity{1.000000}
\draw (1.000000\du,9.000000\du)--(1.000000\du,11.511111\du)--(15.045000\du,11.511111\du)--(15.045000\du,9.000000\du)--cycle;
}% setfont left to latex
\definecolor{dialinecolor}{rgb}{0.000000, 0.000000, 0.000000}
\pgfsetstrokecolor{dialinecolor}
\pgfsetstrokeopacity{1.000000}
\definecolor{diafillcolor}{rgb}{0.000000, 0.000000, 0.000000}
\pgfsetfillcolor{diafillcolor}
\pgfsetfillopacity{1.000000}
\node[anchor=base,inner sep=0pt, outer sep=0pt,color=dialinecolor] at (8.022500\du,10.600000\du){payload};
\pgfsetlinewidth{0.100000\du}
\pgfsetdash{}{0pt}
\pgfsetmiterjoin
{\pgfsetcornersarced{\pgfpoint{0.000000\du}{0.000000\du}}\definecolor{diafillcolor}{rgb}{1.000000, 0.000000, 0.000000}
\pgfsetfillcolor{diafillcolor}
\pgfsetfillopacity{1.000000}
\fill (23.000000\du,9.000000\du)--(23.000000\du,11.511111\du)--(29.000000\du,11.511111\du)--(29.000000\du,9.000000\du)--cycle;
}{\pgfsetcornersarced{\pgfpoint{0.000000\du}{0.000000\du}}\definecolor{dialinecolor}{rgb}{0.000000, 0.000000, 0.000000}
\pgfsetstrokecolor{dialinecolor}
\pgfsetstrokeopacity{1.000000}
\draw (23.000000\du,9.000000\du)--(23.000000\du,11.511111\du)--(29.000000\du,11.511111\du)--(29.000000\du,9.000000\du)--cycle;
}% setfont left to latex
\definecolor{dialinecolor}{rgb}{0.000000, 0.000000, 0.000000}
\pgfsetstrokecolor{dialinecolor}
\pgfsetstrokeopacity{1.000000}
\definecolor{diafillcolor}{rgb}{0.000000, 0.000000, 0.000000}
\pgfsetfillcolor{diafillcolor}
\pgfsetfillopacity{1.000000}
\node[anchor=base,inner sep=0pt, outer sep=0pt,color=dialinecolor] at (26.000000\du,10.600000\du){padding};
\pgfsetlinewidth{0.100000\du}
\pgfsetdash{}{0pt}
\pgfsetmiterjoin
{\pgfsetcornersarced{\pgfpoint{0.000000\du}{0.000000\du}}\definecolor{diafillcolor}{rgb}{0.788235, 0.745098, 0.921569}
\pgfsetfillcolor{diafillcolor}
\pgfsetfillopacity{1.000000}
\fill (12.000000\du,13.000000\du)--(12.000000\du,15.511111\du)--(18.000000\du,15.511111\du)--(18.000000\du,13.000000\du)--cycle;
}{\pgfsetcornersarced{\pgfpoint{0.000000\du}{0.000000\du}}\definecolor{dialinecolor}{rgb}{0.000000, 0.000000, 0.000000}
\pgfsetstrokecolor{dialinecolor}
\pgfsetstrokeopacity{1.000000}
\draw (12.000000\du,13.000000\du)--(12.000000\du,15.511111\du)--(18.000000\du,15.511111\du)--(18.000000\du,13.000000\du)--cycle;
}% setfont left to latex
\definecolor{dialinecolor}{rgb}{0.000000, 0.000000, 0.000000}
\pgfsetstrokecolor{dialinecolor}
\pgfsetstrokeopacity{1.000000}
\definecolor{diafillcolor}{rgb}{0.000000, 0.000000, 0.000000}
\pgfsetfillcolor{diafillcolor}
\pgfsetfillopacity{1.000000}
\node[anchor=base,inner sep=0pt, outer sep=0pt,color=dialinecolor] at (15.000000\du,14.600000\du){Enc};
\pgfsetlinewidth{0.100000\du}
\pgfsetdash{}{0pt}
\pgfsetmiterjoin
\pgfsetbuttcap
{
\definecolor{diafillcolor}{rgb}{0.000000, 0.000000, 0.000000}
\pgfsetfillcolor{diafillcolor}
\pgfsetfillopacity{1.000000}
% was here!!!
{\pgfsetcornersarced{\pgfpoint{0.000000\du}{0.000000\du}}\definecolor{dialinecolor}{rgb}{0.000000, 0.000000, 0.000000}
\pgfsetstrokecolor{dialinecolor}
\pgfsetstrokeopacity{1.000000}
\draw (3.000000\du,3.511111\du)--(12.022500\du,5.000000\du)--(12.022500\du,5.000000\du)--(21.045000\du,3.511111\du);
}}
\pgfsetlinewidth{0.100000\du}
\pgfsetdash{}{0pt}
\pgfsetbuttcap
{
\definecolor{diafillcolor}{rgb}{0.000000, 0.000000, 0.000000}
\pgfsetfillcolor{diafillcolor}
\pgfsetfillopacity{1.000000}
% was here!!!
\pgfsetarrowsend{stealth}
\definecolor{dialinecolor}{rgb}{0.000000, 0.000000, 0.000000}
\pgfsetstrokecolor{dialinecolor}
\pgfsetstrokeopacity{1.000000}
\draw (15.049683\du,7.451225\du)--(19.000000\du,9.000000\du);
}
\pgfsetlinewidth{0.100000\du}
\pgfsetdash{}{0pt}
\pgfsetmiterjoin
\pgfsetbuttcap
{
\definecolor{diafillcolor}{rgb}{0.000000, 0.000000, 0.000000}
\pgfsetfillcolor{diafillcolor}
\pgfsetfillopacity{1.000000}
% was here!!!
{\pgfsetcornersarced{\pgfpoint{0.000000\du}{0.000000\du}}\definecolor{dialinecolor}{rgb}{0.000000, 0.000000, 0.000000}
\pgfsetstrokecolor{dialinecolor}
\pgfsetstrokeopacity{1.000000}
\draw (1.000000\du,11.511111\du)--(15.000000\du,13.000000\du)--(15.000000\du,13.000000\du)--(29.000000\du,11.511111\du);
}}
\pgfsetlinewidth{0.100000\du}
\pgfsetdash{}{0pt}
\pgfsetbuttcap
{
\definecolor{diafillcolor}{rgb}{0.000000, 0.000000, 0.000000}
\pgfsetfillcolor{diafillcolor}
\pgfsetfillopacity{1.000000}
% was here!!!
\pgfsetarrowsend{stealth}
\definecolor{dialinecolor}{rgb}{0.000000, 0.000000, 0.000000}
\pgfsetstrokecolor{dialinecolor}
\pgfsetstrokeopacity{1.000000}
\draw (15.000000\du,15.511111\du)--(15.022500\du,17.000000\du);
}
\end{tikzpicture}
\normalsize}
  \caption{TLS1.2 MAC-Encode-Encrypt}
\label{fig:tls12mee}
\end{figure}

In practice, however, protocols are often more complex than simple encryption and MAC phases, lending themselves to various errors.  
Figure~\ref{fig:tls12mee} shows an example of TLS 1.2, which implemented MAC-Encode-Encrypt mode with associated data (shown in the figure as ``AD").

Notice that the padding is not passed to the MAC, meaning that mangling ciphertexts could potentially propagate errors to the padding that lead to 
similar padding oracle attacks as we've seen previously.  Concretely, an adversary can obtain $C = C_0,C_1,C_2,C_3$ for some header $H$; let $R$ be 
arbitrary $n$ bits.  The adversary then queries $H, R, C_0 \oplus i, C_1$ for increasing values of $i$.  Most of the time, the padding will be 
invalid, and a padding error will be returned.  Eventually, the adversary will guess a ciphertext block with valid padding, and will receive a MAC 
error.  Similar to the padding attacks we have seen previously, this leaks information about bits at the end of the original ciphertext block, now 
treated as padidng by the decryption algorithm.  Complicating this attack in practice is the implementation detail that all errors will end TLS 
sessions, requiring an adversary to open multiple sessions that reuse $H, R, C_1$; this is in fact not possible in the implemented protocol, so a 
multi-session attack that attempts the attacks on several different sessions would be required.

Interestingly, this exact construction spawned debate within the academic community on whether TLS 1.2 was secure.  An analysis by Krawczyk 
in~\cite{Krawczyk2001} concluded that this MAC-Encode-Encrypt mode as used in SSL was safe as used in practice, though not perhaps theoretically safe 
in general. Surprisingly, two years later, Canvel et. al showed a password recovery attack in the above protocol using a classic padding oracle 
attack, described in~\cite{Canvel2003}.


\begin{figure}[h]
\centering
\hfpagesss{.2}{.2}{.3}{
\underline{$\RORCCA1^\advA_{\AEAD}$}\\[1pt]
$K \getsr \kg$\\
$b' \getsr \advA^{\EncOracle,\DecOracle}$\\
Ret $b'$\medskip

\underline{$\EncOracle(H,M)$}\\
$C \getsr \enc_K(H,M)$\\
$\calC \gets \calC \cup \{(H,C)\}$\\
Ret $C$\medskip

\underline{$\DecOracle(H,C)$}\\
If $(H,C) \in \calC$ then \\
\myInd Ret $\bot$\\
Ret \color{blue}$\dec_K(H,C)$\color{black}
}{
\underline{$\RORCCA0^\advA_{\AEAD}$}\\[1pt]
$b' \getsr \advA^{\EncOracle,\DecOracle}$\\
Ret $b'$\medskip

\underline{$\EncOracle(H,M)$}\\
$C \getsr \bits^{\ctxtlen(|M|)}$\\
Ret $C$\medskip

\underline{$\DecOracle(H,C)$}\\
Ret \color{red}$\bot$\color{black}
}{
\underline{\color{blue}$Dec(K_1,K_2,H,C):$\color{black}}\\[1pt]
$M \gets$ CBC-Dec($K,C$)\\
$P \gets$  RemoveLastByte($M$)\\
while $i <$ int($P$):\\
\myInd $P’$ <- RemoveLastByte($M$)\\
\myInd If $P’ \neq P$ then \\
\myInd \myInd Return \color{red}PAD\_ERROR\color{black}\\
\myInd $i$++\\
$T$  <- RemoveLast20Bytes($M$)\\
If HMAC($K_1,H,M$) $\neq T$ then\\
\myInd Return \color{red}MAC\_ERROR\color{black}\\
Return $M$\\
}
%\scalebox{0.6}{\Huge
% Graphic for TeX using PGF
% Title: /home/phil/tls.dia
% Creator: Dia v0.97+git
% CreationDate: Fri Mar  8 17:47:30 2019
% For: phil
% \usepackage{tikz}
% The following commands are not supported in PSTricks at present
% We define them conditionally, so when they are implemented,
% this pgf file will use them.
\ifx\du\undefined
  \newlength{\du}
\fi
\setlength{\du}{15\unitlength}
\begin{tikzpicture}[even odd rule]
\pgftransformxscale{1.000000}
\pgftransformyscale{-1.000000}
\definecolor{dialinecolor}{rgb}{0.000000, 0.000000, 0.000000}
\pgfsetstrokecolor{dialinecolor}
\pgfsetstrokeopacity{1.000000}
\definecolor{diafillcolor}{rgb}{1.000000, 1.000000, 1.000000}
\pgfsetfillcolor{diafillcolor}
\pgfsetfillopacity{1.000000}
\pgfsetlinewidth{0.100000\du}
\pgfsetdash{}{0pt}
\pgfsetmiterjoin
{\pgfsetcornersarced{\pgfpoint{0.000000\du}{0.000000\du}}\definecolor{diafillcolor}{rgb}{1.000000, 1.000000, 1.000000}
\pgfsetfillcolor{diafillcolor}
\pgfsetfillopacity{1.000000}
\fill (3.000000\du,1.000000\du)--(3.000000\du,3.511111\du)--(7.250000\du,3.511111\du)--(7.250000\du,1.000000\du)--cycle;
}{\pgfsetcornersarced{\pgfpoint{0.000000\du}{0.000000\du}}\definecolor{dialinecolor}{rgb}{0.000000, 0.000000, 0.000000}
\pgfsetstrokecolor{dialinecolor}
\pgfsetstrokeopacity{1.000000}
\draw (3.000000\du,1.000000\du)--(3.000000\du,3.511111\du)--(7.250000\du,3.511111\du)--(7.250000\du,1.000000\du)--cycle;
}% setfont left to latex
\definecolor{dialinecolor}{rgb}{0.000000, 0.000000, 0.000000}
\pgfsetstrokecolor{dialinecolor}
\pgfsetstrokeopacity{1.000000}
\definecolor{diafillcolor}{rgb}{0.000000, 0.000000, 0.000000}
\pgfsetfillcolor{diafillcolor}
\pgfsetfillopacity{1.000000}
\node[anchor=base,inner sep=0pt, outer sep=0pt,color=dialinecolor] at (5.125000\du,2.600000\du){AD};
\pgfsetlinewidth{0.100000\du}
\pgfsetdash{}{0pt}
\pgfsetmiterjoin
{\pgfsetcornersarced{\pgfpoint{0.000000\du}{0.000000\du}}\definecolor{diafillcolor}{rgb}{0.784314, 0.960784, 0.772549}
\pgfsetfillcolor{diafillcolor}
\pgfsetfillopacity{1.000000}
\fill (7.000000\du,1.000000\du)--(7.000000\du,3.511111\du)--(21.045000\du,3.511111\du)--(21.045000\du,1.000000\du)--cycle;
}{\pgfsetcornersarced{\pgfpoint{0.000000\du}{0.000000\du}}\definecolor{dialinecolor}{rgb}{0.000000, 0.000000, 0.000000}
\pgfsetstrokecolor{dialinecolor}
\pgfsetstrokeopacity{1.000000}
\draw (7.000000\du,1.000000\du)--(7.000000\du,3.511111\du)--(21.045000\du,3.511111\du)--(21.045000\du,1.000000\du)--cycle;
}% setfont left to latex
\definecolor{dialinecolor}{rgb}{0.000000, 0.000000, 0.000000}
\pgfsetstrokecolor{dialinecolor}
\pgfsetstrokeopacity{1.000000}
\definecolor{diafillcolor}{rgb}{0.000000, 0.000000, 0.000000}
\pgfsetfillcolor{diafillcolor}
\pgfsetfillopacity{1.000000}
\node[anchor=base,inner sep=0pt, outer sep=0pt,color=dialinecolor] at (14.022500\du,2.600000\du){payload};
\pgfsetlinewidth{0.100000\du}
\pgfsetdash{}{0pt}
\pgfsetmiterjoin
{\pgfsetcornersarced{\pgfpoint{0.000000\du}{0.000000\du}}\definecolor{diafillcolor}{rgb}{0.384314, 0.905882, 0.470588}
\pgfsetfillcolor{diafillcolor}
\pgfsetfillopacity{1.000000}
\fill (9.000000\du,5.000000\du)--(9.000000\du,7.511111\du)--(15.000000\du,7.511111\du)--(15.000000\du,5.000000\du)--cycle;
}{\pgfsetcornersarced{\pgfpoint{0.000000\du}{0.000000\du}}\definecolor{dialinecolor}{rgb}{0.000000, 0.000000, 0.000000}
\pgfsetstrokecolor{dialinecolor}
\pgfsetstrokeopacity{1.000000}
\draw (9.000000\du,5.000000\du)--(9.000000\du,7.511111\du)--(15.000000\du,7.511111\du)--(15.000000\du,5.000000\du)--cycle;
}% setfont left to latex
\definecolor{dialinecolor}{rgb}{0.000000, 0.000000, 0.000000}
\pgfsetstrokecolor{dialinecolor}
\pgfsetstrokeopacity{1.000000}
\definecolor{diafillcolor}{rgb}{0.000000, 0.000000, 0.000000}
\pgfsetfillcolor{diafillcolor}
\pgfsetfillopacity{1.000000}
\node[anchor=base,inner sep=0pt, outer sep=0pt,color=dialinecolor] at (12.000000\du,6.600000\du){MAC};
\pgfsetlinewidth{0.100000\du}
\pgfsetdash{}{0pt}
\pgfsetmiterjoin
{\pgfsetcornersarced{\pgfpoint{0.000000\du}{0.000000\du}}\definecolor{diafillcolor}{rgb}{1.000000, 1.000000, 1.000000}
\pgfsetfillcolor{diafillcolor}
\pgfsetfillopacity{1.000000}
\fill (3.000000\du,17.000000\du)--(3.000000\du,19.511111\du)--(8.080000\du,19.511111\du)--(8.080000\du,17.000000\du)--cycle;
}{\pgfsetcornersarced{\pgfpoint{0.000000\du}{0.000000\du}}\definecolor{dialinecolor}{rgb}{0.000000, 0.000000, 0.000000}
\pgfsetstrokecolor{dialinecolor}
\pgfsetstrokeopacity{1.000000}
\draw (3.000000\du,17.000000\du)--(3.000000\du,19.511111\du)--(8.080000\du,19.511111\du)--(8.080000\du,17.000000\du)--cycle;
}% setfont left to latex
\definecolor{dialinecolor}{rgb}{0.000000, 0.000000, 0.000000}
\pgfsetstrokecolor{dialinecolor}
\pgfsetstrokeopacity{1.000000}
\definecolor{diafillcolor}{rgb}{0.000000, 0.000000, 0.000000}
\pgfsetfillcolor{diafillcolor}
\pgfsetfillopacity{1.000000}
\node[anchor=base,inner sep=0pt, outer sep=0pt,color=dialinecolor] at (5.540000\du,18.600000\du){header};
\pgfsetlinewidth{0.100000\du}
\pgfsetdash{}{0pt}
\pgfsetmiterjoin
{\pgfsetcornersarced{\pgfpoint{0.000000\du}{0.000000\du}}\definecolor{diafillcolor}{rgb}{1.000000, 1.000000, 1.000000}
\pgfsetfillcolor{diafillcolor}
\pgfsetfillopacity{1.000000}
\fill (8.000000\du,17.000000\du)--(8.000000\du,19.511111\du)--(22.045000\du,19.511111\du)--(22.045000\du,17.000000\du)--cycle;
}{\pgfsetcornersarced{\pgfpoint{0.000000\du}{0.000000\du}}\definecolor{dialinecolor}{rgb}{0.000000, 0.000000, 0.000000}
\pgfsetstrokecolor{dialinecolor}
\pgfsetstrokeopacity{1.000000}
\draw (8.000000\du,17.000000\du)--(8.000000\du,19.511111\du)--(22.045000\du,19.511111\du)--(22.045000\du,17.000000\du)--cycle;
}% setfont left to latex
\definecolor{dialinecolor}{rgb}{0.000000, 0.000000, 0.000000}
\pgfsetstrokecolor{dialinecolor}
\pgfsetstrokeopacity{1.000000}
\definecolor{diafillcolor}{rgb}{0.000000, 0.000000, 0.000000}
\pgfsetfillcolor{diafillcolor}
\pgfsetfillopacity{1.000000}
\node[anchor=base,inner sep=0pt, outer sep=0pt,color=dialinecolor] at (15.022500\du,18.600000\du){ciphertext};
\pgfsetlinewidth{0.100000\du}
\pgfsetdash{}{0pt}
\pgfsetmiterjoin
{\pgfsetcornersarced{\pgfpoint{0.000000\du}{0.000000\du}}\definecolor{diafillcolor}{rgb}{0.980392, 0.917647, 0.596078}
\pgfsetfillcolor{diafillcolor}
\pgfsetfillopacity{1.000000}
\fill (15.000000\du,9.000000\du)--(15.000000\du,11.511111\du)--(23.000000\du,11.511111\du)--(23.000000\du,9.000000\du)--cycle;
}{\pgfsetcornersarced{\pgfpoint{0.000000\du}{0.000000\du}}\definecolor{dialinecolor}{rgb}{0.000000, 0.000000, 0.000000}
\pgfsetstrokecolor{dialinecolor}
\pgfsetstrokeopacity{1.000000}
\draw (15.000000\du,9.000000\du)--(15.000000\du,11.511111\du)--(23.000000\du,11.511111\du)--(23.000000\du,9.000000\du)--cycle;
}% setfont left to latex
\definecolor{dialinecolor}{rgb}{0.000000, 0.000000, 0.000000}
\pgfsetstrokecolor{dialinecolor}
\pgfsetstrokeopacity{1.000000}
\definecolor{diafillcolor}{rgb}{0.000000, 0.000000, 0.000000}
\pgfsetfillcolor{diafillcolor}
\pgfsetfillopacity{1.000000}
\node[anchor=base,inner sep=0pt, outer sep=0pt,color=dialinecolor] at (19.000000\du,10.600000\du){MAC tag};
\pgfsetlinewidth{0.100000\du}
\pgfsetdash{}{0pt}
\pgfsetmiterjoin
{\pgfsetcornersarced{\pgfpoint{0.000000\du}{0.000000\du}}\definecolor{diafillcolor}{rgb}{0.784314, 0.960784, 0.772549}
\pgfsetfillcolor{diafillcolor}
\pgfsetfillopacity{1.000000}
\fill (1.000000\du,9.000000\du)--(1.000000\du,11.511111\du)--(15.045000\du,11.511111\du)--(15.045000\du,9.000000\du)--cycle;
}{\pgfsetcornersarced{\pgfpoint{0.000000\du}{0.000000\du}}\definecolor{dialinecolor}{rgb}{0.000000, 0.000000, 0.000000}
\pgfsetstrokecolor{dialinecolor}
\pgfsetstrokeopacity{1.000000}
\draw (1.000000\du,9.000000\du)--(1.000000\du,11.511111\du)--(15.045000\du,11.511111\du)--(15.045000\du,9.000000\du)--cycle;
}% setfont left to latex
\definecolor{dialinecolor}{rgb}{0.000000, 0.000000, 0.000000}
\pgfsetstrokecolor{dialinecolor}
\pgfsetstrokeopacity{1.000000}
\definecolor{diafillcolor}{rgb}{0.000000, 0.000000, 0.000000}
\pgfsetfillcolor{diafillcolor}
\pgfsetfillopacity{1.000000}
\node[anchor=base,inner sep=0pt, outer sep=0pt,color=dialinecolor] at (8.022500\du,10.600000\du){payload};
\pgfsetlinewidth{0.100000\du}
\pgfsetdash{}{0pt}
\pgfsetmiterjoin
{\pgfsetcornersarced{\pgfpoint{0.000000\du}{0.000000\du}}\definecolor{diafillcolor}{rgb}{1.000000, 0.000000, 0.000000}
\pgfsetfillcolor{diafillcolor}
\pgfsetfillopacity{1.000000}
\fill (23.000000\du,9.000000\du)--(23.000000\du,11.511111\du)--(29.000000\du,11.511111\du)--(29.000000\du,9.000000\du)--cycle;
}{\pgfsetcornersarced{\pgfpoint{0.000000\du}{0.000000\du}}\definecolor{dialinecolor}{rgb}{0.000000, 0.000000, 0.000000}
\pgfsetstrokecolor{dialinecolor}
\pgfsetstrokeopacity{1.000000}
\draw (23.000000\du,9.000000\du)--(23.000000\du,11.511111\du)--(29.000000\du,11.511111\du)--(29.000000\du,9.000000\du)--cycle;
}% setfont left to latex
\definecolor{dialinecolor}{rgb}{0.000000, 0.000000, 0.000000}
\pgfsetstrokecolor{dialinecolor}
\pgfsetstrokeopacity{1.000000}
\definecolor{diafillcolor}{rgb}{0.000000, 0.000000, 0.000000}
\pgfsetfillcolor{diafillcolor}
\pgfsetfillopacity{1.000000}
\node[anchor=base,inner sep=0pt, outer sep=0pt,color=dialinecolor] at (26.000000\du,10.600000\du){padding};
\pgfsetlinewidth{0.100000\du}
\pgfsetdash{}{0pt}
\pgfsetmiterjoin
{\pgfsetcornersarced{\pgfpoint{0.000000\du}{0.000000\du}}\definecolor{diafillcolor}{rgb}{0.788235, 0.745098, 0.921569}
\pgfsetfillcolor{diafillcolor}
\pgfsetfillopacity{1.000000}
\fill (12.000000\du,13.000000\du)--(12.000000\du,15.511111\du)--(18.000000\du,15.511111\du)--(18.000000\du,13.000000\du)--cycle;
}{\pgfsetcornersarced{\pgfpoint{0.000000\du}{0.000000\du}}\definecolor{dialinecolor}{rgb}{0.000000, 0.000000, 0.000000}
\pgfsetstrokecolor{dialinecolor}
\pgfsetstrokeopacity{1.000000}
\draw (12.000000\du,13.000000\du)--(12.000000\du,15.511111\du)--(18.000000\du,15.511111\du)--(18.000000\du,13.000000\du)--cycle;
}% setfont left to latex
\definecolor{dialinecolor}{rgb}{0.000000, 0.000000, 0.000000}
\pgfsetstrokecolor{dialinecolor}
\pgfsetstrokeopacity{1.000000}
\definecolor{diafillcolor}{rgb}{0.000000, 0.000000, 0.000000}
\pgfsetfillcolor{diafillcolor}
\pgfsetfillopacity{1.000000}
\node[anchor=base,inner sep=0pt, outer sep=0pt,color=dialinecolor] at (15.000000\du,14.600000\du){Enc};
\pgfsetlinewidth{0.100000\du}
\pgfsetdash{}{0pt}
\pgfsetmiterjoin
\pgfsetbuttcap
{
\definecolor{diafillcolor}{rgb}{0.000000, 0.000000, 0.000000}
\pgfsetfillcolor{diafillcolor}
\pgfsetfillopacity{1.000000}
% was here!!!
{\pgfsetcornersarced{\pgfpoint{0.000000\du}{0.000000\du}}\definecolor{dialinecolor}{rgb}{0.000000, 0.000000, 0.000000}
\pgfsetstrokecolor{dialinecolor}
\pgfsetstrokeopacity{1.000000}
\draw (3.000000\du,3.511111\du)--(12.022500\du,5.000000\du)--(12.022500\du,5.000000\du)--(21.045000\du,3.511111\du);
}}
\pgfsetlinewidth{0.100000\du}
\pgfsetdash{}{0pt}
\pgfsetbuttcap
{
\definecolor{diafillcolor}{rgb}{0.000000, 0.000000, 0.000000}
\pgfsetfillcolor{diafillcolor}
\pgfsetfillopacity{1.000000}
% was here!!!
\pgfsetarrowsend{stealth}
\definecolor{dialinecolor}{rgb}{0.000000, 0.000000, 0.000000}
\pgfsetstrokecolor{dialinecolor}
\pgfsetstrokeopacity{1.000000}
\draw (15.049683\du,7.451225\du)--(19.000000\du,9.000000\du);
}
\pgfsetlinewidth{0.100000\du}
\pgfsetdash{}{0pt}
\pgfsetmiterjoin
\pgfsetbuttcap
{
\definecolor{diafillcolor}{rgb}{0.000000, 0.000000, 0.000000}
\pgfsetfillcolor{diafillcolor}
\pgfsetfillopacity{1.000000}
% was here!!!
{\pgfsetcornersarced{\pgfpoint{0.000000\du}{0.000000\du}}\definecolor{dialinecolor}{rgb}{0.000000, 0.000000, 0.000000}
\pgfsetstrokecolor{dialinecolor}
\pgfsetstrokeopacity{1.000000}
\draw (1.000000\du,11.511111\du)--(15.000000\du,13.000000\du)--(15.000000\du,13.000000\du)--(29.000000\du,11.511111\du);
}}
\pgfsetlinewidth{0.100000\du}
\pgfsetdash{}{0pt}
\pgfsetbuttcap
{
\definecolor{diafillcolor}{rgb}{0.000000, 0.000000, 0.000000}
\pgfsetfillcolor{diafillcolor}
\pgfsetfillopacity{1.000000}
% was here!!!
\pgfsetarrowsend{stealth}
\definecolor{dialinecolor}{rgb}{0.000000, 0.000000, 0.000000}
\pgfsetstrokecolor{dialinecolor}
\pgfsetstrokeopacity{1.000000}
\draw (15.000000\du,15.511111\du)--(15.022500\du,17.000000\du);
}
\end{tikzpicture}
\normalsize}
  \caption{Differences between ROR-CCA Model and Implementation Leakage}
\label{fig:modeldifferences}
\end{figure}


This disparity in two different analyses of the same protocol came down to a small difference in the model used to prove security, shown in 
Figure~\ref{fig:modeldifferences}.  Note that in the implementation of the model (rightmost box), two different errors are returned for a padding or 
MAC error, allowing for padding oracle attacks.  In the more abstract theoretical model (left two boxes), however, only one kind of error, $\bot$, 
could be returned by the decryption $\DecOracle(H,C)$ (and therefore by the subroutine $\dec_K(H,C)$ in cases of failure).  This means that, 
implicitly, any security analysis at the level of abstraction of the model implicitly assumed no adversary could distinguish these kinds of errors in 
decryption, a crucial ability an adversary would requires for a padding attack.  The implementation, unlike the model, provided adversaries this 
ability.

The failure of a security proof of TLS1.2 to consider this potential flaw due to restrictions in the model shows that not even formally proved and 
correctly implemented cryptography is a silver bullet: models always come with limitations, and it is critical to introspect on whether they 
accurately reflect the systems that implement them.


\begin{figure}[h]
\centering
\scalebox{.7}{\includegraphics{aeinpractice/timing_tls}}
  \caption{Clear timing differences across error types in TLS provide a side-channel}
\label{fig:timing}
\end{figure}

The particular attack highighted in~\cite{Canvel2003} was fixed by RFC 5246, which required all padding errors to emit a MAC error to bring them in 
line with the secure theoretical model.  Despite this, channels for leaking information outside the scope of the model persist, including timing 
errors highlighted by the same work.  A clear separation in the timing of various error sources in TLS 1.2 was described by~\cite{Canvel2003}, and is 
shown in Figure~\ref{fig:timing}.  So, even returning the same error message may not suffice, and indeed RFC 5246 attempted to address this issue by 
requiring that ``implementations MUST ensure that record processing time is essentially the same whether or not the padding is correct."  The RFC 
noted that the MAC's performance differences under different data sizes still left a timing channel in place, but this was not believed to be 
exploitable.  In 2013, this was shown to be false, and another timing attack was shown in RFC 5246-compliant implementations~\cite{Fardan2013}.

TLS is not unique; SSH protocols underwent a similar series of attacks and ad-hoc countermeasures.  Bellare et al showed an attack as a case study of 
their original work on secure authenticated encryption in~\cite{Bellare2004}.  Specifically, the SSH binary packet protocol leaks the initialization 
vector of the next message to be encrypted to an adversary.  If the adversary can also control the full first block of input, the adversary 
essentially chooses the full input to the underlying deterministic block cipher, which can leak input about whether a queried message block was 
encrypted before, leaking plaintext information. The same work proposed a series of modifications to the protocol fixing the attack, claiming the new 
notion satisfied the most rigorous definition of provable security.

Similarly to the TLS case, an attack was later found in the ``fixed" scheme by Albrecht et al. in~\cite{Albrecht2009}, whereby a lack of ciphertext 
integrity in the scheme allowed an adversary to control the length field of the protocol's encoding, which had to be decoded before the message used 
to calculate the MAC was computed.  The value of this length field would then affect timings of SSH packets, which could be measured to infer its 
value.  This attack was missed by the original analysis of Bellare et al. that claimed the SSH protocol was secure, as there was no model for timing 
side channels in their adversary, and all decryption failure errors were treated as indistinguishable failures (sound familiar?).  These gaps were 
finally bridged by a formal analysis in~\cite{Paterson2010}.

By feeding random blocks of ciphertext in as the first block of a new session, and letting the protocol interpret them as a potential length field 
before throwing a MAC error, leakage about the plaintext data that originally formed these ciphertext blocks could be recovered.  Like the TLS case, 
this shows the danger of data encodings interacting with cryptographic primitives that do not provide appropriate authentication guarantees; as a 
practitioner implementing protocols based on cryptography, it is critical to rigorously evaluate the concrete security provided by any primitives used 
in the context of other interacting components, parsers, etc.

\subsection{Authenticated Encryption Extensions}

Because of the obvious popularity of authenticated encryption in securing communications, and its importance to deployed protocols, a number of 
variants of authenticated encryption and their associated security models have been studied in the literature.  We briefly mention a few, as well as 
provide intuition for key trade-offs in their design.

\textbf{Stateful encryption} All the constructions and games described in this chapter have assumed a random initialization vector used to provide 
security to e.g. repeated plaintext blocks.  The stateful variants of such schemes instead assume a recipient maintains some state, and do not 
transmit the initialization vector as the first block of ciphertext, instead using this state.  Security models for such schemes are complicated, as 
they need to handle e.g. reordering of messages from the sender to the receiver, ensuring no data is leaked.

\textbf{Nonce-Based AE(AD)} One natural question is whether AD schemes can be made deterministic, removing the need for randomness.  Currently, our 
e.g. ROR definitions require a random initialization vector to hide correlations in repeated message values.  An alternative construction first 
explored by Rogaway et. al in 2004~\cite{Rogaway2004} is the use of \emph{nonces}, or values used only once, in the place of initialization vectors. 
In these schemes, extended security definitions make the additional assumption that no nonce is used twice across encryptions.  In the event that a 
nonce is used twice, no security is provided whatsoever, regardless of whether the same or different messages are encrypted.  Additionally, such 
constructions have the benefit that good randomness is not required during encryption, removing many PRNG-based attacks which have led to IV reuse in 
practice.

\textbf{Misuse-resistant nonce-based AE(AD)} A natural extension of nonce-based AE is to extend the security definitions to cases where the nonce 
\emph{is} reused, minimizing the potential for damage in the event of misuse (or equivalently, in the case of IV reuse in a randomized setting).  Any 
nonce that is reused with the same message in a deterministic scheme will obviously leak plaintext equality, so leaking only this is both a natural 
and optimal security definition for such schemes.  To achieve this, we modify the security definitions shown to allow the adversary to make any number 
of encryption and decryption queries that may repeat IVs, and require encryption oracle responses to appear random except in the case of a repetition 
in all of $(header, IV, message)$.  Such schemes were first explored in depth in~\cite{Rogaway2006}.


\textbf{Trade-offs in Misuse-resistant nonce-based AE(AD)} Interestingly, the AE constructions we have explored showcase fundamental security 
trade-offs, as briefly explored in~\cite{Rogaway2006}.  In an encryption scheme used e.g. in an online communication protocol, it may be desireable 
for an encryption algorithm to return partial ciphertext results as plaintext blocks are encrypted, without waiting for the full message.  Such 
constructions are especially useful for resource constrained contexts, where maintaining state may be costly.  In such constructions, the ciphertext 
of the first block $C_1$ cannot be affected by subsequent message $M_n, n>1$ or ciphertext blocks $C_n, n>1$, as the algorithm outputs $C_1$ before 
receiving any such information; we refer to such schemes as \emph{online} schemes, or \emph{feed forward} schemes.  Obviously, this characteristic of 
an online deterministic encryption scheme means that plaintext prefix equality can be leaked when nonces are misused.  The intuition for this is 
clear; consider two messages that differ only in the last block, and note that if a nonce is reused in any deterministic feed forward construction 
over these messages, the first output block cannot depend on the last by nature of feed forward.  Indeed, online output will inherently be identical 
up to the first difference in the message input, leaking exactly equality of prefixes.

Offline constructions feed ciphertexts output by the first pass backwards, to ensure that each block of the message depends on every other block; in 
the above example, the last message block now affects every block of ciphertext.  While such constructions are inherently less efficient, as the first 
pass must complete before the second pass which depends on its full output, they are also inherently more secure, achieving the maximum possible 
security for deterministic encryption.  Nothing is leaked except in the case of re-used $(header, IV, message)$, which causes plaintext equality to 
leak.  Such leaks are unavoidable in any deterministic construction where all input is re-used.

\subsection{Choices for Practitioners}

\begin{table}[h]
\scalebox{.8}{
\begin{tabular}{llllll}
\textbf{Mode}     & \textbf{Inventors} & \textbf{Notes}                                                   & \textbf{Nonce-based} & \textbf{Misuse-resistant} & \textbf{Robust} \\
OCB               & Rogaway            & One-pass                                                         & \color{green}Yes\color{black}                  & No                        & No              \\
\specialcell{AES-CTR\\-then-HMAC} &                    &                                                                  & No                   & No                        & \color{green}Yes\color{black}*            \\
GCM               & McGrew, Viega      & \specialcell{CTR mode with Universal\\Hash Function-based MAC}        & \color{green}Yes\color{black}                  & No                        & No              \\
SIV               & Rogaway, Shrimpton & Not online, two full passes                                      & \color{green}Yes\color{black}                  & \color{green}Yes\color{black}                       & No              \\
AES-GCM-SIV       & Gueron, Lindell    & \specialcell{SIV using AES-GCM\\with special tweaks} & \color{green}Yes\color{black}                  & \color{green}Yes\color{black}                       & No              \\
ASCON             & Dobraunig et al.   & Based on lightweight cipher                                      & \color{green}Yes\color{black}                  & \color{orange}Except prefix\color{black}             & \color{green}Yes\color{black}             \\
AEGIS             & Wu, Preneel        & Uses AES round function                                          & \color{green}Yes\color{black}?                 & No                        & ???             \\
COLM              & Andreeva et al.    & Parallelizable cipher      & \color{green}Yes\color{black}                  & \color{orange}Except prefix\color{black}             & No?            
\end{tabular}
}
  \caption{Clear timing differences across error types in TLS provide a side-channel}
\label{table:practicalae}
\end{table}


Table~\ref{table:practicalae} summarizes several modes that have been cryptographically analyzed for security, as well as their inventors and notes on 
algorithm characteristics.

\subsection{Questions [Proofreaders; any suggestions?]}

\begin{itemize}
\item Consider the American Standards Committee ANSX9.102 document available at \url{https://eprint.iacr.org/2004/340.pdf}, which formalizes keywrap protocols using Misuse-resistant nonce-based AE(AD) as described above.  Categorize AKW1 and AKW2 as online or offline algorithms, and state precisely what is leaked by each in the event of both correct operation and misuse.
\item Consider the key reuse adversary described by Figure~\ref{fig:reuseadversary}, and the scheme described by Figure~\ref{fig:reusescheme}.  Extend the provided attack to a version of the scheme which only allows for messages with two message blocks (of length $2n$) by providing and justifying an adversary that wins the CTXT game.
\end{itemize}

\newpage
%%%%%%%%%%%%%%%%%%%%%%%%%%%%%%%%%%%%%%%%%%%%%%%%%%%%%%%%%%%%%%%%%%%%%%%%%%%%%%%%
\section{Cryptographic Hash Functions}
\label{sec:hashfunctions}

A cryptographic hash function is a map $H\Colon\msgspace\rightarrow\bits^n$ for
some set $\msgspace$ and number $n > 0$. Typical values of $n$ are these days
256, 512, etc., and most hash functions support an essentially unlimited messgae
space $\msgspace$, such as all strings of length up to $2^{64}-1$.

\fpage{.25}{
\underline{$\CR^\advA_{H}$}\\[1pt]
$(M,M') \getsr \advA$\\
If $M = M'$ then Ret $\false$\\
Ret $H(M) = H(M')$
}


\bnm
\AdvCR{H}{\advA} = \Prob{\CR^\advA_H\Rightarrow\true}
\enm




\fpage{.25}{
\underline{$\CR^\advA_{H,\ic}$}\\[1pt]
$E \getsr \blockciphers(k,n)$\\
$(M,M') \getsr \advA^{\ic,\icInv}$\\
If $M = M'$ then Ret $\false$\\
Ret $H^\ic(M) = H^\ic(M')$\medskip

\underline{$\ic(K,X)$}\\
Ret $\cipherE(K,X)$\medskip

\underline{$\icInv(K,Y)$}\\
Ret $\cipherD(K,Y)$
}

\begin{theorem*}
Let $f$ be the Davies-Meyer compression function built from a block cipher
$\cipherE\Colon\bits^k\times\bits^n\rightarrow\bits^n$ modeled as an ideal
cipher. For any adversary $\advA$ making at most~$q$ queries it holds that
\bnm
  \AdvCR{f}{\advA} \le \frac{(q+1)(q+2)}{2^n} \;.
\enm
\end{theorem*}

\begin{proof}
Assume $\advA$ that outputs $(Y,M)$ and $(Y',M')$ has already queried $\cipherE$
or $\cipherD$ for associated points. This can be argued easily by reducing to
by making at most $q' = q+2$ queries. We also assume
$\advA$  Then each query defines a triple 
$(K_i,X_i,Y_i)$ where either $Y_i = \cipherE(K_i,X_i)$ was queried or $X_i =
\cipherD(K_i,Y_i)$ was queried. For a collision to occur it must be that there
exist indices $i,j$ such that $X_i \oplus Y_i = X_j \oplus Y_j$, to arrange that
$X_i \oplus \cipherE(K_i,X_i) = X_j \oplus \cipherE(K_j,X_j)$. Let $C_j$ be the
event that such a collision occurs upon a query the $j\thh$ query. Then we have
that $X_i,Y_i$ are at this point fixed values while exactly one of $X_j$ or
$Y_j$ is a fixed value, with the other being a uniformly chosen point subject
only to permutivity. Then we have that 
\begin{align*}
\Prob{\CR_{f,\cipherE}^\advA\Rightarrow\true} 
  &\le \sum_{j=1}^{q'} \Prob{C_j}   \\
  &\le \sum_{j=1}^{q'} \frac{j-1}{2^n-j+1}\\
  &\le \sum_{j=1}^{q'} \frac{j-1}{2^n-q'}\\
  &= \frac{q'(q'-1)}{2(2^n-q')}
  &= \frac{(q+2)(q+1)}{2^n} 
\end{align*}
\end{proof}


\begin{theorem*}
Let $f$ be a compression function and $H$ be the MD hash function built from it. 
Let $\advA$ be a $\CR_H$-adversary outputing messages eacah of length at most $\sigma$
blocks after MD padding. Then we give an $\CR_f$-adversary $\advB$
such that
\bnm
  \AdvCR{\advA}{H} \le \AdvCR{\advB}{f} \;.
\enm
Adversary~$\advB$ runs in time that of $\advA$ plus at most $2\sigma$ computations of
$f$.
\end{theorem*}


\fpage{.25}{
\underline{$\CR^\advA_{H,\ic}$}\\[1pt]
$E \getsr \blockciphers(k,n)$\\
$(M,M') \getsr \advA^{\ic,\icInv}$\\
If $M = M'$ then Ret $\false$\\
Ret $H^\ic(M) = H^\ic(M')$\medskip

\underline{$\ic(K,X)$}\\
If $\TabE[K,X] \ne \bot$ then\\
\myInd $Y \getsr \Range[K]$\\
\myInd $\TabE[K,X] \getsr Y$\\
\myInd $\TabD[K,Y] \getsr X$\\
\myInd $\Domain[K] \getu X$\\
\myInd $\Range[K] \getu Y$\\
Ret $\TabE[K,X]$\medskip

\underline{$\icInv(K,Y)$}\\
If $\TabD[K,Y] \ne \bot$ then\\
\myInd $X \getsr \Domain[K]$\\
\myInd $\TabE[K,X] \gets Y$\\
\myInd $\TabD[K,Y] \gets X$\\
\myInd $\Domain[K] \getu X$\\
\myInd $\Range[K] \getu Y$\\
Ret $\TabD[K,Y]$
}

\newpage
%%%%%%%%%%%%%%%%%%%%%%%%%%%%%%%%%%%%%%%%%%%%%%%%%%%%%%%%%%%%%%%%%%%%%%%%%%%%%%%%
\section{Further Security Properties for Hash Functions}
\label{sec:hashfunctions}

\fpage{.25}{
\underline{$\OWF^\advA_{H}$}\\[1pt]
$M \getsr \msgspace$\\
$Y \gets H(M)$\\
$M' \getsr \advA(Y)$\\
Ret $(H(M') = Y)$
}


\bnm
\AdvOWF{H}{\advA} = \Prob{\OWF^\advA_H\Rightarrow\true}
\enm



\fpage{.25}{
\underline{$\SPR^\advA_{H}$}\\[1pt]
$M \getsr \msgspace$\\
$Y \gets H(M)$\\
$M' \getsr \advA(M)$\\
Ret $(H(M') = H(M)) \land (M \ne M')$
}


\bnm
\AdvSPR{H}{\advA} = \Prob{\SPR^\advA_H\Rightarrow\true}
\enm


\fpage{.25}{
\underline{$\PWR^\advA_{H,p}$}\\[2pt]
$\pw \get{p} \msgspace$\\
$\salt \getsr \bits^n$\\
$Y \gets H(\salt \concat M)$\\
$\pw' \getsr \advA(\salt,Y)$\\
Ret $(\pw' = \pw)$
}

\fpage{.25}{
\underline{$\PWR^\advA_{\Horacle,p}$}\\[2pt]
$\pw \get{p} \msgspace$\\
$\salt \getsr \bits^n$\\
$Y \gets \Horacle(\salt \concat M)$\\
$\pw' \getsr \advA^\Horacle(\salt,Y)$\\
Ret $(\pw' = \pw)$\medskip

\underline{$\Horacle(M)$}\\
If $\TabH[M] = \bot$ then\\
\myInd $\TabH[M] \getsr \bits^n$\\
Ret $\TabH[M]$
}

\begin{theorem*}
Let $H\Colon\msgspace\rightarrow\bits^n$ and $\advA$ be a $\SPR_H$-adversary.
Then we give a $\CR_H$-adversary $\advB$ such that
\bnm 
  \AdvSPR{H}{\advA} \le \AdvCR{H}{\advB}
\enm
Adversary~$\advB$ runs in time that of $\advA$. 
\end{theorem*}


\paragraph{Application: password recovery.}
The min-entropy of a distribution $p$ is the probability that one can guess a
value chosen according to $p$ in a single guess. Mathematically:

\bnm
  \Hinfty(p) = -\max_{x} \log \frac{1}{p(x)}
\enm
Not to be confused with Shannon entropy:

\bnm
  \Hshan(p) =  \sum_x p(x) \log \frac{1}{p(x)}
\enm

\begin{theorem}
Let $\Horacle\Colon\msgspace\rightarrow\bits^n$ be modeled as a random oracle
and $p$ be a distribution over $\msgspace$.
Then for any $\PWR_{\Horacle,p}$-adversary $\advA$ making at most $q$ queries 
it holds that
\bnm
  \AdvPWR{\Horacle,p}{\advA} \le \frac{q}{2^{\mu(p)}} \;.
\enm
\end{theorem}



\hfpages{.25}{
\underline{$\PRF1_{F,\Horacle}^\advA$}\\
$K \getsr \keyspace$\\
$b' \getsr \advA^{\Fn,\Horacle}$\\
Return $b'$\medskip

\underline{$\Fn(M)$}\\
Return $F^\Horacle_K(M)$\medskip

\underline{$\Horacle(X)$}\\
If $\TabH[X] = \bot$ then\\
\myInd $\TabH[X] \getsr \bits^n$\\
Ret $\TabH[X]$

}{
\underline{$\PRF0_{F,\Horacle}^\advA$}\\
$\rho \getsr \Func(\msgspace,n)$\\
$b' \getsr \advA^{\Fn,\Horacle}$\\
Return $b'$\medskip

\underline{$\Fn(M)$}\\
Return $\rho(M)$\\

\underline{$\Horacle(X)$}\\
If $\TabH[X] = \bot$ then\\
\myInd $\TabH[X] \getsr \bits^n$\\
Ret $\TabH[X]$
}


\begin{theorem}
Let  $\Horacle\Colon\msgspace\rightarrow\bits^n$ be modeled as a random oracle
and let $F^\Horacle\Colon\keyspace\times\msgspace\rightarrow\bits^n$ be the
hash-based PRF defined as $F^\Horacle(K,M) = \Horacle(K \concat M)$. 
Then for any $\PRF_{F,\Horacle}$-adversary $\advA$ making at most $q$ queries 
to $\Horacle$ it holds that
\bnm
  \AdvPRF{F,\Horacle}{\advA} \le \frac{q}{|\keyspace|} \;.
\enm
\end{theorem}



\hfpages{.25}{
\underline{$\INDIFF1_{H,\foracle}^\advD$}\\
$b' \getsr \advD^{\Fn,\foracle}$\\
Ret $b'$\medskip

\underline{$\Fn(M)$}\\
Ret $H^\foracle(M)$\medskip

\underline{$\foracle(X)$}\\
If $\Tabf[X] = \bot$ then\\
\myInd $\Tabf[X] \getsr \bits^n$\\
Ret $\Tabf[X]$
}{
\underline{$\INDIFF0_{\Horacle,\simu}^\advD$}\\
$b' \getsr \advD^{\Fn,\simoracle}$\\
Return $b'$\medskip

\underline{$\Fn(M)$}\\
If $\TabH[M] = \bot$ then\\
\myInd $\TabH[M] \getsr \bits^n$\\
Ret $\TabH[M]$\medskip

\underline{$\simoracle(X)$}\\
Ret $\simu^\Fn(X)$
}


\bnm
  \AdvINDIFF{H,\foracle,\simu}{\advA} =
    \left|\Prob{\INDIFF1_{H,\foracle}^\advD\Rightarrow1} -
            \Prob{\INDIFF0_{\Horacle,\simu}^\advD\Rightarrow1}\right|
\enm


\hfpages{.20}{
\underline{$\RKAPRF1_{F,\Phi}^\advA$}\\
$K \getsr \keyspace$\\
$b' \getsr \advA^{\Fn}$\\
Ret $b'$\medskip

\underline{$\Fn(\phi,X)$}\\
If $\phi \notin \Phi$ then\\
\myInd Ret $\bot$\\
Ret $F(\phi(K),X)$
}{
\underline{$\RKAPRF0_{F,\Phi}^\advA$}\\
$K \getsr \keyspace$\\
$\rho \getsr \Func(\keyspace\times\msgspace,n)$\\
$b' \getsr \advA^{\Fn}$\\
Ret $b'$\medskip

\underline{$\Fn(\phi,X)$}\\
If $\phi \notin \Phi$ then\\
\myInd Ret $\bot$\\
Ret $\rho(\phi(K),X)$
}



\newpage
%%%%%%%%%%%%%%%%%%%%%%%%%%%%%%%%%%%%%%%%%%%%%%%%%%%%%%%%%%%%%%%%%%%%%%%%%%%%%%%%
\section{Public Key Encryption}
\label{sec:pke}

A PKE scheme $\AEnc = (\kg,\enc,\dec)$ is a triple of
algorithms. Key generation is randomized and outputs a key pair $(\pk,\sk)$.
Encryption
takes as input a public key $\pk$ and message $M$ and outputs a ciphertext.
Decryption takes as input a secret key $\sk$ and ciphertext $C$ and outputs a
message or a distinguished error symbol $\bot$. 



\fpage{.20}{
		\underline{$\INDCPA_\AEnc^\advA$}\\
    $(\pk,\sk) \getsr \kg$\\
    $b \getsr \bits$\\
    $b' \getsr \advA^\EncOracle(\pk)$\\
		Ret $b'$\medskip

    \underline{$\EncOracle(M_0,M_1)$}\\
    If $|M_0| \ne |M_1|$ then\\
    \myInd Ret $\bot$\\
    $C \getsr \enc(\pk,M_b)$\\
    Ret $C$
	}


\begin{theorem*}
Let $\AEnc$ be a PKE scheme. Let $\advA$ be an $\INDCPA_\AEnc$-adversary making at
most $q$ queries. We give an $\INDCPA_\AEnc$-adversary $\advB$ making one query
such that
\bnm
  \AdvINDCPA{\AEnc}{\advA} \le q\cdotsm\AdvINDCPA{\AEnc}{\advB}
\enm
Adversary~$\advB$ runs in time that of $\advA$ plus the time to perform $(q-1)$
encryptions under $\AEnc$.
\end{theorem*}


\fpage{.20}{
		\underline{$\G_{i^*}$}\\
    $b \getsr \bits$\\
    $(\pk,\sk) \getsr \kg$\\
    $i \gets 1$\\
    $b' \getsr \advA^\EncOracle(\pk)$\\
		Ret $b'$\medskip

    \underline{$\EncSim(M_0,M_1)$}\\
    If $|M_0| \ne |M_1|$ then\\
    \myInd Ret $\bot$\\
    If $i > i^*$ then\\ 
    \myInd $C \getsr \enc(\pk,M_0)$\\
    Else\\
    \myInd $C \getsr \enc(\pk,M_1)$\\
    $i \gets i + 1$\\
    Ret $C$
	}



\fpage{.20}{
		\underline{$\advB^\EncOracle(\pk)$}\\
    $i \gets 1$\\
    $b' \getsr \advA^\EncSim(\pk)$\\
		Ret $b'$\medskip

    \underline{$\EncSim(M_0,M_1)$}\\
    If $|M_0| \ne |M_1|$ then\\
    \myInd Ret $\bot$\\
    If $i > i^*$ then\\ 
    \myInd $C \getsr \enc(\pk,M_1)$\\
    Else if $i = i^*$ then\\
    \myInd $C \gets \EncOracle(M_0,M_1)$\\
    Else\\
    \myInd $C \getsr \enc(\pk,M_0)$\\
    $i \gets i + 1$\\
    Ret $C$
	}


\begin{align*}
\AdvINDCPA{\AEnc}{\advA} 
  &= \left|\Prob{\G_0\Rightarrow1} - \Prob{\G_1\Rightarrow1}\right|\\
  &= \left|\sum_{i=1}^q \Prob{\G_{i-1}\Rightarrow1} - \Prob{\G_i\Rightarrow1}\right| \\
  &= \left|\sum_{i=1}^q 
      \Prob{\INDCPA1^{\advB_i}_\AEnc\Rightarrow 1}
      - \Prob{\INDCPA0^{\advB_i}_\AEnc\Rightarrow 1}\right|
\end{align*}

\begin{align*}
  \AdvINDCPA{\AEnc}{\advB}
    &= \left| \Prob{\INDCPA1^\advB\Rightarrow1} - \Prob{\INDCPA0^\advB\Rightarrow1} \right|\\
    &= \frac{1}{q} \left|
    \sum_{i^*=1}^q\CondProb{\INDCPA1^\advB\Rightarrow1}{j= i^*} -
    \CondProb{\INDCPA0^\advB\Rightarrow1}{j=i^*} \right|\\
    &= \frac{1}{q} \left|
    \sum_{i^*=1}^q\Prob{\INDCPA1^{\advB_{i^*}}\Rightarrow1} - \Prob{\INDCPA0^{\advB_{i^*}}\Rightarrow1} \right|
\end{align*}
  

\begin{align*}
\AdvINDCPA{\AEnc}{\advB} 
  &= \left|\Prob{\INDCPA0_\AEnc^\advB\Rightarrow1} -
                      \Prob{\INDCPA1_\AEnc^\advB\Rightarrow1}\right|\\
  &= \frac{1}{q}\sum_{i=0}^{q-1} \left|\Prob{\G_i\Rightarrow1} - \Prob{\G_{i+1}\Rightarrow1}\right|\\
  &= \frac{1}{q}\left|\Prob{\G_0\Rightarrow1} - \Prob{\G_q\Rightarrow1}\right|
\end{align*}


\fpage{.20}{
		\underline{$\OWF_{\RSAk}$}\\
    $((N,e),(N,d))\getsr \kg(k)$\\
    $X \getsr \Z_N^*$\\
    $Y \gets X^e \bmod N$\\
    $X' \getsr \advA(Y)$\\
		Ret $(X' = X)$
	}

\bnm
  \AdvOWFRSA{\RSAk}{\advA} = \Prob{\OWF_{\RSAk}^\advA\Rightarrow\true}
\enm


\begin{theorem*}
Let $\RSAk$ be the RSA-based scheme using
security parameter $k$, hash function
$\Horacle\Colon\msgspace\rightarrow\bits^n$ modeled as a random oracle, and
symmetric encryption scheme $\SEscheme$. Let $\advA$ be
an $\INDCPA_{\RSAk}$-adversary making at most $q$ queries to
$\Horacle$. Then we give an
$\OWF_{\RSAk}$-adversary $\advB$ and $\INDCPA_\SEscheme$-adversary
$\advC$ such that
\bnm
  \AdvINDCPA{\RSAk,\Horacle}{\advA} \le
      2\cdotsm\AdvOWF{\RSAk}{\advB} +  
        2\cdotsm\AdvROR{\SEscheme}{\advC}  \;.
\enm
Adversaries $\advB,\advC$ run in time that of $\advA$ plus 
the time to simulate $q$ RO queries. Adversary $\advC$ makes a single encryption query.
\end{theorem*}


\hfpagesss{.20}{.20}{.20}{
\underline{$\G_0$}\\
$b \getsr \bits$\\
$((N,e),(N,d)) \getsr \kg(k)$\\
$b' \getsr \advA^{\EncOracle,\Horacle}(N,e)$\\
Ret $b'$\medskip

\underline{$\EncOracle(M_0,M_1)$}\\
$R \getsr \Z^*_N$\\
$C_1 \gets R^e \bmod N$\\
$K \gets \Horacle(R)$\\
$C_2 \getsr \encSym(K,M)$\\
Ret $(C_1,C_2)$\medskip

\underline{$\Horacle(X)$}\\
If $\TabH[X] = \bot$  then\\
\myInd $\TabH[X] \getsr \bits^n$\\
Ret $\TabH[X]$
}{
\underline{\fbox{$\G_1$}\;\;\; $\G_2$}\\
$((N,e),(N,d)) \getsr \kg(k)$\\
$R \getsr \Z^*_N$\\
$C_1 \gets R^e \bmod N$\\
$K \getsr \bits^n$\\
$b \getsr \bits$\\
$b' \getsr \advA^{\EncOracle,\Horacle}(N,e)$\\
Ret $b'$\medskip

\underline{$\EncOracle(M_0,M_1)$}\\
$C_2 \getsr \encSym(K,M)$\\
Ret $(C_1,C_2)$\medskip

\underline{$\Horacle(X)$}\\
If $X = R$ then\\
\myInd $\badtrue$\\
\myInd \fbox{$\TabH[X] \gets K$}\\
If $\TabH[X] = \bot$  then\\
\myInd $\TabH[X] \getsr \bits^n$\\
Ret $\TabH[X]$
}{
\underline{$\G_3$}\\
$((N,e),(N,d)) \getsr \kg(k)$\\
$R \getsr \Z^*_N$\\
$C_1 \gets R^e \bmod N$\\
$K \getsr \bits^n$\\
$b \getsr \bits$\\
$b' \getsr \advA^{\EncOracle,\Horacle}(N,e)$\\
Ret $b'$\medskip

\underline{$\EncOracle(M_0,M_1)$}\\
%$C_2 \getsr \encSym(K,M)$\\
$C_2 \getsr \bits^{\ctxtlen(M_0)}$\\
Ret $(C_1,C_2)$\medskip

\underline{$\Horacle(X)$}\\
If $\TabH[X] = \bot$  then\\
\myInd $\TabH[X] \getsr \bits^n$\\
Ret $\TabH[X]$
}




\hfpages{.2}{
		\underline{$\enc((N,e),M)$:}\\
    $R \getsr \Z^*_N$\\
    $C_1 \gets R^e \bmod N$\\
    $K \gets H(R)$\\
    $C_2 \getsr \encSym(K,M)$\\
		Ret $(C_1,C_2)$
  }{
	  \underline{$\dec((N,d),(C_1,C_2))$:}\\
    $R \gets C_1^d \bmod N$\\
    $K \gets H(R)$\\
    $M \gets \decSym(K,C_2)$\\
		Ret $M$
	}

$\SEscheme = (\kgSym,\encSym,\decSym)$



\printbibliography

%\bibliographystyle{plain}
%\bibliography{notes}


\end{document}
